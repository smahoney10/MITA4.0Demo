% Options for packages loaded elsewhere
% Options for packages loaded elsewhere
\PassOptionsToPackage{unicode}{hyperref}
\PassOptionsToPackage{hyphens}{url}
\PassOptionsToPackage{dvipsnames,svgnames,x11names}{xcolor}
%
\documentclass[
  letterpaper,
  DIV=11,
  numbers=noendperiod]{scrreprt}
\usepackage{xcolor}
\usepackage{amsmath,amssymb}
\setcounter{secnumdepth}{-\maxdimen} % remove section numbering
\usepackage{iftex}
\ifPDFTeX
  \usepackage[T1]{fontenc}
  \usepackage[utf8]{inputenc}
  \usepackage{textcomp} % provide euro and other symbols
\else % if luatex or xetex
  \usepackage{unicode-math} % this also loads fontspec
  \defaultfontfeatures{Scale=MatchLowercase}
  \defaultfontfeatures[\rmfamily]{Ligatures=TeX,Scale=1}
\fi
\usepackage{lmodern}
\ifPDFTeX\else
  % xetex/luatex font selection
\fi
% Use upquote if available, for straight quotes in verbatim environments
\IfFileExists{upquote.sty}{\usepackage{upquote}}{}
\IfFileExists{microtype.sty}{% use microtype if available
  \usepackage[]{microtype}
  \UseMicrotypeSet[protrusion]{basicmath} % disable protrusion for tt fonts
}{}
\makeatletter
\@ifundefined{KOMAClassName}{% if non-KOMA class
  \IfFileExists{parskip.sty}{%
    \usepackage{parskip}
  }{% else
    \setlength{\parindent}{0pt}
    \setlength{\parskip}{6pt plus 2pt minus 1pt}}
}{% if KOMA class
  \KOMAoptions{parskip=half}}
\makeatother
% Make \paragraph and \subparagraph free-standing
\makeatletter
\ifx\paragraph\undefined\else
  \let\oldparagraph\paragraph
  \renewcommand{\paragraph}{
    \@ifstar
      \xxxParagraphStar
      \xxxParagraphNoStar
  }
  \newcommand{\xxxParagraphStar}[1]{\oldparagraph*{#1}\mbox{}}
  \newcommand{\xxxParagraphNoStar}[1]{\oldparagraph{#1}\mbox{}}
\fi
\ifx\subparagraph\undefined\else
  \let\oldsubparagraph\subparagraph
  \renewcommand{\subparagraph}{
    \@ifstar
      \xxxSubParagraphStar
      \xxxSubParagraphNoStar
  }
  \newcommand{\xxxSubParagraphStar}[1]{\oldsubparagraph*{#1}\mbox{}}
  \newcommand{\xxxSubParagraphNoStar}[1]{\oldsubparagraph{#1}\mbox{}}
\fi
\makeatother


\usepackage{longtable,booktabs,array}
\usepackage{calc} % for calculating minipage widths
% Correct order of tables after \paragraph or \subparagraph
\usepackage{etoolbox}
\makeatletter
\patchcmd\longtable{\par}{\if@noskipsec\mbox{}\fi\par}{}{}
\makeatother
% Allow footnotes in longtable head/foot
\IfFileExists{footnotehyper.sty}{\usepackage{footnotehyper}}{\usepackage{footnote}}
\makesavenoteenv{longtable}
\usepackage{graphicx}
\makeatletter
\newsavebox\pandoc@box
\newcommand*\pandocbounded[1]{% scales image to fit in text height/width
  \sbox\pandoc@box{#1}%
  \Gscale@div\@tempa{\textheight}{\dimexpr\ht\pandoc@box+\dp\pandoc@box\relax}%
  \Gscale@div\@tempb{\linewidth}{\wd\pandoc@box}%
  \ifdim\@tempb\p@<\@tempa\p@\let\@tempa\@tempb\fi% select the smaller of both
  \ifdim\@tempa\p@<\p@\scalebox{\@tempa}{\usebox\pandoc@box}%
  \else\usebox{\pandoc@box}%
  \fi%
}
% Set default figure placement to htbp
\def\fps@figure{htbp}
\makeatother





\setlength{\emergencystretch}{3em} % prevent overfull lines

\providecommand{\tightlist}{%
  \setlength{\itemsep}{0pt}\setlength{\parskip}{0pt}}



 


\KOMAoption{captions}{tableheading}
\makeatletter
\@ifpackageloaded{tcolorbox}{}{\usepackage[skins,breakable]{tcolorbox}}
\@ifpackageloaded{fontawesome5}{}{\usepackage{fontawesome5}}
\definecolor{quarto-callout-color}{HTML}{909090}
\definecolor{quarto-callout-note-color}{HTML}{0758E5}
\definecolor{quarto-callout-important-color}{HTML}{CC1914}
\definecolor{quarto-callout-warning-color}{HTML}{EB9113}
\definecolor{quarto-callout-tip-color}{HTML}{00A047}
\definecolor{quarto-callout-caution-color}{HTML}{FC5300}
\definecolor{quarto-callout-color-frame}{HTML}{acacac}
\definecolor{quarto-callout-note-color-frame}{HTML}{4582ec}
\definecolor{quarto-callout-important-color-frame}{HTML}{d9534f}
\definecolor{quarto-callout-warning-color-frame}{HTML}{f0ad4e}
\definecolor{quarto-callout-tip-color-frame}{HTML}{02b875}
\definecolor{quarto-callout-caution-color-frame}{HTML}{fd7e14}
\makeatother
\makeatletter
\@ifpackageloaded{caption}{}{\usepackage{caption}}
\AtBeginDocument{%
\ifdefined\contentsname
  \renewcommand*\contentsname{Table of contents}
\else
  \newcommand\contentsname{Table of contents}
\fi
\ifdefined\listfigurename
  \renewcommand*\listfigurename{List of Figures}
\else
  \newcommand\listfigurename{List of Figures}
\fi
\ifdefined\listtablename
  \renewcommand*\listtablename{List of Tables}
\else
  \newcommand\listtablename{List of Tables}
\fi
\ifdefined\figurename
  \renewcommand*\figurename{Figure}
\else
  \newcommand\figurename{Figure}
\fi
\ifdefined\tablename
  \renewcommand*\tablename{Table}
\else
  \newcommand\tablename{Table}
\fi
}
\@ifpackageloaded{float}{}{\usepackage{float}}
\floatstyle{ruled}
\@ifundefined{c@chapter}{\newfloat{codelisting}{h}{lop}}{\newfloat{codelisting}{h}{lop}[chapter]}
\floatname{codelisting}{Listing}
\newcommand*\listoflistings{\listof{codelisting}{List of Listings}}
\makeatother
\makeatletter
\makeatother
\makeatletter
\@ifpackageloaded{caption}{}{\usepackage{caption}}
\@ifpackageloaded{subcaption}{}{\usepackage{subcaption}}
\makeatother
\usepackage{bookmark}
\IfFileExists{xurl.sty}{\usepackage{xurl}}{} % add URL line breaks if available
\urlstyle{same}
\hypersetup{
  pdftitle={Medicaid Information Technology Architecture},
  colorlinks=true,
  linkcolor={blue},
  filecolor={Maroon},
  citecolor={Blue},
  urlcolor={Blue},
  pdfcreator={LaTeX via pandoc}}


\title{Medicaid Information Technology Architecture}
\author{}
\date{}
\begin{document}
\maketitle

\renewcommand*\contentsname{Table of contents}
{
\hypersetup{linkcolor=}
\setcounter{tocdepth}{2}
\tableofcontents
}

\chapter{What is MITA}\label{what-is-mita}

\pandocbounded{\includegraphics[keepaspectratio]{media/mitaLogo/mitaLogo.png}}

The Medicaid Information Technology Architecture (MITA) Framework is an
initiative by the Centers for Medicare \& Medicaid Services (CMS) in
partnership with State Medicaid Agencies (SMAs) and Medicaid systems
vendors. It aims to establish national guidance and best practice
references for processes, data standards, and technologies that
facilitate planning and enhance program administration for State
Medicaid Enterprises. Building upon the foundation of previous versions,
MITA 4.0 supports the Medicaid mission and goals by facilitating
integrated business and information technology transformations. This
version refines, refocuses, and repurposes MITA to better align with the
evolving needs of SMAs.

MITA 4.0 introduces several new approaches to enhance its relevance and
accessibility, with the goal of making it more meaningful for
stakeholders while streamlining processes to reduce the burden on state
agencies. The framework incorporates various state agency viewpoints to
improve efficiency and better align with the Advanced Planning Document
process, certification outcomes, and state agency acquisition processes.
Additionally, MITA 4.0 provides guidance that reflects current
healthcare and IT trends, ensuring that it remains at the forefront of
technological advancements.

By focusing on these enhancements, MITA 4.0 ensures that technology
decisions align with Medicaid business needs, optimizing adaptability,
flexibility, interoperability, and data sharing. This evolution enables
significant improvements in policy, decision-making, and daily
operations, ultimately advancing the capabilities of State Medicaid
Enterprises.

\begin{itemize}
\tightlist
\item
  MITA Framework is a consolidation of principles, models, and
  guidelines that combine to form a template for the States to use to
  develop their own enterprise architectures.
\item
  MITA processes provide guidance for State Medicaid Enterprise to use
  in adopting the MITA Framework through shared leadership,
  collaboration, and reuse of solutions.
\item
  MITA planning guidelines help States prepare the MITA State
  Self-Assessment (SS-A) and Roadmap to develop enterprise architectures
  to align to and advance increasingly in MITA maturity for business,
  architecture, and data. The guidelines serve as the basis for a
  state's requests for appropriate Federal Financial Participation (FFP)
  for their Medicaid Management Information Systems (MMIS) as well as
  Medicaid Information Technology (IT) system(s) projects related to
  eligibility determination and enrollment functions.
\end{itemize}

\section{Standards and Conditions}\label{standards-and-conditions}

The MITA framework plays an important role in helping states meet the
standards and conditions outlined in
\href{https://www.ecfr.gov/current/title-42/chapter-IV/subchapter-C/part-433/subpart-C/section-433.112}{Standards
and Conditions within 42 CFR 433.112} required for enhanced federal
financial participation (FFP) for Medicaid technology investments funded
through an approved APD. By providing a structured approach to planning
and development, MITA assists states in aligning their technology
solutions with Medicaid business needs and federal requirements. It
encourages the use of modular, flexible systems that promote
interoperability and data sharing, ensuring that states can effectively
coordinate with other health services and systems. MITA also guides
states in preparing their Advanced Planning Document (APD) submissions,
offering a roadmap to compliance and efficiency. This alignment not only
facilitates adherence to federal standards but also supports states in
achieving their Medicaid program goals more efficiently, effectively,
and sustainably, optimizing adaptability and enhancing overall program
administration.

\section{MITA 4.0 Goals}\label{mita-4.0-goals}

\begin{itemize}
\tightlist
\item
  Develop seamless and integrated systems that communicate effectively
  to achieve common Medicaid goals through interoperability and common
  standards.
\item
  Promote an environment that supports flexibility, adaptability, and
  rapid response to changes in programs and technology.
\item
  Promote an enterprise view that supports enabling technologies that
  align with Medicaid business processes and technologies.
\item
  Provide data that is timely, accurate, usable, and easily accessible
  in order to support analysis and decision making for health care
  management and program administration.
\item
  Provide performance measurement for accountability and planning.
\item
  Coordinate with public health and other partners to integrate health
  outcomes within the Medicaid community.
\end{itemize}

\section{MITA 4.0 Design Principles}\label{mita-4.0-design-principles}

During the development of MITA 4.0, workgroup participants identified a
set of core principles designed to prioritize updates MITA 3.0 that make
MITA more meaningful and accessible to SMAs. These principles ensure
that MITA remains a relevant and valuable tool for State Medicaid
Agencies and their stakeholders.

\textbf{Define Clear Linkages Between Capabilities and Outcomes}:
Establish clear definitions for both MITA Capabilities and Outcomes, and
articulate the relationship between outcomes and the MITA architectures.
This principle ensures that every capability is directly aligned with
desired outcomes, linking strategic objectives with operational
execution in a coherent framework.

\textbf{Business-Driven Transformation}: Define business transformations
with a focus on aligning IT solutions with both common and unique state
needs. This principle ensures that technology initiatives are directly
informed by business objectives, leading to more effective and tailored
solutions.

\textbf{Standards First}: Promote data and technical standards to
improve IT development cost-effectiveness. By prioritizing standards, we
aim to streamline processes and reduce complexity, ultimately enhancing
interoperability and efficiency.

\textbf{Reduce Burden on SMAs}: Simplify processes and requirements to
alleviate the administrative load on State Medicaid Agencies, enabling
them to focus more on service delivery and less on compliance.

\textbf{Enable Automation}: Encourage the adoption of automated
processes to increase efficiency and accuracy in Medicaid operations,
reducing manual intervention and the potential for errors.

\textbf{Release Guidance Aligned with Current Trends}: Provide guidance
that reflects the latest trends in healthcare and IT, ensuring that MITA
4.0 remains relevant and forward-looking in its approach to Medicaid
management.

\textbf{Integrate with Other Activities}: Enhance the integration of
MITA with related activities such as APD development, Certification,
T-MSIS reporting, and state procurement processes. This principle aims
to create a cohesive framework that supports comprehensive Medicaid
management and aligns with broader state and federal initiatives.

\textbf{Transition to a Web-Enabled Presentation}: Move from the static,
PDF-based MITA 3.0 to a dynamic, web-enabled format. This principle
facilitates easier maintenance and continuous improvement by the
community, allowing for real-time updates and enabling stakeholders to
access the most current information. By fostering a collaborative
environment, we encourage the sharing of insights and innovations,
making MITA a more robust and adaptable framework for all users.

These principles are foundational to the development and implementation
of MITA 4.0, ensuring it effectively supports the evolving needs of
Medicaid operations.

\chapter{Versions of MITA}\label{versions-of-mita}

--- dev a change matrix showing dif between versions ---

\section{Concepts}\label{concepts}

\begin{itemize}
\tightlist
\item
  \textbf{Outcomes}: Define what SMAs aim to achieve through the
  implementation of MITA.
\item
  \textbf{Business Process Model}: Define common business processes for
  the Medicaid Enterprise.
\item
  \textbf{Maturity Model}: Illustrate the maturation of Medicaid
  operations over time.
\item
  \textbf{Capability Matrices}: Align business, information, and
  technical capabilities with the Maturity Model.
\item
  \textbf{State Self-Assessment (SS-A)}: Represent current and future
  business, information, and technical capabilities.
\end{itemize}

MITA 4.0 ensures that technology decisions align with Medicaid business
needs, optimizing adaptability, flexibility, interoperability, and data
sharing. This evolution enables significant improvements in policy,
decision-making, and daily operations. Explore the MITA 4.0 Framework to
advance your State Medicaid Enterprise.

\part{Introduction to MITA 4.0}

\chapter{Why Adopt MITA?}\label{why-adopt-mita}

The MITA Initiative provides significant benefits to Medicaid
stakeholders, including the public, states, and the federal government.

Why Adopt MITA?

The MITA Initiative provides significant benefits to Medicaid
stakeholders, including the public, states, and the federal government.

Helping the Public

MITA helps the public by making it easier for people to access
healthcare. MITA helps improve the quality of care and drives data for
decision-making to improve health outcomes and public safety. It also
makes Medicaid enterprise systems more efficient, reduces fraud, and
saves money by using standardized practices and reusable tools.

Helping the States

MITA helps states by supporting Medicaid program management and health
reform efforts. MITA promotes prevention and wellness through
collaboration and aligning technology with Medicaid priorities to get
the most out of investments and support national health initiatives.

\begin{verbatim}
<br>
\end{verbatim}

Helping CMS

MITA helps the federal government by streamlining how CMS reviews state
Medicaid IT plans, aligning with national health goals and making better
use of resources. MITA enhances coordination across agencies to improve
processes, inform decisions, and lower costs for developing systems.

\chapter{}\label{section}

How is MITA 4.0 Different?

MITA 4.0 builds on MITA 3.0 to support Medicaid's mission by helping
states modernize their business processes and technology systems to
improve the administration of their Medicaid programs. MITA 4.0 evolves
and refines the MITA framework to better meet the changing needs of
State Medicaid Agencies.

Click the image below to learn more:

\part{State Self-Assessment}

\chapter{State Self-Assessment
Playbook}\label{state-self-assessment-playbook}

CMS Medicaid Information Technology Architecture (MITA) 4.0

\hfill\break

\section{Introduction}\label{introduction}

The MITA State Self-Assessment (SS-A) guides your State Medicaid Agency
(SMA) to describe current operations and plan for future improvements.
The SS-A helps align your Medicaid Enterprise System (MES) with CMS and
State prioritized outcomes, MITA's capabilities (ORBIT), and Federal
requirements for enhanced funding.

\pandocbounded{\includegraphics[keepaspectratio]{index_files/mediabag/media/playbookCircle.pdf}}

\section{The SS-A Plays}\label{ssa-plays}

\begin{enumerate}
\def\labelenumi{\arabic{enumi}.}
\tightlist
\item
  \hyperref[play-1]{Align MITA to your Medicaid enterprise}
\item
  \hyperref[play-2]{Prepare your team to complete the SS-A}
\item
  \hyperref[play-3]{Focus on the capabilities that drive your goals}
\item
  \hyperref[play-4]{Assess the maturity of your MES capabilities}
\item
  \hyperref[play-5]{Connect the capabilities to your MES modules}
\item
  \hyperref[play-6]{Envision the ideal state of your MES modules}
\item
  \hyperref[play-7]{Map your path to transformation}
\item
  \hyperref[play-8]{Request advanced funding to support your
  transformation}
\item
  \hyperref[play-9]{Take off on your journey to transformation}
\end{enumerate}

\section{Play 1: Align MITA to your Medicaid Enterprise}\label{play-1}

\subsection{Objectives}\label{objectives}

\begin{itemize}
\tightlist
\item
  SMA staff have a common understanding of the SS-A
\item
  SMA understands how MITA aligns with agency vision and strategic goals
\item
  SMA staff are aligned on five-year goals for MES transformation
\end{itemize}

\subsection{Key Actions}\label{key-actions}

\begin{itemize}
\tightlist
\item
  Review influences on modernization initiatives (mission, plans,
  priorities, statutes, regulations)
\item
  Determine five-year MES transformation goals and scope
\item
  Develop objectives and milestones to measure progress
\end{itemize}

\subsection{Key Players}\label{key-players}

\begin{itemize}
\tightlist
\item
  State Medicaid executives
\item
  Relevant SMA staff (Director, IT, Compliance, Procurement, Division
  Heads)
\end{itemize}

\subsection{Key Questions}\label{key-questions}

\begin{itemize}
\tightlist
\item
  How will policies and programs influence MES transformation?
\item
  How will state priorities shape transformation goals?
\item
  Is the transformation enterprise-wide or capability-specific?
\item
  What funding is available and how will it impact scope?
\item
  What are the main drivers for transformation?
\item
  Which stakeholders need to be engaged?
\item
  How will progress be measured?
\end{itemize}

\subsection{References}\label{references}

\begin{itemize}
\tightlist
\item
  SS-A Playbook
\item
  MITA Maturity Model
\item
  Medicaid State Plan
\item
  Other State Medicaid Program Documents
\end{itemize}

\section{Play 2: Prepare your team to complete the SS-A}\label{play-2}

\subsection{Objectives}\label{objectives-1}

\begin{itemize}
\tightlist
\item
  SS-A team is established, with a team lead selected
\item
  Team has approval to move forward
\item
  Team follows SMA processes for project startup
\end{itemize}

\subsection{Key Actions}\label{key-actions-1}

\begin{itemize}
\tightlist
\item
  Establish SS-A team and lead
\item
  Complete and submit APD for funding and approval
\item
  Initiate SS-A project per SMA processes
\end{itemize}

\subsection{Key Players}\label{key-players-1}

\begin{itemize}
\tightlist
\item
  State Medicaid executives
\item
  SS-A team
\end{itemize}

\subsection{Key Questions}\label{key-questions-1}

\begin{itemize}
\tightlist
\item
  Who has the needed information?
\item
  Who will lead the SS-A team?
\item
  How will oversight and alignment be managed?
\item
  What communication strategies will be used?
\item
  Is additional support required?
\end{itemize}

\subsection{References}\label{references-1}

\begin{itemize}
\tightlist
\item
  SS-A Playbook
\item
  SMA project processes
\item
  State procurement policies
\item
  APD template and guidance
\end{itemize}

\section{Play 3: Focus on the capabilities that drive your
goals}\label{play-3}

\subsection{Objectives}\label{objectives-2}

\begin{itemize}
\tightlist
\item
  Identify relevant capability domains for transformation goals
\item
  Identify subject matter experts for SS-A
\item
  Gather people and information needed for SS-A
\end{itemize}

\subsection{Key Actions}\label{key-actions-2}

\begin{itemize}
\tightlist
\item
  Select capability domains using MITA Capability Domain Map and
  transformation goals
\item
  Map state-specific capabilities using MITA Capability Meta-Model
\end{itemize}

\subsection{Key Players}\label{key-players-2}

\begin{itemize}
\tightlist
\item
  SS-A team
\item
  Enterprise architects
\item
  Capability domain subject matter experts
\end{itemize}

\subsection{Key Questions}\label{key-questions-2}

\begin{itemize}
\tightlist
\item
  Which subject matter experts are needed?
\item
  Which capability domains to focus on?
\item
  What state-specific capabilities should be included?
\end{itemize}

\subsection{References}\label{references-2}

\begin{itemize}
\tightlist
\item
  SS-A Playbook
\item
  MITA Capability Domain Map
\item
  MITA Capability Meta-Model
\item
  SMA transformation goals
\end{itemize}

\section{Play 4: Assess the maturity of your MES
Capabilities}\label{play-4}

\subsection{Objectives}\label{objectives-3}

\begin{itemize}
\tightlist
\item
  Know current and desired maturity levels for capability domains
\item
  Document maturity for each domain
\item
  Identify gaps between current and desired maturity
\end{itemize}

\subsection{Key Actions}\label{key-actions-3}

\begin{itemize}
\tightlist
\item
  Assess maturity using SS-A Tool or MITA Maturity Models
\item
  Generate outputs for each domain
\item
  Identify gaps between current and desired maturity
\end{itemize}

\subsection{Key Players}\label{key-players-3}

\begin{itemize}
\tightlist
\item
  SS-A team
\item
  Medicaid program subject matter experts
\item
  Capability domain subject matter experts
\end{itemize}

\subsection{Key Questions}\label{key-questions-3}

\begin{itemize}
\tightlist
\item
  What are the current maturity levels?
\item
  What are the desired maturity levels?
\item
  Which experts are needed to validate maturity?
\item
  Where are the gaps?
\end{itemize}

\subsection{References}\label{references-3}

\begin{itemize}
\tightlist
\item
  SS-A Playbook
\item
  SS-A Tool
\item
  MITA Maturity Models
\end{itemize}

\section{Play 5: Connect the capabilities to your MES
modules}\label{play-5}

\subsection{Objectives}\label{objectives-4}

\begin{itemize}
\tightlist
\item
  Know which capability domains support MES modules
\item
  Decide which MES modules to prioritize for transformation
\end{itemize}

\subsection{Key Actions}\label{key-actions-4}

\begin{itemize}
\tightlist
\item
  Decide which MES modules to update
\item
  Determine which capability domains support each module
\item
  Prioritize modules for updating
\end{itemize}

\subsection{Key Players}\label{key-players-4}

\begin{itemize}
\tightlist
\item
  SS-A team
\item
  IT team
\item
  Capability domain subject matter experts
\item
  MES module subject matter experts
\end{itemize}

\subsection{Key Questions}\label{key-questions-4}

\begin{itemize}
\tightlist
\item
  Which MES modules support the needed capabilities?
\item
  Which modules to focus on?
\item
  What criteria for prioritization?
\item
  How do transformation drivers influence prioritization?
\end{itemize}

\subsection{References}\label{references-4}

\begin{itemize}
\tightlist
\item
  SS-A Playbook
\item
  SS-A outputs
\item
  SMA transformation goals
\item
  MITA Capability Domain Map
\end{itemize}

\section{Play 6: Envision the Ideal State of your MES
modules}\label{play-6}

\subsection{Objectives}\label{objectives-5}

\begin{itemize}
\tightlist
\item
  Understand how to get from current to desired operations
\item
  Gather documentation needed for APDs
\end{itemize}

\subsection{Key Actions}\label{key-actions-5}

\begin{itemize}
\tightlist
\item
  Document current state of operations for MES modules
\item
  Document desired state of operations
\item
  Document gaps between current and desired state
\end{itemize}

\subsection{Key Players}\label{key-players-5}

\begin{itemize}
\tightlist
\item
  SS-A team
\item
  MES module subject matter experts
\end{itemize}

\subsection{Key Questions}\label{key-questions-5}

\begin{itemize}
\tightlist
\item
  What is the ideal state of operations?
\item
  What are possible risks or challenges?
\item
  What resources and funding are needed?
\end{itemize}

\subsection{References}\label{references-5}

\begin{itemize}
\tightlist
\item
  SS-A Playbook
\item
  SMA transformation goals
\item
  SS-A outputs
\item
  MITA Concept of Operations guidance
\end{itemize}

\section{Play 7: Map your path to transformation}\label{play-7}

\subsection{Objectives}\label{objectives-6}

\begin{itemize}
\tightlist
\item
  Five-year roadmap to achieve MES transformation goals
\item
  Know which MES modules to transform and when
\item
  Plan to advance MES from current to desired maturity
\end{itemize}

\subsection{Key Actions}\label{key-actions-6}

\begin{itemize}
\tightlist
\item
  Document transformation goals and objectives for MES modules over five
  years
\item
  Explain how requirements for enhanced funding will be met
\item
  Develop proposed budget for roadmap steps
\end{itemize}

\subsection{Key Players}\label{key-players-6}

\begin{itemize}
\tightlist
\item
  SS-A team
\item
  State Medicaid executives
\item
  MES module subject matter experts
\item
  State financial analyst
\end{itemize}

\subsection{Key Questions}\label{key-questions-6}

\begin{itemize}
\tightlist
\item
  How does the roadmap align with transformation goals?
\item
  How will progress be measured?
\item
  How will progress be communicated?
\item
  How to ensure stakeholder support?
\end{itemize}

\subsection{References}\label{references-6}

\begin{itemize}
\tightlist
\item
  SS-A Playbook
\item
  SS-A outputs
\item
  SMA MES transformation goals
\end{itemize}

\section{Play 8: Request Advanced Funding to Support Your
Transformation}\label{play-8}

\subsection{Objectives}\label{objectives-7}

\begin{itemize}
\tightlist
\item
  Complete APDs to request advanced funding for transformation
  activities
\end{itemize}

\subsection{Key Actions}\label{key-actions-7}

\begin{itemize}
\tightlist
\item
  Determine APD type for each activity
\item
  Complete and submit APD using appropriate template and guidance
\end{itemize}

\subsection{Key Players}\label{key-players-7}

\begin{itemize}
\tightlist
\item
  CMS State Officer
\item
  Acquisition subject matter experts
\item
  Policy subject matter experts
\item
  State financial analyst
\end{itemize}

\subsection{Key Questions}\label{key-questions-7}

\begin{itemize}
\tightlist
\item
  How will requirements for enhanced funding be met?
\item
  How does the funding request align with priorities?
\item
  What acquisition and contracting activities are required?
\item
  What are staffing requirements and costs?
\item
  What is the proposed budget?
\end{itemize}

\subsection{References}\label{references-7}

\begin{itemize}
\tightlist
\item
  MITA Roadmap
\item
  CMS APD template and guidance
\end{itemize}

\section{Play 9: Take Off on Your Journey to
Transformation}\label{play-9}

This playbook led you through the SS-A process from aligning on common
goals to requesting advanced funding. You understand your current state,
your destination, and your path forward.

Now, move from planning to development:

\begin{itemize}
\tightlist
\item
  Mature MES modules
\item
  Measure progress
\item
  Update your MITA Roadmap as priorities change
\end{itemize}

Keeping your roadmap up to date ensures your SMA stays on course and
reaches its transformation goals.

\section{Summary}\label{summary}

The State Self-Assessment Playbook provides a structured approach to
MITA implementation through nine distinct plays. Each play builds upon
the previous one, creating a comprehensive framework for Medicaid
Enterprise System transformation that aligns with federal requirements
and state priorities.

\part{MITA Capability Reference Architecture}

\chapter{MITA Capability Model}\label{mita-capability-model}

\section{Model Objective}\label{model-objective}

The MITA 4.0 Capability Reference Model is a key component of the MITA
4.0 Framework. The primary objective of the reference model is to
support the alignment of capabilities across SMA's to enable leverage,
reuse and interoperability across SMA's.

\section{Model Definition}\label{model-definition}

A capability is defined as an ability that an SMA possesses or seeks to
develop to achieve its goals and meet its desired outcomes. It
represents what the SMA can do without attempting to explain how, why or
where the SMA uses the capability. It may be an ability that may exist
within the SMA today or be required to enable a new direction or reach a
new desired outcome. Each capability is composed of the following: •
Outcomes -- The definition of the desired outcomes that require the
capability to be achieved. • Roles -- The individual roles that are
responsible for providing the capability. • Business Processes -- The
business processes that are performed to deliver the capability. •
Information -- The information and the data management capabilities that
are needed to deliver the capability. • Technology -- The technology
that is used to automate the capability.

\section{MITA 3.0 vs.~MITA 4.0
Capabilities}\label{mita-3.0-vs.-mita-4.0-capabilities}

MITA 3.0 defines capabilities as the competence of an individual,
organization or system to perform a function or process. The MITA 4.0
framework changes the definition of a capability to not include the
concept of competence. The capability in MITA 4.0 focuses only on the
ability that an SMA possesses or seeks to develop, agnostic of how well
the SMA performs that capability. The purpose of the state
self-assessment (SS-A) in MITA 4.0 is to assess the maturity of the
capability and how well the SMA performs the capability based on the
outcomes, roles, business processes, information and technology that an
SMA has defined and implemented to support that capability. The MITA 4.0
capabilities are closely aligned to the following concepts in MITA 3.0:
• MITA 3.0 Business Areas and Categories • Technical Service Areas and
Classifications

\begin{figure}[H]

{\centering \pandocbounded{\includegraphics[keepaspectratio]{media/30vs40capabilities.png}}

}

\caption{MITA 4.0 vs.~MITA 3.0 Capabilities}

\end{figure}%

\section{Model Structure}\label{model-structure}

A capability reference model is an abstract framework that defines
concepts used for grouping capabilities that share a common meaning. It
is used to establish a shared definition of capability concepts that
cross organizational boundaries and helps to identify opportunities for
sharing, leveraging and reuse. The MITA 4.0 Capability Reference Model
is designed to identify the key capability concepts that are needed to
support the Medicaid Program and achieve the goals and outcomes
established for MITA 4.0. The MITA 4.0 Capability Model is grouped into
(2) levels including: • Capability Domain - High-level capability used
to group common capabilities • Capability Area -- Detailed capabilities
that decomposes the capability domain into sub-capabilities that can be
used by SMA's to classify their capabilities. There are one to many
distinct Capability Areas for each Capability Domain. Each Capability
Area can have one to many distinct Capabilities defined within it .

\section{Model Application}\label{model-application}

SMA's will use the MITA 4.0 Capability Reference Model to classify their
own capabilities. The MITA 4.0 Capability Reference Model provides SMA's
the freedom to define their capabilities based on their own
state-specific needs. The SMA's should use the capability domains and
areas in the reference model to classify their state-specific
capabilities. Using the reference model enables alignment of
capabilities across SMA's and enables states and CMS the ability to do
the following: • Identify opportunities to collaborate, leverage and
reuse. • Consistently assess and report capability maturity.

\begin{figure}[H]

{\centering \pandocbounded{\includegraphics[keepaspectratio]{media/modelExample.png}}

}

\caption{MITA 4.0 Reference Model Application Example}

\end{figure}%

\section{Model Contents}\label{model-contents}

The capability domains identified in the MITA 4.0 Capability Reference
Model are organized into (3) distinct groups including: • Strategic --
Identifies capabilities that an SMA possesses or seeks to develop that
enable them to establish and maintain its enterprise strategy. • Core --
identifies capabilities that an SMA possesses or seeks to develop that
enable them to meet its mission and achieve its desired outcomes. •
Support -- Identifies capabilities that an SMA possesses or seeks to
develop that are not specific to its mission but are critical for the
core and strategic capabilities.

\subsection{Purpose}\label{purpose}

\begin{tcolorbox}[enhanced jigsaw, toprule=.15mm, colback=white, colframe=quarto-callout-note-color-frame, left=2mm, arc=.35mm, opacityback=0, rightrule=.15mm, breakable, bottomrule=.15mm, leftrule=.75mm]
\begin{minipage}[t]{5.5mm}
\textcolor{quarto-callout-note-color}{\faInfo}
\end{minipage}%
\begin{minipage}[t]{\textwidth - 5.5mm}

\vspace{-3mm}\textbf{Note}\vspace{3mm}

MITA 4.0 does not endeavor to specify all of the capabilities SMA's may
need to administer Medicaid programs; instead, this version of MITA
focuses on the capabilities that are most closely oriented towards
achieving the CMS-required outcomes.

\end{minipage}%
\end{tcolorbox}

Understanding the how the MITA Capability Model works is important to
obtaining the most value out of many of the other tools and artifacts in
the MITA framework, such as the MITA Maturity Model (MMM) and the
Business Process Model (BPM). The MITA Capability Model provides a
structured way for SMAs to identify, conceptually model, and improve the
capabilities needed for efficient Medicaid operations.

It is important to note that MITA 4.0 does not endeavor to specify all
of the capabilities SMA's may need to administer Medicaid programs;
instead, this version of MITA focuses on the capabilities that are most
closely oriented towards achieving the CMS-required outcomes. In this
way MITA 4.0 provides a reference model for SMAs to model other
capabilities that may be needed to achieve their other goals such as
state specific outcomes, or other state priorities while providing more
guidance within the MITA Framework to support modular.

\subsection{Update to MITA 3.0}\label{update-to-mita-3.0}

MITA 3.0 defined a capability as the level of maturity of a set of
business processes within a business category. By focusing on ``how''
MES operate MITA 3.0 helped SMA's identify ways to improve and mature
their business processes, but it did not link those processes with the
outcomes they are intended to achieve or ensure better alignment of the
information and technical architectures to business outcomes. The
addition of the MITA capability model to the MITA 4.0 business
architecture addresses that by providing the conceptual linkages needed
to elevate the strategic vantage point of the MITA Framework. To guide
this change, we present within this chapter a definition, description,
and approach to modeling business capabilities, based on the widely used
capability models contextualized for Medicaid Enterprises.

The business processes that operationalize MITA capabilities remain
foundational to characterizing the business architecture, and are by
definition a constituent part of any MITA capability. They provide
essential information on how capabilities are operationalize and should
continue to be a routinely utilized reference model for SMA business
process mapping. They are found with in the Business Process Model
chapter of this version of MITA.

\section{The MITA Definition of
Capability}\label{the-mita-definition-of-capability}

Within the context of MITA, a capability can be defined as the ability
or capacity of a State Medicaid Agency to achieve a desired outcome in
compliance with the
\href{https://www.ecfr.gov/current/title-42/chapter-IV/subchapter-C/part-433/subpart-C/section-433.112}{Standards
and Conditions within 42 CFR 433.112}. A capability may currently exist
in an operational state or be envisioned for future development. Through
careful planning, capabilities defined in this way can be matured and
refined over time to become more effective and efficient. They can be
organized and detailed at various levels of abstraction, providing
precise descriptions for operational purposes or more generalized views
for strategic planning.

\begin{tcolorbox}[enhanced jigsaw, toprule=.15mm, colback=white, colframe=quarto-callout-note-color-frame, left=2mm, arc=.35mm, opacityback=0, rightrule=.15mm, breakable, bottomrule=.15mm, leftrule=.75mm]

\vspace{-3mm}\textbf{Key Definition}\vspace{3mm}

\ldots a capability is defined as the ability or capacity of a SMA to
achieve a desired outcome\ldots{}

\end{tcolorbox}

To fully define a business capability, it is essential to understand how
it is realized through the integration of people, processes,
information, and technology resources of an SMA. While these elements of
the capability can change regularly, the capability itself is should
endure over longer planning horizons, supporting the long-term alignment
of business and IT and the achievement of increasingly beneficial
business outcomes.

\subsection{Structure of the MITA Capability
Model}\label{structure-of-the-mita-capability-model}

As depicted in the model below, the MITA Capability Model orients the
people, process, technology, and information resources to define a MITA
Capability. This means that to model a capability the appropriate
components of the information architecture and the technical
architecture must be brought together with the business architecture to
fully formulate any MITA Capability.

\begin{figure}[H]

{\centering \pandocbounded{\includegraphics[keepaspectratio]{media/capabilityModel/topLevelCapabilityMetamodelGraphic1.png}}

}

\caption{MITA Capability Relationship Diagram}

\end{figure}%

\subsubsection{Business Roles}\label{business-roles}

Business roles represent individual actors, stakeholders, or partners
involved in delivering a business capability. A single organizational
group or team may be wholly responsible for delivering the capability,
or multiple business entities may share the delivery of a particular
business capability. Business Roles perform Business Processes using
Technology Resources. They require skills and knowledge resources to
achieve outcomes, and should be actively engaged as partners in the
development or enhancement of any capability they help deliver.

\subsubsection{Business Processes}\label{business-processes}

Individual business capabilities may be enabled or delivered through a
range of business processes that detail the activities (the how)
associated with delivering the capability. Identifying and analyzing the
efficiency of the underlying processes helps to optimize the business
capability's effectiveness. Identifying the processes within a business
capability provides a focus for maturing the capability in concert with
the other capability components. Business Processes operationalize
Business Capabilities.

\subsubsection{Information/Data}\label{informationdata}

Information/data represents the business data, knowledge, and insight
consumed or produced by the business capability (as distinct from
IT-related data entities). This may also include information that the
capability exchanges with other capabilities to support the execution of
value streams. Examples include information about customers and
prospects, products and services, business policies and rules, sales
reports, and performance metrics. Information/data inform the Business
Capability, answering questions and supporting business rules.

\subsubsection{Technology Resources}\label{technology-resources}

Business capabilities rely on a range of tools, applications, systems,
and services for successful execution. Technology Resources use
Information/data to facilitate Business Processes. Such resources may
include:

\begin{itemize}
\tightlist
\item
  Modular software applications

  \begin{itemize}
  \tightlist
  \item
    Cloud or on-premise infrastructure
  \item
    Microservices
  \item
    Analytics
  \item
    Customer portal
  \end{itemize}
\end{itemize}

In this way we can clearly interrelate all of the MITA architecture
models and their individual components which allows us to reveal gaps
not only in the individual components of the architecture, but also
understand their impact on the integration of the architecture
components at the capability level.

\subsection{Relationship of MITA Capabilities to
Outcomes}\label{relationship-of-mita-capabilities-to-outcomes}

In the context of the Medicaid Information Technology Architecture
(MITA), outcomes are intrinsically linked to capabilities, as they
represent the tangible results achieved through the effective
integration and execution of various elements that constitute a
capability. In this sense, outcomes and capabilities define each other.

\begin{figure}[H]

{\centering \pandocbounded{\includegraphics[keepaspectratio]{media/capabilityModel/topLevelCapabilityMetamodelOutcomes.png}}

}

\caption{MITA Capability and Outcome Relationship Diagram}

\end{figure}%

\subsubsection{Outcomes}\label{outcomes}

MITA defines outcomes broadly to encompass CMS-required outcomes,
state-specific outcomes, and other outcomes not mandated as part of the
Advance Planning Document (APD) process. The sole criterion for an
outcome to meet this definition is that it must be a goal of a State
Medicaid Agency (SMA) and be achieved through a Medicaid Enterprise
System (MES) capability.

\begin{tcolorbox}[enhanced jigsaw, toprule=.15mm, colback=white, colframe=quarto-callout-note-color-frame, left=2mm, arc=.35mm, opacityback=0, rightrule=.15mm, breakable, bottomrule=.15mm, leftrule=.75mm]

\vspace{-3mm}\textbf{Key Definition}\vspace{3mm}

A MITA outcome is a goal of a State Medicaid Agency (SMA) that is
achieved by a Medicaid Enterprise System (MES) capability.

\end{tcolorbox}

\subsubsection{Measure}\label{measure}

Measure is a quantifiable metric used to assess the effectiveness and
efficiency of capabilities within a Medicaid Enterprise System (MES).
Measures provide quantifiable and qualitative values that help State
Medicaid Agencies (SMAs) track progress toward achieving specific
outcomes, such as CMS-required or state-specific goals. These indicators
might include metrics like processing times, error rates, or compliance
levels.

Measures are a measurement threshold by establishing a specific value or
level that must be met or exceeded to demonstrate successful
performance. For instance, a KPI might set a threshold for the maximum
allowable processing time for claims, ensuring that they are handled
within a specified timeframe to maintain compliance and eligibility for
enhanced federal funding. By monitoring these thresholds, organizations
can ensure they are meeting regulatory requirements and delivering
high-quality services to beneficiaries, while also identifying areas for
improvement.

\subsubsection{Measure Threshold}\label{measure-threshold}

A specific value or level of a measure that must be met or exceeded to
demonstrate the effective achievement of a capability's intended
outcome. This threshold serves as a benchmark for assessing whether the
processes, roles, and resources integrated within a Medicaid Enterprise
System (MES) are functioning optimally to meet the goals of a State
Medicaid Agency (SMA). For example, a measurement threshold might be set
for processing times, where claims must be processed within a certain
number of days to ensure compliance with CMS-required outcomes and
maintain eligibility for enhanced federal funding. By establishing and
monitoring these thresholds, organizations can ensure they are meeting
regulatory requirements and delivering high-quality services to
beneficiaries.

\subsubsection{Measurement}\label{measurement}

These outcomes and metrics are also used to ensure that healthcare
systems or modules comply with applicable federal regulations, forming
the baseline for system or module functionality. Achieving these
outcomes is essential for continuing to receive enhanced federal funding
for operations. Regular measurement and analysis of KPIs help
organizations demonstrate compliance and effectiveness, ensuring that
they meet regulatory requirements and continue to deliver high-quality
services to beneficiaries.

In this way we can clearly interrelate all of the MITA architecture
models and their individual components with the KPIs, thresholds, and
measurements that indicate whether our capability achieves our desired
outcome.

While models that help conceptualize the capabilities that achieve
CMS-required outcomes are the ones modeled for this version of MITA,
SMAs are encouraged to use these models as a reference to model
capabilities.

\section{Capability Mapping}\label{capability-mapping}

Capability mapping is a strategic tool that enables organizations, such
as State Medicaid Agencies (SMAs), to systematically identify, organize,
and visualize the key capabilities necessary to achieve their
objectives. Within the MITA framework, capability mapping provides SMAs
with a method of developing comprehensive views of the functions and
processes required to deliver Medicaid services effectively. To begin
the capability mapping process, SMAs should first identify the core
capabilities that align with their strategic objectives, focusing on
what the organization needs to achieve rather than how those goals are
accomplished. This involves listing all necessary capabilities and
understanding the desired outcomes they support. Next, these
capabilities should be organized into domains and areas that reflect
their strategic importance and interrelationships. Visualizing these
capabilities through diagrams or maps provides all stakeholders a common
view to understand the roles, processes, technology resources, and
information/data involved in executing each capability, as well as the
outcome each capability is designed to achieve. This structured approach
not only highlights areas for improvement or investment but also ensures
that organizational efforts are strategically aligned with desired
outcomes.

The benefits of capability mapping are multifaceted, offering SMAs a
clear pathway to strategic alignment and gap analysis. By visualizing
capabilities, organizations can identify operational gaps and determine
what new or enhanced capabilities are needed to close those gaps. This
visualization also improves communication among stakeholders by
providing a clear and concise representation of the organization's
functions. To refine capabilities, SMAs should analyze current
operations, assess the efficiency of underlying processes, and optimize
them to enhance capability effectiveness. Additionally, capability
mapping serves as a foundation for heat mapping, which assesses the MITA
Framework will utilize to visualize the maturity of each capability
evaluated in the State Self-Assessment. SMAs can overlay heat maps over
their capability maps to visualize many things other than maturity
levels, using color coding to indicate areas of strength and weakness.
Regular updates to these maps allow SMAs to monitor progress and ensure
resources are allocated effectively to achieve strategic goals. The MITA
framework includes examples of capability maps based on CMS-required
outcomes, serving as a reference model for SMAs to develop their own
capability maps tailored to state-specific goals and priorities. By
leveraging the reference models provided by MITA, SMAs can ensure their
capability mapping efforts are aligned with both federal requirements
and state-specific priorities.

\subsection{Organizing Capabilities}\label{organizing-capabilities}

To enhance the resolution and detail of a capability and provide a
unified view of all its components, a block diagram can be employed to
provide a common view of any MES. This diagram effectively links the
capability to business processes, roles, technical resources, and
information resources through functional decomposition. By breaking down
the capability into its constituent parts, the block diagram offers a
visual representation that highlights the interrelationships and
dependencies among these elements. This approach provides a clearer
understanding of how each component contributes to the overall
capability, facilitating more effective analysis, optimization, and
alignment with organizational objectives.

\pandocbounded{\includegraphics[keepaspectratio]{media/capabilityModel/capabilityOgranizationModel.drawio.png}}

We use this same method to present an this top level view of the
capabilities required to achieve CMS-required outcomes. From this view
increasingly detailed models can be constructed.

\pandocbounded{\includegraphics[keepaspectratio]{media/capabilityModel/mesModuleBasedCapabilities.drawio.png}}

\subsection{MITA Capability Models}\label{mita-capability-models}

The MITA framework represents capabilities visually through a layered
model that represent a capability of being composed of sub-capabilities
and the processes, roles, information and technology resources (PRIT)
that support the business in sustaining the capability. Each layer up
depicts increasingly strategic capabilities and each layer down depicts
the constituent elements that compose a capability in increasing
operational detail. It is not the intention of this version of MITA to
provide a full operational or tactical view of a capability, though SMAs
may consider using this approach to improve their organizational
awareness of their operations by developing further layers of their
capabilities through functional decomposition.

\pandocbounded{\includegraphics[keepaspectratio]{media/capabilityModel/capabilityLevels.png}}

\begin{itemize}
\tightlist
\item
  \textbf{Capability Domains:} The first layer of this model aims to
  group capabilities to organize the strategic view of an SMA's
  capabilities. In this view one or many capabilities can be grouped
  within a domain to indicate the pursuit of common outcomes. Each
  domain is denoted with a single number to help annotate each
  capability.

  \begin{itemize}
  \tightlist
  \item
    \textbf{Capability Areas:} The second layer of this model aims to
    provide a view of the groups of capabilities that compose a domain.
    They are organized to show capabilities that serve a specific group
    of similar outcomes and essential
  \item
    \textbf{Capabilities:} The third layer of this model provides a more
    detailed view view of
  \end{itemize}
\end{itemize}

\pandocbounded{\includegraphics[keepaspectratio]{media/capabilityModel/capabilityLevels2.png}}

\subsection{Relationship of MITA Capabilities to
Maturity}\label{relationship-of-mita-capabilities-to-maturity}

\begin{tcolorbox}[enhanced jigsaw, toprule=.15mm, leftrule=.75mm, colframe=quarto-callout-warning-color-frame, left=2mm, arc=.35mm, titlerule=0mm, rightrule=.15mm, opacitybacktitle=0.6, bottomtitle=1mm, toptitle=1mm, colbacktitle=quarto-callout-warning-color!10!white, bottomrule=.15mm, title=\textcolor{quarto-callout-warning-color}{\faExclamationTriangle}\hspace{0.5em}{Warning}, opacityback=0, breakable, colback=white, coltitle=black]

Under development.

\end{tcolorbox}

\begin{itemize}
\tightlist
\item
  \textbf{Levels of Maturity}

  \begin{itemize}
  \tightlist
  \item
    Description of the five levels of maturity in the MITA framework
  \item
    How capabilities evolve and mature over time
  \end{itemize}
\end{itemize}

\pandocbounded{\includegraphics[keepaspectratio]{media/capabilityModel/maturityModel.png}}

\subsection{Using Capability Maps for Heat Mapping Strategic Priorities
and Identifying Gaps with the MITA Maturity
Model}\label{using-capability-maps-for-heat-mapping-strategic-priorities-and-identifying-gaps-with-the-mita-maturity-model}

Capability maps are powerful tools that not only provide a visual
representation of an SMA's key capabilities but also serve as a
foundation for strategic analysis and planning. There are many
approaches to heat mapping capabilities, each offering unique insights
into organizational priorities and gaps. Here, we describe two
approaches: assessing maturity levels using the MITA Maturity Model and
prioritizing strategic outcomes.

\subsubsection{Identifying Gaps with the MITA Maturity
Model}\label{identifying-gaps-with-the-mita-maturity-model}

The MITA Maturity Model provides a framework for assessing the maturity
of an organization's capabilities across various dimensions, such as
business processes, information, and technology. By integrating the
maturity model with capability maps, SMAs can identify gaps between
their current state and desired maturity levels.

\paragraph{Example 1: Identifying Gaps in Data Management Maturity Using
the PRIT
Model}\label{example-1-identifying-gaps-in-data-management-maturity-using-the-prit-model}

An SMA is conducting an assessment of its data management capabilities
using the MITA Maturity Model, with a focus on the PRIT (Processes,
Roles, Information, and Technology) framework. The capability map
includes various data-related capabilities, such as ``Data
Integration,'' ``Data Quality Management,'' and ``Data Analytics.'' Each
of these capabilities is evaluated across the PRIT dimensions to
determine their maturity levels using the revised scale:

Processes: Level 1: Ad-Hoc Roles: Level 2: Compliant Information: Level
2: Compliant Technology: Level 2: Compliant The capability map is
updated to reflect the maturity assessment, with each dimension marked
with a color code: red for Level 1: Ad-Hoc, yellow for Level 2:
Compliant, green for Level 3: Efficient, blue for Level 4: Optimized,
and purple for Level 5: Pioneering. This visualization helps the SMA
prioritize strategic actions to enhance the ``Data Integration''
capability, such as standardizing processes, refining roles, improving
data quality, and upgrading technology.

\subsubsection{Heat Mapping Strategic
Priorities}\label{heat-mapping-strategic-priorities}

Heat mapping involves applying a color-coded overlay to a capability map
to visually represent the status or priority level of each capability.
This technique can be used to highlight areas of strength, weakness, or
strategic importance. For example, capabilities that are critical to
achieving CMS-required outcomes might be marked in one color, while
those needing immediate attention or improvement could be marked in
another. This visual representation helps stakeholders quickly grasp the
strategic landscape and make informed decisions about where to allocate
resources and focus efforts.

\paragraph{Example 2: Prioritizing Capabilities for CMS-Required
Outcomes}\label{example-2-prioritizing-capabilities-for-cms-required-outcomes}

An SMA is focused on achieving specific CMS-required outcomes related to
improving patient care and reducing administrative costs. The agency
creates a capability map that outlines all the capabilities necessary to
meet these outcomes. By applying a heat map, the SMA highlights
capabilities that are directly linked to these outcomes in green,
indicating they are of high strategic priority. Capabilities that are
indirectly related or less critical are marked in yellow, while those
that are currently underperforming or not aligned with strategic goals
are marked in red.

This visual representation allows the SMA to quickly identify which
capabilities require immediate attention and resources to ensure
compliance with CMS requirements. For instance, if the capability
related to ``Claims Processing Efficiency'' is marked in red, the agency
can prioritize initiatives to enhance this capability, such as investing
in new technology or streamlining processes.

\subsubsection{Other Uses for Capability Heat
Mapping}\label{other-uses-for-capability-heat-mapping}

Beyond assessing maturity levels and prioritizing strategic initiatives,
capability heat mapping can be applied in various other contexts to
enhance organizational effectiveness and alignment.

\paragraph{Example 3: Aligning Capabilities with State-Specific
Initiatives}\label{example-3-aligning-capabilities-with-state-specific-initiatives}

An SMA is working on a state-specific initiative to enhance telehealth
services for rural populations. The capability map includes capabilities
related to telehealth, such as ``Telehealth Infrastructure,'' ``Provider
Engagement,'' and ``Patient Access.'' The SMA uses a heat map to
highlight these capabilities in blue, indicating their alignment with
the state-specific initiative.

By analyzing the capability map, the SMA identifies that ``Provider
Engagement'' is a critical capability that requires further development
to support the telehealth initiative. The agency decides to invest in
training programs and outreach efforts to engage providers in rural
areas, ensuring that the telehealth services are effectively delivered
to the target population.

These examples demonstrate how capability maps, combined with heat
mapping and the MITA Maturity Model, can provide valuable insights for
strategic planning and gap analysis. By visualizing priorities and
maturity levels, SMAs can make informed decisions about where to focus
resources and efforts, ultimately enhancing their Medicaid Enterprise
Systems and achieving strategic objectives.

\begin{itemize}
\tightlist
\item
  \textbf{Capability Mapping}

  \begin{itemize}
  \tightlist
  \item
    Introduction to capability mapping and its significance
  \item
    How capabilities are organized and detailed at various levels of
    abstraction
  \end{itemize}
\end{itemize}

\section{Guidance on reuse of the MITA Capability
Model}\label{guidance-on-reuse-of-the-mita-capability-model}

\begin{itemize}
\tightlist
\item
  \textbf{Practical Application}

  \begin{itemize}
  \tightlist
  \item
    How to integrate the capability model into daily operations and
    strategic planning
  \item
    Tips for maximizing the benefits of the model
  \end{itemize}
\item
  \textbf{Continuous Improvement}

  \begin{itemize}
  \tightlist
  \item
    Encouragement for ongoing assessment and refinement of capabilities
  \item
    Leveraging feedback and performance data for model enhancement
  \end{itemize}
\item
  \textbf{Implementation Guidance}

  \begin{itemize}
  \tightlist
  \item
    Steps for adopting the capability model
  \item
    Resources and support available for SMAs
  \end{itemize}
\item
  \textbf{Performance Monitoring and Reporting}

  \begin{itemize}
  \tightlist
  \item
    Role of the capability model in tracking and enhancing performance
  \item
    Use of metrics and standards to measure capability effectiveness
  \end{itemize}
\end{itemize}

\chapter{MITA Capability Model}\label{mita-capability-model-1}

\section{Introduction to Business Capability
Models}\label{introduction-to-business-capability-models}

A capability model is a conceptual framework that outlines the key
capabilities an organization needs to achieve its strategic objectives.
It provides a comprehensive view of what an organization can do and
helps identify areas for improvement or investment. In the context of an
orchestra, a capability model might help the orchestra identify the set
of skills and resources, or other types of capabilities it needs to
perform a symphony. Just like an orchestra needs well practiced
musicians, sheet music, instruments, a conductor, and an audience to
produce a great symphony, a State Medicaid Agency (SMA) needs its
Medicaid Enterprise System (MES) to employ or develop specific
capabilities to deliver its services effectively, efficiently, and
economically to its enrollees and providers.

The concept of a business capability is extensively used within
enterprise architecture modeling and has been broadly used within
Business Capability Models as a tool to better align the business
strategy and information technology of both private sector and
governmental organizations since they emerged in the mid-2000s. One
example comes from the TOGAF Standard, a well-known standard in
enterprise architecture. Like most architecture frameworks TOGAF defines
a capability as something a business can do to meet its goals. This
focuses a strategic lens of an organization on ``what'' it needs to
achieve its goals, rather than ``how'' those goals are achieved. This
perspective allows for business planning from different viewpoints,
facilitating strategic alignment and operational efficiency.

SMA business architects, technologists, systems analysts, executives,
managers, and program staff can use this same modeling approach to
represent the functional components of their Medicaid Enterprise System
(MES) in ways that can help reveal gaps in their systems and provide
insights on what new or enhanced capabilities might be needed to close
those gaps.

By focusing on capabilities, SMAs can better align their information and
technology resources and processes with their strategic business goals,
ultimately improving their insight into how to improve the outcomes
their Medicaid Enterprise Architecture produces.

\subsection{Purpose}\label{purpose-1}

\begin{tcolorbox}[enhanced jigsaw, toprule=.15mm, colback=white, colframe=quarto-callout-note-color-frame, left=2mm, arc=.35mm, opacityback=0, rightrule=.15mm, breakable, bottomrule=.15mm, leftrule=.75mm]
\begin{minipage}[t]{5.5mm}
\textcolor{quarto-callout-note-color}{\faInfo}
\end{minipage}%
\begin{minipage}[t]{\textwidth - 5.5mm}

\vspace{-3mm}\textbf{Note}\vspace{3mm}

MITA 4.0 does not endeavor to specify all of the capabilities SMA's may
need to administer Medicaid programs; instead, this version of MITA
focuses on the capabilities that are most closely oriented towards
achieving the CMS-required outcomes.

\end{minipage}%
\end{tcolorbox}

Understanding the how the MITA Capability Model works is important to
obtaining the most value out of many of the other tools and artifacts in
the MITA framework, such as the MITA Maturity Model (MMM) and the
Business Process Model (BPM). The MITA Capability Model provides a
structured way for SMAs to identify, conceptually model, and improve the
capabilities needed for efficient Medicaid operations.

It is important to note that MITA 4.0 does not endeavor to specify all
of the capabilities SMA's may need to administer Medicaid programs;
instead, this version of MITA focuses on the capabilities that are most
closely oriented towards achieving the CMS-required outcomes. In this
way MITA 4.0 provides a reference model for SMAs to model other
capabilities that may be needed to achieve their other goals such as
state specific outcomes, or other state priorities while providing more
guidance within the MITA Framework to support modular.

\subsection{Update to MITA 3.0}\label{update-to-mita-3.0-1}

MITA 3.0 defined a capability as the level of maturity of a set of
business processes within a business category. By focusing on ``how''
MES operate MITA 3.0 helped SMA's identify ways to improve and mature
their business processes, but it did not link those processes with the
outcomes they are intended to achieve or ensure better alignment of the
information and technical architectures to business outcomes. The
addition of the MITA capability model to the MITA 4.0 business
architecture addresses that by providing the conceptual linkages needed
to elevate the strategic vantage point of the MITA Framework. To guide
this change, we present within this chapter a definition, description,
and approach to modeling business capabilities, based on the widely used
capability models contextualized for Medicaid Enterprises.

The business processes that operationalize MITA capabilities remain
foundational to characterizing the business architecture, and are by
definition a constituent part of any MITA capability. They provide
essential information on how capabilities are operationalize and should
continue to be a routinely utilized reference model for SMA business
process mapping. They are found with in the Business Process Model
chapter of this version of MITA.

\section{The MITA Definition of
Capability}\label{the-mita-definition-of-capability-1}

Within the context of MITA, a capability can be defined as the ability
or capacity of a State Medicaid Agency to achieve a desired outcome in
compliance with the
\href{https://www.ecfr.gov/current/title-42/chapter-IV/subchapter-C/part-433/subpart-C/section-433.112}{Standards
and Conditions within 42 CFR 433.112}. A capability may currently exist
in an operational state or be envisioned for future development. Through
careful planning, capabilities defined in this way can be matured and
refined over time to become more effective and efficient. They can be
organized and detailed at various levels of abstraction, providing
precise descriptions for operational purposes or more generalized views
for strategic planning.

\begin{tcolorbox}[enhanced jigsaw, toprule=.15mm, colback=white, colframe=quarto-callout-note-color-frame, left=2mm, arc=.35mm, opacityback=0, rightrule=.15mm, breakable, bottomrule=.15mm, leftrule=.75mm]

\vspace{-3mm}\textbf{Key Definition}\vspace{3mm}

\ldots a capability is defined as the ability or capacity of a SMA to
achieve a desired outcome\ldots{}

\end{tcolorbox}

To fully define a business capability, it is essential to understand how
it is realized through the integration of people, processes,
information, and technology resources of an SMA. While these elements of
the capability can change regularly, the capability itself is should
endure over longer planning horizons, supporting the long-term alignment
of business and IT and the achievement of increasingly beneficial
business outcomes.

\subsection{Structure of the MITA Capability
Model}\label{structure-of-the-mita-capability-model-1}

As depicted in the model below, the MITA Capability Model orients the
people, process, technology, and information resources to define a MITA
Capability. This means that to model a capability the appropriate
components of the information architecture and the technical
architecture must be brought together with the business architecture to
fully formulate any MITA Capability.

\begin{figure}[H]

{\centering \pandocbounded{\includegraphics[keepaspectratio]{media/capabilityModel/topLevelCapabilityMetamodelGraphic1.png}}

}

\caption{MITA Capability Relationship Diagram}

\end{figure}%

\subsubsection{Business Roles}\label{business-roles-1}

Business roles represent individual actors, stakeholders, or partners
involved in delivering a business capability. A single organizational
group or team may be wholly responsible for delivering the capability,
or multiple business entities may share the delivery of a particular
business capability. Business Roles perform Business Processes using
Technology Resources. They require skills and knowledge resources to
achieve outcomes, and should be actively engaged as partners in the
development or enhancement of any capability they help deliver.

\subsubsection{Business Processes}\label{business-processes-1}

Individual business capabilities may be enabled or delivered through a
range of business processes that detail the activities (the how)
associated with delivering the capability. Identifying and analyzing the
efficiency of the underlying processes helps to optimize the business
capability's effectiveness. Identifying the processes within a business
capability provides a focus for maturing the capability in concert with
the other capability components. Business Processes operationalize
Business Capabilities.

\subsubsection{Information/Data}\label{informationdata-1}

Information/data represents the business data, knowledge, and insight
consumed or produced by the business capability (as distinct from
IT-related data entities). This may also include information that the
capability exchanges with other capabilities to support the execution of
value streams. Examples include information about customers and
prospects, products and services, business policies and rules, sales
reports, and performance metrics. Information/data inform the Business
Capability, answering questions and supporting business rules.

\subsubsection{Technology Resources}\label{technology-resources-1}

Business capabilities rely on a range of tools, applications, systems,
and services for successful execution. Technology Resources use
Information/data to facilitate Business Processes. Such resources may
include:

\begin{itemize}
\tightlist
\item
  Modular software applications

  \begin{itemize}
  \tightlist
  \item
    Cloud or on-premise infrastructure
  \item
    Microservices
  \item
    Analytics
  \item
    Customer portal
  \end{itemize}
\end{itemize}

In this way we can clearly interrelate all of the MITA architecture
models and their individual components which allows us to reveal gaps
not only in the individual components of the architecture, but also
understand their impact on the integration of the architecture
components at the capability level.

\subsection{Relationship of MITA Capabilities to
Outcomes}\label{relationship-of-mita-capabilities-to-outcomes-1}

In the context of the Medicaid Information Technology Architecture
(MITA), outcomes are intrinsically linked to capabilities, as they
represent the tangible results achieved through the effective
integration and execution of various elements that constitute a
capability. In this sense, outcomes and capabilities define each other.

\begin{figure}[H]

{\centering \pandocbounded{\includegraphics[keepaspectratio]{media/capabilityModel/topLevelCapabilityMetamodelOutcomes.png}}

}

\caption{MITA Capability and Outcome Relationship Diagram}

\end{figure}%

\subsubsection{Outcomes}\label{outcomes-1}

MITA defines outcomes broadly to encompass CMS-required outcomes,
state-specific outcomes, and other outcomes not mandated as part of the
Advance Planning Document (APD) process. The sole criterion for an
outcome to meet this definition is that it must be a goal of a State
Medicaid Agency (SMA) and be achieved through a Medicaid Enterprise
System (MES) capability.

\begin{tcolorbox}[enhanced jigsaw, toprule=.15mm, colback=white, colframe=quarto-callout-note-color-frame, left=2mm, arc=.35mm, opacityback=0, rightrule=.15mm, breakable, bottomrule=.15mm, leftrule=.75mm]

\vspace{-3mm}\textbf{Key Definition}\vspace{3mm}

A MITA outcome is a goal of a State Medicaid Agency (SMA) that is
achieved by a Medicaid Enterprise System (MES) capability.

\end{tcolorbox}

\subsubsection{Measure}\label{measure-1}

Measure is a quantifiable metric used to assess the effectiveness and
efficiency of capabilities within a Medicaid Enterprise System (MES).
Measures provide quantifiable and qualitative values that help State
Medicaid Agencies (SMAs) track progress toward achieving specific
outcomes, such as CMS-required or state-specific goals. These indicators
might include metrics like processing times, error rates, or compliance
levels.

Measures are a measurement threshold by establishing a specific value or
level that must be met or exceeded to demonstrate successful
performance. For instance, a KPI might set a threshold for the maximum
allowable processing time for claims, ensuring that they are handled
within a specified timeframe to maintain compliance and eligibility for
enhanced federal funding. By monitoring these thresholds, organizations
can ensure they are meeting regulatory requirements and delivering
high-quality services to beneficiaries, while also identifying areas for
improvement.

\subsubsection{Measure Threshold}\label{measure-threshold-1}

A specific value or level of a measure that must be met or exceeded to
demonstrate the effective achievement of a capability's intended
outcome. This threshold serves as a benchmark for assessing whether the
processes, roles, and resources integrated within a Medicaid Enterprise
System (MES) are functioning optimally to meet the goals of a State
Medicaid Agency (SMA). For example, a measurement threshold might be set
for processing times, where claims must be processed within a certain
number of days to ensure compliance with CMS-required outcomes and
maintain eligibility for enhanced federal funding. By establishing and
monitoring these thresholds, organizations can ensure they are meeting
regulatory requirements and delivering high-quality services to
beneficiaries.

\subsubsection{Measurement}\label{measurement-1}

These outcomes and metrics are also used to ensure that healthcare
systems or modules comply with applicable federal regulations, forming
the baseline for system or module functionality. Achieving these
outcomes is essential for continuing to receive enhanced federal funding
for operations. Regular measurement and analysis of KPIs help
organizations demonstrate compliance and effectiveness, ensuring that
they meet regulatory requirements and continue to deliver high-quality
services to beneficiaries.

In this way we can clearly interrelate all of the MITA architecture
models and their individual components with the KPIs, thresholds, and
measurements that indicate whether our capability achieves our desired
outcome.

While models that help conceptualize the capabilities that achieve
CMS-required outcomes are the ones modeled for this version of MITA,
SMAs are encouraged to use these models as a reference to model
capabilities.

\section{Capability Mapping}\label{capability-mapping-1}

Capability mapping is a strategic tool that enables organizations, such
as State Medicaid Agencies (SMAs), to systematically identify, organize,
and visualize the key capabilities necessary to achieve their
objectives. Within the MITA framework, capability mapping provides SMAs
with a method of developing comprehensive views of the functions and
processes required to deliver Medicaid services effectively. To begin
the capability mapping process, SMAs should first identify the core
capabilities that align with their strategic objectives, focusing on
what the organization needs to achieve rather than how those goals are
accomplished. This involves listing all necessary capabilities and
understanding the desired outcomes they support. Next, these
capabilities should be organized into domains and areas that reflect
their strategic importance and interrelationships. Visualizing these
capabilities through diagrams or maps provides all stakeholders a common
view to understand the roles, processes, technology resources, and
information/data involved in executing each capability, as well as the
outcome each capability is designed to achieve. This structured approach
not only highlights areas for improvement or investment but also ensures
that organizational efforts are strategically aligned with desired
outcomes.

The benefits of capability mapping are multifaceted, offering SMAs a
clear pathway to strategic alignment and gap analysis. By visualizing
capabilities, organizations can identify operational gaps and determine
what new or enhanced capabilities are needed to close those gaps. This
visualization also improves communication among stakeholders by
providing a clear and concise representation of the organization's
functions. To refine capabilities, SMAs should analyze current
operations, assess the efficiency of underlying processes, and optimize
them to enhance capability effectiveness. Additionally, capability
mapping serves as a foundation for heat mapping, which assesses the MITA
Framework will utilize to visualize the maturity of each capability
evaluated in the State Self-Assessment. SMAs can overlay heat maps over
their capability maps to visualize many things other than maturity
levels, using color coding to indicate areas of strength and weakness.
Regular updates to these maps allow SMAs to monitor progress and ensure
resources are allocated effectively to achieve strategic goals. The MITA
framework includes examples of capability maps based on CMS-required
outcomes, serving as a reference model for SMAs to develop their own
capability maps tailored to state-specific goals and priorities. By
leveraging the reference models provided by MITA, SMAs can ensure their
capability mapping efforts are aligned with both federal requirements
and state-specific priorities.

\subsection{Organizing Capabilities}\label{organizing-capabilities-1}

To enhance the resolution and detail of a capability and provide a
unified view of all its components, a block diagram can be employed to
provide a common view of any MES. This diagram effectively links the
capability to business processes, roles, technical resources, and
information resources through functional decomposition. By breaking down
the capability into its constituent parts, the block diagram offers a
visual representation that highlights the interrelationships and
dependencies among these elements. This approach provides a clearer
understanding of how each component contributes to the overall
capability, facilitating more effective analysis, optimization, and
alignment with organizational objectives.

\pandocbounded{\includegraphics[keepaspectratio]{media/capabilityModel/capabilityOgranizationModel.drawio.png}}

We use this same method to present an this top level view of the
capabilities required to achieve CMS-required outcomes. From this view
increasingly detailed models can be constructed.

\pandocbounded{\includegraphics[keepaspectratio]{media/capabilityModel/mesModuleBasedCapabilities.drawio.png}}

\subsection{MITA Capability Models}\label{mita-capability-models-1}

The MITA framework represents capabilities visually through a layered
model that represent a capability of being composed of sub-capabilities
and the processes, roles, information and technology resources (PRIT)
that support the business in sustaining the capability. Each layer up
depicts increasingly strategic capabilities and each layer down depicts
the constituent elements that compose a capability in increasing
operational detail. It is not the intention of this version of MITA to
provide a full operational or tactical view of a capability, though SMAs
may consider using this approach to improve their organizational
awareness of their operations by developing further layers of their
capabilities through functional decomposition.

\pandocbounded{\includegraphics[keepaspectratio]{media/capabilityModel/capabilityLevels.png}}

\begin{itemize}
\tightlist
\item
  \textbf{Capability Domains:} The first layer of this model aims to
  group capabilities to organize the strategic view of an SMA's
  capabilities. In this view one or many capabilities can be grouped
  within a domain to indicate the pursuit of common outcomes. Each
  domain is denoted with a single number to help annotate each
  capability.

  \begin{itemize}
  \tightlist
  \item
    \textbf{Capability Areas:} The second layer of this model aims to
    provide a view of the groups of capabilities that compose a domain.
    They are organized to show capabilities that serve a specific group
    of similar outcomes and essential
  \item
    \textbf{Capabilities:} The third layer of this model provides a more
    detailed view view of
  \end{itemize}
\end{itemize}

\pandocbounded{\includegraphics[keepaspectratio]{media/capabilityModel/capabilityLevels2.png}}

\subsection{Relationship of MITA Capabilities to
Maturity}\label{relationship-of-mita-capabilities-to-maturity-1}

\begin{tcolorbox}[enhanced jigsaw, toprule=.15mm, leftrule=.75mm, colframe=quarto-callout-warning-color-frame, left=2mm, arc=.35mm, titlerule=0mm, rightrule=.15mm, opacitybacktitle=0.6, bottomtitle=1mm, toptitle=1mm, colbacktitle=quarto-callout-warning-color!10!white, bottomrule=.15mm, title=\textcolor{quarto-callout-warning-color}{\faExclamationTriangle}\hspace{0.5em}{Warning}, opacityback=0, breakable, colback=white, coltitle=black]

Under development.

\end{tcolorbox}

\begin{itemize}
\tightlist
\item
  \textbf{Levels of Maturity}

  \begin{itemize}
  \tightlist
  \item
    Description of the five levels of maturity in the MITA framework
  \item
    How capabilities evolve and mature over time
  \end{itemize}
\end{itemize}

\pandocbounded{\includegraphics[keepaspectratio]{media/capabilityModel/maturityModel.png}}

\subsection{Using Capability Maps for Heat Mapping Strategic Priorities
and Identifying Gaps with the MITA Maturity
Model}\label{using-capability-maps-for-heat-mapping-strategic-priorities-and-identifying-gaps-with-the-mita-maturity-model-1}

Capability maps are powerful tools that not only provide a visual
representation of an SMA's key capabilities but also serve as a
foundation for strategic analysis and planning. There are many
approaches to heat mapping capabilities, each offering unique insights
into organizational priorities and gaps. Here, we describe two
approaches: assessing maturity levels using the MITA Maturity Model and
prioritizing strategic outcomes.

\subsubsection{Identifying Gaps with the MITA Maturity
Model}\label{identifying-gaps-with-the-mita-maturity-model-1}

The MITA Maturity Model provides a framework for assessing the maturity
of an organization's capabilities across various dimensions, such as
business processes, information, and technology. By integrating the
maturity model with capability maps, SMAs can identify gaps between
their current state and desired maturity levels.

\paragraph{Example 1: Identifying Gaps in Data Management Maturity Using
the PRIT
Model}\label{example-1-identifying-gaps-in-data-management-maturity-using-the-prit-model-1}

An SMA is conducting an assessment of its data management capabilities
using the MITA Maturity Model, with a focus on the PRIT (Processes,
Roles, Information, and Technology) framework. The capability map
includes various data-related capabilities, such as ``Data
Integration,'' ``Data Quality Management,'' and ``Data Analytics.'' Each
of these capabilities is evaluated across the PRIT dimensions to
determine their maturity levels using the revised scale:

Processes: Level 1: Ad-Hoc Roles: Level 2: Compliant Information: Level
2: Compliant Technology: Level 2: Compliant The capability map is
updated to reflect the maturity assessment, with each dimension marked
with a color code: red for Level 1: Ad-Hoc, yellow for Level 2:
Compliant, green for Level 3: Efficient, blue for Level 4: Optimized,
and purple for Level 5: Pioneering. This visualization helps the SMA
prioritize strategic actions to enhance the ``Data Integration''
capability, such as standardizing processes, refining roles, improving
data quality, and upgrading technology.

\subsubsection{Heat Mapping Strategic
Priorities}\label{heat-mapping-strategic-priorities-1}

Heat mapping involves applying a color-coded overlay to a capability map
to visually represent the status or priority level of each capability.
This technique can be used to highlight areas of strength, weakness, or
strategic importance. For example, capabilities that are critical to
achieving CMS-required outcomes might be marked in one color, while
those needing immediate attention or improvement could be marked in
another. This visual representation helps stakeholders quickly grasp the
strategic landscape and make informed decisions about where to allocate
resources and focus efforts.

\paragraph{Example 2: Prioritizing Capabilities for CMS-Required
Outcomes}\label{example-2-prioritizing-capabilities-for-cms-required-outcomes-1}

An SMA is focused on achieving specific CMS-required outcomes related to
improving patient care and reducing administrative costs. The agency
creates a capability map that outlines all the capabilities necessary to
meet these outcomes. By applying a heat map, the SMA highlights
capabilities that are directly linked to these outcomes in green,
indicating they are of high strategic priority. Capabilities that are
indirectly related or less critical are marked in yellow, while those
that are currently underperforming or not aligned with strategic goals
are marked in red.

This visual representation allows the SMA to quickly identify which
capabilities require immediate attention and resources to ensure
compliance with CMS requirements. For instance, if the capability
related to ``Claims Processing Efficiency'' is marked in red, the agency
can prioritize initiatives to enhance this capability, such as investing
in new technology or streamlining processes.

\subsubsection{Other Uses for Capability Heat
Mapping}\label{other-uses-for-capability-heat-mapping-1}

Beyond assessing maturity levels and prioritizing strategic initiatives,
capability heat mapping can be applied in various other contexts to
enhance organizational effectiveness and alignment.

\paragraph{Example 3: Aligning Capabilities with State-Specific
Initiatives}\label{example-3-aligning-capabilities-with-state-specific-initiatives-1}

An SMA is working on a state-specific initiative to enhance telehealth
services for rural populations. The capability map includes capabilities
related to telehealth, such as ``Telehealth Infrastructure,'' ``Provider
Engagement,'' and ``Patient Access.'' The SMA uses a heat map to
highlight these capabilities in blue, indicating their alignment with
the state-specific initiative.

By analyzing the capability map, the SMA identifies that ``Provider
Engagement'' is a critical capability that requires further development
to support the telehealth initiative. The agency decides to invest in
training programs and outreach efforts to engage providers in rural
areas, ensuring that the telehealth services are effectively delivered
to the target population.

These examples demonstrate how capability maps, combined with heat
mapping and the MITA Maturity Model, can provide valuable insights for
strategic planning and gap analysis. By visualizing priorities and
maturity levels, SMAs can make informed decisions about where to focus
resources and efforts, ultimately enhancing their Medicaid Enterprise
Systems and achieving strategic objectives.

\begin{itemize}
\tightlist
\item
  \textbf{Capability Mapping}

  \begin{itemize}
  \tightlist
  \item
    Introduction to capability mapping and its significance
  \item
    How capabilities are organized and detailed at various levels of
    abstraction
  \end{itemize}
\end{itemize}

\section{Guidance on reuse of the MITA Capability
Model}\label{guidance-on-reuse-of-the-mita-capability-model-1}

\begin{itemize}
\tightlist
\item
  \textbf{Practical Application}

  \begin{itemize}
  \tightlist
  \item
    How to integrate the capability model into daily operations and
    strategic planning
  \item
    Tips for maximizing the benefits of the model
  \end{itemize}
\item
  \textbf{Continuous Improvement}

  \begin{itemize}
  \tightlist
  \item
    Encouragement for ongoing assessment and refinement of capabilities
  \item
    Leveraging feedback and performance data for model enhancement
  \end{itemize}
\item
  \textbf{Implementation Guidance}

  \begin{itemize}
  \tightlist
  \item
    Steps for adopting the capability model
  \item
    Resources and support available for SMAs
  \end{itemize}
\item
  \textbf{Performance Monitoring and Reporting}

  \begin{itemize}
  \tightlist
  \item
    Role of the capability model in tracking and enhancing performance
  \item
    Use of metrics and standards to measure capability effectiveness
  \end{itemize}
\end{itemize}

\chapter{MITA Capability Model}\label{mita-capability-model-2}

\section{Introduction to Business Capability
Models}\label{introduction-to-business-capability-models-1}

A capability model is a conceptual framework that outlines the key
capabilities an organization needs to achieve its strategic objectives.
It provides a comprehensive view of what an organization can do and
helps identify areas for improvement or investment. In the context of an
orchestra, a capability model might help the orchestra identify the set
of skills and resources, or other types of capabilities it needs to
perform a symphony. Just like an orchestra needs well practiced
musicians, sheet music, instruments, a conductor, and an audience to
produce a great symphony, a State Medicaid Agency (SMA) needs its
Medicaid Enterprise System (MES) to employ or develop specific
capabilities to deliver its services effectively, efficiently, and
economically to its enrollees and providers.

The concept of a business capability is extensively used within
enterprise architecture modeling and has been broadly used within
Business Capability Models as a tool to better align the business
strategy and information technology of both private sector and
governmental organizations since they emerged in the mid-2000s. One
example comes from the TOGAF Standard, a well-known standard in
enterprise architecture. Like most architecture frameworks TOGAF defines
a capability as something a business can do to meet its goals. This
focuses a strategic lens of an organization on ``what'' it needs to
achieve its goals, rather than ``how'' those goals are achieved. This
perspective allows for business planning from different viewpoints,
facilitating strategic alignment and operational efficiency.

SMA business architects, technologists, systems analysts, executives,
managers, and program staff can use this same modeling approach to
represent the functional components of their Medicaid Enterprise System
(MES) in ways that can help reveal gaps in their systems and provide
insights on what new or enhanced capabilities might be needed to close
those gaps.

By focusing on capabilities, SMAs can better align their information and
technology resources and processes with their strategic business goals,
ultimately improving their insight into how to improve the outcomes
their Medicaid Enterprise Architecture produces.

\subsection{Purpose}\label{purpose-2}

\begin{tcolorbox}[enhanced jigsaw, toprule=.15mm, colback=white, colframe=quarto-callout-note-color-frame, left=2mm, arc=.35mm, opacityback=0, rightrule=.15mm, breakable, bottomrule=.15mm, leftrule=.75mm]
\begin{minipage}[t]{5.5mm}
\textcolor{quarto-callout-note-color}{\faInfo}
\end{minipage}%
\begin{minipage}[t]{\textwidth - 5.5mm}

\vspace{-3mm}\textbf{Note}\vspace{3mm}

MITA 4.0 does not endeavor to specify all of the capabilities SMA's may
need to administer Medicaid programs; instead, this version of MITA
focuses on the capabilities that are most closely oriented towards
achieving the CMS-required outcomes.

\end{minipage}%
\end{tcolorbox}

Understanding the how the MITA Capability Model works is important to
obtaining the most value out of many of the other tools and artifacts in
the MITA framework, such as the MITA Maturity Model (MMM) and the
Business Process Model (BPM). The MITA Capability Model provides a
structured way for SMAs to identify, conceptually model, and improve the
capabilities needed for efficient Medicaid operations.

It is important to note that MITA 4.0 does not endeavor to specify all
of the capabilities SMA's may need to administer Medicaid programs;
instead, this version of MITA focuses on the capabilities that are most
closely oriented towards achieving the CMS-required outcomes. In this
way MITA 4.0 provides a reference model for SMAs to model other
capabilities that may be needed to achieve their other goals such as
state specific outcomes, or other state priorities while providing more
guidance within the MITA Framework to support modular.

\subsection{Update to MITA 3.0}\label{update-to-mita-3.0-2}

MITA 3.0 defined a capability as the level of maturity of a set of
business processes within a business category. By focusing on ``how''
MES operate MITA 3.0 helped SMA's identify ways to improve and mature
their business processes, but it did not link those processes with the
outcomes they are intended to achieve or ensure better alignment of the
information and technical architectures to business outcomes. The
addition of the MITA capability model to the MITA 4.0 business
architecture addresses that by providing the conceptual linkages needed
to elevate the strategic vantage point of the MITA Framework. To guide
this change, we present within this chapter a definition, description,
and approach to modeling business capabilities, based on the widely used
capability models contextualized for Medicaid Enterprises.

The business processes that operationalize MITA capabilities remain
foundational to characterizing the business architecture, and are by
definition a constituent part of any MITA capability. They provide
essential information on how capabilities are operationalize and should
continue to be a routinely utilized reference model for SMA business
process mapping. They are found with in the Business Process Model
chapter of this version of MITA.

\section{The MITA Definition of
Capability}\label{the-mita-definition-of-capability-2}

Within the context of MITA, a capability can be defined as the ability
or capacity of a State Medicaid Agency to achieve a desired outcome in
compliance with the
\href{https://www.ecfr.gov/current/title-42/chapter-IV/subchapter-C/part-433/subpart-C/section-433.112}{Standards
and Conditions within 42 CFR 433.112}. A capability may currently exist
in an operational state or be envisioned for future development. Through
careful planning, capabilities defined in this way can be matured and
refined over time to become more effective and efficient. They can be
organized and detailed at various levels of abstraction, providing
precise descriptions for operational purposes or more generalized views
for strategic planning.

\begin{tcolorbox}[enhanced jigsaw, toprule=.15mm, colback=white, colframe=quarto-callout-note-color-frame, left=2mm, arc=.35mm, opacityback=0, rightrule=.15mm, breakable, bottomrule=.15mm, leftrule=.75mm]

\vspace{-3mm}\textbf{Key Definition}\vspace{3mm}

\ldots a capability is defined as the ability or capacity of a SMA to
achieve a desired outcome\ldots{}

\end{tcolorbox}

To fully define a business capability, it is essential to understand how
it is realized through the integration of people, processes,
information, and technology resources of an SMA. While these elements of
the capability can change regularly, the capability itself is should
endure over longer planning horizons, supporting the long-term alignment
of business and IT and the achievement of increasingly beneficial
business outcomes.

\subsection{Structure of the MITA Capability
Model}\label{structure-of-the-mita-capability-model-2}

As depicted in the model below, the MITA Capability Model orients the
people, process, technology, and information resources to define a MITA
Capability. This means that to model a capability the appropriate
components of the information architecture and the technical
architecture must be brought together with the business architecture to
fully formulate any MITA Capability.

\begin{figure}[H]

{\centering \pandocbounded{\includegraphics[keepaspectratio]{media/capabilityModel/topLevelCapabilityMetamodelGraphic1.png}}

}

\caption{MITA Capability Relationship Diagram}

\end{figure}%

\subsubsection{Business Roles}\label{business-roles-2}

Business roles represent individual actors, stakeholders, or partners
involved in delivering a business capability. A single organizational
group or team may be wholly responsible for delivering the capability,
or multiple business entities may share the delivery of a particular
business capability. Business Roles perform Business Processes using
Technology Resources. They require skills and knowledge resources to
achieve outcomes, and should be actively engaged as partners in the
development or enhancement of any capability they help deliver.

\subsubsection{Business Processes}\label{business-processes-2}

Individual business capabilities may be enabled or delivered through a
range of business processes that detail the activities (the how)
associated with delivering the capability. Identifying and analyzing the
efficiency of the underlying processes helps to optimize the business
capability's effectiveness. Identifying the processes within a business
capability provides a focus for maturing the capability in concert with
the other capability components. Business Processes operationalize
Business Capabilities.

\subsubsection{Information/Data}\label{informationdata-2}

Information/data represents the business data, knowledge, and insight
consumed or produced by the business capability (as distinct from
IT-related data entities). This may also include information that the
capability exchanges with other capabilities to support the execution of
value streams. Examples include information about customers and
prospects, products and services, business policies and rules, sales
reports, and performance metrics. Information/data inform the Business
Capability, answering questions and supporting business rules.

\subsubsection{Technology Resources}\label{technology-resources-2}

Business capabilities rely on a range of tools, applications, systems,
and services for successful execution. Technology Resources use
Information/data to facilitate Business Processes. Such resources may
include:

\begin{itemize}
\tightlist
\item
  Modular software applications

  \begin{itemize}
  \tightlist
  \item
    Cloud or on-premise infrastructure
  \item
    Microservices
  \item
    Analytics
  \item
    Customer portal
  \end{itemize}
\end{itemize}

In this way we can clearly interrelate all of the MITA architecture
models and their individual components which allows us to reveal gaps
not only in the individual components of the architecture, but also
understand their impact on the integration of the architecture
components at the capability level.

\subsection{Relationship of MITA Capabilities to
Outcomes}\label{relationship-of-mita-capabilities-to-outcomes-2}

In the context of the Medicaid Information Technology Architecture
(MITA), outcomes are intrinsically linked to capabilities, as they
represent the tangible results achieved through the effective
integration and execution of various elements that constitute a
capability. In this sense, outcomes and capabilities define each other.

\begin{figure}[H]

{\centering \pandocbounded{\includegraphics[keepaspectratio]{media/capabilityModel/topLevelCapabilityMetamodelOutcomes.png}}

}

\caption{MITA Capability and Outcome Relationship Diagram}

\end{figure}%

\subsubsection{Outcomes}\label{outcomes-2}

MITA defines outcomes broadly to encompass CMS-required outcomes,
state-specific outcomes, and other outcomes not mandated as part of the
Advance Planning Document (APD) process. The sole criterion for an
outcome to meet this definition is that it must be a goal of a State
Medicaid Agency (SMA) and be achieved through a Medicaid Enterprise
System (MES) capability.

\begin{tcolorbox}[enhanced jigsaw, toprule=.15mm, colback=white, colframe=quarto-callout-note-color-frame, left=2mm, arc=.35mm, opacityback=0, rightrule=.15mm, breakable, bottomrule=.15mm, leftrule=.75mm]

\vspace{-3mm}\textbf{Key Definition}\vspace{3mm}

A MITA outcome is a goal of a State Medicaid Agency (SMA) that is
achieved by a Medicaid Enterprise System (MES) capability.

\end{tcolorbox}

\subsubsection{Measure}\label{measure-2}

Measure is a quantifiable metric used to assess the effectiveness and
efficiency of capabilities within a Medicaid Enterprise System (MES).
Measures provide quantifiable and qualitative values that help State
Medicaid Agencies (SMAs) track progress toward achieving specific
outcomes, such as CMS-required or state-specific goals. These indicators
might include metrics like processing times, error rates, or compliance
levels.

Measures are a measurement threshold by establishing a specific value or
level that must be met or exceeded to demonstrate successful
performance. For instance, a KPI might set a threshold for the maximum
allowable processing time for claims, ensuring that they are handled
within a specified timeframe to maintain compliance and eligibility for
enhanced federal funding. By monitoring these thresholds, organizations
can ensure they are meeting regulatory requirements and delivering
high-quality services to beneficiaries, while also identifying areas for
improvement.

\subsubsection{Measure Threshold}\label{measure-threshold-2}

A specific value or level of a measure that must be met or exceeded to
demonstrate the effective achievement of a capability's intended
outcome. This threshold serves as a benchmark for assessing whether the
processes, roles, and resources integrated within a Medicaid Enterprise
System (MES) are functioning optimally to meet the goals of a State
Medicaid Agency (SMA). For example, a measurement threshold might be set
for processing times, where claims must be processed within a certain
number of days to ensure compliance with CMS-required outcomes and
maintain eligibility for enhanced federal funding. By establishing and
monitoring these thresholds, organizations can ensure they are meeting
regulatory requirements and delivering high-quality services to
beneficiaries.

\subsubsection{Measurement}\label{measurement-2}

These outcomes and metrics are also used to ensure that healthcare
systems or modules comply with applicable federal regulations, forming
the baseline for system or module functionality. Achieving these
outcomes is essential for continuing to receive enhanced federal funding
for operations. Regular measurement and analysis of KPIs help
organizations demonstrate compliance and effectiveness, ensuring that
they meet regulatory requirements and continue to deliver high-quality
services to beneficiaries.

In this way we can clearly interrelate all of the MITA architecture
models and their individual components with the KPIs, thresholds, and
measurements that indicate whether our capability achieves our desired
outcome.

While models that help conceptualize the capabilities that achieve
CMS-required outcomes are the ones modeled for this version of MITA,
SMAs are encouraged to use these models as a reference to model
capabilities.

\section{Capability Mapping}\label{capability-mapping-2}

Capability mapping is a strategic tool that enables organizations, such
as State Medicaid Agencies (SMAs), to systematically identify, organize,
and visualize the key capabilities necessary to achieve their
objectives. Within the MITA framework, capability mapping provides SMAs
with a method of developing comprehensive views of the functions and
processes required to deliver Medicaid services effectively. To begin
the capability mapping process, SMAs should first identify the core
capabilities that align with their strategic objectives, focusing on
what the organization needs to achieve rather than how those goals are
accomplished. This involves listing all necessary capabilities and
understanding the desired outcomes they support. Next, these
capabilities should be organized into domains and areas that reflect
their strategic importance and interrelationships. Visualizing these
capabilities through diagrams or maps provides all stakeholders a common
view to understand the roles, processes, technology resources, and
information/data involved in executing each capability, as well as the
outcome each capability is designed to achieve. This structured approach
not only highlights areas for improvement or investment but also ensures
that organizational efforts are strategically aligned with desired
outcomes.

The benefits of capability mapping are multifaceted, offering SMAs a
clear pathway to strategic alignment and gap analysis. By visualizing
capabilities, organizations can identify operational gaps and determine
what new or enhanced capabilities are needed to close those gaps. This
visualization also improves communication among stakeholders by
providing a clear and concise representation of the organization's
functions. To refine capabilities, SMAs should analyze current
operations, assess the efficiency of underlying processes, and optimize
them to enhance capability effectiveness. Additionally, capability
mapping serves as a foundation for heat mapping, which assesses the MITA
Framework will utilize to visualize the maturity of each capability
evaluated in the State Self-Assessment. SMAs can overlay heat maps over
their capability maps to visualize many things other than maturity
levels, using color coding to indicate areas of strength and weakness.
Regular updates to these maps allow SMAs to monitor progress and ensure
resources are allocated effectively to achieve strategic goals. The MITA
framework includes examples of capability maps based on CMS-required
outcomes, serving as a reference model for SMAs to develop their own
capability maps tailored to state-specific goals and priorities. By
leveraging the reference models provided by MITA, SMAs can ensure their
capability mapping efforts are aligned with both federal requirements
and state-specific priorities.

\subsection{Organizing Capabilities}\label{organizing-capabilities-2}

To enhance the resolution and detail of a capability and provide a
unified view of all its components, a block diagram can be employed to
provide a common view of any MES. This diagram effectively links the
capability to business processes, roles, technical resources, and
information resources through functional decomposition. By breaking down
the capability into its constituent parts, the block diagram offers a
visual representation that highlights the interrelationships and
dependencies among these elements. This approach provides a clearer
understanding of how each component contributes to the overall
capability, facilitating more effective analysis, optimization, and
alignment with organizational objectives.

\pandocbounded{\includegraphics[keepaspectratio]{media/capabilityModel/capabilityOgranizationModel.drawio.png}}

We use this same method to present an this top level view of the
capabilities required to achieve CMS-required outcomes. From this view
increasingly detailed models can be constructed.

\pandocbounded{\includegraphics[keepaspectratio]{media/capabilityModel/mesModuleBasedCapabilities.drawio.png}}

\subsection{MITA Capability Models}\label{mita-capability-models-2}

The MITA framework represents capabilities visually through a layered
model that represent a capability of being composed of sub-capabilities
and the processes, roles, information and technology resources (PRIT)
that support the business in sustaining the capability. Each layer up
depicts increasingly strategic capabilities and each layer down depicts
the constituent elements that compose a capability in increasing
operational detail. It is not the intention of this version of MITA to
provide a full operational or tactical view of a capability, though SMAs
may consider using this approach to improve their organizational
awareness of their operations by developing further layers of their
capabilities through functional decomposition.

\pandocbounded{\includegraphics[keepaspectratio]{media/capabilityModel/capabilityLevels.png}}

\begin{itemize}
\tightlist
\item
  \textbf{Capability Domains:} The first layer of this model aims to
  group capabilities to organize the strategic view of an SMA's
  capabilities. In this view one or many capabilities can be grouped
  within a domain to indicate the pursuit of common outcomes. Each
  domain is denoted with a single number to help annotate each
  capability.

  \begin{itemize}
  \tightlist
  \item
    \textbf{Capability Areas:} The second layer of this model aims to
    provide a view of the groups of capabilities that compose a domain.
    They are organized to show capabilities that serve a specific group
    of similar outcomes and essential
  \item
    \textbf{Capabilities:} The third layer of this model provides a more
    detailed view view of
  \end{itemize}
\end{itemize}

\pandocbounded{\includegraphics[keepaspectratio]{media/capabilityModel/capabilityLevels2.png}}

\subsection{Relationship of MITA Capabilities to
Maturity}\label{relationship-of-mita-capabilities-to-maturity-2}

\begin{tcolorbox}[enhanced jigsaw, toprule=.15mm, leftrule=.75mm, colframe=quarto-callout-warning-color-frame, left=2mm, arc=.35mm, titlerule=0mm, rightrule=.15mm, opacitybacktitle=0.6, bottomtitle=1mm, toptitle=1mm, colbacktitle=quarto-callout-warning-color!10!white, bottomrule=.15mm, title=\textcolor{quarto-callout-warning-color}{\faExclamationTriangle}\hspace{0.5em}{Warning}, opacityback=0, breakable, colback=white, coltitle=black]

Under development.

\end{tcolorbox}

\begin{itemize}
\tightlist
\item
  \textbf{Levels of Maturity}

  \begin{itemize}
  \tightlist
  \item
    Description of the five levels of maturity in the MITA framework
  \item
    How capabilities evolve and mature over time
  \end{itemize}
\end{itemize}

\pandocbounded{\includegraphics[keepaspectratio]{media/capabilityModel/maturityModel.png}}

\subsection{Using Capability Maps for Heat Mapping Strategic Priorities
and Identifying Gaps with the MITA Maturity
Model}\label{using-capability-maps-for-heat-mapping-strategic-priorities-and-identifying-gaps-with-the-mita-maturity-model-2}

Capability maps are powerful tools that not only provide a visual
representation of an SMA's key capabilities but also serve as a
foundation for strategic analysis and planning. There are many
approaches to heat mapping capabilities, each offering unique insights
into organizational priorities and gaps. Here, we describe two
approaches: assessing maturity levels using the MITA Maturity Model and
prioritizing strategic outcomes.

\subsubsection{Identifying Gaps with the MITA Maturity
Model}\label{identifying-gaps-with-the-mita-maturity-model-2}

The MITA Maturity Model provides a framework for assessing the maturity
of an organization's capabilities across various dimensions, such as
business processes, information, and technology. By integrating the
maturity model with capability maps, SMAs can identify gaps between
their current state and desired maturity levels.

\paragraph{Example 1: Identifying Gaps in Data Management Maturity Using
the PRIT
Model}\label{example-1-identifying-gaps-in-data-management-maturity-using-the-prit-model-2}

An SMA is conducting an assessment of its data management capabilities
using the MITA Maturity Model, with a focus on the PRIT (Processes,
Roles, Information, and Technology) framework. The capability map
includes various data-related capabilities, such as ``Data
Integration,'' ``Data Quality Management,'' and ``Data Analytics.'' Each
of these capabilities is evaluated across the PRIT dimensions to
determine their maturity levels using the revised scale:

Processes: Level 1: Ad-Hoc Roles: Level 2: Compliant Information: Level
2: Compliant Technology: Level 2: Compliant The capability map is
updated to reflect the maturity assessment, with each dimension marked
with a color code: red for Level 1: Ad-Hoc, yellow for Level 2:
Compliant, green for Level 3: Efficient, blue for Level 4: Optimized,
and purple for Level 5: Pioneering. This visualization helps the SMA
prioritize strategic actions to enhance the ``Data Integration''
capability, such as standardizing processes, refining roles, improving
data quality, and upgrading technology.

\subsubsection{Heat Mapping Strategic
Priorities}\label{heat-mapping-strategic-priorities-2}

Heat mapping involves applying a color-coded overlay to a capability map
to visually represent the status or priority level of each capability.
This technique can be used to highlight areas of strength, weakness, or
strategic importance. For example, capabilities that are critical to
achieving CMS-required outcomes might be marked in one color, while
those needing immediate attention or improvement could be marked in
another. This visual representation helps stakeholders quickly grasp the
strategic landscape and make informed decisions about where to allocate
resources and focus efforts.

\paragraph{Example 2: Prioritizing Capabilities for CMS-Required
Outcomes}\label{example-2-prioritizing-capabilities-for-cms-required-outcomes-2}

An SMA is focused on achieving specific CMS-required outcomes related to
improving patient care and reducing administrative costs. The agency
creates a capability map that outlines all the capabilities necessary to
meet these outcomes. By applying a heat map, the SMA highlights
capabilities that are directly linked to these outcomes in green,
indicating they are of high strategic priority. Capabilities that are
indirectly related or less critical are marked in yellow, while those
that are currently underperforming or not aligned with strategic goals
are marked in red.

This visual representation allows the SMA to quickly identify which
capabilities require immediate attention and resources to ensure
compliance with CMS requirements. For instance, if the capability
related to ``Claims Processing Efficiency'' is marked in red, the agency
can prioritize initiatives to enhance this capability, such as investing
in new technology or streamlining processes.

\subsubsection{Other Uses for Capability Heat
Mapping}\label{other-uses-for-capability-heat-mapping-2}

Beyond assessing maturity levels and prioritizing strategic initiatives,
capability heat mapping can be applied in various other contexts to
enhance organizational effectiveness and alignment.

\paragraph{Example 3: Aligning Capabilities with State-Specific
Initiatives}\label{example-3-aligning-capabilities-with-state-specific-initiatives-2}

An SMA is working on a state-specific initiative to enhance telehealth
services for rural populations. The capability map includes capabilities
related to telehealth, such as ``Telehealth Infrastructure,'' ``Provider
Engagement,'' and ``Patient Access.'' The SMA uses a heat map to
highlight these capabilities in blue, indicating their alignment with
the state-specific initiative.

By analyzing the capability map, the SMA identifies that ``Provider
Engagement'' is a critical capability that requires further development
to support the telehealth initiative. The agency decides to invest in
training programs and outreach efforts to engage providers in rural
areas, ensuring that the telehealth services are effectively delivered
to the target population.

These examples demonstrate how capability maps, combined with heat
mapping and the MITA Maturity Model, can provide valuable insights for
strategic planning and gap analysis. By visualizing priorities and
maturity levels, SMAs can make informed decisions about where to focus
resources and efforts, ultimately enhancing their Medicaid Enterprise
Systems and achieving strategic objectives.

\begin{itemize}
\tightlist
\item
  \textbf{Capability Mapping}

  \begin{itemize}
  \tightlist
  \item
    Introduction to capability mapping and its significance
  \item
    How capabilities are organized and detailed at various levels of
    abstraction
  \end{itemize}
\end{itemize}

\section{Guidance on reuse of the MITA Capability
Model}\label{guidance-on-reuse-of-the-mita-capability-model-2}

\begin{itemize}
\tightlist
\item
  \textbf{Practical Application}

  \begin{itemize}
  \tightlist
  \item
    How to integrate the capability model into daily operations and
    strategic planning
  \item
    Tips for maximizing the benefits of the model
  \end{itemize}
\item
  \textbf{Continuous Improvement}

  \begin{itemize}
  \tightlist
  \item
    Encouragement for ongoing assessment and refinement of capabilities
  \item
    Leveraging feedback and performance data for model enhancement
  \end{itemize}
\item
  \textbf{Implementation Guidance}

  \begin{itemize}
  \tightlist
  \item
    Steps for adopting the capability model
  \item
    Resources and support available for SMAs
  \end{itemize}
\item
  \textbf{Performance Monitoring and Reporting}

  \begin{itemize}
  \tightlist
  \item
    Role of the capability model in tracking and enhancing performance
  \item
    Use of metrics and standards to measure capability effectiveness
  \end{itemize}
\end{itemize}

\chapter{MITA Capability Model}\label{mita-capability-model-3}

\section{Introduction to Business Capability
Models}\label{introduction-to-business-capability-models-2}

A capability model is a conceptual framework that outlines the key
capabilities an organization needs to achieve its strategic objectives.
It provides a comprehensive view of what an organization can do and
helps identify areas for improvement or investment. In the context of an
orchestra, a capability model might help the orchestra identify the set
of skills and resources, or other types of capabilities it needs to
perform a symphony. Just like an orchestra needs well practiced
musicians, sheet music, instruments, a conductor, and an audience to
produce a great symphony, a State Medicaid Agency (SMA) needs its
Medicaid Enterprise System (MES) to employ or develop specific
capabilities to deliver its services effectively, efficiently, and
economically to its enrollees and providers.

The concept of a business capability is extensively used within
enterprise architecture modeling and has been broadly used within
Business Capability Models as a tool to better align the business
strategy and information technology of both private sector and
governmental organizations since they emerged in the mid-2000s. One
example comes from the TOGAF Standard, a well-known standard in
enterprise architecture. Like most architecture frameworks TOGAF defines
a capability as something a business can do to meet its goals. This
focuses a strategic lens of an organization on ``what'' it needs to
achieve its goals, rather than ``how'' those goals are achieved. This
perspective allows for business planning from different viewpoints,
facilitating strategic alignment and operational efficiency.

SMA business architects, technologists, systems analysts, executives,
managers, and program staff can use this same modeling approach to
represent the functional components of their Medicaid Enterprise System
(MES) in ways that can help reveal gaps in their systems and provide
insights on what new or enhanced capabilities might be needed to close
those gaps.

By focusing on capabilities, SMAs can better align their information and
technology resources and processes with their strategic business goals,
ultimately improving their insight into how to improve the outcomes
their Medicaid Enterprise Architecture produces.

\subsection{Purpose}\label{purpose-3}

\begin{tcolorbox}[enhanced jigsaw, toprule=.15mm, colback=white, colframe=quarto-callout-note-color-frame, left=2mm, arc=.35mm, opacityback=0, rightrule=.15mm, breakable, bottomrule=.15mm, leftrule=.75mm]
\begin{minipage}[t]{5.5mm}
\textcolor{quarto-callout-note-color}{\faInfo}
\end{minipage}%
\begin{minipage}[t]{\textwidth - 5.5mm}

\vspace{-3mm}\textbf{Note}\vspace{3mm}

MITA 4.0 does not endeavor to specify all of the capabilities SMA's may
need to administer Medicaid programs; instead, this version of MITA
focuses on the capabilities that are most closely oriented towards
achieving the CMS-required outcomes.

\end{minipage}%
\end{tcolorbox}

Understanding the how the MITA Capability Model works is important to
obtaining the most value out of many of the other tools and artifacts in
the MITA framework, such as the MITA Maturity Model (MMM) and the
Business Process Model (BPM). The MITA Capability Model provides a
structured way for SMAs to identify, conceptually model, and improve the
capabilities needed for efficient Medicaid operations.

It is important to note that MITA 4.0 does not endeavor to specify all
of the capabilities SMA's may need to administer Medicaid programs;
instead, this version of MITA focuses on the capabilities that are most
closely oriented towards achieving the CMS-required outcomes. In this
way MITA 4.0 provides a reference model for SMAs to model other
capabilities that may be needed to achieve their other goals such as
state specific outcomes, or other state priorities while providing more
guidance within the MITA Framework to support modular.

\subsection{Update to MITA 3.0}\label{update-to-mita-3.0-3}

MITA 3.0 defined a capability as the level of maturity of a set of
business processes within a business category. By focusing on ``how''
MES operate MITA 3.0 helped SMA's identify ways to improve and mature
their business processes, but it did not link those processes with the
outcomes they are intended to achieve or ensure better alignment of the
information and technical architectures to business outcomes. The
addition of the MITA capability model to the MITA 4.0 business
architecture addresses that by providing the conceptual linkages needed
to elevate the strategic vantage point of the MITA Framework. To guide
this change, we present within this chapter a definition, description,
and approach to modeling business capabilities, based on the widely used
capability models contextualized for Medicaid Enterprises.

The business processes that operationalize MITA capabilities remain
foundational to characterizing the business architecture, and are by
definition a constituent part of any MITA capability. They provide
essential information on how capabilities are operationalize and should
continue to be a routinely utilized reference model for SMA business
process mapping. They are found with in the Business Process Model
chapter of this version of MITA.

\section{The MITA Definition of
Capability}\label{the-mita-definition-of-capability-3}

Within the context of MITA, a capability can be defined as the ability
or capacity of a State Medicaid Agency to achieve a desired outcome in
compliance with the
\href{https://www.ecfr.gov/current/title-42/chapter-IV/subchapter-C/part-433/subpart-C/section-433.112}{Standards
and Conditions within 42 CFR 433.112}. A capability may currently exist
in an operational state or be envisioned for future development. Through
careful planning, capabilities defined in this way can be matured and
refined over time to become more effective and efficient. They can be
organized and detailed at various levels of abstraction, providing
precise descriptions for operational purposes or more generalized views
for strategic planning.

\begin{tcolorbox}[enhanced jigsaw, toprule=.15mm, colback=white, colframe=quarto-callout-note-color-frame, left=2mm, arc=.35mm, opacityback=0, rightrule=.15mm, breakable, bottomrule=.15mm, leftrule=.75mm]

\vspace{-3mm}\textbf{Key Definition}\vspace{3mm}

\ldots a capability is defined as the ability or capacity of a SMA to
achieve a desired outcome\ldots{}

\end{tcolorbox}

To fully define a business capability, it is essential to understand how
it is realized through the integration of people, processes,
information, and technology resources of an SMA. While these elements of
the capability can change regularly, the capability itself is should
endure over longer planning horizons, supporting the long-term alignment
of business and IT and the achievement of increasingly beneficial
business outcomes.

\subsection{Structure of the MITA Capability
Model}\label{structure-of-the-mita-capability-model-3}

As depicted in the model below, the MITA Capability Model orients the
people, process, technology, and information resources to define a MITA
Capability. This means that to model a capability the appropriate
components of the information architecture and the technical
architecture must be brought together with the business architecture to
fully formulate any MITA Capability.

\begin{figure}[H]

{\centering \pandocbounded{\includegraphics[keepaspectratio]{media/capabilityModel/topLevelCapabilityMetamodelGraphic1.png}}

}

\caption{MITA Capability Relationship Diagram}

\end{figure}%

\subsubsection{Business Roles}\label{business-roles-3}

Business roles represent individual actors, stakeholders, or partners
involved in delivering a business capability. A single organizational
group or team may be wholly responsible for delivering the capability,
or multiple business entities may share the delivery of a particular
business capability. Business Roles perform Business Processes using
Technology Resources. They require skills and knowledge resources to
achieve outcomes, and should be actively engaged as partners in the
development or enhancement of any capability they help deliver.

\subsubsection{Business Processes}\label{business-processes-3}

Individual business capabilities may be enabled or delivered through a
range of business processes that detail the activities (the how)
associated with delivering the capability. Identifying and analyzing the
efficiency of the underlying processes helps to optimize the business
capability's effectiveness. Identifying the processes within a business
capability provides a focus for maturing the capability in concert with
the other capability components. Business Processes operationalize
Business Capabilities.

\subsubsection{Information/Data}\label{informationdata-3}

Information/data represents the business data, knowledge, and insight
consumed or produced by the business capability (as distinct from
IT-related data entities). This may also include information that the
capability exchanges with other capabilities to support the execution of
value streams. Examples include information about customers and
prospects, products and services, business policies and rules, sales
reports, and performance metrics. Information/data inform the Business
Capability, answering questions and supporting business rules.

\subsubsection{Technology Resources}\label{technology-resources-3}

Business capabilities rely on a range of tools, applications, systems,
and services for successful execution. Technology Resources use
Information/data to facilitate Business Processes. Such resources may
include:

\begin{itemize}
\tightlist
\item
  Modular software applications

  \begin{itemize}
  \tightlist
  \item
    Cloud or on-premise infrastructure
  \item
    Microservices
  \item
    Analytics
  \item
    Customer portal
  \end{itemize}
\end{itemize}

In this way we can clearly interrelate all of the MITA architecture
models and their individual components which allows us to reveal gaps
not only in the individual components of the architecture, but also
understand their impact on the integration of the architecture
components at the capability level.

\subsection{Relationship of MITA Capabilities to
Outcomes}\label{relationship-of-mita-capabilities-to-outcomes-3}

In the context of the Medicaid Information Technology Architecture
(MITA), outcomes are intrinsically linked to capabilities, as they
represent the tangible results achieved through the effective
integration and execution of various elements that constitute a
capability. In this sense, outcomes and capabilities define each other.

\begin{figure}[H]

{\centering \pandocbounded{\includegraphics[keepaspectratio]{media/capabilityModel/topLevelCapabilityMetamodelOutcomes.png}}

}

\caption{MITA Capability and Outcome Relationship Diagram}

\end{figure}%

\subsubsection{Outcomes}\label{outcomes-3}

MITA defines outcomes broadly to encompass CMS-required outcomes,
state-specific outcomes, and other outcomes not mandated as part of the
Advance Planning Document (APD) process. The sole criterion for an
outcome to meet this definition is that it must be a goal of a State
Medicaid Agency (SMA) and be achieved through a Medicaid Enterprise
System (MES) capability.

\begin{tcolorbox}[enhanced jigsaw, toprule=.15mm, colback=white, colframe=quarto-callout-note-color-frame, left=2mm, arc=.35mm, opacityback=0, rightrule=.15mm, breakable, bottomrule=.15mm, leftrule=.75mm]

\vspace{-3mm}\textbf{Key Definition}\vspace{3mm}

A MITA outcome is a goal of a State Medicaid Agency (SMA) that is
achieved by a Medicaid Enterprise System (MES) capability.

\end{tcolorbox}

\subsubsection{Measure}\label{measure-3}

Measure is a quantifiable metric used to assess the effectiveness and
efficiency of capabilities within a Medicaid Enterprise System (MES).
Measures provide quantifiable and qualitative values that help State
Medicaid Agencies (SMAs) track progress toward achieving specific
outcomes, such as CMS-required or state-specific goals. These indicators
might include metrics like processing times, error rates, or compliance
levels.

Measures are a measurement threshold by establishing a specific value or
level that must be met or exceeded to demonstrate successful
performance. For instance, a KPI might set a threshold for the maximum
allowable processing time for claims, ensuring that they are handled
within a specified timeframe to maintain compliance and eligibility for
enhanced federal funding. By monitoring these thresholds, organizations
can ensure they are meeting regulatory requirements and delivering
high-quality services to beneficiaries, while also identifying areas for
improvement.

\subsubsection{Measure Threshold}\label{measure-threshold-3}

A specific value or level of a measure that must be met or exceeded to
demonstrate the effective achievement of a capability's intended
outcome. This threshold serves as a benchmark for assessing whether the
processes, roles, and resources integrated within a Medicaid Enterprise
System (MES) are functioning optimally to meet the goals of a State
Medicaid Agency (SMA). For example, a measurement threshold might be set
for processing times, where claims must be processed within a certain
number of days to ensure compliance with CMS-required outcomes and
maintain eligibility for enhanced federal funding. By establishing and
monitoring these thresholds, organizations can ensure they are meeting
regulatory requirements and delivering high-quality services to
beneficiaries.

\subsubsection{Measurement}\label{measurement-3}

These outcomes and metrics are also used to ensure that healthcare
systems or modules comply with applicable federal regulations, forming
the baseline for system or module functionality. Achieving these
outcomes is essential for continuing to receive enhanced federal funding
for operations. Regular measurement and analysis of KPIs help
organizations demonstrate compliance and effectiveness, ensuring that
they meet regulatory requirements and continue to deliver high-quality
services to beneficiaries.

In this way we can clearly interrelate all of the MITA architecture
models and their individual components with the KPIs, thresholds, and
measurements that indicate whether our capability achieves our desired
outcome.

While models that help conceptualize the capabilities that achieve
CMS-required outcomes are the ones modeled for this version of MITA,
SMAs are encouraged to use these models as a reference to model
capabilities.

\section{Capability Mapping}\label{capability-mapping-3}

Capability mapping is a strategic tool that enables organizations, such
as State Medicaid Agencies (SMAs), to systematically identify, organize,
and visualize the key capabilities necessary to achieve their
objectives. Within the MITA framework, capability mapping provides SMAs
with a method of developing comprehensive views of the functions and
processes required to deliver Medicaid services effectively. To begin
the capability mapping process, SMAs should first identify the core
capabilities that align with their strategic objectives, focusing on
what the organization needs to achieve rather than how those goals are
accomplished. This involves listing all necessary capabilities and
understanding the desired outcomes they support. Next, these
capabilities should be organized into domains and areas that reflect
their strategic importance and interrelationships. Visualizing these
capabilities through diagrams or maps provides all stakeholders a common
view to understand the roles, processes, technology resources, and
information/data involved in executing each capability, as well as the
outcome each capability is designed to achieve. This structured approach
not only highlights areas for improvement or investment but also ensures
that organizational efforts are strategically aligned with desired
outcomes.

The benefits of capability mapping are multifaceted, offering SMAs a
clear pathway to strategic alignment and gap analysis. By visualizing
capabilities, organizations can identify operational gaps and determine
what new or enhanced capabilities are needed to close those gaps. This
visualization also improves communication among stakeholders by
providing a clear and concise representation of the organization's
functions. To refine capabilities, SMAs should analyze current
operations, assess the efficiency of underlying processes, and optimize
them to enhance capability effectiveness. Additionally, capability
mapping serves as a foundation for heat mapping, which assesses the MITA
Framework will utilize to visualize the maturity of each capability
evaluated in the State Self-Assessment. SMAs can overlay heat maps over
their capability maps to visualize many things other than maturity
levels, using color coding to indicate areas of strength and weakness.
Regular updates to these maps allow SMAs to monitor progress and ensure
resources are allocated effectively to achieve strategic goals. The MITA
framework includes examples of capability maps based on CMS-required
outcomes, serving as a reference model for SMAs to develop their own
capability maps tailored to state-specific goals and priorities. By
leveraging the reference models provided by MITA, SMAs can ensure their
capability mapping efforts are aligned with both federal requirements
and state-specific priorities.

\subsection{Organizing Capabilities}\label{organizing-capabilities-3}

To enhance the resolution and detail of a capability and provide a
unified view of all its components, a block diagram can be employed to
provide a common view of any MES. This diagram effectively links the
capability to business processes, roles, technical resources, and
information resources through functional decomposition. By breaking down
the capability into its constituent parts, the block diagram offers a
visual representation that highlights the interrelationships and
dependencies among these elements. This approach provides a clearer
understanding of how each component contributes to the overall
capability, facilitating more effective analysis, optimization, and
alignment with organizational objectives.

\pandocbounded{\includegraphics[keepaspectratio]{media/capabilityModel/capabilityOgranizationModel.drawio.png}}

We use this same method to present an this top level view of the
capabilities required to achieve CMS-required outcomes. From this view
increasingly detailed models can be constructed.

\pandocbounded{\includegraphics[keepaspectratio]{media/capabilityModel/mesModuleBasedCapabilities.drawio.png}}

\subsection{MITA Capability Models}\label{mita-capability-models-3}

The MITA framework represents capabilities visually through a layered
model that represent a capability of being composed of sub-capabilities
and the processes, roles, information and technology resources (PRIT)
that support the business in sustaining the capability. Each layer up
depicts increasingly strategic capabilities and each layer down depicts
the constituent elements that compose a capability in increasing
operational detail. It is not the intention of this version of MITA to
provide a full operational or tactical view of a capability, though SMAs
may consider using this approach to improve their organizational
awareness of their operations by developing further layers of their
capabilities through functional decomposition.

\pandocbounded{\includegraphics[keepaspectratio]{media/capabilityModel/capabilityLevels.png}}

\begin{itemize}
\tightlist
\item
  \textbf{Capability Domains:} The first layer of this model aims to
  group capabilities to organize the strategic view of an SMA's
  capabilities. In this view one or many capabilities can be grouped
  within a domain to indicate the pursuit of common outcomes. Each
  domain is denoted with a single number to help annotate each
  capability.

  \begin{itemize}
  \tightlist
  \item
    \textbf{Capability Areas:} The second layer of this model aims to
    provide a view of the groups of capabilities that compose a domain.
    They are organized to show capabilities that serve a specific group
    of similar outcomes and essential
  \item
    \textbf{Capabilities:} The third layer of this model provides a more
    detailed view view of
  \end{itemize}
\end{itemize}

\pandocbounded{\includegraphics[keepaspectratio]{media/capabilityModel/capabilityLevels2.png}}

\subsection{Relationship of MITA Capabilities to
Maturity}\label{relationship-of-mita-capabilities-to-maturity-3}

\begin{tcolorbox}[enhanced jigsaw, toprule=.15mm, leftrule=.75mm, colframe=quarto-callout-warning-color-frame, left=2mm, arc=.35mm, titlerule=0mm, rightrule=.15mm, opacitybacktitle=0.6, bottomtitle=1mm, toptitle=1mm, colbacktitle=quarto-callout-warning-color!10!white, bottomrule=.15mm, title=\textcolor{quarto-callout-warning-color}{\faExclamationTriangle}\hspace{0.5em}{Warning}, opacityback=0, breakable, colback=white, coltitle=black]

Under development.

\end{tcolorbox}

\begin{itemize}
\tightlist
\item
  \textbf{Levels of Maturity}

  \begin{itemize}
  \tightlist
  \item
    Description of the five levels of maturity in the MITA framework
  \item
    How capabilities evolve and mature over time
  \end{itemize}
\end{itemize}

\pandocbounded{\includegraphics[keepaspectratio]{media/capabilityModel/maturityModel.png}}

\subsection{Using Capability Maps for Heat Mapping Strategic Priorities
and Identifying Gaps with the MITA Maturity
Model}\label{using-capability-maps-for-heat-mapping-strategic-priorities-and-identifying-gaps-with-the-mita-maturity-model-3}

Capability maps are powerful tools that not only provide a visual
representation of an SMA's key capabilities but also serve as a
foundation for strategic analysis and planning. There are many
approaches to heat mapping capabilities, each offering unique insights
into organizational priorities and gaps. Here, we describe two
approaches: assessing maturity levels using the MITA Maturity Model and
prioritizing strategic outcomes.

\subsubsection{Identifying Gaps with the MITA Maturity
Model}\label{identifying-gaps-with-the-mita-maturity-model-3}

The MITA Maturity Model provides a framework for assessing the maturity
of an organization's capabilities across various dimensions, such as
business processes, information, and technology. By integrating the
maturity model with capability maps, SMAs can identify gaps between
their current state and desired maturity levels.

\paragraph{Example 1: Identifying Gaps in Data Management Maturity Using
the PRIT
Model}\label{example-1-identifying-gaps-in-data-management-maturity-using-the-prit-model-3}

An SMA is conducting an assessment of its data management capabilities
using the MITA Maturity Model, with a focus on the PRIT (Processes,
Roles, Information, and Technology) framework. The capability map
includes various data-related capabilities, such as ``Data
Integration,'' ``Data Quality Management,'' and ``Data Analytics.'' Each
of these capabilities is evaluated across the PRIT dimensions to
determine their maturity levels using the revised scale:

Processes: Level 1: Ad-Hoc Roles: Level 2: Compliant Information: Level
2: Compliant Technology: Level 2: Compliant The capability map is
updated to reflect the maturity assessment, with each dimension marked
with a color code: red for Level 1: Ad-Hoc, yellow for Level 2:
Compliant, green for Level 3: Efficient, blue for Level 4: Optimized,
and purple for Level 5: Pioneering. This visualization helps the SMA
prioritize strategic actions to enhance the ``Data Integration''
capability, such as standardizing processes, refining roles, improving
data quality, and upgrading technology.

\subsubsection{Heat Mapping Strategic
Priorities}\label{heat-mapping-strategic-priorities-3}

Heat mapping involves applying a color-coded overlay to a capability map
to visually represent the status or priority level of each capability.
This technique can be used to highlight areas of strength, weakness, or
strategic importance. For example, capabilities that are critical to
achieving CMS-required outcomes might be marked in one color, while
those needing immediate attention or improvement could be marked in
another. This visual representation helps stakeholders quickly grasp the
strategic landscape and make informed decisions about where to allocate
resources and focus efforts.

\paragraph{Example 2: Prioritizing Capabilities for CMS-Required
Outcomes}\label{example-2-prioritizing-capabilities-for-cms-required-outcomes-3}

An SMA is focused on achieving specific CMS-required outcomes related to
improving patient care and reducing administrative costs. The agency
creates a capability map that outlines all the capabilities necessary to
meet these outcomes. By applying a heat map, the SMA highlights
capabilities that are directly linked to these outcomes in green,
indicating they are of high strategic priority. Capabilities that are
indirectly related or less critical are marked in yellow, while those
that are currently underperforming or not aligned with strategic goals
are marked in red.

This visual representation allows the SMA to quickly identify which
capabilities require immediate attention and resources to ensure
compliance with CMS requirements. For instance, if the capability
related to ``Claims Processing Efficiency'' is marked in red, the agency
can prioritize initiatives to enhance this capability, such as investing
in new technology or streamlining processes.

\subsubsection{Other Uses for Capability Heat
Mapping}\label{other-uses-for-capability-heat-mapping-3}

Beyond assessing maturity levels and prioritizing strategic initiatives,
capability heat mapping can be applied in various other contexts to
enhance organizational effectiveness and alignment.

\paragraph{Example 3: Aligning Capabilities with State-Specific
Initiatives}\label{example-3-aligning-capabilities-with-state-specific-initiatives-3}

An SMA is working on a state-specific initiative to enhance telehealth
services for rural populations. The capability map includes capabilities
related to telehealth, such as ``Telehealth Infrastructure,'' ``Provider
Engagement,'' and ``Patient Access.'' The SMA uses a heat map to
highlight these capabilities in blue, indicating their alignment with
the state-specific initiative.

By analyzing the capability map, the SMA identifies that ``Provider
Engagement'' is a critical capability that requires further development
to support the telehealth initiative. The agency decides to invest in
training programs and outreach efforts to engage providers in rural
areas, ensuring that the telehealth services are effectively delivered
to the target population.

These examples demonstrate how capability maps, combined with heat
mapping and the MITA Maturity Model, can provide valuable insights for
strategic planning and gap analysis. By visualizing priorities and
maturity levels, SMAs can make informed decisions about where to focus
resources and efforts, ultimately enhancing their Medicaid Enterprise
Systems and achieving strategic objectives.

\begin{itemize}
\tightlist
\item
  \textbf{Capability Mapping}

  \begin{itemize}
  \tightlist
  \item
    Introduction to capability mapping and its significance
  \item
    How capabilities are organized and detailed at various levels of
    abstraction
  \end{itemize}
\end{itemize}

\section{Guidance on reuse of the MITA Capability
Model}\label{guidance-on-reuse-of-the-mita-capability-model-3}

\begin{itemize}
\tightlist
\item
  \textbf{Practical Application}

  \begin{itemize}
  \tightlist
  \item
    How to integrate the capability model into daily operations and
    strategic planning
  \item
    Tips for maximizing the benefits of the model
  \end{itemize}
\item
  \textbf{Continuous Improvement}

  \begin{itemize}
  \tightlist
  \item
    Encouragement for ongoing assessment and refinement of capabilities
  \item
    Leveraging feedback and performance data for model enhancement
  \end{itemize}
\item
  \textbf{Implementation Guidance}

  \begin{itemize}
  \tightlist
  \item
    Steps for adopting the capability model
  \item
    Resources and support available for SMAs
  \end{itemize}
\item
  \textbf{Performance Monitoring and Reporting}

  \begin{itemize}
  \tightlist
  \item
    Role of the capability model in tracking and enhancing performance
  \item
    Use of metrics and standards to measure capability effectiveness
  \end{itemize}
\end{itemize}

\part{Business Architecture}

\chapter{Business Architecture
Introduction}\label{business-architecture-introduction}

\phantomsection\label{page-0-1}{}\phantomsection\label{page-0-0}{}\textbf{Part
I -- BUSINESS ARCHITECTURE Chapter 1 -- INTRODUCTION}

\pandocbounded{\includegraphics[keepaspectratio]{_page_0_Picture_1.jpeg}}

\pandocbounded{\includegraphics[keepaspectratio]{_page_0_Picture_2.jpeg}}

\pandocbounded{\includegraphics[keepaspectratio]{_page_0_Picture_3.jpeg}}

\pandocbounded{\includegraphics[keepaspectratio]{_page_0_Picture_4.jpeg}}

\subsubsection{\texorpdfstring{\textbf{Table of
Contents}}{Table of Contents}}\label{table-of-contents}

\begin{longtable}[]{@{}ll@{}}
\toprule\noalign{}
PART I--BUSINESS ARCHITECTURE & 1 \\
\midrule\noalign{}
\endhead
\bottomrule\noalign{}
\endlastfoot
Chapter 1 --Introduction & 1 \\
Introduction & 3 \\
Purpose & 3 \\
Scope & 4 \\
Background & 4 \\
Funding Requirements & 5 \\
BA Seven Standards and Conditions & 6 \\
Business Architecture Components & 7 \\
The Concept of Operations11 & \\
MITA Maturity Model11 & \\
Business Process Model12 & \\
12Business Capability Matrix & \\
14Business Architecture Component Relationships & \\
Connection Between Architectures15 & \\
17Using the Business Architecture & \\
Next Steps in Developing the Business Architecture18 & \\
& \\
\end{longtable}

\subsubsection{\texorpdfstring{\textbf{List of
Figures}}{List of Figures}}\label{list-of-figures}

\begin{longtable}[]{@{}
  >{\raggedright\arraybackslash}p{(\linewidth - 2\tabcolsep) * \real{0.9630}}
  >{\raggedright\arraybackslash}p{(\linewidth - 2\tabcolsep) * \real{0.0370}}@{}}
\toprule\noalign{}
\begin{minipage}[b]{\linewidth}\raggedright
Figure 1-1. MITA Framework Relationship Diagram
\end{minipage} & \begin{minipage}[b]{\linewidth}\raggedright
3
\end{minipage} \\
\midrule\noalign{}
\endhead
\bottomrule\noalign{}
\endlastfoot
Figure 1-2. BA in the Context of the MITA Framework & 8 \\
Figure 1-3. Relationship Among the Components of the Business
Architecture14 & \\
IA, and TA15Figure 1-4. Relationships Among Components of the BA, & \\
\end{longtable}

\subsubsection{\texorpdfstring{\textbf{List of
Tables}}{List of Tables}}\label{list-of-tables}

\begin{longtable}[]{@{}
  >{\raggedright\arraybackslash}p{(\linewidth - 2\tabcolsep) * \real{0.9577}}
  >{\raggedright\arraybackslash}p{(\linewidth - 2\tabcolsep) * \real{0.0423}}@{}}
\toprule\noalign{}
\begin{minipage}[b]{\linewidth}\raggedright
Table 1-1. Correlation of Seven Standards and Conditions with MITA
\end{minipage} & \begin{minipage}[b]{\linewidth}\raggedright
6
\end{minipage} \\
\midrule\noalign{}
\endhead
\bottomrule\noalign{}
\endlastfoot
Table 1-2. The Four Components of the Business Architecture. & 9 \\
13Table 1-3. Business Process Example: Authorize Service & \\
16Table 1-4. Component Relationships ofthe BA, IA, and TA & \\
17Table 1-5. Stakeholder Use of the Business Architecture & \\
\end{longtable}

\pandocbounded{\includegraphics[keepaspectratio]{_page_1_Picture_8.jpeg}}

\chapter[\textbf{Introduction}]{\texorpdfstring{\protect\hypertarget{page-2-0}{}{}\textbf{Introduction}}{Introduction}}\label{introduction-1}

The Medicaid IT Architecture (MITA) Framework contains three (3)
interrelated architectures: Business Architecture (BA), Information
Architecture (IA), and Technical Architecture (TA) shown in
\textbf{\hyperref[page-2-2]{Figure 1-1}}. The business capabilities from
BA define the data strategy of IA and design the business and technical
services of TA. MITA uses all three (3) architectures to develop a
business-driven enterprise to provide consistency across the State
Medicaid Enterprise.

\pandocbounded{\includegraphics[keepaspectratio]{_page_2_Figure_4.jpeg}}

\pandocbounded{\includegraphics[keepaspectratio]{_page_2_Figure_5.jpeg}}

\pandocbounded{\includegraphics[keepaspectratio]{_page_2_Figure_6.jpeg}}

\phantomsection\label{page-2-2}{}The topics covered in this chapter
include:

\begin{itemize}
\tightlist
\item
  BA Seven Standards and Conditions
\item
  Business Architecture Components
\item
  Business Architecture Component Relationships
\item
  Connection Between Architectures
\item
  Using the Business Architecture
\item
  Next Steps in Developing the Business Architecture
\end{itemize}

\section[\textbf{Purpose}]{\texorpdfstring{\protect\hypertarget{page-2-1}{}{}\textbf{Purpose}}{Purpose}}\label{purpose-4}

In keeping with the guiding principle that MITA represents a
business-driven enterprise transformation, the BA is the starting point
of the MITA Framework. The BA describes the needs and goals of the
individual State Medicaid Enterprise, and presents a collective vision
of the future.

The BA will accomplish the following:

\begin{itemize}
\tightlist
\item
  Establish a generic business framework for all States while
  recognizing their differences.
\item
  Describe how each state Medicaid Program can mature over a given
  period with the help of stakeholders, leadership, enabling
  legislation, and technology.
\item
  Provide a baseline for the State Medicaid Agency (SMA) to assess their
  current business capabilities and measure progress toward improved
  capabilities.
\end{itemize}

\pandocbounded{\includegraphics[keepaspectratio]{_page_2_Picture_20.jpeg}}

\subsection[\textbf{Scope}]{\texorpdfstring{\protect\hypertarget{page-3-0}{}{}\textbf{Scope}}{Scope}}\label{scope}

The BA focuses on the State Medicaid Enterprise that centers on the
Medicaid environment including leveraged systems and interconnections
among Medicaid stakeholders, providers, beneficiaries, insurance
affordability programs (e.g., CHIP, tax credits, Basic Health Program),
Health Insurance Exchange (HIX), Health Information Exchange (HIE),
other state and local agencies, other payers, Centers for Medicare \&
Medicaid Services (CMS), and other federal agencies. The MITA context
defines the Medicaid Enterprise as:

\begin{itemize}
\tightlist
\item
  The domain where federal matching funds apply.
\item
  The interfaces and bridges among Medicaid stakeholders, including
  providers, beneficiaries, other state and local agencies, other
  payers, CMS, and other federal agencies.
\item
  The sphere of influence touched by MITA (e.g., national and federal
  initiatives such as the Nationwide Health Information Network
  {[}NwHIN{]}). (See Front Matter, Chapter 6, Overview of the MITA
  Initiative, for a discussion of the Medicaid Enterprise.)
\end{itemize}

\emph{Enterprise can have other meanings. For instance, Enterprise
Architecture (EA) defines an enterprise-wide integrated set of
components that incorporates strategic business thinking, information
assets, and the technical infrastructure of an enterprise to promote
information sharing across agency and organizational boundaries.}

The BA acknowledges technology as one of several enablers that are
important to growth and transformation, but it does not introduce
technical implementations or solutions into the BA components. All
technical references are found in Part III, Technical Architecture.

\subsection[\textbf{Background}]{\texorpdfstring{\protect\hypertarget{page-3-1}{}{}\textbf{Background}}{Background}}\label{background}

States, territories, and the District of Columbia (hereinafter referred
to as States) are responsible for their individual State Medicaid
Enterprise, and all entities are different in important ways.
Differences include:

\begin{itemize}
\tightlist
\item
  Organizational structure, covered programs, and lines of business
\item
  Business rules, policies, and procedures affecting stakeholders
\item
  Relationships with other state and local agencies
\item
  Revenue sources
\item
  Location of business units
\item
  Workflow
\item
  Range of outsourcing
\item
  Technical solutions
\end{itemize}

\pandocbounded{\includegraphics[keepaspectratio]{_page_3_Picture_19.jpeg}}

These entities also differ in their concept of an enterprise, the roles
and responsibilities of one or more Chief Information Officers (CIO),
adoption of data and technical standards, and the use of legacy versus
state-of-the-art applications.

Given these differences, it is not possible or desirable, in the context
of the MITA Framework, to develop a standalone business and technical
model for each individual Medicaid Enterprise. Instead, MITA establishes
a national framework of common processes and enabling technologies to
support improved program administration in all States.

The BA focuses on areas of common ground (e.g., that all States will
enroll providers and pay for services rendered to eligible beneficiaries
and that all States seek to improve health care outcomes and improve
administrative processes).

There is no ready-made methodology for building the MITA Framework to
accommodate the business needs and transformation strategies of the
States. To meet the special needs of MITA, the components included in
the BA draw upon methodologies commonly in use today across industries
as diverse as financial, transportation, and defense. The MITA team
designed templates and models to help States identify and prioritize
their specific business needs.

The BA section of the MITA Framework shows how MITA incorporates
business-driven design to accomplish the following:

\begin{itemize}
\tightlist
\item
  Support state needs.

  \begin{itemize}
  \tightlist
  \item
    \textbf{o} Align with state strategic goals.
  \item
    \textbf{o} Align with state or Medicaid Agency enterprise
    architecture.
  \end{itemize}
\item
  Support the CMS and common state goals.

  \begin{itemize}
  \tightlist
  \item
    \textbf{o} Align state approaches with MITA.
  \item
    \textbf{o} Accommodate multi-state collaborative initiatives.
  \end{itemize}
\item
  Support national goals through alignment with national initiatives,
  such as the Office of the National Coordinator for Health Information
  Technology (ONC) and federal guidelines (e.g., Federal Health
  Architecture (FHA), the Federal Enterprise Architecture Framework
  (FEAF), and national/international data standards).
\end{itemize}

\section[\textbf{Funding
Requirements}]{\texorpdfstring{\protect\hypertarget{page-4-0}{}{}\textbf{Funding
Requirements}}{Funding Requirements}}\label{funding-requirements}

The Health and Human Services (HHS) CMS 42 CFR Part 433 Medicaid
Program; Federal Funding for Medicaid Eligibility Determination and
Enrollment Activities modifies Medicaid regulations for Mechanized
Claims Processing and Information Retrieval Systems effective April 19,
2011. The Medicaid Management Information System (MMIS) is a mechanized
claims processing and information retrieval system used by the States
for Title XIX of the Social Security Act (The Act); therefore, the
guidance set forth in CMS 42 CFR Part 433 applies to the MMIS as well as
the Medicaid eligibility determination and enrollment activities as set
forth in the Affordable Care Act of 2010.

CMS expects States to meet the standards and conditions specified in
§433.112(b)(10) through §433.112(b)(16). The standards and conditions
are descriptive in nature; however, CMS recognizes that in order for the
States to meet these standards and conditions it is necessary to provide
additional guidance that clearly articulates its criteria for meeting
them

\pandocbounded{\includegraphics[keepaspectratio]{_page_4_Picture_17.jpeg}}

in terms of timeliness, accuracy, efficiency, integrity, and performance
standards for mechanized claims processing. In response to this need,
additional guidance materials include:

\begin{itemize}
\tightlist
\item
  Enhanced Funding Requirements: Seven Conditions and Standards (a.k.a.
  Seven Standards and Conditions)
\item
  Guidance for Exchange and Medicaid Information Technology (IT) Systems
  (a.k.a. IT Guidance)
\end{itemize}

CMS will continue to refine, update and expand this guidance in the
future, based on feedback from stakeholders and with experience over
time.

\chapter[\textbf{BA Seven Standards and
Conditions}]{\texorpdfstring{\protect\hypertarget{page-5-0}{}{}\textbf{BA
Seven Standards and
Conditions}}{BA Seven Standards and Conditions}}\label{ba-seven-standards-and-conditions}

The MITA team evaluated and incorporated the 42 CFR Part 433 Medicaid
Program; Federal Funding for Medicaid Eligibility Determination and
Enrollment Activities in the BA for purposes of guiding the MITA
stakeholders to apply the guidance to the Medicaid Enterprise.

Each of the architectures aligns with the Seven Standards and
Conditions. By utilizing best practices, industry standards, and
technology advancements, the processes, and planning guidelines that
build the MITA framework provide a cohesive method for meeting Medicaid
objectives.

\textbf{\hyperref[page-5-1]{Table 1-1}} depicts the impact of the Seven
Standards and Conditions on the MITA BA, IA, and TA.

\phantomsection\label{page-5-1}{}

\begin{longtable}[]{@{}
  >{\raggedright\arraybackslash}p{(\linewidth - 8\tabcolsep) * \real{0.4581}}
  >{\raggedright\arraybackslash}p{(\linewidth - 8\tabcolsep) * \real{0.1677}}
  >{\raggedright\arraybackslash}p{(\linewidth - 8\tabcolsep) * \real{0.1871}}
  >{\raggedright\arraybackslash}p{(\linewidth - 8\tabcolsep) * \real{0.1742}}
  >{\raggedright\arraybackslash}p{(\linewidth - 8\tabcolsep) * \real{0.0129}}@{}}
\toprule\noalign{}
\begin{minipage}[b]{\linewidth}\raggedright
Correlation of Seven Standards and Conditions with MITA Architectures
\end{minipage} & \begin{minipage}[b]{\linewidth}\raggedright
\end{minipage} & \begin{minipage}[b]{\linewidth}\raggedright
\end{minipage} & \begin{minipage}[b]{\linewidth}\raggedright
\end{minipage} & \begin{minipage}[b]{\linewidth}\raggedright
\end{minipage} \\
\midrule\noalign{}
\endhead
\bottomrule\noalign{}
\endlastfoot
Standards and Conditions & BusinessArchitecture &
InformationArchitecture & TechnicalArchitecture & \\
Modularity Standard & X & X & X & \\
MITA Condition & X & X & X & \\
Industry Standards Condition & X & X & X & \\
Leverage Condition & X & X & X & \\
Business Results Condition & X & X & X & \\
Reporting Condition & X & X & X & \\
Interoperability Condition & X & X & X & \\
\end{longtable}

\subsection{\texorpdfstring{\textbf{Table 1-1. Correlation of Seven
Standards and Conditions with
MITA}}{Table 1-1. Correlation of Seven Standards and Conditions with MITA}}\label{table-1-1.-correlation-of-seven-standards-and-conditions-with-mita}

\pandocbounded{\includegraphics[keepaspectratio]{_page_5_Picture_12.jpeg}}

The BA includes:

\begin{itemize}
\tightlist
\item
  \textbf{Modularity Standard} Uses a modular, flexible approach to
  systems development, including the use of open interfaces and exposed
  Application Programming Interfaces (API); the separation of business
  rules from core programming; and the availability of business rules in
  both human and machine-readable formats. The States commit to formal
  system development methodology and open, reusable system architecture.
\item
  \textbf{MITA Condition} States align to and advance increasingly in
  MITA maturity for business, architecture, and data.
\end{itemize}

\textbf{Industry Standards Condition} - Ensures alignment with, and
incorporation of, industry standards: the Health Insurance Portability
and Accountability Act of 1996 (HIPAA) security, privacy and transaction
standards; accessibility standards established under section 508 of the
Rehabilitation Act, or standards that provide greater accessibility for
individuals with disabilities, and compliance with Federal Civil Rights
laws; standards adopted by the Secretary under section 1104 of the
Affordable Care Act; and standards and protocols adopted by the
Secretary under section 1561 of the Affordable Care Act.

\begin{itemize}
\tightlist
\item
  \textbf{Leverage Condition} States solutions should promote sharing,
  leverage, and reuse of Medicaid technologies and systems within and
  among States.
\item
  \textbf{Business Results Condition} Systems should support accurate
  and timely processing of claims (including claims of eligibility),
  adjudications, and effective communications with providers,
  beneficiaries, and the public.
\item
  \textbf{Reporting Condition} Solutions should produce transaction
  data, reports, and performance information that contribute to program
  evaluation, continuous improvement in business operations, and
  transparency and accountability.
\item
  \textbf{Interoperability Condition} Systems must ensure seamless
  coordination and integration with the Exchange (whether run by the
  state or federal government), and allow interoperability with health
  information exchanges, public health agencies, human services
  programs, and community organizations providing outreach and
  enrollment assistance services.
\end{itemize}

\chapter[\textbf{Business Architecture
Components}]{\texorpdfstring{\protect\hypertarget{page-6-0}{}{}\textbf{Business
Architecture
Components}}{Business Architecture Components}}\label{business-architecture-components}

The BA is a conceptual construct that encompasses models, matrices, and
templates. These components derive from a variety of industry standards
because no single methodology exists that meets the scope of MITA. The
MITA Framework breaks new ground and is a model for other federal,
state, and local entities.

The MITA BA contains the following components:

\begin{itemize}
\tightlist
\item
  Concept of Operations
\item
  MITA Maturity Model
\item
  Business Process Model
\item
  Business Capability Matrix
\end{itemize}

\pandocbounded{\includegraphics[keepaspectratio]{_page_6_Picture_17.jpeg}}

These are living models that evolve with the MITA Framework life cycle.
The MITA team tailored the level of detail in each model to meet the
specific needs of the intended audience.
\textbf{\hyperref[page-7-0]{Figure 1-2}} provides an overview of the
components of the BA. See Part I, Chapters 2 through 5 for a more
detailed description for each of these components.

\pandocbounded{\includegraphics[keepaspectratio]{_page_7_Figure_3.jpeg}}

\textbf{Figure 1-2}. \textbf{BA in the Context of the MITA Framework}

\phantomsection\label{page-7-0}{}The MITA Framework focuses on the
common ground shared by various distinct State Medicaid Enterprises and
yet accommodates their differences. The BA consists of four (4)
components that are summarized in \textbf{\hyperref[page-8-0]{Table}
1-2}. The BA is a composite of interrelated models and templates.

\pandocbounded{\includegraphics[keepaspectratio]{_page_7_Picture_6.jpeg}}

\subsection{\texorpdfstring{\textbf{Table 1-2. The Four Components of
the Business
Architecture.}}{Table 1-2. The Four Components of the Business Architecture.}}\label{table-1-2.-the-four-components-of-the-business-architecture.}

\phantomsection\label{page-8-0}{}

\begin{longtable}[]{@{}
  >{\raggedright\arraybackslash}p{(\linewidth - 10\tabcolsep) * \real{0.0465}}
  >{\raggedright\arraybackslash}p{(\linewidth - 10\tabcolsep) * \real{0.2903}}
  >{\raggedright\arraybackslash}p{(\linewidth - 10\tabcolsep) * \real{0.3977}}
  >{\raggedright\arraybackslash}p{(\linewidth - 10\tabcolsep) * \real{0.2614}}
  >{\raggedright\arraybackslash}p{(\linewidth - 10\tabcolsep) * \real{0.0021}}
  >{\raggedright\arraybackslash}p{(\linewidth - 10\tabcolsep) * \real{0.0021}}@{}}
\toprule\noalign{}
\begin{minipage}[b]{\linewidth}\raggedright
Business Architecture Components
\end{minipage} & \begin{minipage}[b]{\linewidth}\raggedright
\end{minipage} & \begin{minipage}[b]{\linewidth}\raggedright
\end{minipage} & \begin{minipage}[b]{\linewidth}\raggedright
\end{minipage} & \begin{minipage}[b]{\linewidth}\raggedright
\end{minipage} & \begin{minipage}[b]{\linewidth}\raggedright
\end{minipage} \\
\midrule\noalign{}
\endhead
\bottomrule\noalign{}
\endlastfoot
Component & TypeofModel & Function & Relationship & & \\
ConceptofOperations(COO)COO &
TheCOOdescribescurrentoperations,avisionoftransformation,transformationstostakeholderrolesandinformationexchanges,andtheinfluenceofenablers(e.g.,newpolicy,legislation,technology).
&
EstablishesavisionfortransformationoftheStateMedicaidEnterprise.Linksenablerstotheimprovementsinbusinessprocesses.Showshowstakeholders'roleschange.Showshowprocessesanddatachange.FocusesonimprovementsintheSMAoperations.
&
EstablishesthetargetsandvisionthatotherBAcomponentswilladdress.ProvidesaplatformandgroundingfortheMMM
andtheBusinessCapabilityMatrix(BCM). & & \\
MITAMaturityModel(MMM) &
Subdividedintofive(5)levelsofprogressivematurity,theMMMillustrateshowto
transformgoals,objectives,andbusinesscapabilitiesprogress. &
ShowshowtomeetStateMedicaidEnterprisegoalsandobjectivesandhowtoimprove
businessareas.Providesbase,consistency,andmeasuresforspecifyingdetailedbusinesscapabilitiesastheymature.
& MMM providesstructure to theCOO vision tobuild the
BCM.Providesaframeworkandmodelforthebusinesscapabilities.MMM aligns
withthe SevenStandards andConditionsrequirements. & & \\
BusinessProcessModel(BPM) & The BPM is acollection ofcommon
businessprocesses for theoperation ofMedicaidPrograms. &
Providesamodelofmajorbusinessareasandsubareas.Providesdetailed &
OriginatesfromtheSystemsTechnicalAdvisoryGroup(S-TAG)redesignoftheMedicaid
& & \\
\end{longtable}

\pandocbounded{\includegraphics[keepaspectratio]{_page_8_Picture_4.jpeg}}

Part I, Chapter 1 - Page 9 February 2012 Version 3.0

\begin{longtable}[]{@{}
  >{\raggedright\arraybackslash}p{(\linewidth - 8\tabcolsep) * \real{0.0410}}
  >{\raggedright\arraybackslash}p{(\linewidth - 8\tabcolsep) * \real{0.3053}}
  >{\raggedright\arraybackslash}p{(\linewidth - 8\tabcolsep) * \real{0.2462}}
  >{\raggedright\arraybackslash}p{(\linewidth - 8\tabcolsep) * \real{0.4055}}
  >{\raggedright\arraybackslash}p{(\linewidth - 8\tabcolsep) * \real{0.0019}}@{}}
\toprule\noalign{}
\begin{minipage}[b]{\linewidth}\raggedright
Business Architecture Components
\end{minipage} & \begin{minipage}[b]{\linewidth}\raggedright
\end{minipage} & \begin{minipage}[b]{\linewidth}\raggedright
\end{minipage} & \begin{minipage}[b]{\linewidth}\raggedright
\end{minipage} & \begin{minipage}[b]{\linewidth}\raggedright
\end{minipage} \\
\midrule\noalign{}
\endhead
\bottomrule\noalign{}
\endlastfoot
Component & TypeofModel & Function & Relationship & \\
&
Atemplatecapturesthedescriptionofeachbusinessprocess.Thebusinessprocessescovercurrentandnear-termoperations.
&
definitionsofcommonbusinessprocesses.Describesbusinessprocessesusingacommonvocabulary.Renderssomebusinessprocessesobsoleteathigherlevelsofmaturity.
&
ManagementInformationSystem(MMIS)model,variousstatemodels,andtheMedicaidHIPAA-CompliantConceptModel(MHCCM)andfederalregulation.BusinessprocessesunderreviewbytheNationalMedicaidEDIHealthcare(NMEH)workgroups.ReviewandrefinementprocessundercontinualreviewbyStates.
& \\
BusinessCapabilityMatrix(BCM) &
Subdividedintofive(5)levelsofmaturity,theBCMappliestheMMMtotheBPMtoderivecapabilitiesforeachbusinessprocessateachmaturitylevel.TheBCMdescribeshowto
transformand improve abusinessprocess. &
Showshoweachbusinessprocesscanimprove.ProvidesconsistencyandamodelfortheSMAtouseinmeasuringtheirownlevelsofmaturityforeachbusinessprocess.
& The BCM definessix (6) businesscapabilitiesacross five (5)levels of
maturityfor each
businessprocess.AlignswiththeMMMforthedescriptionofthecharacteristicsofthematuritylevels.FormstheevaluationcriteriafortheStateSelfAssessment(SSA).
& \\
\end{longtable}

\pandocbounded{\includegraphics[keepaspectratio]{_page_9_Picture_3.jpeg}}

\section[\textbf{The Concept of
Operations}]{\texorpdfstring{\protect\hypertarget{page-10-0}{}{}\textbf{The
Concept of
Operations}}{The Concept of Operations}}\label{the-concept-of-operations}

\pandocbounded{\includegraphics[keepaspectratio]{_page_10_Picture_3.jpeg}}

The COO is a tool used to describe current business operations and to
develop a future transformation that meets the needs of stakeholders and
responds to enablers (e.g., new policy, legislation, and technology).
Other industries (e.g., the Department of Defense (DOD) or National
Aeronautics and Space Administration (NASA)) use the COO as a
strategic-planning device to capture the As-Is (i.e., current)
operations,

create the To-Be (i.e., future) environment, and level-set expectations
before engaging in major transformation projects. The COO provides a
structure to place information gathered from interviews with States and
visioning sessions conducted at MMIS conferences. The COO structure
provides key information including:

\begin{itemize}
\tightlist
\item
  Definition of the scope of the Medicaid Enterprise.
\item
  Description of the As-Is (current) operations in terms of business,
  architecture, and data.
\item
  Description of the drivers and enablers that propel and support
  transformation.
\item
  Description of the To-Be environment in terms of business,
  architecture, and data.
\item
  Description of operational scenarios with sequence of events and
  activities carried out by stakeholders and the State Medicaid
  Enterprise.
\item
  Description of the impacts on each stakeholder.
\item
  Description of a summary of the improvements to the State Medicaid
  Enterprise and stakeholders.
\end{itemize}

The goal of the COO is to project changes, transformations, and provide
visions of To-Be operations, new roles and data exchanges for
stakeholders. The MITA COO provides a common vision shared by CMS and
the States that preserves individual adaptations at the state level.

Part I, Chapter 2, Concept of Operations, provides more information on
the Medicaid Enterprise COO. Part I, Appendix A, Concept of Operations
Details, contains additional information.

\section[\textbf{MITA Maturity
Model}]{\texorpdfstring{\protect\hypertarget{page-10-1}{}{}\textbf{MITA
Maturity Model}}{MITA Maturity Model}}\label{mita-maturity-model}

\pandocbounded{\includegraphics[keepaspectratio]{_page_10_Picture_16.jpeg}}

The MMM originates from industries that use such models to illustrate
how a business can mature. The MMM adapts the industry model to the
Medicaid Enterprise by describing Medicaid Program goals and objectives
and the maturation of the MITA technical principles. The

transformation through each of the five (5) levels represents
significant business capabilities advances over the previous period.

The MMM describes the five (5) levels of maturity and the measurable
qualities that each level demonstrates. The general description is at a
high enough level to apply to most aspects of State Medicaid Enterprise
operations. For example, the MMM defines, at Level 1, the business area
or process is in compliance with current regulations. At Level 2, the
process matures because of pressures for cost containment and
availability of newer tools. At Level 3, noticeable improvement occurs
in the standardization and sharing of information

\pandocbounded{\includegraphics[keepaspectratio]{_page_10_Picture_20.jpeg}}

and processes among multiple entities, including the beneficiary. At
Level 4, instant availability of clinical information increases the
transformation. By Level 5, States and local agencies have become
interoperable across the United States.

The MMM is the point of reference for the BCM. The BCM aligns with the
MMM to maintain consistency of definition. Part I, Chapter 3, Maturity
Model, presents details of the MMM, and Part I, Appendix B, Maturity
Model Details, contains the complete detailed text.

\subsection[\textbf{Business Process
Model}]{\texorpdfstring{\protect\hypertarget{page-11-0}{}{}\textbf{Business
Process Model}}{Business Process Model}}\label{business-process-model}

\pandocbounded{\includegraphics[keepaspectratio]{_page_11_Figure_5.jpeg}}

The BPM is a collection of common business processes for the operation
of Medicaid Programs. A template describes those processes, including
current and near-term operations as defined in Level 3 of the BCM. The
MITA Framework BPM

derives from multiple sources that create a common model that reflects
most State Medicaid Enterprises -- notable sources include the S-TAG
\emph{Redesign of the Medicaid Management Information System (MMIS),}
and the CMS MHCCM, that consolidates business processes from a dozen
States.

States should develop business workflows for the different business
functions of the state to advance the alignment of the state's
capability maturity with the MMM. These business workflows should align
to any provided by CMS in support of Medicaid and Exchange business
operations and requirements. States should work to streamline and
standardize these operational approaches and business workflows to
minimize customization demands on technology solutions and optimize
business outcomes.

There are those business processes that all States perform (e.g., Enroll
Provider) and those that are voluntary and depend on implementation of
special programs within a state (e.g., pay Managed Care Organization
capitation or enrollment of member in a special program). The BPM
defines common business practices across all State Medicaid Enterprises.
The MITA Framework BPM offers a hierarchy of Tier 1 business areas, Tier
2 business categories and Tier 3 business processes. The MITA Framework
contains ten (10) business areas divided into twenty-one (21) business
categories with eighty (80) individual business processes. See Part I,
Appendix C, Business Process Model Details.

The BPM provides a Business Process Template (BPT) for describing each
business process. The BPT provides a summary of the business process,
trigger event and result, activity steps, data requirements, predecessor
and successor processes, failure points, and other elements. The NMEH
workgroups review business processes, and they stand to benefit from
ongoing review by state workgroups. See Part I, Chapter 4, Business
Process Model, for a detailed presentation of the BPM and Part I,
Appendix C, Business Process Model Details, for the complete set of
business area definitions and business process descriptions.

\section[\textbf{Business Capability
Matrix}]{\texorpdfstring{\protect\hypertarget{page-11-1}{}{}\textbf{Business
Capability
Matrix}}{Business Capability Matrix}}\label{business-capability-matrix}

\pandocbounded{\includegraphics[keepaspectratio]{_page_11_Picture_12.jpeg}}

\pandocbounded{\includegraphics[keepaspectratio]{_page_11_Picture_13.jpeg}}

Applying the MMM to each business process yields the BCM that shows how
the business process matures. The BCM defines six (6) business
capabilities with five (5) levels of maturity to each business process.
The BCM assigns capabilities to an individual business process rather
than to SMA operations taken as a whole. In reality, no SMA is ``all
Level 1'' or ``all Level 2,'' but rather having

a blend of different levels of capability. An example of the
relationship among the business process, the MMM, and the BCM is shown
in \textbf{\hyperref[page-12-0]{Table} 1-3.}

Part I, Chapter 5, Business Capability Matrix, presents more information
on the BCM and Part I, Appendix D, Business Capability Matrix Details,
lists the capabilities defined for business processes contained in MITA
Framework.

\subsection{\texorpdfstring{\textbf{Table 1-3. Business Process Example:
Authorize
Service}}{Table 1-3. Business Process Example: Authorize Service}}\label{table-1-3.-business-process-example-authorize-service}

\phantomsection\label{page-12-0}{}

\begin{longtable}[]{@{}
  >{\raggedright\arraybackslash}p{(\linewidth - 6\tabcolsep) * \real{0.0603}}
  >{\raggedright\arraybackslash}p{(\linewidth - 6\tabcolsep) * \real{0.2496}}
  >{\raggedright\arraybackslash}p{(\linewidth - 6\tabcolsep) * \real{0.6868}}
  >{\raggedright\arraybackslash}p{(\linewidth - 6\tabcolsep) * \real{0.0034}}@{}}
\toprule\noalign{}
\begin{minipage}[b]{\linewidth}\raggedright
Authorize Service Business Process
\end{minipage} & \begin{minipage}[b]{\linewidth}\raggedright
\end{minipage} & \begin{minipage}[b]{\linewidth}\raggedright
\end{minipage} & \begin{minipage}[b]{\linewidth}\raggedright
\end{minipage} \\
\midrule\noalign{}
\endhead
\bottomrule\noalign{}
\endlastfoot
LevelNo. & MITAMaturityModelDefinition & BusinessCapability & \\
1 &
Complieswithregulations;mostlymanualactivities;delaysincommunicatingresults.
&
Receiptofandresponsetorequestsareprimarilyviapaper,fax,andphone;applypolicyguidelinesmanually;complieswithregulationsonturnaroundtimeandaccuracy.
& \\
2 &
Improvementsspearheadedbycostmanagementgoals;improvementsmadeinspeedofcommunicationandresponse.
&
Authorizationofservicegivengreaterpriorityasacost-managementtool;improvementsmadeincommunications;receiptofandresponsestorequestsmadeviaportal;adopt
HIPAAstandards. & \\
3 &
Informationandservicessharedwithotheragenciesandbeneficiary;streamlinedprocess;improvedresults.
&
Solutionsbecomereusableandsharablebecauseofadoptionofstandardsbystateagenciesanddata-sharingagreementstocollaborateonauthorizationofservices.
& \\
4 &
Incorporatesclinicalinformationintotheprocesstofurtherimproveresults. &
Directaccessbytheauthorizingagencytoaccess to
clinicalinformation;automationofrequests;render
decisionsbypayerautomaticallyasprovider
updatesbeneficiary'selectronichealthrecord;improve
accuracybecauseprovider basesdecisionsonclinicalevidence;limits
manualinterventiontoexceptions. & \\
5 &
Demonstrateswidespreadinteroperabilitytoachievemaximumimprovementsenvisionedatthistime.
&
Directaccessbytheauthorizingagencytoclinicalandadministrativeinformationanywhereinthecountrytoconfirmordenytheauthorizationforaservice.
& \\
\end{longtable}

\pandocbounded{\includegraphics[keepaspectratio]{_page_12_Picture_6.jpeg}}

\chapter[\textbf{Business Architecture Component
Relationships}]{\texorpdfstring{\protect\hypertarget{page-13-0}{}{}\textbf{Business
Architecture Component
Relationships}}{Business Architecture Component Relationships}}\label{business-architecture-component-relationships}

The four (4) components of the BA are interrelated:

\begin{itemize}
\tightlist
\item
  The COO serves as a model to frame a vision for Medicaid Program
  health care outcomes and operational efficiencies. It establishes the
  To-Be environment that becomes the goal of the Medicaid Enterprise
  transformation. The COO provides the vision for the MMM. It also
  supplies an overview for the BPM.
\item
  The MMM uses a common industry approach to describe the differences
  between five (5) levels of progressive maturity, ranging from As-Is
  operations to the To-Be environment. The MMM is the point of reference
  used by the BCM to describe the levels of maturity for a business
  process.
\item
  The BPM describes As-Is (i.e., current) Medicaid operations as defined
  for Level 3 of the BCM.
\item
  The BCM uses the five (5) levels of maturity described in the MMM and
  the To-Be environment defined in the COO to create definitions for
  business capabilities at five (5) levels of maturity for each business
  process.
\end{itemize}

\pandocbounded{\includegraphics[keepaspectratio]{_page_13_Figure_8.jpeg}}

\pandocbounded{\includegraphics[keepaspectratio]{_page_13_Figure_9.jpeg}}

\subsection{\texorpdfstring{\textbf{Figure 1-3. Relationship Among the
Components of the Business
Architecture}}{Figure 1-3. Relationship Among the Components of the Business Architecture}}\label{figure-1-3.-relationship-among-the-components-of-the-business-architecture}

\phantomsection\label{page-13-1}{}\pandocbounded{\includegraphics[keepaspectratio]{_page_13_Picture_11.jpeg}}

\chapter[\textbf{Connection Between
Architectures}]{\texorpdfstring{\protect\hypertarget{page-14-0}{}{}\textbf{Connection
Between
Architectures}}{Connection Between Architectures}}\label{connection-between-architectures}

The MITA Framework consists of three (3) interrelated BA, IA, and TA
components that work together to define a business-driven enterprise
transformation. The BA describes the business process activities along
with data input, data output, and required shared data. The IA provides
the bridge between the business need of information and the technical
solution data. The TA describes the technology enablers associated with
the business capabilities and their varied levels of maturity.

\textbf{\hyperref[page-14-1]{Figure 1-4}} illustrates how BA, IA, and TA
components interrelate. This is a high-level view of the primary
components within each architecture. Front Matter, Chapter 6,
Introduction to the MITA Framework, presents a detailed discussion on
the inter-relationship of all three (3) architectures. The BA
categorizes the business processes as business capabilities and assigned
a level of MITA maturity. Based on the level of maturity, the IA defines
the Conceptual Data Model (CDM) and Logical Data Model (LDM) with
necessary data attributes for the design of technical capabilities. The
TA defines the resulting business services and technical services for
the To-Be environment of the State Medicaid Enterprise.

\pandocbounded{\includegraphics[keepaspectratio]{_page_14_Figure_5.jpeg}}

\textbf{Figure 1-4. Relationships Among Components of the BA, IA, and
TA}

\phantomsection\label{page-14-1}{}The BA does not present specific
technical solutions or detailed data requirements. Some of its
components, however, point to specific companion components in the IA
and TA sections of MITA Framework (Parts II and III, respectively).
\textbf{\hyperref[page-15-0]{Table} 1-4} describes the name of the BA
Component and its relationship to the other architecture component as
well as its MITA Framework 3.0 documented location.

\pandocbounded{\includegraphics[keepaspectratio]{_page_14_Picture_8.jpeg}}

\subsubsection{\texorpdfstring{\textbf{Table 1-4. Component
Relationships of the BA, IA, and
TA}}{Table 1-4. Component Relationships of the BA, IA, and TA}}\label{table-1-4.-component-relationships-of-the-ba-ia-and-ta}

\phantomsection\label{page-15-0}{}

\begin{longtable}[]{@{}
  >{\raggedright\arraybackslash}p{(\linewidth - 8\tabcolsep) * \real{0.3900}}
  >{\raggedright\arraybackslash}p{(\linewidth - 8\tabcolsep) * \real{0.1876}}
  >{\raggedright\arraybackslash}p{(\linewidth - 8\tabcolsep) * \real{0.4165}}
  >{\raggedright\arraybackslash}p{(\linewidth - 8\tabcolsep) * \real{0.0030}}
  >{\raggedright\arraybackslash}p{(\linewidth - 8\tabcolsep) * \real{0.0030}}@{}}
\toprule\noalign{}
\begin{minipage}[b]{\linewidth}\raggedright
BA, IA, and TA Component Relationships
\end{minipage} & \begin{minipage}[b]{\linewidth}\raggedright
\end{minipage} & \begin{minipage}[b]{\linewidth}\raggedright
\end{minipage} & \begin{minipage}[b]{\linewidth}\raggedright
\end{minipage} & \begin{minipage}[b]{\linewidth}\raggedright
\end{minipage} \\
\midrule\noalign{}
\endhead
\bottomrule\noalign{}
\endlastfoot
BusinessArchitectureComponent & OtherArchitectureComponent &
Relationship & & \\
COO--DataExchanges & IA(PartII)--Allchapters &
IAchaptersprovidedetailsregardingthetransformationofdataandinformationidentifiedintheCOO.
& & \\
COO--Drivers andEnablers & TA(PartIII), Chapter2,Technical
ManagementStrategy;Chapter7,TechnicalCapabilityMatrix & Service-Oriented
Architectures(SOA)andTechnicalCapabilitiesareenablersreferencedintheCOO.
& & \\
BPM--TriggerEvent,Result,andSharedDataineachbusinessprocessdescribeingeneraltermsthekindofdatareceivedby,usedby,andresultingfromeachbusinessprocess
& IA(PartII),
Chapter2,DataManagementStrategy;Chapter3,ConceptualDataModel &
DataManagementStrategy(DMS)explainshowthedatasupportsthebusinessprocesses.TheCDMidentifiesgroupingsofinformationcommontoMedicaidbusinessareasandclustersofbusinessprocesses.
& & \\
BCM & IA(PartII), Chapter4,LogicalDataModel; Chapter 6Information
Capability Matrix & TheLDM
definesdataclassesandattributesneededtosupportdifferentlevelsofmaturity.AbusinessprocessdescribedataLevel3businesscapabilityrequiresLevel3dataattributes.
& & \\
BCM & TA(PartIII), Chapter7,TechnicalCapabilityMatrix &
TheBCMdrivestheTechnicalCapability Matrix (TCM).TAassociates technical
capabilitieswiththeBCMlevelwherespecifictechnologyisnecessarytosupportthebusinessprocess.
& & \\
BCM--Level3andabove & TA(PartIII), Chapter 2Technical
ManagementStrategy; Chapter3,BusinessServices &
Abusinessserviceisanimplementationofaspecificbusinessprocessataspecificlevelofcapability.TA
associatesbusiness servicesandSOAwithBCMLevel3andabove. & & \\
\end{longtable}

\pandocbounded{\includegraphics[keepaspectratio]{_page_15_Picture_4.jpeg}}

\chapter[\textbf{Using the Business
Architecture}]{\texorpdfstring{\protect\hypertarget{page-16-0}{}{}\textbf{Using
the Business
Architecture}}{Using the Business Architecture}}\label{using-the-business-architecture}

CMS requires States to align to and advance increasingly in MITA
maturity for business, architecture, and data. CMS expects States to use
the BA components to plan for improvements in the State Medicaid
Program, both in the delivery of services to providers and
beneficiaries, and in its internal operations and exchanges of
information with the other external stakeholders. BA provides the COO
and the MMM as background material. States and vendors use the BPM and
the BCM tools. \textbf{\hyperref[page-16-1]{Table} 1-5} summarizes how
stakeholders use the BA.

\phantomsection\label{page-16-1}{}

\begin{longtable}[]{@{}
  >{\raggedright\arraybackslash}p{(\linewidth - 4\tabcolsep) * \real{0.0861}}
  >{\raggedright\arraybackslash}p{(\linewidth - 4\tabcolsep) * \real{0.9104}}
  >{\raggedright\arraybackslash}p{(\linewidth - 4\tabcolsep) * \real{0.0035}}@{}}
\toprule\noalign{}
\begin{minipage}[b]{\linewidth}\raggedright
Stakeholder Useof the Business Architecture
\end{minipage} & \begin{minipage}[b]{\linewidth}\raggedright
\end{minipage} & \begin{minipage}[b]{\linewidth}\raggedright
\end{minipage} \\
\midrule\noalign{}
\endhead
\bottomrule\noalign{}
\endlastfoot
Stakeholder & HowStakeholders Use BA & \\
SMA & The
SMAmapstheiroperationstotheBPMandthenassessesthelevelofmaturityusingtheBCM.Whenthe
SMArequires
informationtechnologyupgradestosupportprogramimprovement,theSMAusestheSS-Atoshowhowit
will use
theenhancedfundingtoachieveaspecificresult(e.g.,movingfromLevel1or2toLevel3).
& \\
CMS &
CMSprovidesleadershipinestablishingtheMITAguidelinesandpromotingthemamongStates.ThroughthereleaseoftheMITAFramework,specialworkshopswithStates,Medicaidconferencematerial,andworkingwithearlyadopterStates,CMSprovides
guidance and principles to achieve the Medicaidvision. & \\
Vendors &
ThevendorcommunityusestheMITAFrameworkasareferenceinplanningtheirresearchanddevelopmentactivities.TheyusetheBA,inparticular,todeterminethematurityleveloffunctionssupportedbytheirsystems.Theyhaveacommonunderstandingofthe
CMS directionfor the Medicaid
Program,andtheycanshowhowtheirproductssupportMITAcapabilities. & \\
Providers & Providersplayanactiveroleintheexchangeofinformationwiththe
SMA.
TheycanlookattheSMABAtounderstandwhatdirectiontheSMAistakingandtokeepthisinmindastheyinvestininformation
technologyupgradesandreengineertheirpractices.Insomecases,the
SMAinvolveprovidersdirectlyinplanningaMedicaidProgramtransformation.
& \\
Beneficiaries & The BA supports the SMAperson-centric outreach,
eligibility and enrollmentactivities across the health and human
services spectrum.Beneficiariesandconsumergroupsare
abletolookattheSMABAandidentify thebenefits.AtLevel3business capability
maturity,beneficiariesareparticipantsinselfdirectedhealthcaredecisions.
& \\
Legislators,Governors &
StatesdeveloppresentationsbasedontheBAtoshowthegovernorandlegislatorswhatgoalsCMSisestablishingforStatesthatrequestenhanced
& \\
\end{longtable}

\subsection{\texorpdfstring{\textbf{Table 1-5. Stakeholder Use of the
Business
Architecture}}{Table 1-5. Stakeholder Use of the Business Architecture}}\label{table-1-5.-stakeholder-use-of-the-business-architecture}

\pandocbounded{\includegraphics[keepaspectratio]{_page_16_Picture_6.jpeg}}

\begin{longtable}[]{@{}
  >{\raggedright\arraybackslash}p{(\linewidth - 6\tabcolsep) * \real{0.1842}}
  >{\raggedright\arraybackslash}p{(\linewidth - 6\tabcolsep) * \real{0.8008}}
  >{\raggedright\arraybackslash}p{(\linewidth - 6\tabcolsep) * \real{0.0075}}
  >{\raggedright\arraybackslash}p{(\linewidth - 6\tabcolsep) * \real{0.0075}}@{}}
\toprule\noalign{}
\begin{minipage}[b]{\linewidth}\raggedright
Stakeholder Useof the Business Architecture
\end{minipage} & \begin{minipage}[b]{\linewidth}\raggedright
\end{minipage} & \begin{minipage}[b]{\linewidth}\raggedright
\end{minipage} & \begin{minipage}[b]{\linewidth}\raggedright
\end{minipage} \\
\midrule\noalign{}
\endhead
\bottomrule\noalign{}
\endlastfoot
Stakeholder & HowStakeholders Use BA & & \\
& funding. & & \\
OtherPayersandOtherAgencies & The MITA team invites other
payersandotheragenciestoreviewtheMITAFramework,especiallytheBA,tolearnabouttheMedicaidEnterprisetransformation.
& & \\
\end{longtable}

In general, MITA supports stakeholder roles and access to information,
technology that eliminates most manual activities, and the
transformation of the Medicaid business with the assistance of the CMS,
the SMA, providers, and beneficiaries. In addition, MITA supports
providers with instant access to patient records no matter what their
location is, patients can view their Personal Health Information (PHI)
and make informed decisions regarding treatment, and payers can view
clinical records nationally to expedite decisions on prior authorization
and payment.

\section[\textbf{Next Steps in Developing the Business
Architecture}]{\texorpdfstring{\protect\hypertarget{page-17-0}{}{}\textbf{Next
Steps in Developing the Business
Architecture}}{Next Steps in Developing the Business Architecture}}\label{next-steps-in-developing-the-business-architecture}

The MITA Framework delivers the starter kit for a controlled State
Medicaid Enterprise transformation. MITA will continue to evolve over
time. The business process defines the input and output of information
but not the details of the process; however the business community will
still decide the requirements for standardized triggers and results. The
CMS MITA team continues to support SMA efforts by serving as a conduit
for improvements to MITA models that all States and vendors can access.

The MITA Framework and the BA are ever evolving so that the SMA can
continuously improve the way they deliver services to beneficiaries and
providers, account for outcomes, reward participants based on
performance, and respond dynamically to requests for information.

\pandocbounded{\includegraphics[keepaspectratio]{_page_17_Picture_7.jpeg}}

\pandocbounded{\includegraphics[keepaspectratio]{_page_17_Picture_9.jpeg}}

\chapter{MITA Capability Model}\label{mita-capability-model-4}

\section{Introduction to Business Capability
Models}\label{introduction-to-business-capability-models-3}

A capability model is a conceptual framework that outlines the key
capabilities an organization needs to achieve its strategic objectives.
It provides a comprehensive view of what an organization can do and
helps identify areas for improvement or investment. In the context of an
orchestra, a capability model might help the orchestra identify the set
of skills and resources, or other types of capabilities it needs to
perform a symphony. Just like an orchestra needs well practiced
musicians, sheet music, instruments, a conductor, and an audience to
produce a great symphony, a State Medicaid Agency (SMA) needs its
Medicaid Enterprise System (MES) to employ or develop specific
capabilities to deliver its services effectively, efficiently, and
economically to its enrollees and providers.

The concept of a business capability is extensively used within
enterprise architecture modeling and has been broadly used within
Business Capability Models as a tool to better align the business
strategy and information technology of both private sector and
governmental organizations since they emerged in the mid-2000s. One
example comes from the TOGAF Standard, a well-known standard in
enterprise architecture. Like most architecture frameworks TOGAF defines
a capability as something a business can do to meet its goals. This
focuses a strategic lens of an organization on ``what'' it needs to
achieve its goals, rather than ``how'' those goals are achieved. This
perspective allows for business planning from different viewpoints,
facilitating strategic alignment and operational efficiency.

SMA business architects, technologists, systems analysts, executives,
managers, and program staff can use this same modeling approach to
represent the functional components of their Medicaid Enterprise System
(MES) in ways that can help reveal gaps in their systems and provide
insights on what new or enhanced capabilities might be needed to close
those gaps.

By focusing on capabilities, SMAs can better align their information and
technology resources and processes with their strategic business goals,
ultimately improving their insight into how to improve the outcomes
their Medicaid Enterprise Architecture produces.

\subsection{Purpose}\label{purpose-5}

\begin{tcolorbox}[enhanced jigsaw, toprule=.15mm, colback=white, colframe=quarto-callout-note-color-frame, left=2mm, arc=.35mm, opacityback=0, rightrule=.15mm, breakable, bottomrule=.15mm, leftrule=.75mm]
\begin{minipage}[t]{5.5mm}
\textcolor{quarto-callout-note-color}{\faInfo}
\end{minipage}%
\begin{minipage}[t]{\textwidth - 5.5mm}

\vspace{-3mm}\textbf{Note}\vspace{3mm}

MITA 4.0 does not endeavor to specify all of the capabilities SMA's may
need to administer Medicaid programs; instead, this version of MITA
focuses on the capabilities that are most closely oriented towards
achieving the CMS-required outcomes.

\end{minipage}%
\end{tcolorbox}

Understanding the how the MITA Capability Model works is important to
obtaining the most value out of many of the other tools and artifacts in
the MITA framework, such as the MITA Maturity Model (MMM) and the
Business Process Model (BPM). The MITA Capability Model provides a
structured way for SMAs to identify, conceptually model, and improve the
capabilities needed for efficient Medicaid operations.

It is important to note that MITA 4.0 does not endeavor to specify all
of the capabilities SMA's may need to administer Medicaid programs;
instead, this version of MITA focuses on the capabilities that are most
closely oriented towards achieving the CMS-required outcomes. In this
way MITA 4.0 provides a reference model for SMAs to model other
capabilities that may be needed to achieve their other goals such as
state specific outcomes, or other state priorities while providing more
guidance within the MITA Framework to support modular.

\subsection{Update to MITA 3.0}\label{update-to-mita-3.0-4}

MITA 3.0 defined a capability as the level of maturity of a set of
business processes within a business category. By focusing on ``how''
MES operate MITA 3.0 helped SMA's identify ways to improve and mature
their business processes, but it did not link those processes with the
outcomes they are intended to achieve or ensure better alignment of the
information and technical architectures to business outcomes. The
addition of the MITA capability model to the MITA 4.0 business
architecture addresses that by providing the conceptual linkages needed
to elevate the strategic vantage point of the MITA Framework. To guide
this change, we present within this chapter a definition, description,
and approach to modeling business capabilities, based on the widely used
capability models contextualized for Medicaid Enterprises.

The business processes that operationalize MITA capabilities remain
foundational to characterizing the business architecture, and are by
definition a constituent part of any MITA capability. They provide
essential information on how capabilities are operationalize and should
continue to be a routinely utilized reference model for SMA business
process mapping. They are found with in the Business Process Model
chapter of this version of MITA.

\section{The MITA Definition of
Capability}\label{the-mita-definition-of-capability-4}

Within the context of MITA, a capability can be defined as the ability
or capacity of a State Medicaid Agency to achieve a desired outcome in
compliance with the
\href{https://www.ecfr.gov/current/title-42/chapter-IV/subchapter-C/part-433/subpart-C/section-433.112}{Standards
and Conditions within 42 CFR 433.112}. A capability may currently exist
in an operational state or be envisioned for future development. Through
careful planning, capabilities defined in this way can be matured and
refined over time to become more effective and efficient. They can be
organized and detailed at various levels of abstraction, providing
precise descriptions for operational purposes or more generalized views
for strategic planning.

\begin{tcolorbox}[enhanced jigsaw, toprule=.15mm, colback=white, colframe=quarto-callout-note-color-frame, left=2mm, arc=.35mm, opacityback=0, rightrule=.15mm, breakable, bottomrule=.15mm, leftrule=.75mm]

\vspace{-3mm}\textbf{Key Definition}\vspace{3mm}

\ldots a capability is defined as the ability or capacity of a SMA to
achieve a desired outcome\ldots{}

\end{tcolorbox}

To fully define a business capability, it is essential to understand how
it is realized through the integration of people, processes,
information, and technology resources of an SMA. While these elements of
the capability can change regularly, the capability itself is should
endure over longer planning horizons, supporting the long-term alignment
of business and IT and the achievement of increasingly beneficial
business outcomes.

\subsection{Structure of the MITA Capability
Model}\label{structure-of-the-mita-capability-model-4}

As depicted in the model below, the MITA Capability Model orients the
people, process, technology, and information resources to define a MITA
Capability. This means that to model a capability the appropriate
components of the information architecture and the technical
architecture must be brought together with the business architecture to
fully formulate any MITA Capability.

\begin{figure}[H]

{\centering \pandocbounded{\includegraphics[keepaspectratio]{media/capabilityModel/topLevelCapabilityMetamodelGraphic1.png}}

}

\caption{MITA Capability Relationship Diagram}

\end{figure}%

\subsubsection{Business Roles}\label{business-roles-4}

Business roles represent individual actors, stakeholders, or partners
involved in delivering a business capability. A single organizational
group or team may be wholly responsible for delivering the capability,
or multiple business entities may share the delivery of a particular
business capability. Business Roles perform Business Processes using
Technology Resources. They require skills and knowledge resources to
achieve outcomes, and should be actively engaged as partners in the
development or enhancement of any capability they help deliver.

\subsubsection{Business Processes}\label{business-processes-4}

Individual business capabilities may be enabled or delivered through a
range of business processes that detail the activities (the how)
associated with delivering the capability. Identifying and analyzing the
efficiency of the underlying processes helps to optimize the business
capability's effectiveness. Identifying the processes within a business
capability provides a focus for maturing the capability in concert with
the other capability components. Business Processes operationalize
Business Capabilities.

\subsubsection{Information/Data}\label{informationdata-4}

Information/data represents the business data, knowledge, and insight
consumed or produced by the business capability (as distinct from
IT-related data entities). This may also include information that the
capability exchanges with other capabilities to support the execution of
value streams. Examples include information about customers and
prospects, products and services, business policies and rules, sales
reports, and performance metrics. Information/data inform the Business
Capability, answering questions and supporting business rules.

\subsubsection{Technology Resources}\label{technology-resources-4}

Business capabilities rely on a range of tools, applications, systems,
and services for successful execution. Technology Resources use
Information/data to facilitate Business Processes. Such resources may
include:

\begin{itemize}
\tightlist
\item
  Modular software applications

  \begin{itemize}
  \tightlist
  \item
    Cloud or on-premise infrastructure
  \item
    Microservices
  \item
    Analytics
  \item
    Customer portal
  \end{itemize}
\end{itemize}

In this way we can clearly interrelate all of the MITA architecture
models and their individual components which allows us to reveal gaps
not only in the individual components of the architecture, but also
understand their impact on the integration of the architecture
components at the capability level.

\subsection{Relationship of MITA Capabilities to
Outcomes}\label{relationship-of-mita-capabilities-to-outcomes-4}

In the context of the Medicaid Information Technology Architecture
(MITA), outcomes are intrinsically linked to capabilities, as they
represent the tangible results achieved through the effective
integration and execution of various elements that constitute a
capability. In this sense, outcomes and capabilities define each other.

\begin{figure}[H]

{\centering \pandocbounded{\includegraphics[keepaspectratio]{media/capabilityModel/topLevelCapabilityMetamodelOutcomes.png}}

}

\caption{MITA Capability and Outcome Relationship Diagram}

\end{figure}%

\subsubsection{Outcomes}\label{outcomes-4}

MITA defines outcomes broadly to encompass CMS-required outcomes,
state-specific outcomes, and other outcomes not mandated as part of the
Advance Planning Document (APD) process. The sole criterion for an
outcome to meet this definition is that it must be a goal of a State
Medicaid Agency (SMA) and be achieved through a Medicaid Enterprise
System (MES) capability.

\begin{tcolorbox}[enhanced jigsaw, toprule=.15mm, colback=white, colframe=quarto-callout-note-color-frame, left=2mm, arc=.35mm, opacityback=0, rightrule=.15mm, breakable, bottomrule=.15mm, leftrule=.75mm]

\vspace{-3mm}\textbf{Key Definition}\vspace{3mm}

A MITA outcome is a goal of a State Medicaid Agency (SMA) that is
achieved by a Medicaid Enterprise System (MES) capability.

\end{tcolorbox}

\subsubsection{Measure}\label{measure-4}

Measure is a quantifiable metric used to assess the effectiveness and
efficiency of capabilities within a Medicaid Enterprise System (MES).
Measures provide quantifiable and qualitative values that help State
Medicaid Agencies (SMAs) track progress toward achieving specific
outcomes, such as CMS-required or state-specific goals. These indicators
might include metrics like processing times, error rates, or compliance
levels.

Measures are a measurement threshold by establishing a specific value or
level that must be met or exceeded to demonstrate successful
performance. For instance, a KPI might set a threshold for the maximum
allowable processing time for claims, ensuring that they are handled
within a specified timeframe to maintain compliance and eligibility for
enhanced federal funding. By monitoring these thresholds, organizations
can ensure they are meeting regulatory requirements and delivering
high-quality services to beneficiaries, while also identifying areas for
improvement.

\subsubsection{Measure Threshold}\label{measure-threshold-4}

A specific value or level of a measure that must be met or exceeded to
demonstrate the effective achievement of a capability's intended
outcome. This threshold serves as a benchmark for assessing whether the
processes, roles, and resources integrated within a Medicaid Enterprise
System (MES) are functioning optimally to meet the goals of a State
Medicaid Agency (SMA). For example, a measurement threshold might be set
for processing times, where claims must be processed within a certain
number of days to ensure compliance with CMS-required outcomes and
maintain eligibility for enhanced federal funding. By establishing and
monitoring these thresholds, organizations can ensure they are meeting
regulatory requirements and delivering high-quality services to
beneficiaries.

\subsubsection{Measurement}\label{measurement-4}

These outcomes and metrics are also used to ensure that healthcare
systems or modules comply with applicable federal regulations, forming
the baseline for system or module functionality. Achieving these
outcomes is essential for continuing to receive enhanced federal funding
for operations. Regular measurement and analysis of KPIs help
organizations demonstrate compliance and effectiveness, ensuring that
they meet regulatory requirements and continue to deliver high-quality
services to beneficiaries.

In this way we can clearly interrelate all of the MITA architecture
models and their individual components with the KPIs, thresholds, and
measurements that indicate whether our capability achieves our desired
outcome.

While models that help conceptualize the capabilities that achieve
CMS-required outcomes are the ones modeled for this version of MITA,
SMAs are encouraged to use these models as a reference to model
capabilities.

\section{Capability Mapping}\label{capability-mapping-4}

Capability mapping is a strategic tool that enables organizations, such
as State Medicaid Agencies (SMAs), to systematically identify, organize,
and visualize the key capabilities necessary to achieve their
objectives. Within the MITA framework, capability mapping provides SMAs
with a method of developing comprehensive views of the functions and
processes required to deliver Medicaid services effectively. To begin
the capability mapping process, SMAs should first identify the core
capabilities that align with their strategic objectives, focusing on
what the organization needs to achieve rather than how those goals are
accomplished. This involves listing all necessary capabilities and
understanding the desired outcomes they support. Next, these
capabilities should be organized into domains and areas that reflect
their strategic importance and interrelationships. Visualizing these
capabilities through diagrams or maps provides all stakeholders a common
view to understand the roles, processes, technology resources, and
information/data involved in executing each capability, as well as the
outcome each capability is designed to achieve. This structured approach
not only highlights areas for improvement or investment but also ensures
that organizational efforts are strategically aligned with desired
outcomes.

The benefits of capability mapping are multifaceted, offering SMAs a
clear pathway to strategic alignment and gap analysis. By visualizing
capabilities, organizations can identify operational gaps and determine
what new or enhanced capabilities are needed to close those gaps. This
visualization also improves communication among stakeholders by
providing a clear and concise representation of the organization's
functions. To refine capabilities, SMAs should analyze current
operations, assess the efficiency of underlying processes, and optimize
them to enhance capability effectiveness. Additionally, capability
mapping serves as a foundation for heat mapping, which assesses the MITA
Framework will utilize to visualize the maturity of each capability
evaluated in the State Self-Assessment. SMAs can overlay heat maps over
their capability maps to visualize many things other than maturity
levels, using color coding to indicate areas of strength and weakness.
Regular updates to these maps allow SMAs to monitor progress and ensure
resources are allocated effectively to achieve strategic goals. The MITA
framework includes examples of capability maps based on CMS-required
outcomes, serving as a reference model for SMAs to develop their own
capability maps tailored to state-specific goals and priorities. By
leveraging the reference models provided by MITA, SMAs can ensure their
capability mapping efforts are aligned with both federal requirements
and state-specific priorities.

\subsection{Organizing Capabilities}\label{organizing-capabilities-4}

To enhance the resolution and detail of a capability and provide a
unified view of all its components, a block diagram can be employed to
provide a common view of any MES. This diagram effectively links the
capability to business processes, roles, technical resources, and
information resources through functional decomposition. By breaking down
the capability into its constituent parts, the block diagram offers a
visual representation that highlights the interrelationships and
dependencies among these elements. This approach provides a clearer
understanding of how each component contributes to the overall
capability, facilitating more effective analysis, optimization, and
alignment with organizational objectives.

\pandocbounded{\includegraphics[keepaspectratio]{media/capabilityModel/capabilityOgranizationModel.drawio.png}}

We use this same method to present an this top level view of the
capabilities required to achieve CMS-required outcomes. From this view
increasingly detailed models can be constructed.

\pandocbounded{\includegraphics[keepaspectratio]{media/capabilityModel/mesModuleBasedCapabilities.drawio.png}}

\subsection{MITA Capability Models}\label{mita-capability-models-4}

The MITA framework represents capabilities visually through a layered
model that represent a capability of being composed of sub-capabilities
and the processes, roles, information and technology resources (PRIT)
that support the business in sustaining the capability. Each layer up
depicts increasingly strategic capabilities and each layer down depicts
the constituent elements that compose a capability in increasing
operational detail. It is not the intention of this version of MITA to
provide a full operational or tactical view of a capability, though SMAs
may consider using this approach to improve their organizational
awareness of their operations by developing further layers of their
capabilities through functional decomposition.

\pandocbounded{\includegraphics[keepaspectratio]{media/capabilityModel/capabilityLevels.png}}

\begin{itemize}
\tightlist
\item
  \textbf{Capability Domains:} The first layer of this model aims to
  group capabilities to organize the strategic view of an SMA's
  capabilities. In this view one or many capabilities can be grouped
  within a domain to indicate the pursuit of common outcomes. Each
  domain is denoted with a single number to help annotate each
  capability.

  \begin{itemize}
  \tightlist
  \item
    \textbf{Capability Areas:} The second layer of this model aims to
    provide a view of the groups of capabilities that compose a domain.
    They are organized to show capabilities that serve a specific group
    of similar outcomes and essential
  \item
    \textbf{Capabilities:} The third layer of this model provides a more
    detailed view view of
  \end{itemize}
\end{itemize}

\pandocbounded{\includegraphics[keepaspectratio]{media/capabilityModel/capabilityLevels2.png}}

\subsection{Relationship of MITA Capabilities to
Maturity}\label{relationship-of-mita-capabilities-to-maturity-4}

\begin{tcolorbox}[enhanced jigsaw, toprule=.15mm, leftrule=.75mm, colframe=quarto-callout-warning-color-frame, left=2mm, arc=.35mm, titlerule=0mm, rightrule=.15mm, opacitybacktitle=0.6, bottomtitle=1mm, toptitle=1mm, colbacktitle=quarto-callout-warning-color!10!white, bottomrule=.15mm, title=\textcolor{quarto-callout-warning-color}{\faExclamationTriangle}\hspace{0.5em}{Warning}, opacityback=0, breakable, colback=white, coltitle=black]

Under development.

\end{tcolorbox}

\begin{itemize}
\tightlist
\item
  \textbf{Levels of Maturity}

  \begin{itemize}
  \tightlist
  \item
    Description of the five levels of maturity in the MITA framework
  \item
    How capabilities evolve and mature over time
  \end{itemize}
\end{itemize}

\pandocbounded{\includegraphics[keepaspectratio]{media/capabilityModel/maturityModel.png}}

\subsection{Using Capability Maps for Heat Mapping Strategic Priorities
and Identifying Gaps with the MITA Maturity
Model}\label{using-capability-maps-for-heat-mapping-strategic-priorities-and-identifying-gaps-with-the-mita-maturity-model-4}

Capability maps are powerful tools that not only provide a visual
representation of an SMA's key capabilities but also serve as a
foundation for strategic analysis and planning. There are many
approaches to heat mapping capabilities, each offering unique insights
into organizational priorities and gaps. Here, we describe two
approaches: assessing maturity levels using the MITA Maturity Model and
prioritizing strategic outcomes.

\subsubsection{Identifying Gaps with the MITA Maturity
Model}\label{identifying-gaps-with-the-mita-maturity-model-4}

The MITA Maturity Model provides a framework for assessing the maturity
of an organization's capabilities across various dimensions, such as
business processes, information, and technology. By integrating the
maturity model with capability maps, SMAs can identify gaps between
their current state and desired maturity levels.

\paragraph{Example 1: Identifying Gaps in Data Management Maturity Using
the PRIT
Model}\label{example-1-identifying-gaps-in-data-management-maturity-using-the-prit-model-4}

An SMA is conducting an assessment of its data management capabilities
using the MITA Maturity Model, with a focus on the PRIT (Processes,
Roles, Information, and Technology) framework. The capability map
includes various data-related capabilities, such as ``Data
Integration,'' ``Data Quality Management,'' and ``Data Analytics.'' Each
of these capabilities is evaluated across the PRIT dimensions to
determine their maturity levels using the revised scale:

Processes: Level 1: Ad-Hoc Roles: Level 2: Compliant Information: Level
2: Compliant Technology: Level 2: Compliant The capability map is
updated to reflect the maturity assessment, with each dimension marked
with a color code: red for Level 1: Ad-Hoc, yellow for Level 2:
Compliant, green for Level 3: Efficient, blue for Level 4: Optimized,
and purple for Level 5: Pioneering. This visualization helps the SMA
prioritize strategic actions to enhance the ``Data Integration''
capability, such as standardizing processes, refining roles, improving
data quality, and upgrading technology.

\subsubsection{Heat Mapping Strategic
Priorities}\label{heat-mapping-strategic-priorities-4}

Heat mapping involves applying a color-coded overlay to a capability map
to visually represent the status or priority level of each capability.
This technique can be used to highlight areas of strength, weakness, or
strategic importance. For example, capabilities that are critical to
achieving CMS-required outcomes might be marked in one color, while
those needing immediate attention or improvement could be marked in
another. This visual representation helps stakeholders quickly grasp the
strategic landscape and make informed decisions about where to allocate
resources and focus efforts.

\paragraph{Example 2: Prioritizing Capabilities for CMS-Required
Outcomes}\label{example-2-prioritizing-capabilities-for-cms-required-outcomes-4}

An SMA is focused on achieving specific CMS-required outcomes related to
improving patient care and reducing administrative costs. The agency
creates a capability map that outlines all the capabilities necessary to
meet these outcomes. By applying a heat map, the SMA highlights
capabilities that are directly linked to these outcomes in green,
indicating they are of high strategic priority. Capabilities that are
indirectly related or less critical are marked in yellow, while those
that are currently underperforming or not aligned with strategic goals
are marked in red.

This visual representation allows the SMA to quickly identify which
capabilities require immediate attention and resources to ensure
compliance with CMS requirements. For instance, if the capability
related to ``Claims Processing Efficiency'' is marked in red, the agency
can prioritize initiatives to enhance this capability, such as investing
in new technology or streamlining processes.

\subsubsection{Other Uses for Capability Heat
Mapping}\label{other-uses-for-capability-heat-mapping-4}

Beyond assessing maturity levels and prioritizing strategic initiatives,
capability heat mapping can be applied in various other contexts to
enhance organizational effectiveness and alignment.

\paragraph{Example 3: Aligning Capabilities with State-Specific
Initiatives}\label{example-3-aligning-capabilities-with-state-specific-initiatives-4}

An SMA is working on a state-specific initiative to enhance telehealth
services for rural populations. The capability map includes capabilities
related to telehealth, such as ``Telehealth Infrastructure,'' ``Provider
Engagement,'' and ``Patient Access.'' The SMA uses a heat map to
highlight these capabilities in blue, indicating their alignment with
the state-specific initiative.

By analyzing the capability map, the SMA identifies that ``Provider
Engagement'' is a critical capability that requires further development
to support the telehealth initiative. The agency decides to invest in
training programs and outreach efforts to engage providers in rural
areas, ensuring that the telehealth services are effectively delivered
to the target population.

These examples demonstrate how capability maps, combined with heat
mapping and the MITA Maturity Model, can provide valuable insights for
strategic planning and gap analysis. By visualizing priorities and
maturity levels, SMAs can make informed decisions about where to focus
resources and efforts, ultimately enhancing their Medicaid Enterprise
Systems and achieving strategic objectives.

\begin{itemize}
\tightlist
\item
  \textbf{Capability Mapping}

  \begin{itemize}
  \tightlist
  \item
    Introduction to capability mapping and its significance
  \item
    How capabilities are organized and detailed at various levels of
    abstraction
  \end{itemize}
\end{itemize}

\section{Guidance on reuse of the MITA Capability
Model}\label{guidance-on-reuse-of-the-mita-capability-model-4}

\begin{itemize}
\tightlist
\item
  \textbf{Practical Application}

  \begin{itemize}
  \tightlist
  \item
    How to integrate the capability model into daily operations and
    strategic planning
  \item
    Tips for maximizing the benefits of the model
  \end{itemize}
\item
  \textbf{Continuous Improvement}

  \begin{itemize}
  \tightlist
  \item
    Encouragement for ongoing assessment and refinement of capabilities
  \item
    Leveraging feedback and performance data for model enhancement
  \end{itemize}
\item
  \textbf{Implementation Guidance}

  \begin{itemize}
  \tightlist
  \item
    Steps for adopting the capability model
  \item
    Resources and support available for SMAs
  \end{itemize}
\item
  \textbf{Performance Monitoring and Reporting}

  \begin{itemize}
  \tightlist
  \item
    Role of the capability model in tracking and enhancing performance
  \item
    Use of metrics and standards to measure capability effectiveness
  \end{itemize}
\end{itemize}

\chapter{CONOPS}\label{conops}

The Medicaid IT Architecture (MITA) Framework contains three (3)
interrelated architectures: Business Architecture (BA), Information
Architecture (IA), and Technical Architecture (TA) shown in
\textbf{\hyperref[page-2-2]{Figure 1-1}}. The business capabilities from
BA define the data strategy of IA and design the business and technical
services of TA. MITA uses all three (3) architectures to develop a
business-driven enterprise to provide consistency across the State
Medicaid Enterprise.

\begin{itemize}
\tightlist
\item
  BA Seven Standards and Conditions
\item
  Business Architecture Components
\item
  Business Architecture Component Relationships
\item
  Connection Between Architectures
\item
  Using the Business Architecture
\item
  Next Steps in Developing the Business Architecture
\end{itemize}

\section[\textbf{Purpose}]{\texorpdfstring{\protect\hypertarget{page-2-1}{}{}\textbf{Purpose}}{Purpose}}\label{purpose-6}

In keeping with the guiding principle that MITA represents a
business-driven enterprise transformation, the BA is the starting point
of the MITA Framework. The BA describes the needs and goals of the
individual State Medicaid Enterprise, and presents a collective vision
of the future.

The BA will accomplish the following:

\begin{itemize}
\tightlist
\item
  Establish a generic business framework for all States while
  recognizing their differences.
\item
  Describe how each state Medicaid Program can mature over a given
  period with the help of stakeholders, leadership, enabling
  legislation, and technology.
\item
  Provide a baseline for the State Medicaid Agency (SMA) to assess their
  current business capabilities and measure progress toward improved
  capabilities.
\end{itemize}

\pandocbounded{\includegraphics[keepaspectratio]{_page_2_Picture_20.jpeg}}

\subsection[\textbf{Scope}]{\texorpdfstring{\protect\hypertarget{page-3-0}{}{}\textbf{Scope}}{Scope}}\label{scope-1}

The BA focuses on the State Medicaid Enterprise that centers on the
Medicaid environment including leveraged systems and interconnections
among Medicaid stakeholders, providers, beneficiaries, insurance
affordability programs (e.g., CHIP, tax credits, Basic Health Program),
Health Insurance Exchange (HIX), Health Information Exchange (HIE),
other state and local agencies, other payers, Centers for Medicare \&
Medicaid Services (CMS), and other federal agencies. The MITA context
defines the Medicaid Enterprise as:

\begin{itemize}
\tightlist
\item
  The domain where federal matching funds apply.
\item
  The interfaces and bridges among Medicaid stakeholders, including
  providers, beneficiaries, other state and local agencies, other
  payers, CMS, and other federal agencies.
\item
  The sphere of influence touched by MITA (e.g., national and federal
  initiatives such as the Nationwide Health Information Network
  {[}NwHIN{]}). (See Front Matter, Chapter 6, Overview of the MITA
  Initiative, for a discussion of the Medicaid Enterprise.)
\end{itemize}

\emph{Enterprise can have other meanings. For instance, Enterprise
Architecture (EA) defines an enterprise-wide integrated set of
components that incorporates strategic business thinking, information
assets, and the technical infrastructure of an enterprise to promote
information sharing across agency and organizational boundaries.}

The BA acknowledges technology as one of several enablers that are
important to growth and transformation, but it does not introduce
technical implementations or solutions into the BA components. All
technical references are found in Part III, Technical Architecture.

\subsection[\textbf{Background}]{\texorpdfstring{\protect\hypertarget{page-3-1}{}{}\textbf{Background}}{Background}}\label{background-1}

States, territories, and the District of Columbia (hereinafter referred
to as States) are responsible for their individual State Medicaid
Enterprise, and all entities are different in important ways.
Differences include:

\begin{itemize}
\tightlist
\item
  Organizational structure, covered programs, and lines of business
\item
  Business rules, policies, and procedures affecting stakeholders
\item
  Relationships with other state and local agencies
\item
  Revenue sources
\item
  Location of business units
\item
  Workflow
\item
  Range of outsourcing
\item
  Technical solutions
\end{itemize}

\pandocbounded{\includegraphics[keepaspectratio]{_page_3_Picture_19.jpeg}}

These entities also differ in their concept of an enterprise, the roles
and responsibilities of one or more Chief Information Officers (CIO),
adoption of data and technical standards, and the use of legacy versus
state-of-the-art applications.

Given these differences, it is not possible or desirable, in the context
of the MITA Framework, to develop a standalone business and technical
model for each individual Medicaid Enterprise. Instead, MITA establishes
a national framework of common processes and enabling technologies to
support improved program administration in all States.

The BA focuses on areas of common ground (e.g., that all States will
enroll providers and pay for services rendered to eligible beneficiaries
and that all States seek to improve health care outcomes and improve
administrative processes).

There is no ready-made methodology for building the MITA Framework to
accommodate the business needs and transformation strategies of the
States. To meet the special needs of MITA, the components included in
the BA draw upon methodologies commonly in use today across industries
as diverse as financial, transportation, and defense. The MITA team
designed templates and models to help States identify and prioritize
their specific business needs.

The BA section of the MITA Framework shows how MITA incorporates
business-driven design to accomplish the following:

\begin{itemize}
\tightlist
\item
  Support state needs.

  \begin{itemize}
  \tightlist
  \item
    \textbf{o} Align with state strategic goals.
  \item
    \textbf{o} Align with state or Medicaid Agency enterprise
    architecture.
  \end{itemize}
\item
  Support the CMS and common state goals.

  \begin{itemize}
  \tightlist
  \item
    \textbf{o} Align state approaches with MITA.
  \item
    \textbf{o} Accommodate multi-state collaborative initiatives.
  \end{itemize}
\item
  Support national goals through alignment with national initiatives,
  such as the Office of the National Coordinator for Health Information
  Technology (ONC) and federal guidelines (e.g., Federal Health
  Architecture (FHA), the Federal Enterprise Architecture Framework
  (FEAF), and national/international data standards).
\end{itemize}

\section[\textbf{Funding
Requirements}]{\texorpdfstring{\protect\hypertarget{page-4-0}{}{}\textbf{Funding
Requirements}}{Funding Requirements}}\label{funding-requirements-1}

The Health and Human Services (HHS) CMS 42 CFR Part 433 Medicaid
Program; Federal Funding for Medicaid Eligibility Determination and
Enrollment Activities modifies Medicaid regulations for Mechanized
Claims Processing and Information Retrieval Systems effective April 19,
2011. The Medicaid Management Information System (MMIS) is a mechanized
claims processing and information retrieval system used by the States
for Title XIX of the Social Security Act (The Act); therefore, the
guidance set forth in CMS 42 CFR Part 433 applies to the MMIS as well as
the Medicaid eligibility determination and enrollment activities as set
forth in the Affordable Care Act of 2010.

CMS expects States to meet the standards and conditions specified in
§433.112(b)(10) through §433.112(b)(16). The standards and conditions
are descriptive in nature; however, CMS recognizes that in order for the
States to meet these standards and conditions it is necessary to provide
additional guidance that clearly articulates its criteria for meeting
them

\pandocbounded{\includegraphics[keepaspectratio]{_page_4_Picture_17.jpeg}}

in terms of timeliness, accuracy, efficiency, integrity, and performance
standards for mechanized claims processing. In response to this need,
additional guidance materials include:

\begin{itemize}
\tightlist
\item
  Enhanced Funding Requirements: Seven Conditions and Standards (a.k.a.
  Seven Standards and Conditions)
\item
  Guidance for Exchange and Medicaid Information Technology (IT) Systems
  (a.k.a. IT Guidance)
\end{itemize}

CMS will continue to refine, update and expand this guidance in the
future, based on feedback from stakeholders and with experience over
time.

\chapter[\textbf{BA Seven Standards and
Conditions}]{\texorpdfstring{\protect\hypertarget{page-5-0}{}{}\textbf{BA
Seven Standards and
Conditions}}{BA Seven Standards and Conditions}}\label{ba-seven-standards-and-conditions-1}

The MITA team evaluated and incorporated the 42 CFR Part 433 Medicaid
Program; Federal Funding for Medicaid Eligibility Determination and
Enrollment Activities in the BA for purposes of guiding the MITA
stakeholders to apply the guidance to the Medicaid Enterprise.

Each of the architectures aligns with the Seven Standards and
Conditions. By utilizing best practices, industry standards, and
technology advancements, the processes, and planning guidelines that
build the MITA framework provide a cohesive method for meeting Medicaid
objectives.

\textbf{\hyperref[page-5-1]{Table 1-1}} depicts the impact of the Seven
Standards and Conditions on the MITA BA, IA, and TA.

\phantomsection\label{page-5-1}{}

\begin{longtable}[]{@{}
  >{\raggedright\arraybackslash}p{(\linewidth - 8\tabcolsep) * \real{0.4581}}
  >{\raggedright\arraybackslash}p{(\linewidth - 8\tabcolsep) * \real{0.1677}}
  >{\raggedright\arraybackslash}p{(\linewidth - 8\tabcolsep) * \real{0.1871}}
  >{\raggedright\arraybackslash}p{(\linewidth - 8\tabcolsep) * \real{0.1742}}
  >{\raggedright\arraybackslash}p{(\linewidth - 8\tabcolsep) * \real{0.0129}}@{}}
\toprule\noalign{}
\begin{minipage}[b]{\linewidth}\raggedright
Correlation of Seven Standards and Conditions with MITA Architectures
\end{minipage} & \begin{minipage}[b]{\linewidth}\raggedright
\end{minipage} & \begin{minipage}[b]{\linewidth}\raggedright
\end{minipage} & \begin{minipage}[b]{\linewidth}\raggedright
\end{minipage} & \begin{minipage}[b]{\linewidth}\raggedright
\end{minipage} \\
\midrule\noalign{}
\endhead
\bottomrule\noalign{}
\endlastfoot
Standards and Conditions & BusinessArchitecture &
InformationArchitecture & TechnicalArchitecture & \\
Modularity Standard & X & X & X & \\
MITA Condition & X & X & X & \\
Industry Standards Condition & X & X & X & \\
Leverage Condition & X & X & X & \\
Business Results Condition & X & X & X & \\
Reporting Condition & X & X & X & \\
Interoperability Condition & X & X & X & \\
\end{longtable}

\subsection{\texorpdfstring{\textbf{Table 1-1. Correlation of Seven
Standards and Conditions with
MITA}}{Table 1-1. Correlation of Seven Standards and Conditions with MITA}}\label{table-1-1.-correlation-of-seven-standards-and-conditions-with-mita-1}

\pandocbounded{\includegraphics[keepaspectratio]{_page_5_Picture_12.jpeg}}

The BA includes:

\begin{itemize}
\tightlist
\item
  \textbf{Modularity Standard} Uses a modular, flexible approach to
  systems development, including the use of open interfaces and exposed
  Application Programming Interfaces (API); the separation of business
  rules from core programming; and the availability of business rules in
  both human and machine-readable formats. The States commit to formal
  system development methodology and open, reusable system architecture.
\item
  \textbf{MITA Condition} States align to and advance increasingly in
  MITA maturity for business, architecture, and data.
\end{itemize}

\textbf{Industry Standards Condition} - Ensures alignment with, and
incorporation of, industry standards: the Health Insurance Portability
and Accountability Act of 1996 (HIPAA) security, privacy and transaction
standards; accessibility standards established under section 508 of the
Rehabilitation Act, or standards that provide greater accessibility for
individuals with disabilities, and compliance with Federal Civil Rights
laws; standards adopted by the Secretary under section 1104 of the
Affordable Care Act; and standards and protocols adopted by the
Secretary under section 1561 of the Affordable Care Act.

\begin{itemize}
\tightlist
\item
  \textbf{Leverage Condition} States solutions should promote sharing,
  leverage, and reuse of Medicaid technologies and systems within and
  among States.
\item
  \textbf{Business Results Condition} Systems should support accurate
  and timely processing of claims (including claims of eligibility),
  adjudications, and effective communications with providers,
  beneficiaries, and the public.
\item
  \textbf{Reporting Condition} Solutions should produce transaction
  data, reports, and performance information that contribute to program
  evaluation, continuous improvement in business operations, and
  transparency and accountability.
\item
  \textbf{Interoperability Condition} Systems must ensure seamless
  coordination and integration with the Exchange (whether run by the
  state or federal government), and allow interoperability with health
  information exchanges, public health agencies, human services
  programs, and community organizations providing outreach and
  enrollment assistance services.
\end{itemize}

\chapter[\textbf{Business Architecture
Components}]{\texorpdfstring{\protect\hypertarget{page-6-0}{}{}\textbf{Business
Architecture
Components}}{Business Architecture Components}}\label{business-architecture-components-1}

The BA is a conceptual construct that encompasses models, matrices, and
templates. These components derive from a variety of industry standards
because no single methodology exists that meets the scope of MITA. The
MITA Framework breaks new ground and is a model for other federal,
state, and local entities.

The MITA BA contains the following components:

\begin{itemize}
\tightlist
\item
  Concept of Operations
\item
  MITA Maturity Model
\item
  Business Process Model
\item
  Business Capability Matrix
\end{itemize}

\pandocbounded{\includegraphics[keepaspectratio]{_page_6_Picture_17.jpeg}}

These are living models that evolve with the MITA Framework life cycle.
The MITA team tailored the level of detail in each model to meet the
specific needs of the intended audience.
\textbf{\hyperref[page-7-0]{Figure 1-2}} provides an overview of the
components of the BA. See Part I, Chapters 2 through 5 for a more
detailed description for each of these components.

\pandocbounded{\includegraphics[keepaspectratio]{_page_7_Figure_3.jpeg}}

\textbf{Figure 1-2}. \textbf{BA in the Context of the MITA Framework}

\phantomsection\label{page-7-0}{}The MITA Framework focuses on the
common ground shared by various distinct State Medicaid Enterprises and
yet accommodates their differences. The BA consists of four (4)
components that are summarized in \textbf{\hyperref[page-8-0]{Table}
1-2}. The BA is a composite of interrelated models and templates.

\pandocbounded{\includegraphics[keepaspectratio]{_page_7_Picture_6.jpeg}}

\subsection{\texorpdfstring{\textbf{Table 1-2. The Four Components of
the Business
Architecture.}}{Table 1-2. The Four Components of the Business Architecture.}}\label{table-1-2.-the-four-components-of-the-business-architecture.-1}

\phantomsection\label{page-8-0}{}

\begin{longtable}[]{@{}
  >{\raggedright\arraybackslash}p{(\linewidth - 10\tabcolsep) * \real{0.0465}}
  >{\raggedright\arraybackslash}p{(\linewidth - 10\tabcolsep) * \real{0.2903}}
  >{\raggedright\arraybackslash}p{(\linewidth - 10\tabcolsep) * \real{0.3977}}
  >{\raggedright\arraybackslash}p{(\linewidth - 10\tabcolsep) * \real{0.2614}}
  >{\raggedright\arraybackslash}p{(\linewidth - 10\tabcolsep) * \real{0.0021}}
  >{\raggedright\arraybackslash}p{(\linewidth - 10\tabcolsep) * \real{0.0021}}@{}}
\toprule\noalign{}
\begin{minipage}[b]{\linewidth}\raggedright
Business Architecture Components
\end{minipage} & \begin{minipage}[b]{\linewidth}\raggedright
\end{minipage} & \begin{minipage}[b]{\linewidth}\raggedright
\end{minipage} & \begin{minipage}[b]{\linewidth}\raggedright
\end{minipage} & \begin{minipage}[b]{\linewidth}\raggedright
\end{minipage} & \begin{minipage}[b]{\linewidth}\raggedright
\end{minipage} \\
\midrule\noalign{}
\endhead
\bottomrule\noalign{}
\endlastfoot
Component & TypeofModel & Function & Relationship & & \\
ConceptofOperations(COO)COO &
TheCOOdescribescurrentoperations,avisionoftransformation,transformationstostakeholderrolesandinformationexchanges,andtheinfluenceofenablers(e.g.,newpolicy,legislation,technology).
&
EstablishesavisionfortransformationoftheStateMedicaidEnterprise.Linksenablerstotheimprovementsinbusinessprocesses.Showshowstakeholders'roleschange.Showshowprocessesanddatachange.FocusesonimprovementsintheSMAoperations.
&
EstablishesthetargetsandvisionthatotherBAcomponentswilladdress.ProvidesaplatformandgroundingfortheMMM
andtheBusinessCapabilityMatrix(BCM). & & \\
MITAMaturityModel(MMM) &
Subdividedintofive(5)levelsofprogressivematurity,theMMMillustrateshowto
transformgoals,objectives,andbusinesscapabilitiesprogress. &
ShowshowtomeetStateMedicaidEnterprisegoalsandobjectivesandhowtoimprove
businessareas.Providesbase,consistency,andmeasuresforspecifyingdetailedbusinesscapabilitiesastheymature.
& MMM providesstructure to theCOO vision tobuild the
BCM.Providesaframeworkandmodelforthebusinesscapabilities.MMM aligns
withthe SevenStandards andConditionsrequirements. & & \\
BusinessProcessModel(BPM) & The BPM is acollection ofcommon
businessprocesses for theoperation ofMedicaidPrograms. &
Providesamodelofmajorbusinessareasandsubareas.Providesdetailed &
OriginatesfromtheSystemsTechnicalAdvisoryGroup(S-TAG)redesignoftheMedicaid
& & \\
\end{longtable}

\pandocbounded{\includegraphics[keepaspectratio]{_page_8_Picture_4.jpeg}}

Part I, Chapter 1 - Page 9 February 2012 Version 3.0

\begin{longtable}[]{@{}
  >{\raggedright\arraybackslash}p{(\linewidth - 8\tabcolsep) * \real{0.0410}}
  >{\raggedright\arraybackslash}p{(\linewidth - 8\tabcolsep) * \real{0.3053}}
  >{\raggedright\arraybackslash}p{(\linewidth - 8\tabcolsep) * \real{0.2462}}
  >{\raggedright\arraybackslash}p{(\linewidth - 8\tabcolsep) * \real{0.4055}}
  >{\raggedright\arraybackslash}p{(\linewidth - 8\tabcolsep) * \real{0.0019}}@{}}
\toprule\noalign{}
\begin{minipage}[b]{\linewidth}\raggedright
Business Architecture Components
\end{minipage} & \begin{minipage}[b]{\linewidth}\raggedright
\end{minipage} & \begin{minipage}[b]{\linewidth}\raggedright
\end{minipage} & \begin{minipage}[b]{\linewidth}\raggedright
\end{minipage} & \begin{minipage}[b]{\linewidth}\raggedright
\end{minipage} \\
\midrule\noalign{}
\endhead
\bottomrule\noalign{}
\endlastfoot
Component & TypeofModel & Function & Relationship & \\
&
Atemplatecapturesthedescriptionofeachbusinessprocess.Thebusinessprocessescovercurrentandnear-termoperations.
&
definitionsofcommonbusinessprocesses.Describesbusinessprocessesusingacommonvocabulary.Renderssomebusinessprocessesobsoleteathigherlevelsofmaturity.
&
ManagementInformationSystem(MMIS)model,variousstatemodels,andtheMedicaidHIPAA-CompliantConceptModel(MHCCM)andfederalregulation.BusinessprocessesunderreviewbytheNationalMedicaidEDIHealthcare(NMEH)workgroups.ReviewandrefinementprocessundercontinualreviewbyStates.
& \\
BusinessCapabilityMatrix(BCM) &
Subdividedintofive(5)levelsofmaturity,theBCMappliestheMMMtotheBPMtoderivecapabilitiesforeachbusinessprocessateachmaturitylevel.TheBCMdescribeshowto
transformand improve abusinessprocess. &
Showshoweachbusinessprocesscanimprove.ProvidesconsistencyandamodelfortheSMAtouseinmeasuringtheirownlevelsofmaturityforeachbusinessprocess.
& The BCM definessix (6) businesscapabilitiesacross five (5)levels of
maturityfor each
businessprocess.AlignswiththeMMMforthedescriptionofthecharacteristicsofthematuritylevels.FormstheevaluationcriteriafortheStateSelfAssessment(SSA).
& \\
\end{longtable}

\pandocbounded{\includegraphics[keepaspectratio]{_page_9_Picture_3.jpeg}}

\section[\textbf{The Concept of
Operations}]{\texorpdfstring{\protect\hypertarget{page-10-0}{}{}\textbf{The
Concept of
Operations}}{The Concept of Operations}}\label{the-concept-of-operations-1}

\pandocbounded{\includegraphics[keepaspectratio]{_page_10_Picture_3.jpeg}}

The COO is a tool used to describe current business operations and to
develop a future transformation that meets the needs of stakeholders and
responds to enablers (e.g., new policy, legislation, and technology).
Other industries (e.g., the Department of Defense (DOD) or National
Aeronautics and Space Administration (NASA)) use the COO as a
strategic-planning device to capture the As-Is (i.e., current)
operations,

create the To-Be (i.e., future) environment, and level-set expectations
before engaging in major transformation projects. The COO provides a
structure to place information gathered from interviews with States and
visioning sessions conducted at MMIS conferences. The COO structure
provides key information including:

\begin{itemize}
\tightlist
\item
  Definition of the scope of the Medicaid Enterprise.
\item
  Description of the As-Is (current) operations in terms of business,
  architecture, and data.
\item
  Description of the drivers and enablers that propel and support
  transformation.
\item
  Description of the To-Be environment in terms of business,
  architecture, and data.
\item
  Description of operational scenarios with sequence of events and
  activities carried out by stakeholders and the State Medicaid
  Enterprise.
\item
  Description of the impacts on each stakeholder.
\item
  Description of a summary of the improvements to the State Medicaid
  Enterprise and stakeholders.
\end{itemize}

The goal of the COO is to project changes, transformations, and provide
visions of To-Be operations, new roles and data exchanges for
stakeholders. The MITA COO provides a common vision shared by CMS and
the States that preserves individual adaptations at the state level.

Part I, Chapter 2, Concept of Operations, provides more information on
the Medicaid Enterprise COO. Part I, Appendix A, Concept of Operations
Details, contains additional information.

\section[\textbf{MITA Maturity
Model}]{\texorpdfstring{\protect\hypertarget{page-10-1}{}{}\textbf{MITA
Maturity Model}}{MITA Maturity Model}}\label{mita-maturity-model-1}

\pandocbounded{\includegraphics[keepaspectratio]{_page_10_Picture_16.jpeg}}

The MMM originates from industries that use such models to illustrate
how a business can mature. The MMM adapts the industry model to the
Medicaid Enterprise by describing Medicaid Program goals and objectives
and the maturation of the MITA technical principles. The

transformation through each of the five (5) levels represents
significant business capabilities advances over the previous period.

The MMM describes the five (5) levels of maturity and the measurable
qualities that each level demonstrates. The general description is at a
high enough level to apply to most aspects of State Medicaid Enterprise
operations. For example, the MMM defines, at Level 1, the business area
or process is in compliance with current regulations. At Level 2, the
process matures because of pressures for cost containment and
availability of newer tools. At Level 3, noticeable improvement occurs
in the standardization and sharing of information

\pandocbounded{\includegraphics[keepaspectratio]{_page_10_Picture_20.jpeg}}

and processes among multiple entities, including the beneficiary. At
Level 4, instant availability of clinical information increases the
transformation. By Level 5, States and local agencies have become
interoperable across the United States.

The MMM is the point of reference for the BCM. The BCM aligns with the
MMM to maintain consistency of definition. Part I, Chapter 3, Maturity
Model, presents details of the MMM, and Part I, Appendix B, Maturity
Model Details, contains the complete detailed text.

\subsection[\textbf{Business Process
Model}]{\texorpdfstring{\protect\hypertarget{page-11-0}{}{}\textbf{Business
Process Model}}{Business Process Model}}\label{business-process-model-1}

\pandocbounded{\includegraphics[keepaspectratio]{_page_11_Figure_5.jpeg}}

The BPM is a collection of common business processes for the operation
of Medicaid Programs. A template describes those processes, including
current and near-term operations as defined in Level 3 of the BCM. The
MITA Framework BPM

derives from multiple sources that create a common model that reflects
most State Medicaid Enterprises -- notable sources include the S-TAG
\emph{Redesign of the Medicaid Management Information System (MMIS),}
and the CMS MHCCM, that consolidates business processes from a dozen
States.

States should develop business workflows for the different business
functions of the state to advance the alignment of the state's
capability maturity with the MMM. These business workflows should align
to any provided by CMS in support of Medicaid and Exchange business
operations and requirements. States should work to streamline and
standardize these operational approaches and business workflows to
minimize customization demands on technology solutions and optimize
business outcomes.

There are those business processes that all States perform (e.g., Enroll
Provider) and those that are voluntary and depend on implementation of
special programs within a state (e.g., pay Managed Care Organization
capitation or enrollment of member in a special program). The BPM
defines common business practices across all State Medicaid Enterprises.
The MITA Framework BPM offers a hierarchy of Tier 1 business areas, Tier
2 business categories and Tier 3 business processes. The MITA Framework
contains ten (10) business areas divided into twenty-one (21) business
categories with eighty (80) individual business processes. See Part I,
Appendix C, Business Process Model Details.

The BPM provides a Business Process Template (BPT) for describing each
business process. The BPT provides a summary of the business process,
trigger event and result, activity steps, data requirements, predecessor
and successor processes, failure points, and other elements. The NMEH
workgroups review business processes, and they stand to benefit from
ongoing review by state workgroups. See Part I, Chapter 4, Business
Process Model, for a detailed presentation of the BPM and Part I,
Appendix C, Business Process Model Details, for the complete set of
business area definitions and business process descriptions.

\section[\textbf{Business Capability
Matrix}]{\texorpdfstring{\protect\hypertarget{page-11-1}{}{}\textbf{Business
Capability
Matrix}}{Business Capability Matrix}}\label{business-capability-matrix-1}

\pandocbounded{\includegraphics[keepaspectratio]{_page_11_Picture_12.jpeg}}

\pandocbounded{\includegraphics[keepaspectratio]{_page_11_Picture_13.jpeg}}

Applying the MMM to each business process yields the BCM that shows how
the business process matures. The BCM defines six (6) business
capabilities with five (5) levels of maturity to each business process.
The BCM assigns capabilities to an individual business process rather
than to SMA operations taken as a whole. In reality, no SMA is ``all
Level 1'' or ``all Level 2,'' but rather having

a blend of different levels of capability. An example of the
relationship among the business process, the MMM, and the BCM is shown
in \textbf{\hyperref[page-12-0]{Table} 1-3.}

Part I, Chapter 5, Business Capability Matrix, presents more information
on the BCM and Part I, Appendix D, Business Capability Matrix Details,
lists the capabilities defined for business processes contained in MITA
Framework.

\subsection{\texorpdfstring{\textbf{Table 1-3. Business Process Example:
Authorize
Service}}{Table 1-3. Business Process Example: Authorize Service}}\label{table-1-3.-business-process-example-authorize-service-1}

\phantomsection\label{page-12-0}{}

\begin{longtable}[]{@{}
  >{\raggedright\arraybackslash}p{(\linewidth - 6\tabcolsep) * \real{0.0603}}
  >{\raggedright\arraybackslash}p{(\linewidth - 6\tabcolsep) * \real{0.2496}}
  >{\raggedright\arraybackslash}p{(\linewidth - 6\tabcolsep) * \real{0.6868}}
  >{\raggedright\arraybackslash}p{(\linewidth - 6\tabcolsep) * \real{0.0034}}@{}}
\toprule\noalign{}
\begin{minipage}[b]{\linewidth}\raggedright
Authorize Service Business Process
\end{minipage} & \begin{minipage}[b]{\linewidth}\raggedright
\end{minipage} & \begin{minipage}[b]{\linewidth}\raggedright
\end{minipage} & \begin{minipage}[b]{\linewidth}\raggedright
\end{minipage} \\
\midrule\noalign{}
\endhead
\bottomrule\noalign{}
\endlastfoot
LevelNo. & MITAMaturityModelDefinition & BusinessCapability & \\
1 &
Complieswithregulations;mostlymanualactivities;delaysincommunicatingresults.
&
Receiptofandresponsetorequestsareprimarilyviapaper,fax,andphone;applypolicyguidelinesmanually;complieswithregulationsonturnaroundtimeandaccuracy.
& \\
2 &
Improvementsspearheadedbycostmanagementgoals;improvementsmadeinspeedofcommunicationandresponse.
&
Authorizationofservicegivengreaterpriorityasacost-managementtool;improvementsmadeincommunications;receiptofandresponsestorequestsmadeviaportal;adopt
HIPAAstandards. & \\
3 &
Informationandservicessharedwithotheragenciesandbeneficiary;streamlinedprocess;improvedresults.
&
Solutionsbecomereusableandsharablebecauseofadoptionofstandardsbystateagenciesanddata-sharingagreementstocollaborateonauthorizationofservices.
& \\
4 &
Incorporatesclinicalinformationintotheprocesstofurtherimproveresults. &
Directaccessbytheauthorizingagencytoaccess to
clinicalinformation;automationofrequests;render
decisionsbypayerautomaticallyasprovider
updatesbeneficiary'selectronichealthrecord;improve
accuracybecauseprovider basesdecisionsonclinicalevidence;limits
manualinterventiontoexceptions. & \\
5 &
Demonstrateswidespreadinteroperabilitytoachievemaximumimprovementsenvisionedatthistime.
&
Directaccessbytheauthorizingagencytoclinicalandadministrativeinformationanywhereinthecountrytoconfirmordenytheauthorizationforaservice.
& \\
\end{longtable}

\pandocbounded{\includegraphics[keepaspectratio]{_page_12_Picture_6.jpeg}}

\chapter[\textbf{Business Architecture Component
Relationships}]{\texorpdfstring{\protect\hypertarget{page-13-0}{}{}\textbf{Business
Architecture Component
Relationships}}{Business Architecture Component Relationships}}\label{business-architecture-component-relationships-1}

The four (4) components of the BA are interrelated:

\begin{itemize}
\tightlist
\item
  The COO serves as a model to frame a vision for Medicaid Program
  health care outcomes and operational efficiencies. It establishes the
  To-Be environment that becomes the goal of the Medicaid Enterprise
  transformation. The COO provides the vision for the MMM. It also
  supplies an overview for the BPM.
\item
  The MMM uses a common industry approach to describe the differences
  between five (5) levels of progressive maturity, ranging from As-Is
  operations to the To-Be environment. The MMM is the point of reference
  used by the BCM to describe the levels of maturity for a business
  process.
\item
  The BPM describes As-Is (i.e., current) Medicaid operations as defined
  for Level 3 of the BCM.
\item
  The BCM uses the five (5) levels of maturity described in the MMM and
  the To-Be environment defined in the COO to create definitions for
  business capabilities at five (5) levels of maturity for each business
  process.
\end{itemize}

\pandocbounded{\includegraphics[keepaspectratio]{_page_13_Figure_8.jpeg}}

\pandocbounded{\includegraphics[keepaspectratio]{_page_13_Figure_9.jpeg}}

\subsection{\texorpdfstring{\textbf{Figure 1-3. Relationship Among the
Components of the Business
Architecture}}{Figure 1-3. Relationship Among the Components of the Business Architecture}}\label{figure-1-3.-relationship-among-the-components-of-the-business-architecture-1}

\phantomsection\label{page-13-1}{}\pandocbounded{\includegraphics[keepaspectratio]{_page_13_Picture_11.jpeg}}

\chapter[\textbf{Connection Between
Architectures}]{\texorpdfstring{\protect\hypertarget{page-14-0}{}{}\textbf{Connection
Between
Architectures}}{Connection Between Architectures}}\label{connection-between-architectures-1}

The MITA Framework consists of three (3) interrelated BA, IA, and TA
components that work together to define a business-driven enterprise
transformation. The BA describes the business process activities along
with data input, data output, and required shared data. The IA provides
the bridge between the business need of information and the technical
solution data. The TA describes the technology enablers associated with
the business capabilities and their varied levels of maturity.

\textbf{\hyperref[page-14-1]{Figure 1-4}} illustrates how BA, IA, and TA
components interrelate. This is a high-level view of the primary
components within each architecture. Front Matter, Chapter 6,
Introduction to the MITA Framework, presents a detailed discussion on
the inter-relationship of all three (3) architectures. The BA
categorizes the business processes as business capabilities and assigned
a level of MITA maturity. Based on the level of maturity, the IA defines
the Conceptual Data Model (CDM) and Logical Data Model (LDM) with
necessary data attributes for the design of technical capabilities. The
TA defines the resulting business services and technical services for
the To-Be environment of the State Medicaid Enterprise.

\pandocbounded{\includegraphics[keepaspectratio]{_page_14_Figure_5.jpeg}}

\textbf{Figure 1-4. Relationships Among Components of the BA, IA, and
TA}

\phantomsection\label{page-14-1}{}The BA does not present specific
technical solutions or detailed data requirements. Some of its
components, however, point to specific companion components in the IA
and TA sections of MITA Framework (Parts II and III, respectively).
\textbf{\hyperref[page-15-0]{Table} 1-4} describes the name of the BA
Component and its relationship to the other architecture component as
well as its MITA Framework 3.0 documented location.

\pandocbounded{\includegraphics[keepaspectratio]{_page_14_Picture_8.jpeg}}

\subsubsection{\texorpdfstring{\textbf{Table 1-4. Component
Relationships of the BA, IA, and
TA}}{Table 1-4. Component Relationships of the BA, IA, and TA}}\label{table-1-4.-component-relationships-of-the-ba-ia-and-ta-1}

\phantomsection\label{page-15-0}{}

\begin{longtable}[]{@{}
  >{\raggedright\arraybackslash}p{(\linewidth - 8\tabcolsep) * \real{0.3900}}
  >{\raggedright\arraybackslash}p{(\linewidth - 8\tabcolsep) * \real{0.1876}}
  >{\raggedright\arraybackslash}p{(\linewidth - 8\tabcolsep) * \real{0.4165}}
  >{\raggedright\arraybackslash}p{(\linewidth - 8\tabcolsep) * \real{0.0030}}
  >{\raggedright\arraybackslash}p{(\linewidth - 8\tabcolsep) * \real{0.0030}}@{}}
\toprule\noalign{}
\begin{minipage}[b]{\linewidth}\raggedright
BA, IA, and TA Component Relationships
\end{minipage} & \begin{minipage}[b]{\linewidth}\raggedright
\end{minipage} & \begin{minipage}[b]{\linewidth}\raggedright
\end{minipage} & \begin{minipage}[b]{\linewidth}\raggedright
\end{minipage} & \begin{minipage}[b]{\linewidth}\raggedright
\end{minipage} \\
\midrule\noalign{}
\endhead
\bottomrule\noalign{}
\endlastfoot
BusinessArchitectureComponent & OtherArchitectureComponent &
Relationship & & \\
COO--DataExchanges & IA(PartII)--Allchapters &
IAchaptersprovidedetailsregardingthetransformationofdataandinformationidentifiedintheCOO.
& & \\
COO--Drivers andEnablers & TA(PartIII), Chapter2,Technical
ManagementStrategy;Chapter7,TechnicalCapabilityMatrix & Service-Oriented
Architectures(SOA)andTechnicalCapabilitiesareenablersreferencedintheCOO.
& & \\
BPM--TriggerEvent,Result,andSharedDataineachbusinessprocessdescribeingeneraltermsthekindofdatareceivedby,usedby,andresultingfromeachbusinessprocess
& IA(PartII),
Chapter2,DataManagementStrategy;Chapter3,ConceptualDataModel &
DataManagementStrategy(DMS)explainshowthedatasupportsthebusinessprocesses.TheCDMidentifiesgroupingsofinformationcommontoMedicaidbusinessareasandclustersofbusinessprocesses.
& & \\
BCM & IA(PartII), Chapter4,LogicalDataModel; Chapter 6Information
Capability Matrix & TheLDM
definesdataclassesandattributesneededtosupportdifferentlevelsofmaturity.AbusinessprocessdescribedataLevel3businesscapabilityrequiresLevel3dataattributes.
& & \\
BCM & TA(PartIII), Chapter7,TechnicalCapabilityMatrix &
TheBCMdrivestheTechnicalCapability Matrix (TCM).TAassociates technical
capabilitieswiththeBCMlevelwherespecifictechnologyisnecessarytosupportthebusinessprocess.
& & \\
BCM--Level3andabove & TA(PartIII), Chapter 2Technical
ManagementStrategy; Chapter3,BusinessServices &
Abusinessserviceisanimplementationofaspecificbusinessprocessataspecificlevelofcapability.TA
associatesbusiness servicesandSOAwithBCMLevel3andabove. & & \\
\end{longtable}

\pandocbounded{\includegraphics[keepaspectratio]{_page_15_Picture_4.jpeg}}

\chapter[\textbf{Using the Business
Architecture}]{\texorpdfstring{\protect\hypertarget{page-16-0}{}{}\textbf{Using
the Business
Architecture}}{Using the Business Architecture}}\label{using-the-business-architecture-1}

CMS requires States to align to and advance increasingly in MITA
maturity for business, architecture, and data. CMS expects States to use
the BA components to plan for improvements in the State Medicaid
Program, both in the delivery of services to providers and
beneficiaries, and in its internal operations and exchanges of
information with the other external stakeholders. BA provides the COO
and the MMM as background material. States and vendors use the BPM and
the BCM tools. \textbf{\hyperref[page-16-1]{Table} 1-5} summarizes how
stakeholders use the BA.

\phantomsection\label{page-16-1}{}

\begin{longtable}[]{@{}
  >{\raggedright\arraybackslash}p{(\linewidth - 4\tabcolsep) * \real{0.0861}}
  >{\raggedright\arraybackslash}p{(\linewidth - 4\tabcolsep) * \real{0.9104}}
  >{\raggedright\arraybackslash}p{(\linewidth - 4\tabcolsep) * \real{0.0035}}@{}}
\toprule\noalign{}
\begin{minipage}[b]{\linewidth}\raggedright
Stakeholder Useof the Business Architecture
\end{minipage} & \begin{minipage}[b]{\linewidth}\raggedright
\end{minipage} & \begin{minipage}[b]{\linewidth}\raggedright
\end{minipage} \\
\midrule\noalign{}
\endhead
\bottomrule\noalign{}
\endlastfoot
Stakeholder & HowStakeholders Use BA & \\
SMA & The
SMAmapstheiroperationstotheBPMandthenassessesthelevelofmaturityusingtheBCM.Whenthe
SMArequires
informationtechnologyupgradestosupportprogramimprovement,theSMAusestheSS-Atoshowhowit
will use
theenhancedfundingtoachieveaspecificresult(e.g.,movingfromLevel1or2toLevel3).
& \\
CMS &
CMSprovidesleadershipinestablishingtheMITAguidelinesandpromotingthemamongStates.ThroughthereleaseoftheMITAFramework,specialworkshopswithStates,Medicaidconferencematerial,andworkingwithearlyadopterStates,CMSprovides
guidance and principles to achieve the Medicaidvision. & \\
Vendors &
ThevendorcommunityusestheMITAFrameworkasareferenceinplanningtheirresearchanddevelopmentactivities.TheyusetheBA,inparticular,todeterminethematurityleveloffunctionssupportedbytheirsystems.Theyhaveacommonunderstandingofthe
CMS directionfor the Medicaid
Program,andtheycanshowhowtheirproductssupportMITAcapabilities. & \\
Providers & Providersplayanactiveroleintheexchangeofinformationwiththe
SMA.
TheycanlookattheSMABAtounderstandwhatdirectiontheSMAistakingandtokeepthisinmindastheyinvestininformation
technologyupgradesandreengineertheirpractices.Insomecases,the
SMAinvolveprovidersdirectlyinplanningaMedicaidProgramtransformation.
& \\
Beneficiaries & The BA supports the SMAperson-centric outreach,
eligibility and enrollmentactivities across the health and human
services spectrum.Beneficiariesandconsumergroupsare
abletolookattheSMABAandidentify thebenefits.AtLevel3business capability
maturity,beneficiariesareparticipantsinselfdirectedhealthcaredecisions.
& \\
Legislators,Governors &
StatesdeveloppresentationsbasedontheBAtoshowthegovernorandlegislatorswhatgoalsCMSisestablishingforStatesthatrequestenhanced
& \\
\end{longtable}

\subsection{\texorpdfstring{\textbf{Table 1-5. Stakeholder Use of the
Business
Architecture}}{Table 1-5. Stakeholder Use of the Business Architecture}}\label{table-1-5.-stakeholder-use-of-the-business-architecture-1}

\pandocbounded{\includegraphics[keepaspectratio]{_page_16_Picture_6.jpeg}}

\begin{longtable}[]{@{}
  >{\raggedright\arraybackslash}p{(\linewidth - 6\tabcolsep) * \real{0.1842}}
  >{\raggedright\arraybackslash}p{(\linewidth - 6\tabcolsep) * \real{0.8008}}
  >{\raggedright\arraybackslash}p{(\linewidth - 6\tabcolsep) * \real{0.0075}}
  >{\raggedright\arraybackslash}p{(\linewidth - 6\tabcolsep) * \real{0.0075}}@{}}
\toprule\noalign{}
\begin{minipage}[b]{\linewidth}\raggedright
Stakeholder Useof the Business Architecture
\end{minipage} & \begin{minipage}[b]{\linewidth}\raggedright
\end{minipage} & \begin{minipage}[b]{\linewidth}\raggedright
\end{minipage} & \begin{minipage}[b]{\linewidth}\raggedright
\end{minipage} \\
\midrule\noalign{}
\endhead
\bottomrule\noalign{}
\endlastfoot
Stakeholder & HowStakeholders Use BA & & \\
& funding. & & \\
OtherPayersandOtherAgencies & The MITA team invites other
payersandotheragenciestoreviewtheMITAFramework,especiallytheBA,tolearnabouttheMedicaidEnterprisetransformation.
& & \\
\end{longtable}

In general, MITA supports stakeholder roles and access to information,
technology that eliminates most manual activities, and the
transformation of the Medicaid business with the assistance of the CMS,
the SMA, providers, and beneficiaries. In addition, MITA supports
providers with instant access to patient records no matter what their
location is, patients can view their Personal Health Information (PHI)
and make informed decisions regarding treatment, and payers can view
clinical records nationally to expedite decisions on prior authorization
and payment.

\section[\textbf{Next Steps in Developing the Business
Architecture}]{\texorpdfstring{\protect\hypertarget{page-17-0}{}{}\textbf{Next
Steps in Developing the Business
Architecture}}{Next Steps in Developing the Business Architecture}}\label{next-steps-in-developing-the-business-architecture-1}

The MITA Framework delivers the starter kit for a controlled State
Medicaid Enterprise transformation. MITA will continue to evolve over
time. The business process defines the input and output of information
but not the details of the process; however the business community will
still decide the requirements for standardized triggers and results. The
CMS MITA team continues to support SMA efforts by serving as a conduit
for improvements to MITA models that all States and vendors can access.

The MITA Framework and the BA are ever evolving so that the SMA can
continuously improve the way they deliver services to beneficiaries and
providers, account for outcomes, reward participants based on
performance, and respond dynamically to requests for information.

\pandocbounded{\includegraphics[keepaspectratio]{_page_17_Picture_7.jpeg}}

\pandocbounded{\includegraphics[keepaspectratio]{_page_17_Picture_9.jpeg}}

\chapter{Business Process Model}\label{business-process-model-2}

\part{Information Architecture}

\chapter{Business Architecture
Introduction}\label{business-architecture-introduction-1}

\phantomsection\label{page-0-1}{}\phantomsection\label{page-0-0}{}\textbf{Part
I -- BUSINESS ARCHITECTURE Chapter 1 -- INTRODUCTION}

\pandocbounded{\includegraphics[keepaspectratio]{_page_0_Picture_1.jpeg}}

\pandocbounded{\includegraphics[keepaspectratio]{_page_0_Picture_2.jpeg}}

\pandocbounded{\includegraphics[keepaspectratio]{_page_0_Picture_3.jpeg}}

\pandocbounded{\includegraphics[keepaspectratio]{_page_0_Picture_4.jpeg}}

\subsubsection{\texorpdfstring{\textbf{Table of
Contents}}{Table of Contents}}\label{table-of-contents-1}

\begin{longtable}[]{@{}ll@{}}
\toprule\noalign{}
PART I--BUSINESS ARCHITECTURE & 1 \\
\midrule\noalign{}
\endhead
\bottomrule\noalign{}
\endlastfoot
Chapter 1 --Introduction & 1 \\
Introduction & 3 \\
Purpose & 3 \\
Scope & 4 \\
Background & 4 \\
Funding Requirements & 5 \\
BA Seven Standards and Conditions & 6 \\
Business Architecture Components & 7 \\
The Concept of Operations11 & \\
MITA Maturity Model11 & \\
Business Process Model12 & \\
12Business Capability Matrix & \\
14Business Architecture Component Relationships & \\
Connection Between Architectures15 & \\
17Using the Business Architecture & \\
Next Steps in Developing the Business Architecture18 & \\
& \\
\end{longtable}

\subsubsection{\texorpdfstring{\textbf{List of
Figures}}{List of Figures}}\label{list-of-figures-1}

\begin{longtable}[]{@{}
  >{\raggedright\arraybackslash}p{(\linewidth - 2\tabcolsep) * \real{0.9630}}
  >{\raggedright\arraybackslash}p{(\linewidth - 2\tabcolsep) * \real{0.0370}}@{}}
\toprule\noalign{}
\begin{minipage}[b]{\linewidth}\raggedright
Figure 1-1. MITA Framework Relationship Diagram
\end{minipage} & \begin{minipage}[b]{\linewidth}\raggedright
3
\end{minipage} \\
\midrule\noalign{}
\endhead
\bottomrule\noalign{}
\endlastfoot
Figure 1-2. BA in the Context of the MITA Framework & 8 \\
Figure 1-3. Relationship Among the Components of the Business
Architecture14 & \\
IA, and TA15Figure 1-4. Relationships Among Components of the BA, & \\
\end{longtable}

\subsubsection{\texorpdfstring{\textbf{List of
Tables}}{List of Tables}}\label{list-of-tables-1}

\begin{longtable}[]{@{}
  >{\raggedright\arraybackslash}p{(\linewidth - 2\tabcolsep) * \real{0.9577}}
  >{\raggedright\arraybackslash}p{(\linewidth - 2\tabcolsep) * \real{0.0423}}@{}}
\toprule\noalign{}
\begin{minipage}[b]{\linewidth}\raggedright
Table 1-1. Correlation of Seven Standards and Conditions with MITA
\end{minipage} & \begin{minipage}[b]{\linewidth}\raggedright
6
\end{minipage} \\
\midrule\noalign{}
\endhead
\bottomrule\noalign{}
\endlastfoot
Table 1-2. The Four Components of the Business Architecture. & 9 \\
13Table 1-3. Business Process Example: Authorize Service & \\
16Table 1-4. Component Relationships ofthe BA, IA, and TA & \\
17Table 1-5. Stakeholder Use of the Business Architecture & \\
\end{longtable}

\pandocbounded{\includegraphics[keepaspectratio]{_page_1_Picture_8.jpeg}}

\chapter[\textbf{Introduction}]{\texorpdfstring{\protect\hypertarget{page-2-0}{}{}\textbf{Introduction}}{Introduction}}\label{introduction-2}

The Medicaid IT Architecture (MITA) Framework contains three (3)
interrelated architectures: Business Architecture (BA), Information
Architecture (IA), and Technical Architecture (TA) shown in
\textbf{\hyperref[page-2-2]{Figure 1-1}}. The business capabilities from
BA define the data strategy of IA and design the business and technical
services of TA. MITA uses all three (3) architectures to develop a
business-driven enterprise to provide consistency across the State
Medicaid Enterprise.

\pandocbounded{\includegraphics[keepaspectratio]{_page_2_Figure_4.jpeg}}

\pandocbounded{\includegraphics[keepaspectratio]{_page_2_Figure_5.jpeg}}

\pandocbounded{\includegraphics[keepaspectratio]{_page_2_Figure_6.jpeg}}

\phantomsection\label{page-2-2}{}The topics covered in this chapter
include:

\begin{itemize}
\tightlist
\item
  BA Seven Standards and Conditions
\item
  Business Architecture Components
\item
  Business Architecture Component Relationships
\item
  Connection Between Architectures
\item
  Using the Business Architecture
\item
  Next Steps in Developing the Business Architecture
\end{itemize}

\section[\textbf{Purpose}]{\texorpdfstring{\protect\hypertarget{page-2-1}{}{}\textbf{Purpose}}{Purpose}}\label{purpose-7}

In keeping with the guiding principle that MITA represents a
business-driven enterprise transformation, the BA is the starting point
of the MITA Framework. The BA describes the needs and goals of the
individual State Medicaid Enterprise, and presents a collective vision
of the future.

The BA will accomplish the following:

\begin{itemize}
\tightlist
\item
  Establish a generic business framework for all States while
  recognizing their differences.
\item
  Describe how each state Medicaid Program can mature over a given
  period with the help of stakeholders, leadership, enabling
  legislation, and technology.
\item
  Provide a baseline for the State Medicaid Agency (SMA) to assess their
  current business capabilities and measure progress toward improved
  capabilities.
\end{itemize}

\pandocbounded{\includegraphics[keepaspectratio]{_page_2_Picture_20.jpeg}}

\subsection[\textbf{Scope}]{\texorpdfstring{\protect\hypertarget{page-3-0}{}{}\textbf{Scope}}{Scope}}\label{scope-2}

The BA focuses on the State Medicaid Enterprise that centers on the
Medicaid environment including leveraged systems and interconnections
among Medicaid stakeholders, providers, beneficiaries, insurance
affordability programs (e.g., CHIP, tax credits, Basic Health Program),
Health Insurance Exchange (HIX), Health Information Exchange (HIE),
other state and local agencies, other payers, Centers for Medicare \&
Medicaid Services (CMS), and other federal agencies. The MITA context
defines the Medicaid Enterprise as:

\begin{itemize}
\tightlist
\item
  The domain where federal matching funds apply.
\item
  The interfaces and bridges among Medicaid stakeholders, including
  providers, beneficiaries, other state and local agencies, other
  payers, CMS, and other federal agencies.
\item
  The sphere of influence touched by MITA (e.g., national and federal
  initiatives such as the Nationwide Health Information Network
  {[}NwHIN{]}). (See Front Matter, Chapter 6, Overview of the MITA
  Initiative, for a discussion of the Medicaid Enterprise.)
\end{itemize}

\emph{Enterprise can have other meanings. For instance, Enterprise
Architecture (EA) defines an enterprise-wide integrated set of
components that incorporates strategic business thinking, information
assets, and the technical infrastructure of an enterprise to promote
information sharing across agency and organizational boundaries.}

The BA acknowledges technology as one of several enablers that are
important to growth and transformation, but it does not introduce
technical implementations or solutions into the BA components. All
technical references are found in Part III, Technical Architecture.

\subsection[\textbf{Background}]{\texorpdfstring{\protect\hypertarget{page-3-1}{}{}\textbf{Background}}{Background}}\label{background-2}

States, territories, and the District of Columbia (hereinafter referred
to as States) are responsible for their individual State Medicaid
Enterprise, and all entities are different in important ways.
Differences include:

\begin{itemize}
\tightlist
\item
  Organizational structure, covered programs, and lines of business
\item
  Business rules, policies, and procedures affecting stakeholders
\item
  Relationships with other state and local agencies
\item
  Revenue sources
\item
  Location of business units
\item
  Workflow
\item
  Range of outsourcing
\item
  Technical solutions
\end{itemize}

\pandocbounded{\includegraphics[keepaspectratio]{_page_3_Picture_19.jpeg}}

These entities also differ in their concept of an enterprise, the roles
and responsibilities of one or more Chief Information Officers (CIO),
adoption of data and technical standards, and the use of legacy versus
state-of-the-art applications.

Given these differences, it is not possible or desirable, in the context
of the MITA Framework, to develop a standalone business and technical
model for each individual Medicaid Enterprise. Instead, MITA establishes
a national framework of common processes and enabling technologies to
support improved program administration in all States.

The BA focuses on areas of common ground (e.g., that all States will
enroll providers and pay for services rendered to eligible beneficiaries
and that all States seek to improve health care outcomes and improve
administrative processes).

There is no ready-made methodology for building the MITA Framework to
accommodate the business needs and transformation strategies of the
States. To meet the special needs of MITA, the components included in
the BA draw upon methodologies commonly in use today across industries
as diverse as financial, transportation, and defense. The MITA team
designed templates and models to help States identify and prioritize
their specific business needs.

The BA section of the MITA Framework shows how MITA incorporates
business-driven design to accomplish the following:

\begin{itemize}
\tightlist
\item
  Support state needs.

  \begin{itemize}
  \tightlist
  \item
    \textbf{o} Align with state strategic goals.
  \item
    \textbf{o} Align with state or Medicaid Agency enterprise
    architecture.
  \end{itemize}
\item
  Support the CMS and common state goals.

  \begin{itemize}
  \tightlist
  \item
    \textbf{o} Align state approaches with MITA.
  \item
    \textbf{o} Accommodate multi-state collaborative initiatives.
  \end{itemize}
\item
  Support national goals through alignment with national initiatives,
  such as the Office of the National Coordinator for Health Information
  Technology (ONC) and federal guidelines (e.g., Federal Health
  Architecture (FHA), the Federal Enterprise Architecture Framework
  (FEAF), and national/international data standards).
\end{itemize}

\section[\textbf{Funding
Requirements}]{\texorpdfstring{\protect\hypertarget{page-4-0}{}{}\textbf{Funding
Requirements}}{Funding Requirements}}\label{funding-requirements-2}

The Health and Human Services (HHS) CMS 42 CFR Part 433 Medicaid
Program; Federal Funding for Medicaid Eligibility Determination and
Enrollment Activities modifies Medicaid regulations for Mechanized
Claims Processing and Information Retrieval Systems effective April 19,
2011. The Medicaid Management Information System (MMIS) is a mechanized
claims processing and information retrieval system used by the States
for Title XIX of the Social Security Act (The Act); therefore, the
guidance set forth in CMS 42 CFR Part 433 applies to the MMIS as well as
the Medicaid eligibility determination and enrollment activities as set
forth in the Affordable Care Act of 2010.

CMS expects States to meet the standards and conditions specified in
§433.112(b)(10) through §433.112(b)(16). The standards and conditions
are descriptive in nature; however, CMS recognizes that in order for the
States to meet these standards and conditions it is necessary to provide
additional guidance that clearly articulates its criteria for meeting
them

\pandocbounded{\includegraphics[keepaspectratio]{_page_4_Picture_17.jpeg}}

in terms of timeliness, accuracy, efficiency, integrity, and performance
standards for mechanized claims processing. In response to this need,
additional guidance materials include:

\begin{itemize}
\tightlist
\item
  Enhanced Funding Requirements: Seven Conditions and Standards (a.k.a.
  Seven Standards and Conditions)
\item
  Guidance for Exchange and Medicaid Information Technology (IT) Systems
  (a.k.a. IT Guidance)
\end{itemize}

CMS will continue to refine, update and expand this guidance in the
future, based on feedback from stakeholders and with experience over
time.

\chapter[\textbf{BA Seven Standards and
Conditions}]{\texorpdfstring{\protect\hypertarget{page-5-0}{}{}\textbf{BA
Seven Standards and
Conditions}}{BA Seven Standards and Conditions}}\label{ba-seven-standards-and-conditions-2}

The MITA team evaluated and incorporated the 42 CFR Part 433 Medicaid
Program; Federal Funding for Medicaid Eligibility Determination and
Enrollment Activities in the BA for purposes of guiding the MITA
stakeholders to apply the guidance to the Medicaid Enterprise.

Each of the architectures aligns with the Seven Standards and
Conditions. By utilizing best practices, industry standards, and
technology advancements, the processes, and planning guidelines that
build the MITA framework provide a cohesive method for meeting Medicaid
objectives.

\textbf{\hyperref[page-5-1]{Table 1-1}} depicts the impact of the Seven
Standards and Conditions on the MITA BA, IA, and TA.

\phantomsection\label{page-5-1}{}

\begin{longtable}[]{@{}
  >{\raggedright\arraybackslash}p{(\linewidth - 8\tabcolsep) * \real{0.4581}}
  >{\raggedright\arraybackslash}p{(\linewidth - 8\tabcolsep) * \real{0.1677}}
  >{\raggedright\arraybackslash}p{(\linewidth - 8\tabcolsep) * \real{0.1871}}
  >{\raggedright\arraybackslash}p{(\linewidth - 8\tabcolsep) * \real{0.1742}}
  >{\raggedright\arraybackslash}p{(\linewidth - 8\tabcolsep) * \real{0.0129}}@{}}
\toprule\noalign{}
\begin{minipage}[b]{\linewidth}\raggedright
Correlation of Seven Standards and Conditions with MITA Architectures
\end{minipage} & \begin{minipage}[b]{\linewidth}\raggedright
\end{minipage} & \begin{minipage}[b]{\linewidth}\raggedright
\end{minipage} & \begin{minipage}[b]{\linewidth}\raggedright
\end{minipage} & \begin{minipage}[b]{\linewidth}\raggedright
\end{minipage} \\
\midrule\noalign{}
\endhead
\bottomrule\noalign{}
\endlastfoot
Standards and Conditions & BusinessArchitecture &
InformationArchitecture & TechnicalArchitecture & \\
Modularity Standard & X & X & X & \\
MITA Condition & X & X & X & \\
Industry Standards Condition & X & X & X & \\
Leverage Condition & X & X & X & \\
Business Results Condition & X & X & X & \\
Reporting Condition & X & X & X & \\
Interoperability Condition & X & X & X & \\
\end{longtable}

\subsection{\texorpdfstring{\textbf{Table 1-1. Correlation of Seven
Standards and Conditions with
MITA}}{Table 1-1. Correlation of Seven Standards and Conditions with MITA}}\label{table-1-1.-correlation-of-seven-standards-and-conditions-with-mita-2}

\pandocbounded{\includegraphics[keepaspectratio]{_page_5_Picture_12.jpeg}}

The BA includes:

\begin{itemize}
\tightlist
\item
  \textbf{Modularity Standard} Uses a modular, flexible approach to
  systems development, including the use of open interfaces and exposed
  Application Programming Interfaces (API); the separation of business
  rules from core programming; and the availability of business rules in
  both human and machine-readable formats. The States commit to formal
  system development methodology and open, reusable system architecture.
\item
  \textbf{MITA Condition} States align to and advance increasingly in
  MITA maturity for business, architecture, and data.
\end{itemize}

\textbf{Industry Standards Condition} - Ensures alignment with, and
incorporation of, industry standards: the Health Insurance Portability
and Accountability Act of 1996 (HIPAA) security, privacy and transaction
standards; accessibility standards established under section 508 of the
Rehabilitation Act, or standards that provide greater accessibility for
individuals with disabilities, and compliance with Federal Civil Rights
laws; standards adopted by the Secretary under section 1104 of the
Affordable Care Act; and standards and protocols adopted by the
Secretary under section 1561 of the Affordable Care Act.

\begin{itemize}
\tightlist
\item
  \textbf{Leverage Condition} States solutions should promote sharing,
  leverage, and reuse of Medicaid technologies and systems within and
  among States.
\item
  \textbf{Business Results Condition} Systems should support accurate
  and timely processing of claims (including claims of eligibility),
  adjudications, and effective communications with providers,
  beneficiaries, and the public.
\item
  \textbf{Reporting Condition} Solutions should produce transaction
  data, reports, and performance information that contribute to program
  evaluation, continuous improvement in business operations, and
  transparency and accountability.
\item
  \textbf{Interoperability Condition} Systems must ensure seamless
  coordination and integration with the Exchange (whether run by the
  state or federal government), and allow interoperability with health
  information exchanges, public health agencies, human services
  programs, and community organizations providing outreach and
  enrollment assistance services.
\end{itemize}

\chapter[\textbf{Business Architecture
Components}]{\texorpdfstring{\protect\hypertarget{page-6-0}{}{}\textbf{Business
Architecture
Components}}{Business Architecture Components}}\label{business-architecture-components-2}

The BA is a conceptual construct that encompasses models, matrices, and
templates. These components derive from a variety of industry standards
because no single methodology exists that meets the scope of MITA. The
MITA Framework breaks new ground and is a model for other federal,
state, and local entities.

The MITA BA contains the following components:

\begin{itemize}
\tightlist
\item
  Concept of Operations
\item
  MITA Maturity Model
\item
  Business Process Model
\item
  Business Capability Matrix
\end{itemize}

\pandocbounded{\includegraphics[keepaspectratio]{_page_6_Picture_17.jpeg}}

These are living models that evolve with the MITA Framework life cycle.
The MITA team tailored the level of detail in each model to meet the
specific needs of the intended audience.
\textbf{\hyperref[page-7-0]{Figure 1-2}} provides an overview of the
components of the BA. See Part I, Chapters 2 through 5 for a more
detailed description for each of these components.

\pandocbounded{\includegraphics[keepaspectratio]{_page_7_Figure_3.jpeg}}

\textbf{Figure 1-2}. \textbf{BA in the Context of the MITA Framework}

\phantomsection\label{page-7-0}{}The MITA Framework focuses on the
common ground shared by various distinct State Medicaid Enterprises and
yet accommodates their differences. The BA consists of four (4)
components that are summarized in \textbf{\hyperref[page-8-0]{Table}
1-2}. The BA is a composite of interrelated models and templates.

\pandocbounded{\includegraphics[keepaspectratio]{_page_7_Picture_6.jpeg}}

\subsection{\texorpdfstring{\textbf{Table 1-2. The Four Components of
the Business
Architecture.}}{Table 1-2. The Four Components of the Business Architecture.}}\label{table-1-2.-the-four-components-of-the-business-architecture.-2}

\phantomsection\label{page-8-0}{}

\begin{longtable}[]{@{}
  >{\raggedright\arraybackslash}p{(\linewidth - 10\tabcolsep) * \real{0.0465}}
  >{\raggedright\arraybackslash}p{(\linewidth - 10\tabcolsep) * \real{0.2903}}
  >{\raggedright\arraybackslash}p{(\linewidth - 10\tabcolsep) * \real{0.3977}}
  >{\raggedright\arraybackslash}p{(\linewidth - 10\tabcolsep) * \real{0.2614}}
  >{\raggedright\arraybackslash}p{(\linewidth - 10\tabcolsep) * \real{0.0021}}
  >{\raggedright\arraybackslash}p{(\linewidth - 10\tabcolsep) * \real{0.0021}}@{}}
\toprule\noalign{}
\begin{minipage}[b]{\linewidth}\raggedright
Business Architecture Components
\end{minipage} & \begin{minipage}[b]{\linewidth}\raggedright
\end{minipage} & \begin{minipage}[b]{\linewidth}\raggedright
\end{minipage} & \begin{minipage}[b]{\linewidth}\raggedright
\end{minipage} & \begin{minipage}[b]{\linewidth}\raggedright
\end{minipage} & \begin{minipage}[b]{\linewidth}\raggedright
\end{minipage} \\
\midrule\noalign{}
\endhead
\bottomrule\noalign{}
\endlastfoot
Component & TypeofModel & Function & Relationship & & \\
ConceptofOperations(COO)COO &
TheCOOdescribescurrentoperations,avisionoftransformation,transformationstostakeholderrolesandinformationexchanges,andtheinfluenceofenablers(e.g.,newpolicy,legislation,technology).
&
EstablishesavisionfortransformationoftheStateMedicaidEnterprise.Linksenablerstotheimprovementsinbusinessprocesses.Showshowstakeholders'roleschange.Showshowprocessesanddatachange.FocusesonimprovementsintheSMAoperations.
&
EstablishesthetargetsandvisionthatotherBAcomponentswilladdress.ProvidesaplatformandgroundingfortheMMM
andtheBusinessCapabilityMatrix(BCM). & & \\
MITAMaturityModel(MMM) &
Subdividedintofive(5)levelsofprogressivematurity,theMMMillustrateshowto
transformgoals,objectives,andbusinesscapabilitiesprogress. &
ShowshowtomeetStateMedicaidEnterprisegoalsandobjectivesandhowtoimprove
businessareas.Providesbase,consistency,andmeasuresforspecifyingdetailedbusinesscapabilitiesastheymature.
& MMM providesstructure to theCOO vision tobuild the
BCM.Providesaframeworkandmodelforthebusinesscapabilities.MMM aligns
withthe SevenStandards andConditionsrequirements. & & \\
BusinessProcessModel(BPM) & The BPM is acollection ofcommon
businessprocesses for theoperation ofMedicaidPrograms. &
Providesamodelofmajorbusinessareasandsubareas.Providesdetailed &
OriginatesfromtheSystemsTechnicalAdvisoryGroup(S-TAG)redesignoftheMedicaid
& & \\
\end{longtable}

\pandocbounded{\includegraphics[keepaspectratio]{_page_8_Picture_4.jpeg}}

Part I, Chapter 1 - Page 9 February 2012 Version 3.0

\begin{longtable}[]{@{}
  >{\raggedright\arraybackslash}p{(\linewidth - 8\tabcolsep) * \real{0.0410}}
  >{\raggedright\arraybackslash}p{(\linewidth - 8\tabcolsep) * \real{0.3053}}
  >{\raggedright\arraybackslash}p{(\linewidth - 8\tabcolsep) * \real{0.2462}}
  >{\raggedright\arraybackslash}p{(\linewidth - 8\tabcolsep) * \real{0.4055}}
  >{\raggedright\arraybackslash}p{(\linewidth - 8\tabcolsep) * \real{0.0019}}@{}}
\toprule\noalign{}
\begin{minipage}[b]{\linewidth}\raggedright
Business Architecture Components
\end{minipage} & \begin{minipage}[b]{\linewidth}\raggedright
\end{minipage} & \begin{minipage}[b]{\linewidth}\raggedright
\end{minipage} & \begin{minipage}[b]{\linewidth}\raggedright
\end{minipage} & \begin{minipage}[b]{\linewidth}\raggedright
\end{minipage} \\
\midrule\noalign{}
\endhead
\bottomrule\noalign{}
\endlastfoot
Component & TypeofModel & Function & Relationship & \\
&
Atemplatecapturesthedescriptionofeachbusinessprocess.Thebusinessprocessescovercurrentandnear-termoperations.
&
definitionsofcommonbusinessprocesses.Describesbusinessprocessesusingacommonvocabulary.Renderssomebusinessprocessesobsoleteathigherlevelsofmaturity.
&
ManagementInformationSystem(MMIS)model,variousstatemodels,andtheMedicaidHIPAA-CompliantConceptModel(MHCCM)andfederalregulation.BusinessprocessesunderreviewbytheNationalMedicaidEDIHealthcare(NMEH)workgroups.ReviewandrefinementprocessundercontinualreviewbyStates.
& \\
BusinessCapabilityMatrix(BCM) &
Subdividedintofive(5)levelsofmaturity,theBCMappliestheMMMtotheBPMtoderivecapabilitiesforeachbusinessprocessateachmaturitylevel.TheBCMdescribeshowto
transformand improve abusinessprocess. &
Showshoweachbusinessprocesscanimprove.ProvidesconsistencyandamodelfortheSMAtouseinmeasuringtheirownlevelsofmaturityforeachbusinessprocess.
& The BCM definessix (6) businesscapabilitiesacross five (5)levels of
maturityfor each
businessprocess.AlignswiththeMMMforthedescriptionofthecharacteristicsofthematuritylevels.FormstheevaluationcriteriafortheStateSelfAssessment(SSA).
& \\
\end{longtable}

\pandocbounded{\includegraphics[keepaspectratio]{_page_9_Picture_3.jpeg}}

\section[\textbf{The Concept of
Operations}]{\texorpdfstring{\protect\hypertarget{page-10-0}{}{}\textbf{The
Concept of
Operations}}{The Concept of Operations}}\label{the-concept-of-operations-2}

\pandocbounded{\includegraphics[keepaspectratio]{_page_10_Picture_3.jpeg}}

The COO is a tool used to describe current business operations and to
develop a future transformation that meets the needs of stakeholders and
responds to enablers (e.g., new policy, legislation, and technology).
Other industries (e.g., the Department of Defense (DOD) or National
Aeronautics and Space Administration (NASA)) use the COO as a
strategic-planning device to capture the As-Is (i.e., current)
operations,

create the To-Be (i.e., future) environment, and level-set expectations
before engaging in major transformation projects. The COO provides a
structure to place information gathered from interviews with States and
visioning sessions conducted at MMIS conferences. The COO structure
provides key information including:

\begin{itemize}
\tightlist
\item
  Definition of the scope of the Medicaid Enterprise.
\item
  Description of the As-Is (current) operations in terms of business,
  architecture, and data.
\item
  Description of the drivers and enablers that propel and support
  transformation.
\item
  Description of the To-Be environment in terms of business,
  architecture, and data.
\item
  Description of operational scenarios with sequence of events and
  activities carried out by stakeholders and the State Medicaid
  Enterprise.
\item
  Description of the impacts on each stakeholder.
\item
  Description of a summary of the improvements to the State Medicaid
  Enterprise and stakeholders.
\end{itemize}

The goal of the COO is to project changes, transformations, and provide
visions of To-Be operations, new roles and data exchanges for
stakeholders. The MITA COO provides a common vision shared by CMS and
the States that preserves individual adaptations at the state level.

Part I, Chapter 2, Concept of Operations, provides more information on
the Medicaid Enterprise COO. Part I, Appendix A, Concept of Operations
Details, contains additional information.

\section[\textbf{MITA Maturity
Model}]{\texorpdfstring{\protect\hypertarget{page-10-1}{}{}\textbf{MITA
Maturity Model}}{MITA Maturity Model}}\label{mita-maturity-model-2}

\pandocbounded{\includegraphics[keepaspectratio]{_page_10_Picture_16.jpeg}}

The MMM originates from industries that use such models to illustrate
how a business can mature. The MMM adapts the industry model to the
Medicaid Enterprise by describing Medicaid Program goals and objectives
and the maturation of the MITA technical principles. The

transformation through each of the five (5) levels represents
significant business capabilities advances over the previous period.

The MMM describes the five (5) levels of maturity and the measurable
qualities that each level demonstrates. The general description is at a
high enough level to apply to most aspects of State Medicaid Enterprise
operations. For example, the MMM defines, at Level 1, the business area
or process is in compliance with current regulations. At Level 2, the
process matures because of pressures for cost containment and
availability of newer tools. At Level 3, noticeable improvement occurs
in the standardization and sharing of information

\pandocbounded{\includegraphics[keepaspectratio]{_page_10_Picture_20.jpeg}}

and processes among multiple entities, including the beneficiary. At
Level 4, instant availability of clinical information increases the
transformation. By Level 5, States and local agencies have become
interoperable across the United States.

The MMM is the point of reference for the BCM. The BCM aligns with the
MMM to maintain consistency of definition. Part I, Chapter 3, Maturity
Model, presents details of the MMM, and Part I, Appendix B, Maturity
Model Details, contains the complete detailed text.

\subsection[\textbf{Business Process
Model}]{\texorpdfstring{\protect\hypertarget{page-11-0}{}{}\textbf{Business
Process Model}}{Business Process Model}}\label{business-process-model-3}

\pandocbounded{\includegraphics[keepaspectratio]{_page_11_Figure_5.jpeg}}

The BPM is a collection of common business processes for the operation
of Medicaid Programs. A template describes those processes, including
current and near-term operations as defined in Level 3 of the BCM. The
MITA Framework BPM

derives from multiple sources that create a common model that reflects
most State Medicaid Enterprises -- notable sources include the S-TAG
\emph{Redesign of the Medicaid Management Information System (MMIS),}
and the CMS MHCCM, that consolidates business processes from a dozen
States.

States should develop business workflows for the different business
functions of the state to advance the alignment of the state's
capability maturity with the MMM. These business workflows should align
to any provided by CMS in support of Medicaid and Exchange business
operations and requirements. States should work to streamline and
standardize these operational approaches and business workflows to
minimize customization demands on technology solutions and optimize
business outcomes.

There are those business processes that all States perform (e.g., Enroll
Provider) and those that are voluntary and depend on implementation of
special programs within a state (e.g., pay Managed Care Organization
capitation or enrollment of member in a special program). The BPM
defines common business practices across all State Medicaid Enterprises.
The MITA Framework BPM offers a hierarchy of Tier 1 business areas, Tier
2 business categories and Tier 3 business processes. The MITA Framework
contains ten (10) business areas divided into twenty-one (21) business
categories with eighty (80) individual business processes. See Part I,
Appendix C, Business Process Model Details.

The BPM provides a Business Process Template (BPT) for describing each
business process. The BPT provides a summary of the business process,
trigger event and result, activity steps, data requirements, predecessor
and successor processes, failure points, and other elements. The NMEH
workgroups review business processes, and they stand to benefit from
ongoing review by state workgroups. See Part I, Chapter 4, Business
Process Model, for a detailed presentation of the BPM and Part I,
Appendix C, Business Process Model Details, for the complete set of
business area definitions and business process descriptions.

\section[\textbf{Business Capability
Matrix}]{\texorpdfstring{\protect\hypertarget{page-11-1}{}{}\textbf{Business
Capability
Matrix}}{Business Capability Matrix}}\label{business-capability-matrix-2}

\pandocbounded{\includegraphics[keepaspectratio]{_page_11_Picture_12.jpeg}}

\pandocbounded{\includegraphics[keepaspectratio]{_page_11_Picture_13.jpeg}}

Applying the MMM to each business process yields the BCM that shows how
the business process matures. The BCM defines six (6) business
capabilities with five (5) levels of maturity to each business process.
The BCM assigns capabilities to an individual business process rather
than to SMA operations taken as a whole. In reality, no SMA is ``all
Level 1'' or ``all Level 2,'' but rather having

a blend of different levels of capability. An example of the
relationship among the business process, the MMM, and the BCM is shown
in \textbf{\hyperref[page-12-0]{Table} 1-3.}

Part I, Chapter 5, Business Capability Matrix, presents more information
on the BCM and Part I, Appendix D, Business Capability Matrix Details,
lists the capabilities defined for business processes contained in MITA
Framework.

\subsection{\texorpdfstring{\textbf{Table 1-3. Business Process Example:
Authorize
Service}}{Table 1-3. Business Process Example: Authorize Service}}\label{table-1-3.-business-process-example-authorize-service-2}

\phantomsection\label{page-12-0}{}

\begin{longtable}[]{@{}
  >{\raggedright\arraybackslash}p{(\linewidth - 6\tabcolsep) * \real{0.0603}}
  >{\raggedright\arraybackslash}p{(\linewidth - 6\tabcolsep) * \real{0.2496}}
  >{\raggedright\arraybackslash}p{(\linewidth - 6\tabcolsep) * \real{0.6868}}
  >{\raggedright\arraybackslash}p{(\linewidth - 6\tabcolsep) * \real{0.0034}}@{}}
\toprule\noalign{}
\begin{minipage}[b]{\linewidth}\raggedright
Authorize Service Business Process
\end{minipage} & \begin{minipage}[b]{\linewidth}\raggedright
\end{minipage} & \begin{minipage}[b]{\linewidth}\raggedright
\end{minipage} & \begin{minipage}[b]{\linewidth}\raggedright
\end{minipage} \\
\midrule\noalign{}
\endhead
\bottomrule\noalign{}
\endlastfoot
LevelNo. & MITAMaturityModelDefinition & BusinessCapability & \\
1 &
Complieswithregulations;mostlymanualactivities;delaysincommunicatingresults.
&
Receiptofandresponsetorequestsareprimarilyviapaper,fax,andphone;applypolicyguidelinesmanually;complieswithregulationsonturnaroundtimeandaccuracy.
& \\
2 &
Improvementsspearheadedbycostmanagementgoals;improvementsmadeinspeedofcommunicationandresponse.
&
Authorizationofservicegivengreaterpriorityasacost-managementtool;improvementsmadeincommunications;receiptofandresponsestorequestsmadeviaportal;adopt
HIPAAstandards. & \\
3 &
Informationandservicessharedwithotheragenciesandbeneficiary;streamlinedprocess;improvedresults.
&
Solutionsbecomereusableandsharablebecauseofadoptionofstandardsbystateagenciesanddata-sharingagreementstocollaborateonauthorizationofservices.
& \\
4 &
Incorporatesclinicalinformationintotheprocesstofurtherimproveresults. &
Directaccessbytheauthorizingagencytoaccess to
clinicalinformation;automationofrequests;render
decisionsbypayerautomaticallyasprovider
updatesbeneficiary'selectronichealthrecord;improve
accuracybecauseprovider basesdecisionsonclinicalevidence;limits
manualinterventiontoexceptions. & \\
5 &
Demonstrateswidespreadinteroperabilitytoachievemaximumimprovementsenvisionedatthistime.
&
Directaccessbytheauthorizingagencytoclinicalandadministrativeinformationanywhereinthecountrytoconfirmordenytheauthorizationforaservice.
& \\
\end{longtable}

\pandocbounded{\includegraphics[keepaspectratio]{_page_12_Picture_6.jpeg}}

\chapter[\textbf{Business Architecture Component
Relationships}]{\texorpdfstring{\protect\hypertarget{page-13-0}{}{}\textbf{Business
Architecture Component
Relationships}}{Business Architecture Component Relationships}}\label{business-architecture-component-relationships-2}

The four (4) components of the BA are interrelated:

\begin{itemize}
\tightlist
\item
  The COO serves as a model to frame a vision for Medicaid Program
  health care outcomes and operational efficiencies. It establishes the
  To-Be environment that becomes the goal of the Medicaid Enterprise
  transformation. The COO provides the vision for the MMM. It also
  supplies an overview for the BPM.
\item
  The MMM uses a common industry approach to describe the differences
  between five (5) levels of progressive maturity, ranging from As-Is
  operations to the To-Be environment. The MMM is the point of reference
  used by the BCM to describe the levels of maturity for a business
  process.
\item
  The BPM describes As-Is (i.e., current) Medicaid operations as defined
  for Level 3 of the BCM.
\item
  The BCM uses the five (5) levels of maturity described in the MMM and
  the To-Be environment defined in the COO to create definitions for
  business capabilities at five (5) levels of maturity for each business
  process.
\end{itemize}

\pandocbounded{\includegraphics[keepaspectratio]{_page_13_Figure_8.jpeg}}

\pandocbounded{\includegraphics[keepaspectratio]{_page_13_Figure_9.jpeg}}

\subsection{\texorpdfstring{\textbf{Figure 1-3. Relationship Among the
Components of the Business
Architecture}}{Figure 1-3. Relationship Among the Components of the Business Architecture}}\label{figure-1-3.-relationship-among-the-components-of-the-business-architecture-2}

\phantomsection\label{page-13-1}{}\pandocbounded{\includegraphics[keepaspectratio]{_page_13_Picture_11.jpeg}}

\chapter[\textbf{Connection Between
Architectures}]{\texorpdfstring{\protect\hypertarget{page-14-0}{}{}\textbf{Connection
Between
Architectures}}{Connection Between Architectures}}\label{connection-between-architectures-2}

The MITA Framework consists of three (3) interrelated BA, IA, and TA
components that work together to define a business-driven enterprise
transformation. The BA describes the business process activities along
with data input, data output, and required shared data. The IA provides
the bridge between the business need of information and the technical
solution data. The TA describes the technology enablers associated with
the business capabilities and their varied levels of maturity.

\textbf{\hyperref[page-14-1]{Figure 1-4}} illustrates how BA, IA, and TA
components interrelate. This is a high-level view of the primary
components within each architecture. Front Matter, Chapter 6,
Introduction to the MITA Framework, presents a detailed discussion on
the inter-relationship of all three (3) architectures. The BA
categorizes the business processes as business capabilities and assigned
a level of MITA maturity. Based on the level of maturity, the IA defines
the Conceptual Data Model (CDM) and Logical Data Model (LDM) with
necessary data attributes for the design of technical capabilities. The
TA defines the resulting business services and technical services for
the To-Be environment of the State Medicaid Enterprise.

\pandocbounded{\includegraphics[keepaspectratio]{_page_14_Figure_5.jpeg}}

\textbf{Figure 1-4. Relationships Among Components of the BA, IA, and
TA}

\phantomsection\label{page-14-1}{}The BA does not present specific
technical solutions or detailed data requirements. Some of its
components, however, point to specific companion components in the IA
and TA sections of MITA Framework (Parts II and III, respectively).
\textbf{\hyperref[page-15-0]{Table} 1-4} describes the name of the BA
Component and its relationship to the other architecture component as
well as its MITA Framework 3.0 documented location.

\pandocbounded{\includegraphics[keepaspectratio]{_page_14_Picture_8.jpeg}}

\subsubsection{\texorpdfstring{\textbf{Table 1-4. Component
Relationships of the BA, IA, and
TA}}{Table 1-4. Component Relationships of the BA, IA, and TA}}\label{table-1-4.-component-relationships-of-the-ba-ia-and-ta-2}

\phantomsection\label{page-15-0}{}

\begin{longtable}[]{@{}
  >{\raggedright\arraybackslash}p{(\linewidth - 8\tabcolsep) * \real{0.3900}}
  >{\raggedright\arraybackslash}p{(\linewidth - 8\tabcolsep) * \real{0.1876}}
  >{\raggedright\arraybackslash}p{(\linewidth - 8\tabcolsep) * \real{0.4165}}
  >{\raggedright\arraybackslash}p{(\linewidth - 8\tabcolsep) * \real{0.0030}}
  >{\raggedright\arraybackslash}p{(\linewidth - 8\tabcolsep) * \real{0.0030}}@{}}
\toprule\noalign{}
\begin{minipage}[b]{\linewidth}\raggedright
BA, IA, and TA Component Relationships
\end{minipage} & \begin{minipage}[b]{\linewidth}\raggedright
\end{minipage} & \begin{minipage}[b]{\linewidth}\raggedright
\end{minipage} & \begin{minipage}[b]{\linewidth}\raggedright
\end{minipage} & \begin{minipage}[b]{\linewidth}\raggedright
\end{minipage} \\
\midrule\noalign{}
\endhead
\bottomrule\noalign{}
\endlastfoot
BusinessArchitectureComponent & OtherArchitectureComponent &
Relationship & & \\
COO--DataExchanges & IA(PartII)--Allchapters &
IAchaptersprovidedetailsregardingthetransformationofdataandinformationidentifiedintheCOO.
& & \\
COO--Drivers andEnablers & TA(PartIII), Chapter2,Technical
ManagementStrategy;Chapter7,TechnicalCapabilityMatrix & Service-Oriented
Architectures(SOA)andTechnicalCapabilitiesareenablersreferencedintheCOO.
& & \\
BPM--TriggerEvent,Result,andSharedDataineachbusinessprocessdescribeingeneraltermsthekindofdatareceivedby,usedby,andresultingfromeachbusinessprocess
& IA(PartII),
Chapter2,DataManagementStrategy;Chapter3,ConceptualDataModel &
DataManagementStrategy(DMS)explainshowthedatasupportsthebusinessprocesses.TheCDMidentifiesgroupingsofinformationcommontoMedicaidbusinessareasandclustersofbusinessprocesses.
& & \\
BCM & IA(PartII), Chapter4,LogicalDataModel; Chapter 6Information
Capability Matrix & TheLDM
definesdataclassesandattributesneededtosupportdifferentlevelsofmaturity.AbusinessprocessdescribedataLevel3businesscapabilityrequiresLevel3dataattributes.
& & \\
BCM & TA(PartIII), Chapter7,TechnicalCapabilityMatrix &
TheBCMdrivestheTechnicalCapability Matrix (TCM).TAassociates technical
capabilitieswiththeBCMlevelwherespecifictechnologyisnecessarytosupportthebusinessprocess.
& & \\
BCM--Level3andabove & TA(PartIII), Chapter 2Technical
ManagementStrategy; Chapter3,BusinessServices &
Abusinessserviceisanimplementationofaspecificbusinessprocessataspecificlevelofcapability.TA
associatesbusiness servicesandSOAwithBCMLevel3andabove. & & \\
\end{longtable}

\pandocbounded{\includegraphics[keepaspectratio]{_page_15_Picture_4.jpeg}}

\chapter[\textbf{Using the Business
Architecture}]{\texorpdfstring{\protect\hypertarget{page-16-0}{}{}\textbf{Using
the Business
Architecture}}{Using the Business Architecture}}\label{using-the-business-architecture-2}

CMS requires States to align to and advance increasingly in MITA
maturity for business, architecture, and data. CMS expects States to use
the BA components to plan for improvements in the State Medicaid
Program, both in the delivery of services to providers and
beneficiaries, and in its internal operations and exchanges of
information with the other external stakeholders. BA provides the COO
and the MMM as background material. States and vendors use the BPM and
the BCM tools. \textbf{\hyperref[page-16-1]{Table} 1-5} summarizes how
stakeholders use the BA.

\phantomsection\label{page-16-1}{}

\begin{longtable}[]{@{}
  >{\raggedright\arraybackslash}p{(\linewidth - 4\tabcolsep) * \real{0.0861}}
  >{\raggedright\arraybackslash}p{(\linewidth - 4\tabcolsep) * \real{0.9104}}
  >{\raggedright\arraybackslash}p{(\linewidth - 4\tabcolsep) * \real{0.0035}}@{}}
\toprule\noalign{}
\begin{minipage}[b]{\linewidth}\raggedright
Stakeholder Useof the Business Architecture
\end{minipage} & \begin{minipage}[b]{\linewidth}\raggedright
\end{minipage} & \begin{minipage}[b]{\linewidth}\raggedright
\end{minipage} \\
\midrule\noalign{}
\endhead
\bottomrule\noalign{}
\endlastfoot
Stakeholder & HowStakeholders Use BA & \\
SMA & The
SMAmapstheiroperationstotheBPMandthenassessesthelevelofmaturityusingtheBCM.Whenthe
SMArequires
informationtechnologyupgradestosupportprogramimprovement,theSMAusestheSS-Atoshowhowit
will use
theenhancedfundingtoachieveaspecificresult(e.g.,movingfromLevel1or2toLevel3).
& \\
CMS &
CMSprovidesleadershipinestablishingtheMITAguidelinesandpromotingthemamongStates.ThroughthereleaseoftheMITAFramework,specialworkshopswithStates,Medicaidconferencematerial,andworkingwithearlyadopterStates,CMSprovides
guidance and principles to achieve the Medicaidvision. & \\
Vendors &
ThevendorcommunityusestheMITAFrameworkasareferenceinplanningtheirresearchanddevelopmentactivities.TheyusetheBA,inparticular,todeterminethematurityleveloffunctionssupportedbytheirsystems.Theyhaveacommonunderstandingofthe
CMS directionfor the Medicaid
Program,andtheycanshowhowtheirproductssupportMITAcapabilities. & \\
Providers & Providersplayanactiveroleintheexchangeofinformationwiththe
SMA.
TheycanlookattheSMABAtounderstandwhatdirectiontheSMAistakingandtokeepthisinmindastheyinvestininformation
technologyupgradesandreengineertheirpractices.Insomecases,the
SMAinvolveprovidersdirectlyinplanningaMedicaidProgramtransformation.
& \\
Beneficiaries & The BA supports the SMAperson-centric outreach,
eligibility and enrollmentactivities across the health and human
services spectrum.Beneficiariesandconsumergroupsare
abletolookattheSMABAandidentify thebenefits.AtLevel3business capability
maturity,beneficiariesareparticipantsinselfdirectedhealthcaredecisions.
& \\
Legislators,Governors &
StatesdeveloppresentationsbasedontheBAtoshowthegovernorandlegislatorswhatgoalsCMSisestablishingforStatesthatrequestenhanced
& \\
\end{longtable}

\subsection{\texorpdfstring{\textbf{Table 1-5. Stakeholder Use of the
Business
Architecture}}{Table 1-5. Stakeholder Use of the Business Architecture}}\label{table-1-5.-stakeholder-use-of-the-business-architecture-2}

\pandocbounded{\includegraphics[keepaspectratio]{_page_16_Picture_6.jpeg}}

\begin{longtable}[]{@{}
  >{\raggedright\arraybackslash}p{(\linewidth - 6\tabcolsep) * \real{0.1842}}
  >{\raggedright\arraybackslash}p{(\linewidth - 6\tabcolsep) * \real{0.8008}}
  >{\raggedright\arraybackslash}p{(\linewidth - 6\tabcolsep) * \real{0.0075}}
  >{\raggedright\arraybackslash}p{(\linewidth - 6\tabcolsep) * \real{0.0075}}@{}}
\toprule\noalign{}
\begin{minipage}[b]{\linewidth}\raggedright
Stakeholder Useof the Business Architecture
\end{minipage} & \begin{minipage}[b]{\linewidth}\raggedright
\end{minipage} & \begin{minipage}[b]{\linewidth}\raggedright
\end{minipage} & \begin{minipage}[b]{\linewidth}\raggedright
\end{minipage} \\
\midrule\noalign{}
\endhead
\bottomrule\noalign{}
\endlastfoot
Stakeholder & HowStakeholders Use BA & & \\
& funding. & & \\
OtherPayersandOtherAgencies & The MITA team invites other
payersandotheragenciestoreviewtheMITAFramework,especiallytheBA,tolearnabouttheMedicaidEnterprisetransformation.
& & \\
\end{longtable}

In general, MITA supports stakeholder roles and access to information,
technology that eliminates most manual activities, and the
transformation of the Medicaid business with the assistance of the CMS,
the SMA, providers, and beneficiaries. In addition, MITA supports
providers with instant access to patient records no matter what their
location is, patients can view their Personal Health Information (PHI)
and make informed decisions regarding treatment, and payers can view
clinical records nationally to expedite decisions on prior authorization
and payment.

\section[\textbf{Next Steps in Developing the Business
Architecture}]{\texorpdfstring{\protect\hypertarget{page-17-0}{}{}\textbf{Next
Steps in Developing the Business
Architecture}}{Next Steps in Developing the Business Architecture}}\label{next-steps-in-developing-the-business-architecture-2}

The MITA Framework delivers the starter kit for a controlled State
Medicaid Enterprise transformation. MITA will continue to evolve over
time. The business process defines the input and output of information
but not the details of the process; however the business community will
still decide the requirements for standardized triggers and results. The
CMS MITA team continues to support SMA efforts by serving as a conduit
for improvements to MITA models that all States and vendors can access.

The MITA Framework and the BA are ever evolving so that the SMA can
continuously improve the way they deliver services to beneficiaries and
providers, account for outcomes, reward participants based on
performance, and respond dynamically to requests for information.

\pandocbounded{\includegraphics[keepaspectratio]{_page_17_Picture_7.jpeg}}

\pandocbounded{\includegraphics[keepaspectratio]{_page_17_Picture_9.jpeg}}

\chapter{MITA Capability Model}\label{mita-capability-model-5}

\section{Introduction to Business Capability
Models}\label{introduction-to-business-capability-models-4}

A capability model is a conceptual framework that outlines the key
capabilities an organization needs to achieve its strategic objectives.
It provides a comprehensive view of what an organization can do and
helps identify areas for improvement or investment. In the context of an
orchestra, a capability model might help the orchestra identify the set
of skills and resources, or other types of capabilities it needs to
perform a symphony. Just like an orchestra needs well practiced
musicians, sheet music, instruments, a conductor, and an audience to
produce a great symphony, a State Medicaid Agency (SMA) needs its
Medicaid Enterprise System (MES) to employ or develop specific
capabilities to deliver its services effectively, efficiently, and
economically to its enrollees and providers.

The concept of a business capability is extensively used within
enterprise architecture modeling and has been broadly used within
Business Capability Models as a tool to better align the business
strategy and information technology of both private sector and
governmental organizations since they emerged in the mid-2000s. One
example comes from the TOGAF Standard, a well-known standard in
enterprise architecture. Like most architecture frameworks TOGAF defines
a capability as something a business can do to meet its goals. This
focuses a strategic lens of an organization on ``what'' it needs to
achieve its goals, rather than ``how'' those goals are achieved. This
perspective allows for business planning from different viewpoints,
facilitating strategic alignment and operational efficiency.

SMA business architects, technologists, systems analysts, executives,
managers, and program staff can use this same modeling approach to
represent the functional components of their Medicaid Enterprise System
(MES) in ways that can help reveal gaps in their systems and provide
insights on what new or enhanced capabilities might be needed to close
those gaps.

By focusing on capabilities, SMAs can better align their information and
technology resources and processes with their strategic business goals,
ultimately improving their insight into how to improve the outcomes
their Medicaid Enterprise Architecture produces.

\subsection{Purpose}\label{purpose-8}

\begin{tcolorbox}[enhanced jigsaw, toprule=.15mm, colback=white, colframe=quarto-callout-note-color-frame, left=2mm, arc=.35mm, opacityback=0, rightrule=.15mm, breakable, bottomrule=.15mm, leftrule=.75mm]
\begin{minipage}[t]{5.5mm}
\textcolor{quarto-callout-note-color}{\faInfo}
\end{minipage}%
\begin{minipage}[t]{\textwidth - 5.5mm}

\vspace{-3mm}\textbf{Note}\vspace{3mm}

MITA 4.0 does not endeavor to specify all of the capabilities SMA's may
need to administer Medicaid programs; instead, this version of MITA
focuses on the capabilities that are most closely oriented towards
achieving the CMS-required outcomes.

\end{minipage}%
\end{tcolorbox}

Understanding the how the MITA Capability Model works is important to
obtaining the most value out of many of the other tools and artifacts in
the MITA framework, such as the MITA Maturity Model (MMM) and the
Business Process Model (BPM). The MITA Capability Model provides a
structured way for SMAs to identify, conceptually model, and improve the
capabilities needed for efficient Medicaid operations.

It is important to note that MITA 4.0 does not endeavor to specify all
of the capabilities SMA's may need to administer Medicaid programs;
instead, this version of MITA focuses on the capabilities that are most
closely oriented towards achieving the CMS-required outcomes. In this
way MITA 4.0 provides a reference model for SMAs to model other
capabilities that may be needed to achieve their other goals such as
state specific outcomes, or other state priorities while providing more
guidance within the MITA Framework to support modular.

\subsection{Update to MITA 3.0}\label{update-to-mita-3.0-5}

MITA 3.0 defined a capability as the level of maturity of a set of
business processes within a business category. By focusing on ``how''
MES operate MITA 3.0 helped SMA's identify ways to improve and mature
their business processes, but it did not link those processes with the
outcomes they are intended to achieve or ensure better alignment of the
information and technical architectures to business outcomes. The
addition of the MITA capability model to the MITA 4.0 business
architecture addresses that by providing the conceptual linkages needed
to elevate the strategic vantage point of the MITA Framework. To guide
this change, we present within this chapter a definition, description,
and approach to modeling business capabilities, based on the widely used
capability models contextualized for Medicaid Enterprises.

The business processes that operationalize MITA capabilities remain
foundational to characterizing the business architecture, and are by
definition a constituent part of any MITA capability. They provide
essential information on how capabilities are operationalize and should
continue to be a routinely utilized reference model for SMA business
process mapping. They are found with in the Business Process Model
chapter of this version of MITA.

\section{The MITA Definition of
Capability}\label{the-mita-definition-of-capability-5}

Within the context of MITA, a capability can be defined as the ability
or capacity of a State Medicaid Agency to achieve a desired outcome in
compliance with the
\href{https://www.ecfr.gov/current/title-42/chapter-IV/subchapter-C/part-433/subpart-C/section-433.112}{Standards
and Conditions within 42 CFR 433.112}. A capability may currently exist
in an operational state or be envisioned for future development. Through
careful planning, capabilities defined in this way can be matured and
refined over time to become more effective and efficient. They can be
organized and detailed at various levels of abstraction, providing
precise descriptions for operational purposes or more generalized views
for strategic planning.

\begin{tcolorbox}[enhanced jigsaw, toprule=.15mm, colback=white, colframe=quarto-callout-note-color-frame, left=2mm, arc=.35mm, opacityback=0, rightrule=.15mm, breakable, bottomrule=.15mm, leftrule=.75mm]

\vspace{-3mm}\textbf{Key Definition}\vspace{3mm}

\ldots a capability is defined as the ability or capacity of a SMA to
achieve a desired outcome\ldots{}

\end{tcolorbox}

To fully define a business capability, it is essential to understand how
it is realized through the integration of people, processes,
information, and technology resources of an SMA. While these elements of
the capability can change regularly, the capability itself is should
endure over longer planning horizons, supporting the long-term alignment
of business and IT and the achievement of increasingly beneficial
business outcomes.

\subsection{Structure of the MITA Capability
Model}\label{structure-of-the-mita-capability-model-5}

As depicted in the model below, the MITA Capability Model orients the
people, process, technology, and information resources to define a MITA
Capability. This means that to model a capability the appropriate
components of the information architecture and the technical
architecture must be brought together with the business architecture to
fully formulate any MITA Capability.

\begin{figure}[H]

{\centering \pandocbounded{\includegraphics[keepaspectratio]{media/capabilityModel/topLevelCapabilityMetamodelGraphic1.png}}

}

\caption{MITA Capability Relationship Diagram}

\end{figure}%

\subsubsection{Business Roles}\label{business-roles-5}

Business roles represent individual actors, stakeholders, or partners
involved in delivering a business capability. A single organizational
group or team may be wholly responsible for delivering the capability,
or multiple business entities may share the delivery of a particular
business capability. Business Roles perform Business Processes using
Technology Resources. They require skills and knowledge resources to
achieve outcomes, and should be actively engaged as partners in the
development or enhancement of any capability they help deliver.

\subsubsection{Business Processes}\label{business-processes-5}

Individual business capabilities may be enabled or delivered through a
range of business processes that detail the activities (the how)
associated with delivering the capability. Identifying and analyzing the
efficiency of the underlying processes helps to optimize the business
capability's effectiveness. Identifying the processes within a business
capability provides a focus for maturing the capability in concert with
the other capability components. Business Processes operationalize
Business Capabilities.

\subsubsection{Information/Data}\label{informationdata-5}

Information/data represents the business data, knowledge, and insight
consumed or produced by the business capability (as distinct from
IT-related data entities). This may also include information that the
capability exchanges with other capabilities to support the execution of
value streams. Examples include information about customers and
prospects, products and services, business policies and rules, sales
reports, and performance metrics. Information/data inform the Business
Capability, answering questions and supporting business rules.

\subsubsection{Technology Resources}\label{technology-resources-5}

Business capabilities rely on a range of tools, applications, systems,
and services for successful execution. Technology Resources use
Information/data to facilitate Business Processes. Such resources may
include:

\begin{itemize}
\tightlist
\item
  Modular software applications

  \begin{itemize}
  \tightlist
  \item
    Cloud or on-premise infrastructure
  \item
    Microservices
  \item
    Analytics
  \item
    Customer portal
  \end{itemize}
\end{itemize}

In this way we can clearly interrelate all of the MITA architecture
models and their individual components which allows us to reveal gaps
not only in the individual components of the architecture, but also
understand their impact on the integration of the architecture
components at the capability level.

\subsection{Relationship of MITA Capabilities to
Outcomes}\label{relationship-of-mita-capabilities-to-outcomes-5}

In the context of the Medicaid Information Technology Architecture
(MITA), outcomes are intrinsically linked to capabilities, as they
represent the tangible results achieved through the effective
integration and execution of various elements that constitute a
capability. In this sense, outcomes and capabilities define each other.

\begin{figure}[H]

{\centering \pandocbounded{\includegraphics[keepaspectratio]{media/capabilityModel/topLevelCapabilityMetamodelOutcomes.png}}

}

\caption{MITA Capability and Outcome Relationship Diagram}

\end{figure}%

\subsubsection{Outcomes}\label{outcomes-5}

MITA defines outcomes broadly to encompass CMS-required outcomes,
state-specific outcomes, and other outcomes not mandated as part of the
Advance Planning Document (APD) process. The sole criterion for an
outcome to meet this definition is that it must be a goal of a State
Medicaid Agency (SMA) and be achieved through a Medicaid Enterprise
System (MES) capability.

\begin{tcolorbox}[enhanced jigsaw, toprule=.15mm, colback=white, colframe=quarto-callout-note-color-frame, left=2mm, arc=.35mm, opacityback=0, rightrule=.15mm, breakable, bottomrule=.15mm, leftrule=.75mm]

\vspace{-3mm}\textbf{Key Definition}\vspace{3mm}

A MITA outcome is a goal of a State Medicaid Agency (SMA) that is
achieved by a Medicaid Enterprise System (MES) capability.

\end{tcolorbox}

\subsubsection{Measure}\label{measure-5}

Measure is a quantifiable metric used to assess the effectiveness and
efficiency of capabilities within a Medicaid Enterprise System (MES).
Measures provide quantifiable and qualitative values that help State
Medicaid Agencies (SMAs) track progress toward achieving specific
outcomes, such as CMS-required or state-specific goals. These indicators
might include metrics like processing times, error rates, or compliance
levels.

Measures are a measurement threshold by establishing a specific value or
level that must be met or exceeded to demonstrate successful
performance. For instance, a KPI might set a threshold for the maximum
allowable processing time for claims, ensuring that they are handled
within a specified timeframe to maintain compliance and eligibility for
enhanced federal funding. By monitoring these thresholds, organizations
can ensure they are meeting regulatory requirements and delivering
high-quality services to beneficiaries, while also identifying areas for
improvement.

\subsubsection{Measure Threshold}\label{measure-threshold-5}

A specific value or level of a measure that must be met or exceeded to
demonstrate the effective achievement of a capability's intended
outcome. This threshold serves as a benchmark for assessing whether the
processes, roles, and resources integrated within a Medicaid Enterprise
System (MES) are functioning optimally to meet the goals of a State
Medicaid Agency (SMA). For example, a measurement threshold might be set
for processing times, where claims must be processed within a certain
number of days to ensure compliance with CMS-required outcomes and
maintain eligibility for enhanced federal funding. By establishing and
monitoring these thresholds, organizations can ensure they are meeting
regulatory requirements and delivering high-quality services to
beneficiaries.

\subsubsection{Measurement}\label{measurement-5}

These outcomes and metrics are also used to ensure that healthcare
systems or modules comply with applicable federal regulations, forming
the baseline for system or module functionality. Achieving these
outcomes is essential for continuing to receive enhanced federal funding
for operations. Regular measurement and analysis of KPIs help
organizations demonstrate compliance and effectiveness, ensuring that
they meet regulatory requirements and continue to deliver high-quality
services to beneficiaries.

In this way we can clearly interrelate all of the MITA architecture
models and their individual components with the KPIs, thresholds, and
measurements that indicate whether our capability achieves our desired
outcome.

While models that help conceptualize the capabilities that achieve
CMS-required outcomes are the ones modeled for this version of MITA,
SMAs are encouraged to use these models as a reference to model
capabilities.

\section{Capability Mapping}\label{capability-mapping-5}

Capability mapping is a strategic tool that enables organizations, such
as State Medicaid Agencies (SMAs), to systematically identify, organize,
and visualize the key capabilities necessary to achieve their
objectives. Within the MITA framework, capability mapping provides SMAs
with a method of developing comprehensive views of the functions and
processes required to deliver Medicaid services effectively. To begin
the capability mapping process, SMAs should first identify the core
capabilities that align with their strategic objectives, focusing on
what the organization needs to achieve rather than how those goals are
accomplished. This involves listing all necessary capabilities and
understanding the desired outcomes they support. Next, these
capabilities should be organized into domains and areas that reflect
their strategic importance and interrelationships. Visualizing these
capabilities through diagrams or maps provides all stakeholders a common
view to understand the roles, processes, technology resources, and
information/data involved in executing each capability, as well as the
outcome each capability is designed to achieve. This structured approach
not only highlights areas for improvement or investment but also ensures
that organizational efforts are strategically aligned with desired
outcomes.

The benefits of capability mapping are multifaceted, offering SMAs a
clear pathway to strategic alignment and gap analysis. By visualizing
capabilities, organizations can identify operational gaps and determine
what new or enhanced capabilities are needed to close those gaps. This
visualization also improves communication among stakeholders by
providing a clear and concise representation of the organization's
functions. To refine capabilities, SMAs should analyze current
operations, assess the efficiency of underlying processes, and optimize
them to enhance capability effectiveness. Additionally, capability
mapping serves as a foundation for heat mapping, which assesses the MITA
Framework will utilize to visualize the maturity of each capability
evaluated in the State Self-Assessment. SMAs can overlay heat maps over
their capability maps to visualize many things other than maturity
levels, using color coding to indicate areas of strength and weakness.
Regular updates to these maps allow SMAs to monitor progress and ensure
resources are allocated effectively to achieve strategic goals. The MITA
framework includes examples of capability maps based on CMS-required
outcomes, serving as a reference model for SMAs to develop their own
capability maps tailored to state-specific goals and priorities. By
leveraging the reference models provided by MITA, SMAs can ensure their
capability mapping efforts are aligned with both federal requirements
and state-specific priorities.

\subsection{Organizing Capabilities}\label{organizing-capabilities-5}

To enhance the resolution and detail of a capability and provide a
unified view of all its components, a block diagram can be employed to
provide a common view of any MES. This diagram effectively links the
capability to business processes, roles, technical resources, and
information resources through functional decomposition. By breaking down
the capability into its constituent parts, the block diagram offers a
visual representation that highlights the interrelationships and
dependencies among these elements. This approach provides a clearer
understanding of how each component contributes to the overall
capability, facilitating more effective analysis, optimization, and
alignment with organizational objectives.

\pandocbounded{\includegraphics[keepaspectratio]{media/capabilityModel/capabilityOgranizationModel.drawio.png}}

We use this same method to present an this top level view of the
capabilities required to achieve CMS-required outcomes. From this view
increasingly detailed models can be constructed.

\pandocbounded{\includegraphics[keepaspectratio]{media/capabilityModel/mesModuleBasedCapabilities.drawio.png}}

\subsection{MITA Capability Models}\label{mita-capability-models-5}

The MITA framework represents capabilities visually through a layered
model that represent a capability of being composed of sub-capabilities
and the processes, roles, information and technology resources (PRIT)
that support the business in sustaining the capability. Each layer up
depicts increasingly strategic capabilities and each layer down depicts
the constituent elements that compose a capability in increasing
operational detail. It is not the intention of this version of MITA to
provide a full operational or tactical view of a capability, though SMAs
may consider using this approach to improve their organizational
awareness of their operations by developing further layers of their
capabilities through functional decomposition.

\pandocbounded{\includegraphics[keepaspectratio]{media/capabilityModel/capabilityLevels.png}}

\begin{itemize}
\tightlist
\item
  \textbf{Capability Domains:} The first layer of this model aims to
  group capabilities to organize the strategic view of an SMA's
  capabilities. In this view one or many capabilities can be grouped
  within a domain to indicate the pursuit of common outcomes. Each
  domain is denoted with a single number to help annotate each
  capability.

  \begin{itemize}
  \tightlist
  \item
    \textbf{Capability Areas:} The second layer of this model aims to
    provide a view of the groups of capabilities that compose a domain.
    They are organized to show capabilities that serve a specific group
    of similar outcomes and essential
  \item
    \textbf{Capabilities:} The third layer of this model provides a more
    detailed view view of
  \end{itemize}
\end{itemize}

\pandocbounded{\includegraphics[keepaspectratio]{media/capabilityModel/capabilityLevels2.png}}

\subsection{Relationship of MITA Capabilities to
Maturity}\label{relationship-of-mita-capabilities-to-maturity-5}

\begin{tcolorbox}[enhanced jigsaw, toprule=.15mm, leftrule=.75mm, colframe=quarto-callout-warning-color-frame, left=2mm, arc=.35mm, titlerule=0mm, rightrule=.15mm, opacitybacktitle=0.6, bottomtitle=1mm, toptitle=1mm, colbacktitle=quarto-callout-warning-color!10!white, bottomrule=.15mm, title=\textcolor{quarto-callout-warning-color}{\faExclamationTriangle}\hspace{0.5em}{Warning}, opacityback=0, breakable, colback=white, coltitle=black]

Under development.

\end{tcolorbox}

\begin{itemize}
\tightlist
\item
  \textbf{Levels of Maturity}

  \begin{itemize}
  \tightlist
  \item
    Description of the five levels of maturity in the MITA framework
  \item
    How capabilities evolve and mature over time
  \end{itemize}
\end{itemize}

\pandocbounded{\includegraphics[keepaspectratio]{media/capabilityModel/maturityModel.png}}

\subsection{Using Capability Maps for Heat Mapping Strategic Priorities
and Identifying Gaps with the MITA Maturity
Model}\label{using-capability-maps-for-heat-mapping-strategic-priorities-and-identifying-gaps-with-the-mita-maturity-model-5}

Capability maps are powerful tools that not only provide a visual
representation of an SMA's key capabilities but also serve as a
foundation for strategic analysis and planning. There are many
approaches to heat mapping capabilities, each offering unique insights
into organizational priorities and gaps. Here, we describe two
approaches: assessing maturity levels using the MITA Maturity Model and
prioritizing strategic outcomes.

\subsubsection{Identifying Gaps with the MITA Maturity
Model}\label{identifying-gaps-with-the-mita-maturity-model-5}

The MITA Maturity Model provides a framework for assessing the maturity
of an organization's capabilities across various dimensions, such as
business processes, information, and technology. By integrating the
maturity model with capability maps, SMAs can identify gaps between
their current state and desired maturity levels.

\paragraph{Example 1: Identifying Gaps in Data Management Maturity Using
the PRIT
Model}\label{example-1-identifying-gaps-in-data-management-maturity-using-the-prit-model-5}

An SMA is conducting an assessment of its data management capabilities
using the MITA Maturity Model, with a focus on the PRIT (Processes,
Roles, Information, and Technology) framework. The capability map
includes various data-related capabilities, such as ``Data
Integration,'' ``Data Quality Management,'' and ``Data Analytics.'' Each
of these capabilities is evaluated across the PRIT dimensions to
determine their maturity levels using the revised scale:

Processes: Level 1: Ad-Hoc Roles: Level 2: Compliant Information: Level
2: Compliant Technology: Level 2: Compliant The capability map is
updated to reflect the maturity assessment, with each dimension marked
with a color code: red for Level 1: Ad-Hoc, yellow for Level 2:
Compliant, green for Level 3: Efficient, blue for Level 4: Optimized,
and purple for Level 5: Pioneering. This visualization helps the SMA
prioritize strategic actions to enhance the ``Data Integration''
capability, such as standardizing processes, refining roles, improving
data quality, and upgrading technology.

\subsubsection{Heat Mapping Strategic
Priorities}\label{heat-mapping-strategic-priorities-5}

Heat mapping involves applying a color-coded overlay to a capability map
to visually represent the status or priority level of each capability.
This technique can be used to highlight areas of strength, weakness, or
strategic importance. For example, capabilities that are critical to
achieving CMS-required outcomes might be marked in one color, while
those needing immediate attention or improvement could be marked in
another. This visual representation helps stakeholders quickly grasp the
strategic landscape and make informed decisions about where to allocate
resources and focus efforts.

\paragraph{Example 2: Prioritizing Capabilities for CMS-Required
Outcomes}\label{example-2-prioritizing-capabilities-for-cms-required-outcomes-5}

An SMA is focused on achieving specific CMS-required outcomes related to
improving patient care and reducing administrative costs. The agency
creates a capability map that outlines all the capabilities necessary to
meet these outcomes. By applying a heat map, the SMA highlights
capabilities that are directly linked to these outcomes in green,
indicating they are of high strategic priority. Capabilities that are
indirectly related or less critical are marked in yellow, while those
that are currently underperforming or not aligned with strategic goals
are marked in red.

This visual representation allows the SMA to quickly identify which
capabilities require immediate attention and resources to ensure
compliance with CMS requirements. For instance, if the capability
related to ``Claims Processing Efficiency'' is marked in red, the agency
can prioritize initiatives to enhance this capability, such as investing
in new technology or streamlining processes.

\subsubsection{Other Uses for Capability Heat
Mapping}\label{other-uses-for-capability-heat-mapping-5}

Beyond assessing maturity levels and prioritizing strategic initiatives,
capability heat mapping can be applied in various other contexts to
enhance organizational effectiveness and alignment.

\paragraph{Example 3: Aligning Capabilities with State-Specific
Initiatives}\label{example-3-aligning-capabilities-with-state-specific-initiatives-5}

An SMA is working on a state-specific initiative to enhance telehealth
services for rural populations. The capability map includes capabilities
related to telehealth, such as ``Telehealth Infrastructure,'' ``Provider
Engagement,'' and ``Patient Access.'' The SMA uses a heat map to
highlight these capabilities in blue, indicating their alignment with
the state-specific initiative.

By analyzing the capability map, the SMA identifies that ``Provider
Engagement'' is a critical capability that requires further development
to support the telehealth initiative. The agency decides to invest in
training programs and outreach efforts to engage providers in rural
areas, ensuring that the telehealth services are effectively delivered
to the target population.

These examples demonstrate how capability maps, combined with heat
mapping and the MITA Maturity Model, can provide valuable insights for
strategic planning and gap analysis. By visualizing priorities and
maturity levels, SMAs can make informed decisions about where to focus
resources and efforts, ultimately enhancing their Medicaid Enterprise
Systems and achieving strategic objectives.

\begin{itemize}
\tightlist
\item
  \textbf{Capability Mapping}

  \begin{itemize}
  \tightlist
  \item
    Introduction to capability mapping and its significance
  \item
    How capabilities are organized and detailed at various levels of
    abstraction
  \end{itemize}
\end{itemize}

\section{Guidance on reuse of the MITA Capability
Model}\label{guidance-on-reuse-of-the-mita-capability-model-5}

\begin{itemize}
\tightlist
\item
  \textbf{Practical Application}

  \begin{itemize}
  \tightlist
  \item
    How to integrate the capability model into daily operations and
    strategic planning
  \item
    Tips for maximizing the benefits of the model
  \end{itemize}
\item
  \textbf{Continuous Improvement}

  \begin{itemize}
  \tightlist
  \item
    Encouragement for ongoing assessment and refinement of capabilities
  \item
    Leveraging feedback and performance data for model enhancement
  \end{itemize}
\item
  \textbf{Implementation Guidance}

  \begin{itemize}
  \tightlist
  \item
    Steps for adopting the capability model
  \item
    Resources and support available for SMAs
  \end{itemize}
\item
  \textbf{Performance Monitoring and Reporting}

  \begin{itemize}
  \tightlist
  \item
    Role of the capability model in tracking and enhancing performance
  \item
    Use of metrics and standards to measure capability effectiveness
  \end{itemize}
\end{itemize}

\chapter{MITA Capability Model}\label{mita-capability-model-6}

\section{Introduction to Business Capability
Models}\label{introduction-to-business-capability-models-5}

A capability model is a conceptual framework that outlines the key
capabilities an organization needs to achieve its strategic objectives.
It provides a comprehensive view of what an organization can do and
helps identify areas for improvement or investment. In the context of an
orchestra, a capability model might help the orchestra identify the set
of skills and resources, or other types of capabilities it needs to
perform a symphony. Just like an orchestra needs well practiced
musicians, sheet music, instruments, a conductor, and an audience to
produce a great symphony, a State Medicaid Agency (SMA) needs its
Medicaid Enterprise System (MES) to employ or develop specific
capabilities to deliver its services effectively, efficiently, and
economically to its enrollees and providers.

The concept of a business capability is extensively used within
enterprise architecture modeling and has been broadly used within
Business Capability Models as a tool to better align the business
strategy and information technology of both private sector and
governmental organizations since they emerged in the mid-2000s. One
example comes from the TOGAF Standard, a well-known standard in
enterprise architecture. Like most architecture frameworks TOGAF defines
a capability as something a business can do to meet its goals. This
focuses a strategic lens of an organization on ``what'' it needs to
achieve its goals, rather than ``how'' those goals are achieved. This
perspective allows for business planning from different viewpoints,
facilitating strategic alignment and operational efficiency.

SMA business architects, technologists, systems analysts, executives,
managers, and program staff can use this same modeling approach to
represent the functional components of their Medicaid Enterprise System
(MES) in ways that can help reveal gaps in their systems and provide
insights on what new or enhanced capabilities might be needed to close
those gaps.

By focusing on capabilities, SMAs can better align their information and
technology resources and processes with their strategic business goals,
ultimately improving their insight into how to improve the outcomes
their Medicaid Enterprise Architecture produces.

\subsection{Purpose}\label{purpose-9}

\begin{tcolorbox}[enhanced jigsaw, toprule=.15mm, colback=white, colframe=quarto-callout-note-color-frame, left=2mm, arc=.35mm, opacityback=0, rightrule=.15mm, breakable, bottomrule=.15mm, leftrule=.75mm]
\begin{minipage}[t]{5.5mm}
\textcolor{quarto-callout-note-color}{\faInfo}
\end{minipage}%
\begin{minipage}[t]{\textwidth - 5.5mm}

\vspace{-3mm}\textbf{Note}\vspace{3mm}

MITA 4.0 does not endeavor to specify all of the capabilities SMA's may
need to administer Medicaid programs; instead, this version of MITA
focuses on the capabilities that are most closely oriented towards
achieving the CMS-required outcomes.

\end{minipage}%
\end{tcolorbox}

Understanding the how the MITA Capability Model works is important to
obtaining the most value out of many of the other tools and artifacts in
the MITA framework, such as the MITA Maturity Model (MMM) and the
Business Process Model (BPM). The MITA Capability Model provides a
structured way for SMAs to identify, conceptually model, and improve the
capabilities needed for efficient Medicaid operations.

It is important to note that MITA 4.0 does not endeavor to specify all
of the capabilities SMA's may need to administer Medicaid programs;
instead, this version of MITA focuses on the capabilities that are most
closely oriented towards achieving the CMS-required outcomes. In this
way MITA 4.0 provides a reference model for SMAs to model other
capabilities that may be needed to achieve their other goals such as
state specific outcomes, or other state priorities while providing more
guidance within the MITA Framework to support modular.

\subsection{Update to MITA 3.0}\label{update-to-mita-3.0-6}

MITA 3.0 defined a capability as the level of maturity of a set of
business processes within a business category. By focusing on ``how''
MES operate MITA 3.0 helped SMA's identify ways to improve and mature
their business processes, but it did not link those processes with the
outcomes they are intended to achieve or ensure better alignment of the
information and technical architectures to business outcomes. The
addition of the MITA capability model to the MITA 4.0 business
architecture addresses that by providing the conceptual linkages needed
to elevate the strategic vantage point of the MITA Framework. To guide
this change, we present within this chapter a definition, description,
and approach to modeling business capabilities, based on the widely used
capability models contextualized for Medicaid Enterprises.

The business processes that operationalize MITA capabilities remain
foundational to characterizing the business architecture, and are by
definition a constituent part of any MITA capability. They provide
essential information on how capabilities are operationalize and should
continue to be a routinely utilized reference model for SMA business
process mapping. They are found with in the Business Process Model
chapter of this version of MITA.

\section{The MITA Definition of
Capability}\label{the-mita-definition-of-capability-6}

Within the context of MITA, a capability can be defined as the ability
or capacity of a State Medicaid Agency to achieve a desired outcome in
compliance with the
\href{https://www.ecfr.gov/current/title-42/chapter-IV/subchapter-C/part-433/subpart-C/section-433.112}{Standards
and Conditions within 42 CFR 433.112}. A capability may currently exist
in an operational state or be envisioned for future development. Through
careful planning, capabilities defined in this way can be matured and
refined over time to become more effective and efficient. They can be
organized and detailed at various levels of abstraction, providing
precise descriptions for operational purposes or more generalized views
for strategic planning.

\begin{tcolorbox}[enhanced jigsaw, toprule=.15mm, colback=white, colframe=quarto-callout-note-color-frame, left=2mm, arc=.35mm, opacityback=0, rightrule=.15mm, breakable, bottomrule=.15mm, leftrule=.75mm]

\vspace{-3mm}\textbf{Key Definition}\vspace{3mm}

\ldots a capability is defined as the ability or capacity of a SMA to
achieve a desired outcome\ldots{}

\end{tcolorbox}

To fully define a business capability, it is essential to understand how
it is realized through the integration of people, processes,
information, and technology resources of an SMA. While these elements of
the capability can change regularly, the capability itself is should
endure over longer planning horizons, supporting the long-term alignment
of business and IT and the achievement of increasingly beneficial
business outcomes.

\subsection{Structure of the MITA Capability
Model}\label{structure-of-the-mita-capability-model-6}

As depicted in the model below, the MITA Capability Model orients the
people, process, technology, and information resources to define a MITA
Capability. This means that to model a capability the appropriate
components of the information architecture and the technical
architecture must be brought together with the business architecture to
fully formulate any MITA Capability.

\begin{figure}[H]

{\centering \pandocbounded{\includegraphics[keepaspectratio]{media/capabilityModel/topLevelCapabilityMetamodelGraphic1.png}}

}

\caption{MITA Capability Relationship Diagram}

\end{figure}%

\subsubsection{Business Roles}\label{business-roles-6}

Business roles represent individual actors, stakeholders, or partners
involved in delivering a business capability. A single organizational
group or team may be wholly responsible for delivering the capability,
or multiple business entities may share the delivery of a particular
business capability. Business Roles perform Business Processes using
Technology Resources. They require skills and knowledge resources to
achieve outcomes, and should be actively engaged as partners in the
development or enhancement of any capability they help deliver.

\subsubsection{Business Processes}\label{business-processes-6}

Individual business capabilities may be enabled or delivered through a
range of business processes that detail the activities (the how)
associated with delivering the capability. Identifying and analyzing the
efficiency of the underlying processes helps to optimize the business
capability's effectiveness. Identifying the processes within a business
capability provides a focus for maturing the capability in concert with
the other capability components. Business Processes operationalize
Business Capabilities.

\subsubsection{Information/Data}\label{informationdata-6}

Information/data represents the business data, knowledge, and insight
consumed or produced by the business capability (as distinct from
IT-related data entities). This may also include information that the
capability exchanges with other capabilities to support the execution of
value streams. Examples include information about customers and
prospects, products and services, business policies and rules, sales
reports, and performance metrics. Information/data inform the Business
Capability, answering questions and supporting business rules.

\subsubsection{Technology Resources}\label{technology-resources-6}

Business capabilities rely on a range of tools, applications, systems,
and services for successful execution. Technology Resources use
Information/data to facilitate Business Processes. Such resources may
include:

\begin{itemize}
\tightlist
\item
  Modular software applications

  \begin{itemize}
  \tightlist
  \item
    Cloud or on-premise infrastructure
  \item
    Microservices
  \item
    Analytics
  \item
    Customer portal
  \end{itemize}
\end{itemize}

In this way we can clearly interrelate all of the MITA architecture
models and their individual components which allows us to reveal gaps
not only in the individual components of the architecture, but also
understand their impact on the integration of the architecture
components at the capability level.

\subsection{Relationship of MITA Capabilities to
Outcomes}\label{relationship-of-mita-capabilities-to-outcomes-6}

In the context of the Medicaid Information Technology Architecture
(MITA), outcomes are intrinsically linked to capabilities, as they
represent the tangible results achieved through the effective
integration and execution of various elements that constitute a
capability. In this sense, outcomes and capabilities define each other.

\begin{figure}[H]

{\centering \pandocbounded{\includegraphics[keepaspectratio]{media/capabilityModel/topLevelCapabilityMetamodelOutcomes.png}}

}

\caption{MITA Capability and Outcome Relationship Diagram}

\end{figure}%

\subsubsection{Outcomes}\label{outcomes-6}

MITA defines outcomes broadly to encompass CMS-required outcomes,
state-specific outcomes, and other outcomes not mandated as part of the
Advance Planning Document (APD) process. The sole criterion for an
outcome to meet this definition is that it must be a goal of a State
Medicaid Agency (SMA) and be achieved through a Medicaid Enterprise
System (MES) capability.

\begin{tcolorbox}[enhanced jigsaw, toprule=.15mm, colback=white, colframe=quarto-callout-note-color-frame, left=2mm, arc=.35mm, opacityback=0, rightrule=.15mm, breakable, bottomrule=.15mm, leftrule=.75mm]

\vspace{-3mm}\textbf{Key Definition}\vspace{3mm}

A MITA outcome is a goal of a State Medicaid Agency (SMA) that is
achieved by a Medicaid Enterprise System (MES) capability.

\end{tcolorbox}

\subsubsection{Measure}\label{measure-6}

Measure is a quantifiable metric used to assess the effectiveness and
efficiency of capabilities within a Medicaid Enterprise System (MES).
Measures provide quantifiable and qualitative values that help State
Medicaid Agencies (SMAs) track progress toward achieving specific
outcomes, such as CMS-required or state-specific goals. These indicators
might include metrics like processing times, error rates, or compliance
levels.

Measures are a measurement threshold by establishing a specific value or
level that must be met or exceeded to demonstrate successful
performance. For instance, a KPI might set a threshold for the maximum
allowable processing time for claims, ensuring that they are handled
within a specified timeframe to maintain compliance and eligibility for
enhanced federal funding. By monitoring these thresholds, organizations
can ensure they are meeting regulatory requirements and delivering
high-quality services to beneficiaries, while also identifying areas for
improvement.

\subsubsection{Measure Threshold}\label{measure-threshold-6}

A specific value or level of a measure that must be met or exceeded to
demonstrate the effective achievement of a capability's intended
outcome. This threshold serves as a benchmark for assessing whether the
processes, roles, and resources integrated within a Medicaid Enterprise
System (MES) are functioning optimally to meet the goals of a State
Medicaid Agency (SMA). For example, a measurement threshold might be set
for processing times, where claims must be processed within a certain
number of days to ensure compliance with CMS-required outcomes and
maintain eligibility for enhanced federal funding. By establishing and
monitoring these thresholds, organizations can ensure they are meeting
regulatory requirements and delivering high-quality services to
beneficiaries.

\subsubsection{Measurement}\label{measurement-6}

These outcomes and metrics are also used to ensure that healthcare
systems or modules comply with applicable federal regulations, forming
the baseline for system or module functionality. Achieving these
outcomes is essential for continuing to receive enhanced federal funding
for operations. Regular measurement and analysis of KPIs help
organizations demonstrate compliance and effectiveness, ensuring that
they meet regulatory requirements and continue to deliver high-quality
services to beneficiaries.

In this way we can clearly interrelate all of the MITA architecture
models and their individual components with the KPIs, thresholds, and
measurements that indicate whether our capability achieves our desired
outcome.

While models that help conceptualize the capabilities that achieve
CMS-required outcomes are the ones modeled for this version of MITA,
SMAs are encouraged to use these models as a reference to model
capabilities.

\section{Capability Mapping}\label{capability-mapping-6}

Capability mapping is a strategic tool that enables organizations, such
as State Medicaid Agencies (SMAs), to systematically identify, organize,
and visualize the key capabilities necessary to achieve their
objectives. Within the MITA framework, capability mapping provides SMAs
with a method of developing comprehensive views of the functions and
processes required to deliver Medicaid services effectively. To begin
the capability mapping process, SMAs should first identify the core
capabilities that align with their strategic objectives, focusing on
what the organization needs to achieve rather than how those goals are
accomplished. This involves listing all necessary capabilities and
understanding the desired outcomes they support. Next, these
capabilities should be organized into domains and areas that reflect
their strategic importance and interrelationships. Visualizing these
capabilities through diagrams or maps provides all stakeholders a common
view to understand the roles, processes, technology resources, and
information/data involved in executing each capability, as well as the
outcome each capability is designed to achieve. This structured approach
not only highlights areas for improvement or investment but also ensures
that organizational efforts are strategically aligned with desired
outcomes.

The benefits of capability mapping are multifaceted, offering SMAs a
clear pathway to strategic alignment and gap analysis. By visualizing
capabilities, organizations can identify operational gaps and determine
what new or enhanced capabilities are needed to close those gaps. This
visualization also improves communication among stakeholders by
providing a clear and concise representation of the organization's
functions. To refine capabilities, SMAs should analyze current
operations, assess the efficiency of underlying processes, and optimize
them to enhance capability effectiveness. Additionally, capability
mapping serves as a foundation for heat mapping, which assesses the MITA
Framework will utilize to visualize the maturity of each capability
evaluated in the State Self-Assessment. SMAs can overlay heat maps over
their capability maps to visualize many things other than maturity
levels, using color coding to indicate areas of strength and weakness.
Regular updates to these maps allow SMAs to monitor progress and ensure
resources are allocated effectively to achieve strategic goals. The MITA
framework includes examples of capability maps based on CMS-required
outcomes, serving as a reference model for SMAs to develop their own
capability maps tailored to state-specific goals and priorities. By
leveraging the reference models provided by MITA, SMAs can ensure their
capability mapping efforts are aligned with both federal requirements
and state-specific priorities.

\subsection{Organizing Capabilities}\label{organizing-capabilities-6}

To enhance the resolution and detail of a capability and provide a
unified view of all its components, a block diagram can be employed to
provide a common view of any MES. This diagram effectively links the
capability to business processes, roles, technical resources, and
information resources through functional decomposition. By breaking down
the capability into its constituent parts, the block diagram offers a
visual representation that highlights the interrelationships and
dependencies among these elements. This approach provides a clearer
understanding of how each component contributes to the overall
capability, facilitating more effective analysis, optimization, and
alignment with organizational objectives.

\pandocbounded{\includegraphics[keepaspectratio]{media/capabilityModel/capabilityOgranizationModel.drawio.png}}

We use this same method to present an this top level view of the
capabilities required to achieve CMS-required outcomes. From this view
increasingly detailed models can be constructed.

\pandocbounded{\includegraphics[keepaspectratio]{media/capabilityModel/mesModuleBasedCapabilities.drawio.png}}

\subsection{MITA Capability Models}\label{mita-capability-models-6}

The MITA framework represents capabilities visually through a layered
model that represent a capability of being composed of sub-capabilities
and the processes, roles, information and technology resources (PRIT)
that support the business in sustaining the capability. Each layer up
depicts increasingly strategic capabilities and each layer down depicts
the constituent elements that compose a capability in increasing
operational detail. It is not the intention of this version of MITA to
provide a full operational or tactical view of a capability, though SMAs
may consider using this approach to improve their organizational
awareness of their operations by developing further layers of their
capabilities through functional decomposition.

\pandocbounded{\includegraphics[keepaspectratio]{media/capabilityModel/capabilityLevels.png}}

\begin{itemize}
\tightlist
\item
  \textbf{Capability Domains:} The first layer of this model aims to
  group capabilities to organize the strategic view of an SMA's
  capabilities. In this view one or many capabilities can be grouped
  within a domain to indicate the pursuit of common outcomes. Each
  domain is denoted with a single number to help annotate each
  capability.

  \begin{itemize}
  \tightlist
  \item
    \textbf{Capability Areas:} The second layer of this model aims to
    provide a view of the groups of capabilities that compose a domain.
    They are organized to show capabilities that serve a specific group
    of similar outcomes and essential
  \item
    \textbf{Capabilities:} The third layer of this model provides a more
    detailed view view of
  \end{itemize}
\end{itemize}

\pandocbounded{\includegraphics[keepaspectratio]{media/capabilityModel/capabilityLevels2.png}}

\subsection{Relationship of MITA Capabilities to
Maturity}\label{relationship-of-mita-capabilities-to-maturity-6}

\begin{tcolorbox}[enhanced jigsaw, toprule=.15mm, leftrule=.75mm, colframe=quarto-callout-warning-color-frame, left=2mm, arc=.35mm, titlerule=0mm, rightrule=.15mm, opacitybacktitle=0.6, bottomtitle=1mm, toptitle=1mm, colbacktitle=quarto-callout-warning-color!10!white, bottomrule=.15mm, title=\textcolor{quarto-callout-warning-color}{\faExclamationTriangle}\hspace{0.5em}{Warning}, opacityback=0, breakable, colback=white, coltitle=black]

Under development.

\end{tcolorbox}

\begin{itemize}
\tightlist
\item
  \textbf{Levels of Maturity}

  \begin{itemize}
  \tightlist
  \item
    Description of the five levels of maturity in the MITA framework
  \item
    How capabilities evolve and mature over time
  \end{itemize}
\end{itemize}

\pandocbounded{\includegraphics[keepaspectratio]{media/capabilityModel/maturityModel.png}}

\subsection{Using Capability Maps for Heat Mapping Strategic Priorities
and Identifying Gaps with the MITA Maturity
Model}\label{using-capability-maps-for-heat-mapping-strategic-priorities-and-identifying-gaps-with-the-mita-maturity-model-6}

Capability maps are powerful tools that not only provide a visual
representation of an SMA's key capabilities but also serve as a
foundation for strategic analysis and planning. There are many
approaches to heat mapping capabilities, each offering unique insights
into organizational priorities and gaps. Here, we describe two
approaches: assessing maturity levels using the MITA Maturity Model and
prioritizing strategic outcomes.

\subsubsection{Identifying Gaps with the MITA Maturity
Model}\label{identifying-gaps-with-the-mita-maturity-model-6}

The MITA Maturity Model provides a framework for assessing the maturity
of an organization's capabilities across various dimensions, such as
business processes, information, and technology. By integrating the
maturity model with capability maps, SMAs can identify gaps between
their current state and desired maturity levels.

\paragraph{Example 1: Identifying Gaps in Data Management Maturity Using
the PRIT
Model}\label{example-1-identifying-gaps-in-data-management-maturity-using-the-prit-model-6}

An SMA is conducting an assessment of its data management capabilities
using the MITA Maturity Model, with a focus on the PRIT (Processes,
Roles, Information, and Technology) framework. The capability map
includes various data-related capabilities, such as ``Data
Integration,'' ``Data Quality Management,'' and ``Data Analytics.'' Each
of these capabilities is evaluated across the PRIT dimensions to
determine their maturity levels using the revised scale:

Processes: Level 1: Ad-Hoc Roles: Level 2: Compliant Information: Level
2: Compliant Technology: Level 2: Compliant The capability map is
updated to reflect the maturity assessment, with each dimension marked
with a color code: red for Level 1: Ad-Hoc, yellow for Level 2:
Compliant, green for Level 3: Efficient, blue for Level 4: Optimized,
and purple for Level 5: Pioneering. This visualization helps the SMA
prioritize strategic actions to enhance the ``Data Integration''
capability, such as standardizing processes, refining roles, improving
data quality, and upgrading technology.

\subsubsection{Heat Mapping Strategic
Priorities}\label{heat-mapping-strategic-priorities-6}

Heat mapping involves applying a color-coded overlay to a capability map
to visually represent the status or priority level of each capability.
This technique can be used to highlight areas of strength, weakness, or
strategic importance. For example, capabilities that are critical to
achieving CMS-required outcomes might be marked in one color, while
those needing immediate attention or improvement could be marked in
another. This visual representation helps stakeholders quickly grasp the
strategic landscape and make informed decisions about where to allocate
resources and focus efforts.

\paragraph{Example 2: Prioritizing Capabilities for CMS-Required
Outcomes}\label{example-2-prioritizing-capabilities-for-cms-required-outcomes-6}

An SMA is focused on achieving specific CMS-required outcomes related to
improving patient care and reducing administrative costs. The agency
creates a capability map that outlines all the capabilities necessary to
meet these outcomes. By applying a heat map, the SMA highlights
capabilities that are directly linked to these outcomes in green,
indicating they are of high strategic priority. Capabilities that are
indirectly related or less critical are marked in yellow, while those
that are currently underperforming or not aligned with strategic goals
are marked in red.

This visual representation allows the SMA to quickly identify which
capabilities require immediate attention and resources to ensure
compliance with CMS requirements. For instance, if the capability
related to ``Claims Processing Efficiency'' is marked in red, the agency
can prioritize initiatives to enhance this capability, such as investing
in new technology or streamlining processes.

\subsubsection{Other Uses for Capability Heat
Mapping}\label{other-uses-for-capability-heat-mapping-6}

Beyond assessing maturity levels and prioritizing strategic initiatives,
capability heat mapping can be applied in various other contexts to
enhance organizational effectiveness and alignment.

\paragraph{Example 3: Aligning Capabilities with State-Specific
Initiatives}\label{example-3-aligning-capabilities-with-state-specific-initiatives-6}

An SMA is working on a state-specific initiative to enhance telehealth
services for rural populations. The capability map includes capabilities
related to telehealth, such as ``Telehealth Infrastructure,'' ``Provider
Engagement,'' and ``Patient Access.'' The SMA uses a heat map to
highlight these capabilities in blue, indicating their alignment with
the state-specific initiative.

By analyzing the capability map, the SMA identifies that ``Provider
Engagement'' is a critical capability that requires further development
to support the telehealth initiative. The agency decides to invest in
training programs and outreach efforts to engage providers in rural
areas, ensuring that the telehealth services are effectively delivered
to the target population.

These examples demonstrate how capability maps, combined with heat
mapping and the MITA Maturity Model, can provide valuable insights for
strategic planning and gap analysis. By visualizing priorities and
maturity levels, SMAs can make informed decisions about where to focus
resources and efforts, ultimately enhancing their Medicaid Enterprise
Systems and achieving strategic objectives.

\begin{itemize}
\tightlist
\item
  \textbf{Capability Mapping}

  \begin{itemize}
  \tightlist
  \item
    Introduction to capability mapping and its significance
  \item
    How capabilities are organized and detailed at various levels of
    abstraction
  \end{itemize}
\end{itemize}

\section{Guidance on reuse of the MITA Capability
Model}\label{guidance-on-reuse-of-the-mita-capability-model-6}

\begin{itemize}
\tightlist
\item
  \textbf{Practical Application}

  \begin{itemize}
  \tightlist
  \item
    How to integrate the capability model into daily operations and
    strategic planning
  \item
    Tips for maximizing the benefits of the model
  \end{itemize}
\item
  \textbf{Continuous Improvement}

  \begin{itemize}
  \tightlist
  \item
    Encouragement for ongoing assessment and refinement of capabilities
  \item
    Leveraging feedback and performance data for model enhancement
  \end{itemize}
\item
  \textbf{Implementation Guidance}

  \begin{itemize}
  \tightlist
  \item
    Steps for adopting the capability model
  \item
    Resources and support available for SMAs
  \end{itemize}
\item
  \textbf{Performance Monitoring and Reporting}

  \begin{itemize}
  \tightlist
  \item
    Role of the capability model in tracking and enhancing performance
  \item
    Use of metrics and standards to measure capability effectiveness
  \end{itemize}
\end{itemize}

\chapter{MITA Capability Model}\label{mita-capability-model-7}

\section{Introduction to Business Capability
Models}\label{introduction-to-business-capability-models-6}

A capability model is a conceptual framework that outlines the key
capabilities an organization needs to achieve its strategic objectives.
It provides a comprehensive view of what an organization can do and
helps identify areas for improvement or investment. In the context of an
orchestra, a capability model might help the orchestra identify the set
of skills and resources, or other types of capabilities it needs to
perform a symphony. Just like an orchestra needs well practiced
musicians, sheet music, instruments, a conductor, and an audience to
produce a great symphony, a State Medicaid Agency (SMA) needs its
Medicaid Enterprise System (MES) to employ or develop specific
capabilities to deliver its services effectively, efficiently, and
economically to its enrollees and providers.

The concept of a business capability is extensively used within
enterprise architecture modeling and has been broadly used within
Business Capability Models as a tool to better align the business
strategy and information technology of both private sector and
governmental organizations since they emerged in the mid-2000s. One
example comes from the TOGAF Standard, a well-known standard in
enterprise architecture. Like most architecture frameworks TOGAF defines
a capability as something a business can do to meet its goals. This
focuses a strategic lens of an organization on ``what'' it needs to
achieve its goals, rather than ``how'' those goals are achieved. This
perspective allows for business planning from different viewpoints,
facilitating strategic alignment and operational efficiency.

SMA business architects, technologists, systems analysts, executives,
managers, and program staff can use this same modeling approach to
represent the functional components of their Medicaid Enterprise System
(MES) in ways that can help reveal gaps in their systems and provide
insights on what new or enhanced capabilities might be needed to close
those gaps.

By focusing on capabilities, SMAs can better align their information and
technology resources and processes with their strategic business goals,
ultimately improving their insight into how to improve the outcomes
their Medicaid Enterprise Architecture produces.

\subsection{Purpose}\label{purpose-10}

\begin{tcolorbox}[enhanced jigsaw, toprule=.15mm, colback=white, colframe=quarto-callout-note-color-frame, left=2mm, arc=.35mm, opacityback=0, rightrule=.15mm, breakable, bottomrule=.15mm, leftrule=.75mm]
\begin{minipage}[t]{5.5mm}
\textcolor{quarto-callout-note-color}{\faInfo}
\end{minipage}%
\begin{minipage}[t]{\textwidth - 5.5mm}

\vspace{-3mm}\textbf{Note}\vspace{3mm}

MITA 4.0 does not endeavor to specify all of the capabilities SMA's may
need to administer Medicaid programs; instead, this version of MITA
focuses on the capabilities that are most closely oriented towards
achieving the CMS-required outcomes.

\end{minipage}%
\end{tcolorbox}

Understanding the how the MITA Capability Model works is important to
obtaining the most value out of many of the other tools and artifacts in
the MITA framework, such as the MITA Maturity Model (MMM) and the
Business Process Model (BPM). The MITA Capability Model provides a
structured way for SMAs to identify, conceptually model, and improve the
capabilities needed for efficient Medicaid operations.

It is important to note that MITA 4.0 does not endeavor to specify all
of the capabilities SMA's may need to administer Medicaid programs;
instead, this version of MITA focuses on the capabilities that are most
closely oriented towards achieving the CMS-required outcomes. In this
way MITA 4.0 provides a reference model for SMAs to model other
capabilities that may be needed to achieve their other goals such as
state specific outcomes, or other state priorities while providing more
guidance within the MITA Framework to support modular.

\subsection{Update to MITA 3.0}\label{update-to-mita-3.0-7}

MITA 3.0 defined a capability as the level of maturity of a set of
business processes within a business category. By focusing on ``how''
MES operate MITA 3.0 helped SMA's identify ways to improve and mature
their business processes, but it did not link those processes with the
outcomes they are intended to achieve or ensure better alignment of the
information and technical architectures to business outcomes. The
addition of the MITA capability model to the MITA 4.0 business
architecture addresses that by providing the conceptual linkages needed
to elevate the strategic vantage point of the MITA Framework. To guide
this change, we present within this chapter a definition, description,
and approach to modeling business capabilities, based on the widely used
capability models contextualized for Medicaid Enterprises.

The business processes that operationalize MITA capabilities remain
foundational to characterizing the business architecture, and are by
definition a constituent part of any MITA capability. They provide
essential information on how capabilities are operationalize and should
continue to be a routinely utilized reference model for SMA business
process mapping. They are found with in the Business Process Model
chapter of this version of MITA.

\section{The MITA Definition of
Capability}\label{the-mita-definition-of-capability-7}

Within the context of MITA, a capability can be defined as the ability
or capacity of a State Medicaid Agency to achieve a desired outcome in
compliance with the
\href{https://www.ecfr.gov/current/title-42/chapter-IV/subchapter-C/part-433/subpart-C/section-433.112}{Standards
and Conditions within 42 CFR 433.112}. A capability may currently exist
in an operational state or be envisioned for future development. Through
careful planning, capabilities defined in this way can be matured and
refined over time to become more effective and efficient. They can be
organized and detailed at various levels of abstraction, providing
precise descriptions for operational purposes or more generalized views
for strategic planning.

\begin{tcolorbox}[enhanced jigsaw, toprule=.15mm, colback=white, colframe=quarto-callout-note-color-frame, left=2mm, arc=.35mm, opacityback=0, rightrule=.15mm, breakable, bottomrule=.15mm, leftrule=.75mm]

\vspace{-3mm}\textbf{Key Definition}\vspace{3mm}

\ldots a capability is defined as the ability or capacity of a SMA to
achieve a desired outcome\ldots{}

\end{tcolorbox}

To fully define a business capability, it is essential to understand how
it is realized through the integration of people, processes,
information, and technology resources of an SMA. While these elements of
the capability can change regularly, the capability itself is should
endure over longer planning horizons, supporting the long-term alignment
of business and IT and the achievement of increasingly beneficial
business outcomes.

\subsection{Structure of the MITA Capability
Model}\label{structure-of-the-mita-capability-model-7}

As depicted in the model below, the MITA Capability Model orients the
people, process, technology, and information resources to define a MITA
Capability. This means that to model a capability the appropriate
components of the information architecture and the technical
architecture must be brought together with the business architecture to
fully formulate any MITA Capability.

\begin{figure}[H]

{\centering \pandocbounded{\includegraphics[keepaspectratio]{media/capabilityModel/topLevelCapabilityMetamodelGraphic1.png}}

}

\caption{MITA Capability Relationship Diagram}

\end{figure}%

\subsubsection{Business Roles}\label{business-roles-7}

Business roles represent individual actors, stakeholders, or partners
involved in delivering a business capability. A single organizational
group or team may be wholly responsible for delivering the capability,
or multiple business entities may share the delivery of a particular
business capability. Business Roles perform Business Processes using
Technology Resources. They require skills and knowledge resources to
achieve outcomes, and should be actively engaged as partners in the
development or enhancement of any capability they help deliver.

\subsubsection{Business Processes}\label{business-processes-7}

Individual business capabilities may be enabled or delivered through a
range of business processes that detail the activities (the how)
associated with delivering the capability. Identifying and analyzing the
efficiency of the underlying processes helps to optimize the business
capability's effectiveness. Identifying the processes within a business
capability provides a focus for maturing the capability in concert with
the other capability components. Business Processes operationalize
Business Capabilities.

\subsubsection{Information/Data}\label{informationdata-7}

Information/data represents the business data, knowledge, and insight
consumed or produced by the business capability (as distinct from
IT-related data entities). This may also include information that the
capability exchanges with other capabilities to support the execution of
value streams. Examples include information about customers and
prospects, products and services, business policies and rules, sales
reports, and performance metrics. Information/data inform the Business
Capability, answering questions and supporting business rules.

\subsubsection{Technology Resources}\label{technology-resources-7}

Business capabilities rely on a range of tools, applications, systems,
and services for successful execution. Technology Resources use
Information/data to facilitate Business Processes. Such resources may
include:

\begin{itemize}
\tightlist
\item
  Modular software applications

  \begin{itemize}
  \tightlist
  \item
    Cloud or on-premise infrastructure
  \item
    Microservices
  \item
    Analytics
  \item
    Customer portal
  \end{itemize}
\end{itemize}

In this way we can clearly interrelate all of the MITA architecture
models and their individual components which allows us to reveal gaps
not only in the individual components of the architecture, but also
understand their impact on the integration of the architecture
components at the capability level.

\subsection{Relationship of MITA Capabilities to
Outcomes}\label{relationship-of-mita-capabilities-to-outcomes-7}

In the context of the Medicaid Information Technology Architecture
(MITA), outcomes are intrinsically linked to capabilities, as they
represent the tangible results achieved through the effective
integration and execution of various elements that constitute a
capability. In this sense, outcomes and capabilities define each other.

\begin{figure}[H]

{\centering \pandocbounded{\includegraphics[keepaspectratio]{media/capabilityModel/topLevelCapabilityMetamodelOutcomes.png}}

}

\caption{MITA Capability and Outcome Relationship Diagram}

\end{figure}%

\subsubsection{Outcomes}\label{outcomes-7}

MITA defines outcomes broadly to encompass CMS-required outcomes,
state-specific outcomes, and other outcomes not mandated as part of the
Advance Planning Document (APD) process. The sole criterion for an
outcome to meet this definition is that it must be a goal of a State
Medicaid Agency (SMA) and be achieved through a Medicaid Enterprise
System (MES) capability.

\begin{tcolorbox}[enhanced jigsaw, toprule=.15mm, colback=white, colframe=quarto-callout-note-color-frame, left=2mm, arc=.35mm, opacityback=0, rightrule=.15mm, breakable, bottomrule=.15mm, leftrule=.75mm]

\vspace{-3mm}\textbf{Key Definition}\vspace{3mm}

A MITA outcome is a goal of a State Medicaid Agency (SMA) that is
achieved by a Medicaid Enterprise System (MES) capability.

\end{tcolorbox}

\subsubsection{Measure}\label{measure-7}

Measure is a quantifiable metric used to assess the effectiveness and
efficiency of capabilities within a Medicaid Enterprise System (MES).
Measures provide quantifiable and qualitative values that help State
Medicaid Agencies (SMAs) track progress toward achieving specific
outcomes, such as CMS-required or state-specific goals. These indicators
might include metrics like processing times, error rates, or compliance
levels.

Measures are a measurement threshold by establishing a specific value or
level that must be met or exceeded to demonstrate successful
performance. For instance, a KPI might set a threshold for the maximum
allowable processing time for claims, ensuring that they are handled
within a specified timeframe to maintain compliance and eligibility for
enhanced federal funding. By monitoring these thresholds, organizations
can ensure they are meeting regulatory requirements and delivering
high-quality services to beneficiaries, while also identifying areas for
improvement.

\subsubsection{Measure Threshold}\label{measure-threshold-7}

A specific value or level of a measure that must be met or exceeded to
demonstrate the effective achievement of a capability's intended
outcome. This threshold serves as a benchmark for assessing whether the
processes, roles, and resources integrated within a Medicaid Enterprise
System (MES) are functioning optimally to meet the goals of a State
Medicaid Agency (SMA). For example, a measurement threshold might be set
for processing times, where claims must be processed within a certain
number of days to ensure compliance with CMS-required outcomes and
maintain eligibility for enhanced federal funding. By establishing and
monitoring these thresholds, organizations can ensure they are meeting
regulatory requirements and delivering high-quality services to
beneficiaries.

\subsubsection{Measurement}\label{measurement-7}

These outcomes and metrics are also used to ensure that healthcare
systems or modules comply with applicable federal regulations, forming
the baseline for system or module functionality. Achieving these
outcomes is essential for continuing to receive enhanced federal funding
for operations. Regular measurement and analysis of KPIs help
organizations demonstrate compliance and effectiveness, ensuring that
they meet regulatory requirements and continue to deliver high-quality
services to beneficiaries.

In this way we can clearly interrelate all of the MITA architecture
models and their individual components with the KPIs, thresholds, and
measurements that indicate whether our capability achieves our desired
outcome.

While models that help conceptualize the capabilities that achieve
CMS-required outcomes are the ones modeled for this version of MITA,
SMAs are encouraged to use these models as a reference to model
capabilities.

\section{Capability Mapping}\label{capability-mapping-7}

Capability mapping is a strategic tool that enables organizations, such
as State Medicaid Agencies (SMAs), to systematically identify, organize,
and visualize the key capabilities necessary to achieve their
objectives. Within the MITA framework, capability mapping provides SMAs
with a method of developing comprehensive views of the functions and
processes required to deliver Medicaid services effectively. To begin
the capability mapping process, SMAs should first identify the core
capabilities that align with their strategic objectives, focusing on
what the organization needs to achieve rather than how those goals are
accomplished. This involves listing all necessary capabilities and
understanding the desired outcomes they support. Next, these
capabilities should be organized into domains and areas that reflect
their strategic importance and interrelationships. Visualizing these
capabilities through diagrams or maps provides all stakeholders a common
view to understand the roles, processes, technology resources, and
information/data involved in executing each capability, as well as the
outcome each capability is designed to achieve. This structured approach
not only highlights areas for improvement or investment but also ensures
that organizational efforts are strategically aligned with desired
outcomes.

The benefits of capability mapping are multifaceted, offering SMAs a
clear pathway to strategic alignment and gap analysis. By visualizing
capabilities, organizations can identify operational gaps and determine
what new or enhanced capabilities are needed to close those gaps. This
visualization also improves communication among stakeholders by
providing a clear and concise representation of the organization's
functions. To refine capabilities, SMAs should analyze current
operations, assess the efficiency of underlying processes, and optimize
them to enhance capability effectiveness. Additionally, capability
mapping serves as a foundation for heat mapping, which assesses the MITA
Framework will utilize to visualize the maturity of each capability
evaluated in the State Self-Assessment. SMAs can overlay heat maps over
their capability maps to visualize many things other than maturity
levels, using color coding to indicate areas of strength and weakness.
Regular updates to these maps allow SMAs to monitor progress and ensure
resources are allocated effectively to achieve strategic goals. The MITA
framework includes examples of capability maps based on CMS-required
outcomes, serving as a reference model for SMAs to develop their own
capability maps tailored to state-specific goals and priorities. By
leveraging the reference models provided by MITA, SMAs can ensure their
capability mapping efforts are aligned with both federal requirements
and state-specific priorities.

\subsection{Organizing Capabilities}\label{organizing-capabilities-7}

To enhance the resolution and detail of a capability and provide a
unified view of all its components, a block diagram can be employed to
provide a common view of any MES. This diagram effectively links the
capability to business processes, roles, technical resources, and
information resources through functional decomposition. By breaking down
the capability into its constituent parts, the block diagram offers a
visual representation that highlights the interrelationships and
dependencies among these elements. This approach provides a clearer
understanding of how each component contributes to the overall
capability, facilitating more effective analysis, optimization, and
alignment with organizational objectives.

\pandocbounded{\includegraphics[keepaspectratio]{media/capabilityModel/capabilityOgranizationModel.drawio.png}}

We use this same method to present an this top level view of the
capabilities required to achieve CMS-required outcomes. From this view
increasingly detailed models can be constructed.

\pandocbounded{\includegraphics[keepaspectratio]{media/capabilityModel/mesModuleBasedCapabilities.drawio.png}}

\subsection{MITA Capability Models}\label{mita-capability-models-7}

The MITA framework represents capabilities visually through a layered
model that represent a capability of being composed of sub-capabilities
and the processes, roles, information and technology resources (PRIT)
that support the business in sustaining the capability. Each layer up
depicts increasingly strategic capabilities and each layer down depicts
the constituent elements that compose a capability in increasing
operational detail. It is not the intention of this version of MITA to
provide a full operational or tactical view of a capability, though SMAs
may consider using this approach to improve their organizational
awareness of their operations by developing further layers of their
capabilities through functional decomposition.

\pandocbounded{\includegraphics[keepaspectratio]{media/capabilityModel/capabilityLevels.png}}

\begin{itemize}
\tightlist
\item
  \textbf{Capability Domains:} The first layer of this model aims to
  group capabilities to organize the strategic view of an SMA's
  capabilities. In this view one or many capabilities can be grouped
  within a domain to indicate the pursuit of common outcomes. Each
  domain is denoted with a single number to help annotate each
  capability.

  \begin{itemize}
  \tightlist
  \item
    \textbf{Capability Areas:} The second layer of this model aims to
    provide a view of the groups of capabilities that compose a domain.
    They are organized to show capabilities that serve a specific group
    of similar outcomes and essential
  \item
    \textbf{Capabilities:} The third layer of this model provides a more
    detailed view view of
  \end{itemize}
\end{itemize}

\pandocbounded{\includegraphics[keepaspectratio]{media/capabilityModel/capabilityLevels2.png}}

\subsection{Relationship of MITA Capabilities to
Maturity}\label{relationship-of-mita-capabilities-to-maturity-7}

\begin{tcolorbox}[enhanced jigsaw, toprule=.15mm, leftrule=.75mm, colframe=quarto-callout-warning-color-frame, left=2mm, arc=.35mm, titlerule=0mm, rightrule=.15mm, opacitybacktitle=0.6, bottomtitle=1mm, toptitle=1mm, colbacktitle=quarto-callout-warning-color!10!white, bottomrule=.15mm, title=\textcolor{quarto-callout-warning-color}{\faExclamationTriangle}\hspace{0.5em}{Warning}, opacityback=0, breakable, colback=white, coltitle=black]

Under development.

\end{tcolorbox}

\begin{itemize}
\tightlist
\item
  \textbf{Levels of Maturity}

  \begin{itemize}
  \tightlist
  \item
    Description of the five levels of maturity in the MITA framework
  \item
    How capabilities evolve and mature over time
  \end{itemize}
\end{itemize}

\pandocbounded{\includegraphics[keepaspectratio]{media/capabilityModel/maturityModel.png}}

\subsection{Using Capability Maps for Heat Mapping Strategic Priorities
and Identifying Gaps with the MITA Maturity
Model}\label{using-capability-maps-for-heat-mapping-strategic-priorities-and-identifying-gaps-with-the-mita-maturity-model-7}

Capability maps are powerful tools that not only provide a visual
representation of an SMA's key capabilities but also serve as a
foundation for strategic analysis and planning. There are many
approaches to heat mapping capabilities, each offering unique insights
into organizational priorities and gaps. Here, we describe two
approaches: assessing maturity levels using the MITA Maturity Model and
prioritizing strategic outcomes.

\subsubsection{Identifying Gaps with the MITA Maturity
Model}\label{identifying-gaps-with-the-mita-maturity-model-7}

The MITA Maturity Model provides a framework for assessing the maturity
of an organization's capabilities across various dimensions, such as
business processes, information, and technology. By integrating the
maturity model with capability maps, SMAs can identify gaps between
their current state and desired maturity levels.

\paragraph{Example 1: Identifying Gaps in Data Management Maturity Using
the PRIT
Model}\label{example-1-identifying-gaps-in-data-management-maturity-using-the-prit-model-7}

An SMA is conducting an assessment of its data management capabilities
using the MITA Maturity Model, with a focus on the PRIT (Processes,
Roles, Information, and Technology) framework. The capability map
includes various data-related capabilities, such as ``Data
Integration,'' ``Data Quality Management,'' and ``Data Analytics.'' Each
of these capabilities is evaluated across the PRIT dimensions to
determine their maturity levels using the revised scale:

Processes: Level 1: Ad-Hoc Roles: Level 2: Compliant Information: Level
2: Compliant Technology: Level 2: Compliant The capability map is
updated to reflect the maturity assessment, with each dimension marked
with a color code: red for Level 1: Ad-Hoc, yellow for Level 2:
Compliant, green for Level 3: Efficient, blue for Level 4: Optimized,
and purple for Level 5: Pioneering. This visualization helps the SMA
prioritize strategic actions to enhance the ``Data Integration''
capability, such as standardizing processes, refining roles, improving
data quality, and upgrading technology.

\subsubsection{Heat Mapping Strategic
Priorities}\label{heat-mapping-strategic-priorities-7}

Heat mapping involves applying a color-coded overlay to a capability map
to visually represent the status or priority level of each capability.
This technique can be used to highlight areas of strength, weakness, or
strategic importance. For example, capabilities that are critical to
achieving CMS-required outcomes might be marked in one color, while
those needing immediate attention or improvement could be marked in
another. This visual representation helps stakeholders quickly grasp the
strategic landscape and make informed decisions about where to allocate
resources and focus efforts.

\paragraph{Example 2: Prioritizing Capabilities for CMS-Required
Outcomes}\label{example-2-prioritizing-capabilities-for-cms-required-outcomes-7}

An SMA is focused on achieving specific CMS-required outcomes related to
improving patient care and reducing administrative costs. The agency
creates a capability map that outlines all the capabilities necessary to
meet these outcomes. By applying a heat map, the SMA highlights
capabilities that are directly linked to these outcomes in green,
indicating they are of high strategic priority. Capabilities that are
indirectly related or less critical are marked in yellow, while those
that are currently underperforming or not aligned with strategic goals
are marked in red.

This visual representation allows the SMA to quickly identify which
capabilities require immediate attention and resources to ensure
compliance with CMS requirements. For instance, if the capability
related to ``Claims Processing Efficiency'' is marked in red, the agency
can prioritize initiatives to enhance this capability, such as investing
in new technology or streamlining processes.

\subsubsection{Other Uses for Capability Heat
Mapping}\label{other-uses-for-capability-heat-mapping-7}

Beyond assessing maturity levels and prioritizing strategic initiatives,
capability heat mapping can be applied in various other contexts to
enhance organizational effectiveness and alignment.

\paragraph{Example 3: Aligning Capabilities with State-Specific
Initiatives}\label{example-3-aligning-capabilities-with-state-specific-initiatives-7}

An SMA is working on a state-specific initiative to enhance telehealth
services for rural populations. The capability map includes capabilities
related to telehealth, such as ``Telehealth Infrastructure,'' ``Provider
Engagement,'' and ``Patient Access.'' The SMA uses a heat map to
highlight these capabilities in blue, indicating their alignment with
the state-specific initiative.

By analyzing the capability map, the SMA identifies that ``Provider
Engagement'' is a critical capability that requires further development
to support the telehealth initiative. The agency decides to invest in
training programs and outreach efforts to engage providers in rural
areas, ensuring that the telehealth services are effectively delivered
to the target population.

These examples demonstrate how capability maps, combined with heat
mapping and the MITA Maturity Model, can provide valuable insights for
strategic planning and gap analysis. By visualizing priorities and
maturity levels, SMAs can make informed decisions about where to focus
resources and efforts, ultimately enhancing their Medicaid Enterprise
Systems and achieving strategic objectives.

\begin{itemize}
\tightlist
\item
  \textbf{Capability Mapping}

  \begin{itemize}
  \tightlist
  \item
    Introduction to capability mapping and its significance
  \item
    How capabilities are organized and detailed at various levels of
    abstraction
  \end{itemize}
\end{itemize}

\section{Guidance on reuse of the MITA Capability
Model}\label{guidance-on-reuse-of-the-mita-capability-model-7}

\begin{itemize}
\tightlist
\item
  \textbf{Practical Application}

  \begin{itemize}
  \tightlist
  \item
    How to integrate the capability model into daily operations and
    strategic planning
  \item
    Tips for maximizing the benefits of the model
  \end{itemize}
\item
  \textbf{Continuous Improvement}

  \begin{itemize}
  \tightlist
  \item
    Encouragement for ongoing assessment and refinement of capabilities
  \item
    Leveraging feedback and performance data for model enhancement
  \end{itemize}
\item
  \textbf{Implementation Guidance}

  \begin{itemize}
  \tightlist
  \item
    Steps for adopting the capability model
  \item
    Resources and support available for SMAs
  \end{itemize}
\item
  \textbf{Performance Monitoring and Reporting}

  \begin{itemize}
  \tightlist
  \item
    Role of the capability model in tracking and enhancing performance
  \item
    Use of metrics and standards to measure capability effectiveness
  \end{itemize}
\end{itemize}

\part{Technical Architecture}

\chapter{Business Architecture
Introduction}\label{business-architecture-introduction-2}

\phantomsection\label{page-0-1}{}\phantomsection\label{page-0-0}{}\textbf{Part
I -- BUSINESS ARCHITECTURE Chapter 1 -- INTRODUCTION}

\pandocbounded{\includegraphics[keepaspectratio]{_page_0_Picture_1.jpeg}}

\pandocbounded{\includegraphics[keepaspectratio]{_page_0_Picture_2.jpeg}}

\pandocbounded{\includegraphics[keepaspectratio]{_page_0_Picture_3.jpeg}}

\pandocbounded{\includegraphics[keepaspectratio]{_page_0_Picture_4.jpeg}}

\subsubsection{\texorpdfstring{\textbf{Table of
Contents}}{Table of Contents}}\label{table-of-contents-2}

\begin{longtable}[]{@{}ll@{}}
\toprule\noalign{}
PART I--BUSINESS ARCHITECTURE & 1 \\
\midrule\noalign{}
\endhead
\bottomrule\noalign{}
\endlastfoot
Chapter 1 --Introduction & 1 \\
Introduction & 3 \\
Purpose & 3 \\
Scope & 4 \\
Background & 4 \\
Funding Requirements & 5 \\
BA Seven Standards and Conditions & 6 \\
Business Architecture Components & 7 \\
The Concept of Operations11 & \\
MITA Maturity Model11 & \\
Business Process Model12 & \\
12Business Capability Matrix & \\
14Business Architecture Component Relationships & \\
Connection Between Architectures15 & \\
17Using the Business Architecture & \\
Next Steps in Developing the Business Architecture18 & \\
& \\
\end{longtable}

\subsubsection{\texorpdfstring{\textbf{List of
Figures}}{List of Figures}}\label{list-of-figures-2}

\begin{longtable}[]{@{}
  >{\raggedright\arraybackslash}p{(\linewidth - 2\tabcolsep) * \real{0.9630}}
  >{\raggedright\arraybackslash}p{(\linewidth - 2\tabcolsep) * \real{0.0370}}@{}}
\toprule\noalign{}
\begin{minipage}[b]{\linewidth}\raggedright
Figure 1-1. MITA Framework Relationship Diagram
\end{minipage} & \begin{minipage}[b]{\linewidth}\raggedright
3
\end{minipage} \\
\midrule\noalign{}
\endhead
\bottomrule\noalign{}
\endlastfoot
Figure 1-2. BA in the Context of the MITA Framework & 8 \\
Figure 1-3. Relationship Among the Components of the Business
Architecture14 & \\
IA, and TA15Figure 1-4. Relationships Among Components of the BA, & \\
\end{longtable}

\subsubsection{\texorpdfstring{\textbf{List of
Tables}}{List of Tables}}\label{list-of-tables-2}

\begin{longtable}[]{@{}
  >{\raggedright\arraybackslash}p{(\linewidth - 2\tabcolsep) * \real{0.9577}}
  >{\raggedright\arraybackslash}p{(\linewidth - 2\tabcolsep) * \real{0.0423}}@{}}
\toprule\noalign{}
\begin{minipage}[b]{\linewidth}\raggedright
Table 1-1. Correlation of Seven Standards and Conditions with MITA
\end{minipage} & \begin{minipage}[b]{\linewidth}\raggedright
6
\end{minipage} \\
\midrule\noalign{}
\endhead
\bottomrule\noalign{}
\endlastfoot
Table 1-2. The Four Components of the Business Architecture. & 9 \\
13Table 1-3. Business Process Example: Authorize Service & \\
16Table 1-4. Component Relationships ofthe BA, IA, and TA & \\
17Table 1-5. Stakeholder Use of the Business Architecture & \\
\end{longtable}

\pandocbounded{\includegraphics[keepaspectratio]{_page_1_Picture_8.jpeg}}

\chapter[\textbf{Introduction}]{\texorpdfstring{\protect\hypertarget{page-2-0}{}{}\textbf{Introduction}}{Introduction}}\label{introduction-3}

The Medicaid IT Architecture (MITA) Framework contains three (3)
interrelated architectures: Business Architecture (BA), Information
Architecture (IA), and Technical Architecture (TA) shown in
\textbf{\hyperref[page-2-2]{Figure 1-1}}. The business capabilities from
BA define the data strategy of IA and design the business and technical
services of TA. MITA uses all three (3) architectures to develop a
business-driven enterprise to provide consistency across the State
Medicaid Enterprise.

\pandocbounded{\includegraphics[keepaspectratio]{_page_2_Figure_4.jpeg}}

\pandocbounded{\includegraphics[keepaspectratio]{_page_2_Figure_5.jpeg}}

\pandocbounded{\includegraphics[keepaspectratio]{_page_2_Figure_6.jpeg}}

\phantomsection\label{page-2-2}{}The topics covered in this chapter
include:

\begin{itemize}
\tightlist
\item
  BA Seven Standards and Conditions
\item
  Business Architecture Components
\item
  Business Architecture Component Relationships
\item
  Connection Between Architectures
\item
  Using the Business Architecture
\item
  Next Steps in Developing the Business Architecture
\end{itemize}

\section[\textbf{Purpose}]{\texorpdfstring{\protect\hypertarget{page-2-1}{}{}\textbf{Purpose}}{Purpose}}\label{purpose-11}

In keeping with the guiding principle that MITA represents a
business-driven enterprise transformation, the BA is the starting point
of the MITA Framework. The BA describes the needs and goals of the
individual State Medicaid Enterprise, and presents a collective vision
of the future.

The BA will accomplish the following:

\begin{itemize}
\tightlist
\item
  Establish a generic business framework for all States while
  recognizing their differences.
\item
  Describe how each state Medicaid Program can mature over a given
  period with the help of stakeholders, leadership, enabling
  legislation, and technology.
\item
  Provide a baseline for the State Medicaid Agency (SMA) to assess their
  current business capabilities and measure progress toward improved
  capabilities.
\end{itemize}

\pandocbounded{\includegraphics[keepaspectratio]{_page_2_Picture_20.jpeg}}

\subsection[\textbf{Scope}]{\texorpdfstring{\protect\hypertarget{page-3-0}{}{}\textbf{Scope}}{Scope}}\label{scope-3}

The BA focuses on the State Medicaid Enterprise that centers on the
Medicaid environment including leveraged systems and interconnections
among Medicaid stakeholders, providers, beneficiaries, insurance
affordability programs (e.g., CHIP, tax credits, Basic Health Program),
Health Insurance Exchange (HIX), Health Information Exchange (HIE),
other state and local agencies, other payers, Centers for Medicare \&
Medicaid Services (CMS), and other federal agencies. The MITA context
defines the Medicaid Enterprise as:

\begin{itemize}
\tightlist
\item
  The domain where federal matching funds apply.
\item
  The interfaces and bridges among Medicaid stakeholders, including
  providers, beneficiaries, other state and local agencies, other
  payers, CMS, and other federal agencies.
\item
  The sphere of influence touched by MITA (e.g., national and federal
  initiatives such as the Nationwide Health Information Network
  {[}NwHIN{]}). (See Front Matter, Chapter 6, Overview of the MITA
  Initiative, for a discussion of the Medicaid Enterprise.)
\end{itemize}

\emph{Enterprise can have other meanings. For instance, Enterprise
Architecture (EA) defines an enterprise-wide integrated set of
components that incorporates strategic business thinking, information
assets, and the technical infrastructure of an enterprise to promote
information sharing across agency and organizational boundaries.}

The BA acknowledges technology as one of several enablers that are
important to growth and transformation, but it does not introduce
technical implementations or solutions into the BA components. All
technical references are found in Part III, Technical Architecture.

\subsection[\textbf{Background}]{\texorpdfstring{\protect\hypertarget{page-3-1}{}{}\textbf{Background}}{Background}}\label{background-3}

States, territories, and the District of Columbia (hereinafter referred
to as States) are responsible for their individual State Medicaid
Enterprise, and all entities are different in important ways.
Differences include:

\begin{itemize}
\tightlist
\item
  Organizational structure, covered programs, and lines of business
\item
  Business rules, policies, and procedures affecting stakeholders
\item
  Relationships with other state and local agencies
\item
  Revenue sources
\item
  Location of business units
\item
  Workflow
\item
  Range of outsourcing
\item
  Technical solutions
\end{itemize}

\pandocbounded{\includegraphics[keepaspectratio]{_page_3_Picture_19.jpeg}}

These entities also differ in their concept of an enterprise, the roles
and responsibilities of one or more Chief Information Officers (CIO),
adoption of data and technical standards, and the use of legacy versus
state-of-the-art applications.

Given these differences, it is not possible or desirable, in the context
of the MITA Framework, to develop a standalone business and technical
model for each individual Medicaid Enterprise. Instead, MITA establishes
a national framework of common processes and enabling technologies to
support improved program administration in all States.

The BA focuses on areas of common ground (e.g., that all States will
enroll providers and pay for services rendered to eligible beneficiaries
and that all States seek to improve health care outcomes and improve
administrative processes).

There is no ready-made methodology for building the MITA Framework to
accommodate the business needs and transformation strategies of the
States. To meet the special needs of MITA, the components included in
the BA draw upon methodologies commonly in use today across industries
as diverse as financial, transportation, and defense. The MITA team
designed templates and models to help States identify and prioritize
their specific business needs.

The BA section of the MITA Framework shows how MITA incorporates
business-driven design to accomplish the following:

\begin{itemize}
\tightlist
\item
  Support state needs.

  \begin{itemize}
  \tightlist
  \item
    \textbf{o} Align with state strategic goals.
  \item
    \textbf{o} Align with state or Medicaid Agency enterprise
    architecture.
  \end{itemize}
\item
  Support the CMS and common state goals.

  \begin{itemize}
  \tightlist
  \item
    \textbf{o} Align state approaches with MITA.
  \item
    \textbf{o} Accommodate multi-state collaborative initiatives.
  \end{itemize}
\item
  Support national goals through alignment with national initiatives,
  such as the Office of the National Coordinator for Health Information
  Technology (ONC) and federal guidelines (e.g., Federal Health
  Architecture (FHA), the Federal Enterprise Architecture Framework
  (FEAF), and national/international data standards).
\end{itemize}

\section[\textbf{Funding
Requirements}]{\texorpdfstring{\protect\hypertarget{page-4-0}{}{}\textbf{Funding
Requirements}}{Funding Requirements}}\label{funding-requirements-3}

The Health and Human Services (HHS) CMS 42 CFR Part 433 Medicaid
Program; Federal Funding for Medicaid Eligibility Determination and
Enrollment Activities modifies Medicaid regulations for Mechanized
Claims Processing and Information Retrieval Systems effective April 19,
2011. The Medicaid Management Information System (MMIS) is a mechanized
claims processing and information retrieval system used by the States
for Title XIX of the Social Security Act (The Act); therefore, the
guidance set forth in CMS 42 CFR Part 433 applies to the MMIS as well as
the Medicaid eligibility determination and enrollment activities as set
forth in the Affordable Care Act of 2010.

CMS expects States to meet the standards and conditions specified in
§433.112(b)(10) through §433.112(b)(16). The standards and conditions
are descriptive in nature; however, CMS recognizes that in order for the
States to meet these standards and conditions it is necessary to provide
additional guidance that clearly articulates its criteria for meeting
them

\pandocbounded{\includegraphics[keepaspectratio]{_page_4_Picture_17.jpeg}}

in terms of timeliness, accuracy, efficiency, integrity, and performance
standards for mechanized claims processing. In response to this need,
additional guidance materials include:

\begin{itemize}
\tightlist
\item
  Enhanced Funding Requirements: Seven Conditions and Standards (a.k.a.
  Seven Standards and Conditions)
\item
  Guidance for Exchange and Medicaid Information Technology (IT) Systems
  (a.k.a. IT Guidance)
\end{itemize}

CMS will continue to refine, update and expand this guidance in the
future, based on feedback from stakeholders and with experience over
time.

\chapter[\textbf{BA Seven Standards and
Conditions}]{\texorpdfstring{\protect\hypertarget{page-5-0}{}{}\textbf{BA
Seven Standards and
Conditions}}{BA Seven Standards and Conditions}}\label{ba-seven-standards-and-conditions-3}

The MITA team evaluated and incorporated the 42 CFR Part 433 Medicaid
Program; Federal Funding for Medicaid Eligibility Determination and
Enrollment Activities in the BA for purposes of guiding the MITA
stakeholders to apply the guidance to the Medicaid Enterprise.

Each of the architectures aligns with the Seven Standards and
Conditions. By utilizing best practices, industry standards, and
technology advancements, the processes, and planning guidelines that
build the MITA framework provide a cohesive method for meeting Medicaid
objectives.

\textbf{\hyperref[page-5-1]{Table 1-1}} depicts the impact of the Seven
Standards and Conditions on the MITA BA, IA, and TA.

\phantomsection\label{page-5-1}{}

\begin{longtable}[]{@{}
  >{\raggedright\arraybackslash}p{(\linewidth - 8\tabcolsep) * \real{0.4581}}
  >{\raggedright\arraybackslash}p{(\linewidth - 8\tabcolsep) * \real{0.1677}}
  >{\raggedright\arraybackslash}p{(\linewidth - 8\tabcolsep) * \real{0.1871}}
  >{\raggedright\arraybackslash}p{(\linewidth - 8\tabcolsep) * \real{0.1742}}
  >{\raggedright\arraybackslash}p{(\linewidth - 8\tabcolsep) * \real{0.0129}}@{}}
\toprule\noalign{}
\begin{minipage}[b]{\linewidth}\raggedright
Correlation of Seven Standards and Conditions with MITA Architectures
\end{minipage} & \begin{minipage}[b]{\linewidth}\raggedright
\end{minipage} & \begin{minipage}[b]{\linewidth}\raggedright
\end{minipage} & \begin{minipage}[b]{\linewidth}\raggedright
\end{minipage} & \begin{minipage}[b]{\linewidth}\raggedright
\end{minipage} \\
\midrule\noalign{}
\endhead
\bottomrule\noalign{}
\endlastfoot
Standards and Conditions & BusinessArchitecture &
InformationArchitecture & TechnicalArchitecture & \\
Modularity Standard & X & X & X & \\
MITA Condition & X & X & X & \\
Industry Standards Condition & X & X & X & \\
Leverage Condition & X & X & X & \\
Business Results Condition & X & X & X & \\
Reporting Condition & X & X & X & \\
Interoperability Condition & X & X & X & \\
\end{longtable}

\subsection{\texorpdfstring{\textbf{Table 1-1. Correlation of Seven
Standards and Conditions with
MITA}}{Table 1-1. Correlation of Seven Standards and Conditions with MITA}}\label{table-1-1.-correlation-of-seven-standards-and-conditions-with-mita-3}

\pandocbounded{\includegraphics[keepaspectratio]{_page_5_Picture_12.jpeg}}

The BA includes:

\begin{itemize}
\tightlist
\item
  \textbf{Modularity Standard} Uses a modular, flexible approach to
  systems development, including the use of open interfaces and exposed
  Application Programming Interfaces (API); the separation of business
  rules from core programming; and the availability of business rules in
  both human and machine-readable formats. The States commit to formal
  system development methodology and open, reusable system architecture.
\item
  \textbf{MITA Condition} States align to and advance increasingly in
  MITA maturity for business, architecture, and data.
\end{itemize}

\textbf{Industry Standards Condition} - Ensures alignment with, and
incorporation of, industry standards: the Health Insurance Portability
and Accountability Act of 1996 (HIPAA) security, privacy and transaction
standards; accessibility standards established under section 508 of the
Rehabilitation Act, or standards that provide greater accessibility for
individuals with disabilities, and compliance with Federal Civil Rights
laws; standards adopted by the Secretary under section 1104 of the
Affordable Care Act; and standards and protocols adopted by the
Secretary under section 1561 of the Affordable Care Act.

\begin{itemize}
\tightlist
\item
  \textbf{Leverage Condition} States solutions should promote sharing,
  leverage, and reuse of Medicaid technologies and systems within and
  among States.
\item
  \textbf{Business Results Condition} Systems should support accurate
  and timely processing of claims (including claims of eligibility),
  adjudications, and effective communications with providers,
  beneficiaries, and the public.
\item
  \textbf{Reporting Condition} Solutions should produce transaction
  data, reports, and performance information that contribute to program
  evaluation, continuous improvement in business operations, and
  transparency and accountability.
\item
  \textbf{Interoperability Condition} Systems must ensure seamless
  coordination and integration with the Exchange (whether run by the
  state or federal government), and allow interoperability with health
  information exchanges, public health agencies, human services
  programs, and community organizations providing outreach and
  enrollment assistance services.
\end{itemize}

\chapter[\textbf{Business Architecture
Components}]{\texorpdfstring{\protect\hypertarget{page-6-0}{}{}\textbf{Business
Architecture
Components}}{Business Architecture Components}}\label{business-architecture-components-3}

The BA is a conceptual construct that encompasses models, matrices, and
templates. These components derive from a variety of industry standards
because no single methodology exists that meets the scope of MITA. The
MITA Framework breaks new ground and is a model for other federal,
state, and local entities.

The MITA BA contains the following components:

\begin{itemize}
\tightlist
\item
  Concept of Operations
\item
  MITA Maturity Model
\item
  Business Process Model
\item
  Business Capability Matrix
\end{itemize}

\pandocbounded{\includegraphics[keepaspectratio]{_page_6_Picture_17.jpeg}}

These are living models that evolve with the MITA Framework life cycle.
The MITA team tailored the level of detail in each model to meet the
specific needs of the intended audience.
\textbf{\hyperref[page-7-0]{Figure 1-2}} provides an overview of the
components of the BA. See Part I, Chapters 2 through 5 for a more
detailed description for each of these components.

\pandocbounded{\includegraphics[keepaspectratio]{_page_7_Figure_3.jpeg}}

\textbf{Figure 1-2}. \textbf{BA in the Context of the MITA Framework}

\phantomsection\label{page-7-0}{}The MITA Framework focuses on the
common ground shared by various distinct State Medicaid Enterprises and
yet accommodates their differences. The BA consists of four (4)
components that are summarized in \textbf{\hyperref[page-8-0]{Table}
1-2}. The BA is a composite of interrelated models and templates.

\pandocbounded{\includegraphics[keepaspectratio]{_page_7_Picture_6.jpeg}}

\subsection{\texorpdfstring{\textbf{Table 1-2. The Four Components of
the Business
Architecture.}}{Table 1-2. The Four Components of the Business Architecture.}}\label{table-1-2.-the-four-components-of-the-business-architecture.-3}

\phantomsection\label{page-8-0}{}

\begin{longtable}[]{@{}
  >{\raggedright\arraybackslash}p{(\linewidth - 10\tabcolsep) * \real{0.0465}}
  >{\raggedright\arraybackslash}p{(\linewidth - 10\tabcolsep) * \real{0.2903}}
  >{\raggedright\arraybackslash}p{(\linewidth - 10\tabcolsep) * \real{0.3977}}
  >{\raggedright\arraybackslash}p{(\linewidth - 10\tabcolsep) * \real{0.2614}}
  >{\raggedright\arraybackslash}p{(\linewidth - 10\tabcolsep) * \real{0.0021}}
  >{\raggedright\arraybackslash}p{(\linewidth - 10\tabcolsep) * \real{0.0021}}@{}}
\toprule\noalign{}
\begin{minipage}[b]{\linewidth}\raggedright
Business Architecture Components
\end{minipage} & \begin{minipage}[b]{\linewidth}\raggedright
\end{minipage} & \begin{minipage}[b]{\linewidth}\raggedright
\end{minipage} & \begin{minipage}[b]{\linewidth}\raggedright
\end{minipage} & \begin{minipage}[b]{\linewidth}\raggedright
\end{minipage} & \begin{minipage}[b]{\linewidth}\raggedright
\end{minipage} \\
\midrule\noalign{}
\endhead
\bottomrule\noalign{}
\endlastfoot
Component & TypeofModel & Function & Relationship & & \\
ConceptofOperations(COO)COO &
TheCOOdescribescurrentoperations,avisionoftransformation,transformationstostakeholderrolesandinformationexchanges,andtheinfluenceofenablers(e.g.,newpolicy,legislation,technology).
&
EstablishesavisionfortransformationoftheStateMedicaidEnterprise.Linksenablerstotheimprovementsinbusinessprocesses.Showshowstakeholders'roleschange.Showshowprocessesanddatachange.FocusesonimprovementsintheSMAoperations.
&
EstablishesthetargetsandvisionthatotherBAcomponentswilladdress.ProvidesaplatformandgroundingfortheMMM
andtheBusinessCapabilityMatrix(BCM). & & \\
MITAMaturityModel(MMM) &
Subdividedintofive(5)levelsofprogressivematurity,theMMMillustrateshowto
transformgoals,objectives,andbusinesscapabilitiesprogress. &
ShowshowtomeetStateMedicaidEnterprisegoalsandobjectivesandhowtoimprove
businessareas.Providesbase,consistency,andmeasuresforspecifyingdetailedbusinesscapabilitiesastheymature.
& MMM providesstructure to theCOO vision tobuild the
BCM.Providesaframeworkandmodelforthebusinesscapabilities.MMM aligns
withthe SevenStandards andConditionsrequirements. & & \\
BusinessProcessModel(BPM) & The BPM is acollection ofcommon
businessprocesses for theoperation ofMedicaidPrograms. &
Providesamodelofmajorbusinessareasandsubareas.Providesdetailed &
OriginatesfromtheSystemsTechnicalAdvisoryGroup(S-TAG)redesignoftheMedicaid
& & \\
\end{longtable}

\pandocbounded{\includegraphics[keepaspectratio]{_page_8_Picture_4.jpeg}}

Part I, Chapter 1 - Page 9 February 2012 Version 3.0

\begin{longtable}[]{@{}
  >{\raggedright\arraybackslash}p{(\linewidth - 8\tabcolsep) * \real{0.0410}}
  >{\raggedright\arraybackslash}p{(\linewidth - 8\tabcolsep) * \real{0.3053}}
  >{\raggedright\arraybackslash}p{(\linewidth - 8\tabcolsep) * \real{0.2462}}
  >{\raggedright\arraybackslash}p{(\linewidth - 8\tabcolsep) * \real{0.4055}}
  >{\raggedright\arraybackslash}p{(\linewidth - 8\tabcolsep) * \real{0.0019}}@{}}
\toprule\noalign{}
\begin{minipage}[b]{\linewidth}\raggedright
Business Architecture Components
\end{minipage} & \begin{minipage}[b]{\linewidth}\raggedright
\end{minipage} & \begin{minipage}[b]{\linewidth}\raggedright
\end{minipage} & \begin{minipage}[b]{\linewidth}\raggedright
\end{minipage} & \begin{minipage}[b]{\linewidth}\raggedright
\end{minipage} \\
\midrule\noalign{}
\endhead
\bottomrule\noalign{}
\endlastfoot
Component & TypeofModel & Function & Relationship & \\
&
Atemplatecapturesthedescriptionofeachbusinessprocess.Thebusinessprocessescovercurrentandnear-termoperations.
&
definitionsofcommonbusinessprocesses.Describesbusinessprocessesusingacommonvocabulary.Renderssomebusinessprocessesobsoleteathigherlevelsofmaturity.
&
ManagementInformationSystem(MMIS)model,variousstatemodels,andtheMedicaidHIPAA-CompliantConceptModel(MHCCM)andfederalregulation.BusinessprocessesunderreviewbytheNationalMedicaidEDIHealthcare(NMEH)workgroups.ReviewandrefinementprocessundercontinualreviewbyStates.
& \\
BusinessCapabilityMatrix(BCM) &
Subdividedintofive(5)levelsofmaturity,theBCMappliestheMMMtotheBPMtoderivecapabilitiesforeachbusinessprocessateachmaturitylevel.TheBCMdescribeshowto
transformand improve abusinessprocess. &
Showshoweachbusinessprocesscanimprove.ProvidesconsistencyandamodelfortheSMAtouseinmeasuringtheirownlevelsofmaturityforeachbusinessprocess.
& The BCM definessix (6) businesscapabilitiesacross five (5)levels of
maturityfor each
businessprocess.AlignswiththeMMMforthedescriptionofthecharacteristicsofthematuritylevels.FormstheevaluationcriteriafortheStateSelfAssessment(SSA).
& \\
\end{longtable}

\pandocbounded{\includegraphics[keepaspectratio]{_page_9_Picture_3.jpeg}}

\section[\textbf{The Concept of
Operations}]{\texorpdfstring{\protect\hypertarget{page-10-0}{}{}\textbf{The
Concept of
Operations}}{The Concept of Operations}}\label{the-concept-of-operations-3}

\pandocbounded{\includegraphics[keepaspectratio]{_page_10_Picture_3.jpeg}}

The COO is a tool used to describe current business operations and to
develop a future transformation that meets the needs of stakeholders and
responds to enablers (e.g., new policy, legislation, and technology).
Other industries (e.g., the Department of Defense (DOD) or National
Aeronautics and Space Administration (NASA)) use the COO as a
strategic-planning device to capture the As-Is (i.e., current)
operations,

create the To-Be (i.e., future) environment, and level-set expectations
before engaging in major transformation projects. The COO provides a
structure to place information gathered from interviews with States and
visioning sessions conducted at MMIS conferences. The COO structure
provides key information including:

\begin{itemize}
\tightlist
\item
  Definition of the scope of the Medicaid Enterprise.
\item
  Description of the As-Is (current) operations in terms of business,
  architecture, and data.
\item
  Description of the drivers and enablers that propel and support
  transformation.
\item
  Description of the To-Be environment in terms of business,
  architecture, and data.
\item
  Description of operational scenarios with sequence of events and
  activities carried out by stakeholders and the State Medicaid
  Enterprise.
\item
  Description of the impacts on each stakeholder.
\item
  Description of a summary of the improvements to the State Medicaid
  Enterprise and stakeholders.
\end{itemize}

The goal of the COO is to project changes, transformations, and provide
visions of To-Be operations, new roles and data exchanges for
stakeholders. The MITA COO provides a common vision shared by CMS and
the States that preserves individual adaptations at the state level.

Part I, Chapter 2, Concept of Operations, provides more information on
the Medicaid Enterprise COO. Part I, Appendix A, Concept of Operations
Details, contains additional information.

\section[\textbf{MITA Maturity
Model}]{\texorpdfstring{\protect\hypertarget{page-10-1}{}{}\textbf{MITA
Maturity Model}}{MITA Maturity Model}}\label{mita-maturity-model-3}

\pandocbounded{\includegraphics[keepaspectratio]{_page_10_Picture_16.jpeg}}

The MMM originates from industries that use such models to illustrate
how a business can mature. The MMM adapts the industry model to the
Medicaid Enterprise by describing Medicaid Program goals and objectives
and the maturation of the MITA technical principles. The

transformation through each of the five (5) levels represents
significant business capabilities advances over the previous period.

The MMM describes the five (5) levels of maturity and the measurable
qualities that each level demonstrates. The general description is at a
high enough level to apply to most aspects of State Medicaid Enterprise
operations. For example, the MMM defines, at Level 1, the business area
or process is in compliance with current regulations. At Level 2, the
process matures because of pressures for cost containment and
availability of newer tools. At Level 3, noticeable improvement occurs
in the standardization and sharing of information

\pandocbounded{\includegraphics[keepaspectratio]{_page_10_Picture_20.jpeg}}

and processes among multiple entities, including the beneficiary. At
Level 4, instant availability of clinical information increases the
transformation. By Level 5, States and local agencies have become
interoperable across the United States.

The MMM is the point of reference for the BCM. The BCM aligns with the
MMM to maintain consistency of definition. Part I, Chapter 3, Maturity
Model, presents details of the MMM, and Part I, Appendix B, Maturity
Model Details, contains the complete detailed text.

\subsection[\textbf{Business Process
Model}]{\texorpdfstring{\protect\hypertarget{page-11-0}{}{}\textbf{Business
Process Model}}{Business Process Model}}\label{business-process-model-4}

\pandocbounded{\includegraphics[keepaspectratio]{_page_11_Figure_5.jpeg}}

The BPM is a collection of common business processes for the operation
of Medicaid Programs. A template describes those processes, including
current and near-term operations as defined in Level 3 of the BCM. The
MITA Framework BPM

derives from multiple sources that create a common model that reflects
most State Medicaid Enterprises -- notable sources include the S-TAG
\emph{Redesign of the Medicaid Management Information System (MMIS),}
and the CMS MHCCM, that consolidates business processes from a dozen
States.

States should develop business workflows for the different business
functions of the state to advance the alignment of the state's
capability maturity with the MMM. These business workflows should align
to any provided by CMS in support of Medicaid and Exchange business
operations and requirements. States should work to streamline and
standardize these operational approaches and business workflows to
minimize customization demands on technology solutions and optimize
business outcomes.

There are those business processes that all States perform (e.g., Enroll
Provider) and those that are voluntary and depend on implementation of
special programs within a state (e.g., pay Managed Care Organization
capitation or enrollment of member in a special program). The BPM
defines common business practices across all State Medicaid Enterprises.
The MITA Framework BPM offers a hierarchy of Tier 1 business areas, Tier
2 business categories and Tier 3 business processes. The MITA Framework
contains ten (10) business areas divided into twenty-one (21) business
categories with eighty (80) individual business processes. See Part I,
Appendix C, Business Process Model Details.

The BPM provides a Business Process Template (BPT) for describing each
business process. The BPT provides a summary of the business process,
trigger event and result, activity steps, data requirements, predecessor
and successor processes, failure points, and other elements. The NMEH
workgroups review business processes, and they stand to benefit from
ongoing review by state workgroups. See Part I, Chapter 4, Business
Process Model, for a detailed presentation of the BPM and Part I,
Appendix C, Business Process Model Details, for the complete set of
business area definitions and business process descriptions.

\section[\textbf{Business Capability
Matrix}]{\texorpdfstring{\protect\hypertarget{page-11-1}{}{}\textbf{Business
Capability
Matrix}}{Business Capability Matrix}}\label{business-capability-matrix-3}

\pandocbounded{\includegraphics[keepaspectratio]{_page_11_Picture_12.jpeg}}

\pandocbounded{\includegraphics[keepaspectratio]{_page_11_Picture_13.jpeg}}

Applying the MMM to each business process yields the BCM that shows how
the business process matures. The BCM defines six (6) business
capabilities with five (5) levels of maturity to each business process.
The BCM assigns capabilities to an individual business process rather
than to SMA operations taken as a whole. In reality, no SMA is ``all
Level 1'' or ``all Level 2,'' but rather having

a blend of different levels of capability. An example of the
relationship among the business process, the MMM, and the BCM is shown
in \textbf{\hyperref[page-12-0]{Table} 1-3.}

Part I, Chapter 5, Business Capability Matrix, presents more information
on the BCM and Part I, Appendix D, Business Capability Matrix Details,
lists the capabilities defined for business processes contained in MITA
Framework.

\subsection{\texorpdfstring{\textbf{Table 1-3. Business Process Example:
Authorize
Service}}{Table 1-3. Business Process Example: Authorize Service}}\label{table-1-3.-business-process-example-authorize-service-3}

\phantomsection\label{page-12-0}{}

\begin{longtable}[]{@{}
  >{\raggedright\arraybackslash}p{(\linewidth - 6\tabcolsep) * \real{0.0603}}
  >{\raggedright\arraybackslash}p{(\linewidth - 6\tabcolsep) * \real{0.2496}}
  >{\raggedright\arraybackslash}p{(\linewidth - 6\tabcolsep) * \real{0.6868}}
  >{\raggedright\arraybackslash}p{(\linewidth - 6\tabcolsep) * \real{0.0034}}@{}}
\toprule\noalign{}
\begin{minipage}[b]{\linewidth}\raggedright
Authorize Service Business Process
\end{minipage} & \begin{minipage}[b]{\linewidth}\raggedright
\end{minipage} & \begin{minipage}[b]{\linewidth}\raggedright
\end{minipage} & \begin{minipage}[b]{\linewidth}\raggedright
\end{minipage} \\
\midrule\noalign{}
\endhead
\bottomrule\noalign{}
\endlastfoot
LevelNo. & MITAMaturityModelDefinition & BusinessCapability & \\
1 &
Complieswithregulations;mostlymanualactivities;delaysincommunicatingresults.
&
Receiptofandresponsetorequestsareprimarilyviapaper,fax,andphone;applypolicyguidelinesmanually;complieswithregulationsonturnaroundtimeandaccuracy.
& \\
2 &
Improvementsspearheadedbycostmanagementgoals;improvementsmadeinspeedofcommunicationandresponse.
&
Authorizationofservicegivengreaterpriorityasacost-managementtool;improvementsmadeincommunications;receiptofandresponsestorequestsmadeviaportal;adopt
HIPAAstandards. & \\
3 &
Informationandservicessharedwithotheragenciesandbeneficiary;streamlinedprocess;improvedresults.
&
Solutionsbecomereusableandsharablebecauseofadoptionofstandardsbystateagenciesanddata-sharingagreementstocollaborateonauthorizationofservices.
& \\
4 &
Incorporatesclinicalinformationintotheprocesstofurtherimproveresults. &
Directaccessbytheauthorizingagencytoaccess to
clinicalinformation;automationofrequests;render
decisionsbypayerautomaticallyasprovider
updatesbeneficiary'selectronichealthrecord;improve
accuracybecauseprovider basesdecisionsonclinicalevidence;limits
manualinterventiontoexceptions. & \\
5 &
Demonstrateswidespreadinteroperabilitytoachievemaximumimprovementsenvisionedatthistime.
&
Directaccessbytheauthorizingagencytoclinicalandadministrativeinformationanywhereinthecountrytoconfirmordenytheauthorizationforaservice.
& \\
\end{longtable}

\pandocbounded{\includegraphics[keepaspectratio]{_page_12_Picture_6.jpeg}}

\chapter[\textbf{Business Architecture Component
Relationships}]{\texorpdfstring{\protect\hypertarget{page-13-0}{}{}\textbf{Business
Architecture Component
Relationships}}{Business Architecture Component Relationships}}\label{business-architecture-component-relationships-3}

The four (4) components of the BA are interrelated:

\begin{itemize}
\tightlist
\item
  The COO serves as a model to frame a vision for Medicaid Program
  health care outcomes and operational efficiencies. It establishes the
  To-Be environment that becomes the goal of the Medicaid Enterprise
  transformation. The COO provides the vision for the MMM. It also
  supplies an overview for the BPM.
\item
  The MMM uses a common industry approach to describe the differences
  between five (5) levels of progressive maturity, ranging from As-Is
  operations to the To-Be environment. The MMM is the point of reference
  used by the BCM to describe the levels of maturity for a business
  process.
\item
  The BPM describes As-Is (i.e., current) Medicaid operations as defined
  for Level 3 of the BCM.
\item
  The BCM uses the five (5) levels of maturity described in the MMM and
  the To-Be environment defined in the COO to create definitions for
  business capabilities at five (5) levels of maturity for each business
  process.
\end{itemize}

\pandocbounded{\includegraphics[keepaspectratio]{_page_13_Figure_8.jpeg}}

\pandocbounded{\includegraphics[keepaspectratio]{_page_13_Figure_9.jpeg}}

\subsection{\texorpdfstring{\textbf{Figure 1-3. Relationship Among the
Components of the Business
Architecture}}{Figure 1-3. Relationship Among the Components of the Business Architecture}}\label{figure-1-3.-relationship-among-the-components-of-the-business-architecture-3}

\phantomsection\label{page-13-1}{}\pandocbounded{\includegraphics[keepaspectratio]{_page_13_Picture_11.jpeg}}

\chapter[\textbf{Connection Between
Architectures}]{\texorpdfstring{\protect\hypertarget{page-14-0}{}{}\textbf{Connection
Between
Architectures}}{Connection Between Architectures}}\label{connection-between-architectures-3}

The MITA Framework consists of three (3) interrelated BA, IA, and TA
components that work together to define a business-driven enterprise
transformation. The BA describes the business process activities along
with data input, data output, and required shared data. The IA provides
the bridge between the business need of information and the technical
solution data. The TA describes the technology enablers associated with
the business capabilities and their varied levels of maturity.

\textbf{\hyperref[page-14-1]{Figure 1-4}} illustrates how BA, IA, and TA
components interrelate. This is a high-level view of the primary
components within each architecture. Front Matter, Chapter 6,
Introduction to the MITA Framework, presents a detailed discussion on
the inter-relationship of all three (3) architectures. The BA
categorizes the business processes as business capabilities and assigned
a level of MITA maturity. Based on the level of maturity, the IA defines
the Conceptual Data Model (CDM) and Logical Data Model (LDM) with
necessary data attributes for the design of technical capabilities. The
TA defines the resulting business services and technical services for
the To-Be environment of the State Medicaid Enterprise.

\pandocbounded{\includegraphics[keepaspectratio]{_page_14_Figure_5.jpeg}}

\textbf{Figure 1-4. Relationships Among Components of the BA, IA, and
TA}

\phantomsection\label{page-14-1}{}The BA does not present specific
technical solutions or detailed data requirements. Some of its
components, however, point to specific companion components in the IA
and TA sections of MITA Framework (Parts II and III, respectively).
\textbf{\hyperref[page-15-0]{Table} 1-4} describes the name of the BA
Component and its relationship to the other architecture component as
well as its MITA Framework 3.0 documented location.

\pandocbounded{\includegraphics[keepaspectratio]{_page_14_Picture_8.jpeg}}

\subsubsection{\texorpdfstring{\textbf{Table 1-4. Component
Relationships of the BA, IA, and
TA}}{Table 1-4. Component Relationships of the BA, IA, and TA}}\label{table-1-4.-component-relationships-of-the-ba-ia-and-ta-3}

\phantomsection\label{page-15-0}{}

\begin{longtable}[]{@{}
  >{\raggedright\arraybackslash}p{(\linewidth - 8\tabcolsep) * \real{0.3900}}
  >{\raggedright\arraybackslash}p{(\linewidth - 8\tabcolsep) * \real{0.1876}}
  >{\raggedright\arraybackslash}p{(\linewidth - 8\tabcolsep) * \real{0.4165}}
  >{\raggedright\arraybackslash}p{(\linewidth - 8\tabcolsep) * \real{0.0030}}
  >{\raggedright\arraybackslash}p{(\linewidth - 8\tabcolsep) * \real{0.0030}}@{}}
\toprule\noalign{}
\begin{minipage}[b]{\linewidth}\raggedright
BA, IA, and TA Component Relationships
\end{minipage} & \begin{minipage}[b]{\linewidth}\raggedright
\end{minipage} & \begin{minipage}[b]{\linewidth}\raggedright
\end{minipage} & \begin{minipage}[b]{\linewidth}\raggedright
\end{minipage} & \begin{minipage}[b]{\linewidth}\raggedright
\end{minipage} \\
\midrule\noalign{}
\endhead
\bottomrule\noalign{}
\endlastfoot
BusinessArchitectureComponent & OtherArchitectureComponent &
Relationship & & \\
COO--DataExchanges & IA(PartII)--Allchapters &
IAchaptersprovidedetailsregardingthetransformationofdataandinformationidentifiedintheCOO.
& & \\
COO--Drivers andEnablers & TA(PartIII), Chapter2,Technical
ManagementStrategy;Chapter7,TechnicalCapabilityMatrix & Service-Oriented
Architectures(SOA)andTechnicalCapabilitiesareenablersreferencedintheCOO.
& & \\
BPM--TriggerEvent,Result,andSharedDataineachbusinessprocessdescribeingeneraltermsthekindofdatareceivedby,usedby,andresultingfromeachbusinessprocess
& IA(PartII),
Chapter2,DataManagementStrategy;Chapter3,ConceptualDataModel &
DataManagementStrategy(DMS)explainshowthedatasupportsthebusinessprocesses.TheCDMidentifiesgroupingsofinformationcommontoMedicaidbusinessareasandclustersofbusinessprocesses.
& & \\
BCM & IA(PartII), Chapter4,LogicalDataModel; Chapter 6Information
Capability Matrix & TheLDM
definesdataclassesandattributesneededtosupportdifferentlevelsofmaturity.AbusinessprocessdescribedataLevel3businesscapabilityrequiresLevel3dataattributes.
& & \\
BCM & TA(PartIII), Chapter7,TechnicalCapabilityMatrix &
TheBCMdrivestheTechnicalCapability Matrix (TCM).TAassociates technical
capabilitieswiththeBCMlevelwherespecifictechnologyisnecessarytosupportthebusinessprocess.
& & \\
BCM--Level3andabove & TA(PartIII), Chapter 2Technical
ManagementStrategy; Chapter3,BusinessServices &
Abusinessserviceisanimplementationofaspecificbusinessprocessataspecificlevelofcapability.TA
associatesbusiness servicesandSOAwithBCMLevel3andabove. & & \\
\end{longtable}

\pandocbounded{\includegraphics[keepaspectratio]{_page_15_Picture_4.jpeg}}

\chapter[\textbf{Using the Business
Architecture}]{\texorpdfstring{\protect\hypertarget{page-16-0}{}{}\textbf{Using
the Business
Architecture}}{Using the Business Architecture}}\label{using-the-business-architecture-3}

CMS requires States to align to and advance increasingly in MITA
maturity for business, architecture, and data. CMS expects States to use
the BA components to plan for improvements in the State Medicaid
Program, both in the delivery of services to providers and
beneficiaries, and in its internal operations and exchanges of
information with the other external stakeholders. BA provides the COO
and the MMM as background material. States and vendors use the BPM and
the BCM tools. \textbf{\hyperref[page-16-1]{Table} 1-5} summarizes how
stakeholders use the BA.

\phantomsection\label{page-16-1}{}

\begin{longtable}[]{@{}
  >{\raggedright\arraybackslash}p{(\linewidth - 4\tabcolsep) * \real{0.0861}}
  >{\raggedright\arraybackslash}p{(\linewidth - 4\tabcolsep) * \real{0.9104}}
  >{\raggedright\arraybackslash}p{(\linewidth - 4\tabcolsep) * \real{0.0035}}@{}}
\toprule\noalign{}
\begin{minipage}[b]{\linewidth}\raggedright
Stakeholder Useof the Business Architecture
\end{minipage} & \begin{minipage}[b]{\linewidth}\raggedright
\end{minipage} & \begin{minipage}[b]{\linewidth}\raggedright
\end{minipage} \\
\midrule\noalign{}
\endhead
\bottomrule\noalign{}
\endlastfoot
Stakeholder & HowStakeholders Use BA & \\
SMA & The
SMAmapstheiroperationstotheBPMandthenassessesthelevelofmaturityusingtheBCM.Whenthe
SMArequires
informationtechnologyupgradestosupportprogramimprovement,theSMAusestheSS-Atoshowhowit
will use
theenhancedfundingtoachieveaspecificresult(e.g.,movingfromLevel1or2toLevel3).
& \\
CMS &
CMSprovidesleadershipinestablishingtheMITAguidelinesandpromotingthemamongStates.ThroughthereleaseoftheMITAFramework,specialworkshopswithStates,Medicaidconferencematerial,andworkingwithearlyadopterStates,CMSprovides
guidance and principles to achieve the Medicaidvision. & \\
Vendors &
ThevendorcommunityusestheMITAFrameworkasareferenceinplanningtheirresearchanddevelopmentactivities.TheyusetheBA,inparticular,todeterminethematurityleveloffunctionssupportedbytheirsystems.Theyhaveacommonunderstandingofthe
CMS directionfor the Medicaid
Program,andtheycanshowhowtheirproductssupportMITAcapabilities. & \\
Providers & Providersplayanactiveroleintheexchangeofinformationwiththe
SMA.
TheycanlookattheSMABAtounderstandwhatdirectiontheSMAistakingandtokeepthisinmindastheyinvestininformation
technologyupgradesandreengineertheirpractices.Insomecases,the
SMAinvolveprovidersdirectlyinplanningaMedicaidProgramtransformation.
& \\
Beneficiaries & The BA supports the SMAperson-centric outreach,
eligibility and enrollmentactivities across the health and human
services spectrum.Beneficiariesandconsumergroupsare
abletolookattheSMABAandidentify thebenefits.AtLevel3business capability
maturity,beneficiariesareparticipantsinselfdirectedhealthcaredecisions.
& \\
Legislators,Governors &
StatesdeveloppresentationsbasedontheBAtoshowthegovernorandlegislatorswhatgoalsCMSisestablishingforStatesthatrequestenhanced
& \\
\end{longtable}

\subsection{\texorpdfstring{\textbf{Table 1-5. Stakeholder Use of the
Business
Architecture}}{Table 1-5. Stakeholder Use of the Business Architecture}}\label{table-1-5.-stakeholder-use-of-the-business-architecture-3}

\pandocbounded{\includegraphics[keepaspectratio]{_page_16_Picture_6.jpeg}}

\begin{longtable}[]{@{}
  >{\raggedright\arraybackslash}p{(\linewidth - 6\tabcolsep) * \real{0.1842}}
  >{\raggedright\arraybackslash}p{(\linewidth - 6\tabcolsep) * \real{0.8008}}
  >{\raggedright\arraybackslash}p{(\linewidth - 6\tabcolsep) * \real{0.0075}}
  >{\raggedright\arraybackslash}p{(\linewidth - 6\tabcolsep) * \real{0.0075}}@{}}
\toprule\noalign{}
\begin{minipage}[b]{\linewidth}\raggedright
Stakeholder Useof the Business Architecture
\end{minipage} & \begin{minipage}[b]{\linewidth}\raggedright
\end{minipage} & \begin{minipage}[b]{\linewidth}\raggedright
\end{minipage} & \begin{minipage}[b]{\linewidth}\raggedright
\end{minipage} \\
\midrule\noalign{}
\endhead
\bottomrule\noalign{}
\endlastfoot
Stakeholder & HowStakeholders Use BA & & \\
& funding. & & \\
OtherPayersandOtherAgencies & The MITA team invites other
payersandotheragenciestoreviewtheMITAFramework,especiallytheBA,tolearnabouttheMedicaidEnterprisetransformation.
& & \\
\end{longtable}

In general, MITA supports stakeholder roles and access to information,
technology that eliminates most manual activities, and the
transformation of the Medicaid business with the assistance of the CMS,
the SMA, providers, and beneficiaries. In addition, MITA supports
providers with instant access to patient records no matter what their
location is, patients can view their Personal Health Information (PHI)
and make informed decisions regarding treatment, and payers can view
clinical records nationally to expedite decisions on prior authorization
and payment.

\section[\textbf{Next Steps in Developing the Business
Architecture}]{\texorpdfstring{\protect\hypertarget{page-17-0}{}{}\textbf{Next
Steps in Developing the Business
Architecture}}{Next Steps in Developing the Business Architecture}}\label{next-steps-in-developing-the-business-architecture-3}

The MITA Framework delivers the starter kit for a controlled State
Medicaid Enterprise transformation. MITA will continue to evolve over
time. The business process defines the input and output of information
but not the details of the process; however the business community will
still decide the requirements for standardized triggers and results. The
CMS MITA team continues to support SMA efforts by serving as a conduit
for improvements to MITA models that all States and vendors can access.

The MITA Framework and the BA are ever evolving so that the SMA can
continuously improve the way they deliver services to beneficiaries and
providers, account for outcomes, reward participants based on
performance, and respond dynamically to requests for information.

\pandocbounded{\includegraphics[keepaspectratio]{_page_17_Picture_7.jpeg}}

\pandocbounded{\includegraphics[keepaspectratio]{_page_17_Picture_9.jpeg}}

\chapter{MITA Capability Model}\label{mita-capability-model-8}

\section{Introduction to Business Capability
Models}\label{introduction-to-business-capability-models-7}

A capability model is a conceptual framework that outlines the key
capabilities an organization needs to achieve its strategic objectives.
It provides a comprehensive view of what an organization can do and
helps identify areas for improvement or investment. In the context of an
orchestra, a capability model might help the orchestra identify the set
of skills and resources, or other types of capabilities it needs to
perform a symphony. Just like an orchestra needs well practiced
musicians, sheet music, instruments, a conductor, and an audience to
produce a great symphony, a State Medicaid Agency (SMA) needs its
Medicaid Enterprise System (MES) to employ or develop specific
capabilities to deliver its services effectively, efficiently, and
economically to its enrollees and providers.

The concept of a business capability is extensively used within
enterprise architecture modeling and has been broadly used within
Business Capability Models as a tool to better align the business
strategy and information technology of both private sector and
governmental organizations since they emerged in the mid-2000s. One
example comes from the TOGAF Standard, a well-known standard in
enterprise architecture. Like most architecture frameworks TOGAF defines
a capability as something a business can do to meet its goals. This
focuses a strategic lens of an organization on ``what'' it needs to
achieve its goals, rather than ``how'' those goals are achieved. This
perspective allows for business planning from different viewpoints,
facilitating strategic alignment and operational efficiency.

SMA business architects, technologists, systems analysts, executives,
managers, and program staff can use this same modeling approach to
represent the functional components of their Medicaid Enterprise System
(MES) in ways that can help reveal gaps in their systems and provide
insights on what new or enhanced capabilities might be needed to close
those gaps.

By focusing on capabilities, SMAs can better align their information and
technology resources and processes with their strategic business goals,
ultimately improving their insight into how to improve the outcomes
their Medicaid Enterprise Architecture produces.

\subsection{Purpose}\label{purpose-12}

\begin{tcolorbox}[enhanced jigsaw, toprule=.15mm, colback=white, colframe=quarto-callout-note-color-frame, left=2mm, arc=.35mm, opacityback=0, rightrule=.15mm, breakable, bottomrule=.15mm, leftrule=.75mm]
\begin{minipage}[t]{5.5mm}
\textcolor{quarto-callout-note-color}{\faInfo}
\end{minipage}%
\begin{minipage}[t]{\textwidth - 5.5mm}

\vspace{-3mm}\textbf{Note}\vspace{3mm}

MITA 4.0 does not endeavor to specify all of the capabilities SMA's may
need to administer Medicaid programs; instead, this version of MITA
focuses on the capabilities that are most closely oriented towards
achieving the CMS-required outcomes.

\end{minipage}%
\end{tcolorbox}

Understanding the how the MITA Capability Model works is important to
obtaining the most value out of many of the other tools and artifacts in
the MITA framework, such as the MITA Maturity Model (MMM) and the
Business Process Model (BPM). The MITA Capability Model provides a
structured way for SMAs to identify, conceptually model, and improve the
capabilities needed for efficient Medicaid operations.

It is important to note that MITA 4.0 does not endeavor to specify all
of the capabilities SMA's may need to administer Medicaid programs;
instead, this version of MITA focuses on the capabilities that are most
closely oriented towards achieving the CMS-required outcomes. In this
way MITA 4.0 provides a reference model for SMAs to model other
capabilities that may be needed to achieve their other goals such as
state specific outcomes, or other state priorities while providing more
guidance within the MITA Framework to support modular.

\subsection{Update to MITA 3.0}\label{update-to-mita-3.0-8}

MITA 3.0 defined a capability as the level of maturity of a set of
business processes within a business category. By focusing on ``how''
MES operate MITA 3.0 helped SMA's identify ways to improve and mature
their business processes, but it did not link those processes with the
outcomes they are intended to achieve or ensure better alignment of the
information and technical architectures to business outcomes. The
addition of the MITA capability model to the MITA 4.0 business
architecture addresses that by providing the conceptual linkages needed
to elevate the strategic vantage point of the MITA Framework. To guide
this change, we present within this chapter a definition, description,
and approach to modeling business capabilities, based on the widely used
capability models contextualized for Medicaid Enterprises.

The business processes that operationalize MITA capabilities remain
foundational to characterizing the business architecture, and are by
definition a constituent part of any MITA capability. They provide
essential information on how capabilities are operationalize and should
continue to be a routinely utilized reference model for SMA business
process mapping. They are found with in the Business Process Model
chapter of this version of MITA.

\section{The MITA Definition of
Capability}\label{the-mita-definition-of-capability-8}

Within the context of MITA, a capability can be defined as the ability
or capacity of a State Medicaid Agency to achieve a desired outcome in
compliance with the
\href{https://www.ecfr.gov/current/title-42/chapter-IV/subchapter-C/part-433/subpart-C/section-433.112}{Standards
and Conditions within 42 CFR 433.112}. A capability may currently exist
in an operational state or be envisioned for future development. Through
careful planning, capabilities defined in this way can be matured and
refined over time to become more effective and efficient. They can be
organized and detailed at various levels of abstraction, providing
precise descriptions for operational purposes or more generalized views
for strategic planning.

\begin{tcolorbox}[enhanced jigsaw, toprule=.15mm, colback=white, colframe=quarto-callout-note-color-frame, left=2mm, arc=.35mm, opacityback=0, rightrule=.15mm, breakable, bottomrule=.15mm, leftrule=.75mm]

\vspace{-3mm}\textbf{Key Definition}\vspace{3mm}

\ldots a capability is defined as the ability or capacity of a SMA to
achieve a desired outcome\ldots{}

\end{tcolorbox}

To fully define a business capability, it is essential to understand how
it is realized through the integration of people, processes,
information, and technology resources of an SMA. While these elements of
the capability can change regularly, the capability itself is should
endure over longer planning horizons, supporting the long-term alignment
of business and IT and the achievement of increasingly beneficial
business outcomes.

\subsection{Structure of the MITA Capability
Model}\label{structure-of-the-mita-capability-model-8}

As depicted in the model below, the MITA Capability Model orients the
people, process, technology, and information resources to define a MITA
Capability. This means that to model a capability the appropriate
components of the information architecture and the technical
architecture must be brought together with the business architecture to
fully formulate any MITA Capability.

\begin{figure}[H]

{\centering \pandocbounded{\includegraphics[keepaspectratio]{media/capabilityModel/topLevelCapabilityMetamodelGraphic1.png}}

}

\caption{MITA Capability Relationship Diagram}

\end{figure}%

\subsubsection{Business Roles}\label{business-roles-8}

Business roles represent individual actors, stakeholders, or partners
involved in delivering a business capability. A single organizational
group or team may be wholly responsible for delivering the capability,
or multiple business entities may share the delivery of a particular
business capability. Business Roles perform Business Processes using
Technology Resources. They require skills and knowledge resources to
achieve outcomes, and should be actively engaged as partners in the
development or enhancement of any capability they help deliver.

\subsubsection{Business Processes}\label{business-processes-8}

Individual business capabilities may be enabled or delivered through a
range of business processes that detail the activities (the how)
associated with delivering the capability. Identifying and analyzing the
efficiency of the underlying processes helps to optimize the business
capability's effectiveness. Identifying the processes within a business
capability provides a focus for maturing the capability in concert with
the other capability components. Business Processes operationalize
Business Capabilities.

\subsubsection{Information/Data}\label{informationdata-8}

Information/data represents the business data, knowledge, and insight
consumed or produced by the business capability (as distinct from
IT-related data entities). This may also include information that the
capability exchanges with other capabilities to support the execution of
value streams. Examples include information about customers and
prospects, products and services, business policies and rules, sales
reports, and performance metrics. Information/data inform the Business
Capability, answering questions and supporting business rules.

\subsubsection{Technology Resources}\label{technology-resources-8}

Business capabilities rely on a range of tools, applications, systems,
and services for successful execution. Technology Resources use
Information/data to facilitate Business Processes. Such resources may
include:

\begin{itemize}
\tightlist
\item
  Modular software applications

  \begin{itemize}
  \tightlist
  \item
    Cloud or on-premise infrastructure
  \item
    Microservices
  \item
    Analytics
  \item
    Customer portal
  \end{itemize}
\end{itemize}

In this way we can clearly interrelate all of the MITA architecture
models and their individual components which allows us to reveal gaps
not only in the individual components of the architecture, but also
understand their impact on the integration of the architecture
components at the capability level.

\subsection{Relationship of MITA Capabilities to
Outcomes}\label{relationship-of-mita-capabilities-to-outcomes-8}

In the context of the Medicaid Information Technology Architecture
(MITA), outcomes are intrinsically linked to capabilities, as they
represent the tangible results achieved through the effective
integration and execution of various elements that constitute a
capability. In this sense, outcomes and capabilities define each other.

\begin{figure}[H]

{\centering \pandocbounded{\includegraphics[keepaspectratio]{media/capabilityModel/topLevelCapabilityMetamodelOutcomes.png}}

}

\caption{MITA Capability and Outcome Relationship Diagram}

\end{figure}%

\subsubsection{Outcomes}\label{outcomes-8}

MITA defines outcomes broadly to encompass CMS-required outcomes,
state-specific outcomes, and other outcomes not mandated as part of the
Advance Planning Document (APD) process. The sole criterion for an
outcome to meet this definition is that it must be a goal of a State
Medicaid Agency (SMA) and be achieved through a Medicaid Enterprise
System (MES) capability.

\begin{tcolorbox}[enhanced jigsaw, toprule=.15mm, colback=white, colframe=quarto-callout-note-color-frame, left=2mm, arc=.35mm, opacityback=0, rightrule=.15mm, breakable, bottomrule=.15mm, leftrule=.75mm]

\vspace{-3mm}\textbf{Key Definition}\vspace{3mm}

A MITA outcome is a goal of a State Medicaid Agency (SMA) that is
achieved by a Medicaid Enterprise System (MES) capability.

\end{tcolorbox}

\subsubsection{Measure}\label{measure-8}

Measure is a quantifiable metric used to assess the effectiveness and
efficiency of capabilities within a Medicaid Enterprise System (MES).
Measures provide quantifiable and qualitative values that help State
Medicaid Agencies (SMAs) track progress toward achieving specific
outcomes, such as CMS-required or state-specific goals. These indicators
might include metrics like processing times, error rates, or compliance
levels.

Measures are a measurement threshold by establishing a specific value or
level that must be met or exceeded to demonstrate successful
performance. For instance, a KPI might set a threshold for the maximum
allowable processing time for claims, ensuring that they are handled
within a specified timeframe to maintain compliance and eligibility for
enhanced federal funding. By monitoring these thresholds, organizations
can ensure they are meeting regulatory requirements and delivering
high-quality services to beneficiaries, while also identifying areas for
improvement.

\subsubsection{Measure Threshold}\label{measure-threshold-8}

A specific value or level of a measure that must be met or exceeded to
demonstrate the effective achievement of a capability's intended
outcome. This threshold serves as a benchmark for assessing whether the
processes, roles, and resources integrated within a Medicaid Enterprise
System (MES) are functioning optimally to meet the goals of a State
Medicaid Agency (SMA). For example, a measurement threshold might be set
for processing times, where claims must be processed within a certain
number of days to ensure compliance with CMS-required outcomes and
maintain eligibility for enhanced federal funding. By establishing and
monitoring these thresholds, organizations can ensure they are meeting
regulatory requirements and delivering high-quality services to
beneficiaries.

\subsubsection{Measurement}\label{measurement-8}

These outcomes and metrics are also used to ensure that healthcare
systems or modules comply with applicable federal regulations, forming
the baseline for system or module functionality. Achieving these
outcomes is essential for continuing to receive enhanced federal funding
for operations. Regular measurement and analysis of KPIs help
organizations demonstrate compliance and effectiveness, ensuring that
they meet regulatory requirements and continue to deliver high-quality
services to beneficiaries.

In this way we can clearly interrelate all of the MITA architecture
models and their individual components with the KPIs, thresholds, and
measurements that indicate whether our capability achieves our desired
outcome.

While models that help conceptualize the capabilities that achieve
CMS-required outcomes are the ones modeled for this version of MITA,
SMAs are encouraged to use these models as a reference to model
capabilities.

\section{Capability Mapping}\label{capability-mapping-8}

Capability mapping is a strategic tool that enables organizations, such
as State Medicaid Agencies (SMAs), to systematically identify, organize,
and visualize the key capabilities necessary to achieve their
objectives. Within the MITA framework, capability mapping provides SMAs
with a method of developing comprehensive views of the functions and
processes required to deliver Medicaid services effectively. To begin
the capability mapping process, SMAs should first identify the core
capabilities that align with their strategic objectives, focusing on
what the organization needs to achieve rather than how those goals are
accomplished. This involves listing all necessary capabilities and
understanding the desired outcomes they support. Next, these
capabilities should be organized into domains and areas that reflect
their strategic importance and interrelationships. Visualizing these
capabilities through diagrams or maps provides all stakeholders a common
view to understand the roles, processes, technology resources, and
information/data involved in executing each capability, as well as the
outcome each capability is designed to achieve. This structured approach
not only highlights areas for improvement or investment but also ensures
that organizational efforts are strategically aligned with desired
outcomes.

The benefits of capability mapping are multifaceted, offering SMAs a
clear pathway to strategic alignment and gap analysis. By visualizing
capabilities, organizations can identify operational gaps and determine
what new or enhanced capabilities are needed to close those gaps. This
visualization also improves communication among stakeholders by
providing a clear and concise representation of the organization's
functions. To refine capabilities, SMAs should analyze current
operations, assess the efficiency of underlying processes, and optimize
them to enhance capability effectiveness. Additionally, capability
mapping serves as a foundation for heat mapping, which assesses the MITA
Framework will utilize to visualize the maturity of each capability
evaluated in the State Self-Assessment. SMAs can overlay heat maps over
their capability maps to visualize many things other than maturity
levels, using color coding to indicate areas of strength and weakness.
Regular updates to these maps allow SMAs to monitor progress and ensure
resources are allocated effectively to achieve strategic goals. The MITA
framework includes examples of capability maps based on CMS-required
outcomes, serving as a reference model for SMAs to develop their own
capability maps tailored to state-specific goals and priorities. By
leveraging the reference models provided by MITA, SMAs can ensure their
capability mapping efforts are aligned with both federal requirements
and state-specific priorities.

\subsection{Organizing Capabilities}\label{organizing-capabilities-8}

To enhance the resolution and detail of a capability and provide a
unified view of all its components, a block diagram can be employed to
provide a common view of any MES. This diagram effectively links the
capability to business processes, roles, technical resources, and
information resources through functional decomposition. By breaking down
the capability into its constituent parts, the block diagram offers a
visual representation that highlights the interrelationships and
dependencies among these elements. This approach provides a clearer
understanding of how each component contributes to the overall
capability, facilitating more effective analysis, optimization, and
alignment with organizational objectives.

\pandocbounded{\includegraphics[keepaspectratio]{media/capabilityModel/capabilityOgranizationModel.drawio.png}}

We use this same method to present an this top level view of the
capabilities required to achieve CMS-required outcomes. From this view
increasingly detailed models can be constructed.

\pandocbounded{\includegraphics[keepaspectratio]{media/capabilityModel/mesModuleBasedCapabilities.drawio.png}}

\subsection{MITA Capability Models}\label{mita-capability-models-8}

The MITA framework represents capabilities visually through a layered
model that represent a capability of being composed of sub-capabilities
and the processes, roles, information and technology resources (PRIT)
that support the business in sustaining the capability. Each layer up
depicts increasingly strategic capabilities and each layer down depicts
the constituent elements that compose a capability in increasing
operational detail. It is not the intention of this version of MITA to
provide a full operational or tactical view of a capability, though SMAs
may consider using this approach to improve their organizational
awareness of their operations by developing further layers of their
capabilities through functional decomposition.

\pandocbounded{\includegraphics[keepaspectratio]{media/capabilityModel/capabilityLevels.png}}

\begin{itemize}
\tightlist
\item
  \textbf{Capability Domains:} The first layer of this model aims to
  group capabilities to organize the strategic view of an SMA's
  capabilities. In this view one or many capabilities can be grouped
  within a domain to indicate the pursuit of common outcomes. Each
  domain is denoted with a single number to help annotate each
  capability.

  \begin{itemize}
  \tightlist
  \item
    \textbf{Capability Areas:} The second layer of this model aims to
    provide a view of the groups of capabilities that compose a domain.
    They are organized to show capabilities that serve a specific group
    of similar outcomes and essential
  \item
    \textbf{Capabilities:} The third layer of this model provides a more
    detailed view view of
  \end{itemize}
\end{itemize}

\pandocbounded{\includegraphics[keepaspectratio]{media/capabilityModel/capabilityLevels2.png}}

\subsection{Relationship of MITA Capabilities to
Maturity}\label{relationship-of-mita-capabilities-to-maturity-8}

\begin{tcolorbox}[enhanced jigsaw, toprule=.15mm, leftrule=.75mm, colframe=quarto-callout-warning-color-frame, left=2mm, arc=.35mm, titlerule=0mm, rightrule=.15mm, opacitybacktitle=0.6, bottomtitle=1mm, toptitle=1mm, colbacktitle=quarto-callout-warning-color!10!white, bottomrule=.15mm, title=\textcolor{quarto-callout-warning-color}{\faExclamationTriangle}\hspace{0.5em}{Warning}, opacityback=0, breakable, colback=white, coltitle=black]

Under development.

\end{tcolorbox}

\begin{itemize}
\tightlist
\item
  \textbf{Levels of Maturity}

  \begin{itemize}
  \tightlist
  \item
    Description of the five levels of maturity in the MITA framework
  \item
    How capabilities evolve and mature over time
  \end{itemize}
\end{itemize}

\pandocbounded{\includegraphics[keepaspectratio]{media/capabilityModel/maturityModel.png}}

\subsection{Using Capability Maps for Heat Mapping Strategic Priorities
and Identifying Gaps with the MITA Maturity
Model}\label{using-capability-maps-for-heat-mapping-strategic-priorities-and-identifying-gaps-with-the-mita-maturity-model-8}

Capability maps are powerful tools that not only provide a visual
representation of an SMA's key capabilities but also serve as a
foundation for strategic analysis and planning. There are many
approaches to heat mapping capabilities, each offering unique insights
into organizational priorities and gaps. Here, we describe two
approaches: assessing maturity levels using the MITA Maturity Model and
prioritizing strategic outcomes.

\subsubsection{Identifying Gaps with the MITA Maturity
Model}\label{identifying-gaps-with-the-mita-maturity-model-8}

The MITA Maturity Model provides a framework for assessing the maturity
of an organization's capabilities across various dimensions, such as
business processes, information, and technology. By integrating the
maturity model with capability maps, SMAs can identify gaps between
their current state and desired maturity levels.

\paragraph{Example 1: Identifying Gaps in Data Management Maturity Using
the PRIT
Model}\label{example-1-identifying-gaps-in-data-management-maturity-using-the-prit-model-8}

An SMA is conducting an assessment of its data management capabilities
using the MITA Maturity Model, with a focus on the PRIT (Processes,
Roles, Information, and Technology) framework. The capability map
includes various data-related capabilities, such as ``Data
Integration,'' ``Data Quality Management,'' and ``Data Analytics.'' Each
of these capabilities is evaluated across the PRIT dimensions to
determine their maturity levels using the revised scale:

Processes: Level 1: Ad-Hoc Roles: Level 2: Compliant Information: Level
2: Compliant Technology: Level 2: Compliant The capability map is
updated to reflect the maturity assessment, with each dimension marked
with a color code: red for Level 1: Ad-Hoc, yellow for Level 2:
Compliant, green for Level 3: Efficient, blue for Level 4: Optimized,
and purple for Level 5: Pioneering. This visualization helps the SMA
prioritize strategic actions to enhance the ``Data Integration''
capability, such as standardizing processes, refining roles, improving
data quality, and upgrading technology.

\subsubsection{Heat Mapping Strategic
Priorities}\label{heat-mapping-strategic-priorities-8}

Heat mapping involves applying a color-coded overlay to a capability map
to visually represent the status or priority level of each capability.
This technique can be used to highlight areas of strength, weakness, or
strategic importance. For example, capabilities that are critical to
achieving CMS-required outcomes might be marked in one color, while
those needing immediate attention or improvement could be marked in
another. This visual representation helps stakeholders quickly grasp the
strategic landscape and make informed decisions about where to allocate
resources and focus efforts.

\paragraph{Example 2: Prioritizing Capabilities for CMS-Required
Outcomes}\label{example-2-prioritizing-capabilities-for-cms-required-outcomes-8}

An SMA is focused on achieving specific CMS-required outcomes related to
improving patient care and reducing administrative costs. The agency
creates a capability map that outlines all the capabilities necessary to
meet these outcomes. By applying a heat map, the SMA highlights
capabilities that are directly linked to these outcomes in green,
indicating they are of high strategic priority. Capabilities that are
indirectly related or less critical are marked in yellow, while those
that are currently underperforming or not aligned with strategic goals
are marked in red.

This visual representation allows the SMA to quickly identify which
capabilities require immediate attention and resources to ensure
compliance with CMS requirements. For instance, if the capability
related to ``Claims Processing Efficiency'' is marked in red, the agency
can prioritize initiatives to enhance this capability, such as investing
in new technology or streamlining processes.

\subsubsection{Other Uses for Capability Heat
Mapping}\label{other-uses-for-capability-heat-mapping-8}

Beyond assessing maturity levels and prioritizing strategic initiatives,
capability heat mapping can be applied in various other contexts to
enhance organizational effectiveness and alignment.

\paragraph{Example 3: Aligning Capabilities with State-Specific
Initiatives}\label{example-3-aligning-capabilities-with-state-specific-initiatives-8}

An SMA is working on a state-specific initiative to enhance telehealth
services for rural populations. The capability map includes capabilities
related to telehealth, such as ``Telehealth Infrastructure,'' ``Provider
Engagement,'' and ``Patient Access.'' The SMA uses a heat map to
highlight these capabilities in blue, indicating their alignment with
the state-specific initiative.

By analyzing the capability map, the SMA identifies that ``Provider
Engagement'' is a critical capability that requires further development
to support the telehealth initiative. The agency decides to invest in
training programs and outreach efforts to engage providers in rural
areas, ensuring that the telehealth services are effectively delivered
to the target population.

These examples demonstrate how capability maps, combined with heat
mapping and the MITA Maturity Model, can provide valuable insights for
strategic planning and gap analysis. By visualizing priorities and
maturity levels, SMAs can make informed decisions about where to focus
resources and efforts, ultimately enhancing their Medicaid Enterprise
Systems and achieving strategic objectives.

\begin{itemize}
\tightlist
\item
  \textbf{Capability Mapping}

  \begin{itemize}
  \tightlist
  \item
    Introduction to capability mapping and its significance
  \item
    How capabilities are organized and detailed at various levels of
    abstraction
  \end{itemize}
\end{itemize}

\section{Guidance on reuse of the MITA Capability
Model}\label{guidance-on-reuse-of-the-mita-capability-model-8}

\begin{itemize}
\tightlist
\item
  \textbf{Practical Application}

  \begin{itemize}
  \tightlist
  \item
    How to integrate the capability model into daily operations and
    strategic planning
  \item
    Tips for maximizing the benefits of the model
  \end{itemize}
\item
  \textbf{Continuous Improvement}

  \begin{itemize}
  \tightlist
  \item
    Encouragement for ongoing assessment and refinement of capabilities
  \item
    Leveraging feedback and performance data for model enhancement
  \end{itemize}
\item
  \textbf{Implementation Guidance}

  \begin{itemize}
  \tightlist
  \item
    Steps for adopting the capability model
  \item
    Resources and support available for SMAs
  \end{itemize}
\item
  \textbf{Performance Monitoring and Reporting}

  \begin{itemize}
  \tightlist
  \item
    Role of the capability model in tracking and enhancing performance
  \item
    Use of metrics and standards to measure capability effectiveness
  \end{itemize}
\end{itemize}

\chapter{MITA Capability Model}\label{mita-capability-model-9}

\section{Introduction to Business Capability
Models}\label{introduction-to-business-capability-models-8}

A capability model is a conceptual framework that outlines the key
capabilities an organization needs to achieve its strategic objectives.
It provides a comprehensive view of what an organization can do and
helps identify areas for improvement or investment. In the context of an
orchestra, a capability model might help the orchestra identify the set
of skills and resources, or other types of capabilities it needs to
perform a symphony. Just like an orchestra needs well practiced
musicians, sheet music, instruments, a conductor, and an audience to
produce a great symphony, a State Medicaid Agency (SMA) needs its
Medicaid Enterprise System (MES) to employ or develop specific
capabilities to deliver its services effectively, efficiently, and
economically to its enrollees and providers.

The concept of a business capability is extensively used within
enterprise architecture modeling and has been broadly used within
Business Capability Models as a tool to better align the business
strategy and information technology of both private sector and
governmental organizations since they emerged in the mid-2000s. One
example comes from the TOGAF Standard, a well-known standard in
enterprise architecture. Like most architecture frameworks TOGAF defines
a capability as something a business can do to meet its goals. This
focuses a strategic lens of an organization on ``what'' it needs to
achieve its goals, rather than ``how'' those goals are achieved. This
perspective allows for business planning from different viewpoints,
facilitating strategic alignment and operational efficiency.

SMA business architects, technologists, systems analysts, executives,
managers, and program staff can use this same modeling approach to
represent the functional components of their Medicaid Enterprise System
(MES) in ways that can help reveal gaps in their systems and provide
insights on what new or enhanced capabilities might be needed to close
those gaps.

By focusing on capabilities, SMAs can better align their information and
technology resources and processes with their strategic business goals,
ultimately improving their insight into how to improve the outcomes
their Medicaid Enterprise Architecture produces.

\subsection{Purpose}\label{purpose-13}

\begin{tcolorbox}[enhanced jigsaw, toprule=.15mm, colback=white, colframe=quarto-callout-note-color-frame, left=2mm, arc=.35mm, opacityback=0, rightrule=.15mm, breakable, bottomrule=.15mm, leftrule=.75mm]
\begin{minipage}[t]{5.5mm}
\textcolor{quarto-callout-note-color}{\faInfo}
\end{minipage}%
\begin{minipage}[t]{\textwidth - 5.5mm}

\vspace{-3mm}\textbf{Note}\vspace{3mm}

MITA 4.0 does not endeavor to specify all of the capabilities SMA's may
need to administer Medicaid programs; instead, this version of MITA
focuses on the capabilities that are most closely oriented towards
achieving the CMS-required outcomes.

\end{minipage}%
\end{tcolorbox}

Understanding the how the MITA Capability Model works is important to
obtaining the most value out of many of the other tools and artifacts in
the MITA framework, such as the MITA Maturity Model (MMM) and the
Business Process Model (BPM). The MITA Capability Model provides a
structured way for SMAs to identify, conceptually model, and improve the
capabilities needed for efficient Medicaid operations.

It is important to note that MITA 4.0 does not endeavor to specify all
of the capabilities SMA's may need to administer Medicaid programs;
instead, this version of MITA focuses on the capabilities that are most
closely oriented towards achieving the CMS-required outcomes. In this
way MITA 4.0 provides a reference model for SMAs to model other
capabilities that may be needed to achieve their other goals such as
state specific outcomes, or other state priorities while providing more
guidance within the MITA Framework to support modular.

\subsection{Update to MITA 3.0}\label{update-to-mita-3.0-9}

MITA 3.0 defined a capability as the level of maturity of a set of
business processes within a business category. By focusing on ``how''
MES operate MITA 3.0 helped SMA's identify ways to improve and mature
their business processes, but it did not link those processes with the
outcomes they are intended to achieve or ensure better alignment of the
information and technical architectures to business outcomes. The
addition of the MITA capability model to the MITA 4.0 business
architecture addresses that by providing the conceptual linkages needed
to elevate the strategic vantage point of the MITA Framework. To guide
this change, we present within this chapter a definition, description,
and approach to modeling business capabilities, based on the widely used
capability models contextualized for Medicaid Enterprises.

The business processes that operationalize MITA capabilities remain
foundational to characterizing the business architecture, and are by
definition a constituent part of any MITA capability. They provide
essential information on how capabilities are operationalize and should
continue to be a routinely utilized reference model for SMA business
process mapping. They are found with in the Business Process Model
chapter of this version of MITA.

\section{The MITA Definition of
Capability}\label{the-mita-definition-of-capability-9}

Within the context of MITA, a capability can be defined as the ability
or capacity of a State Medicaid Agency to achieve a desired outcome in
compliance with the
\href{https://www.ecfr.gov/current/title-42/chapter-IV/subchapter-C/part-433/subpart-C/section-433.112}{Standards
and Conditions within 42 CFR 433.112}. A capability may currently exist
in an operational state or be envisioned for future development. Through
careful planning, capabilities defined in this way can be matured and
refined over time to become more effective and efficient. They can be
organized and detailed at various levels of abstraction, providing
precise descriptions for operational purposes or more generalized views
for strategic planning.

\begin{tcolorbox}[enhanced jigsaw, toprule=.15mm, colback=white, colframe=quarto-callout-note-color-frame, left=2mm, arc=.35mm, opacityback=0, rightrule=.15mm, breakable, bottomrule=.15mm, leftrule=.75mm]

\vspace{-3mm}\textbf{Key Definition}\vspace{3mm}

\ldots a capability is defined as the ability or capacity of a SMA to
achieve a desired outcome\ldots{}

\end{tcolorbox}

To fully define a business capability, it is essential to understand how
it is realized through the integration of people, processes,
information, and technology resources of an SMA. While these elements of
the capability can change regularly, the capability itself is should
endure over longer planning horizons, supporting the long-term alignment
of business and IT and the achievement of increasingly beneficial
business outcomes.

\subsection{Structure of the MITA Capability
Model}\label{structure-of-the-mita-capability-model-9}

As depicted in the model below, the MITA Capability Model orients the
people, process, technology, and information resources to define a MITA
Capability. This means that to model a capability the appropriate
components of the information architecture and the technical
architecture must be brought together with the business architecture to
fully formulate any MITA Capability.

\begin{figure}[H]

{\centering \pandocbounded{\includegraphics[keepaspectratio]{media/capabilityModel/topLevelCapabilityMetamodelGraphic1.png}}

}

\caption{MITA Capability Relationship Diagram}

\end{figure}%

\subsubsection{Business Roles}\label{business-roles-9}

Business roles represent individual actors, stakeholders, or partners
involved in delivering a business capability. A single organizational
group or team may be wholly responsible for delivering the capability,
or multiple business entities may share the delivery of a particular
business capability. Business Roles perform Business Processes using
Technology Resources. They require skills and knowledge resources to
achieve outcomes, and should be actively engaged as partners in the
development or enhancement of any capability they help deliver.

\subsubsection{Business Processes}\label{business-processes-9}

Individual business capabilities may be enabled or delivered through a
range of business processes that detail the activities (the how)
associated with delivering the capability. Identifying and analyzing the
efficiency of the underlying processes helps to optimize the business
capability's effectiveness. Identifying the processes within a business
capability provides a focus for maturing the capability in concert with
the other capability components. Business Processes operationalize
Business Capabilities.

\subsubsection{Information/Data}\label{informationdata-9}

Information/data represents the business data, knowledge, and insight
consumed or produced by the business capability (as distinct from
IT-related data entities). This may also include information that the
capability exchanges with other capabilities to support the execution of
value streams. Examples include information about customers and
prospects, products and services, business policies and rules, sales
reports, and performance metrics. Information/data inform the Business
Capability, answering questions and supporting business rules.

\subsubsection{Technology Resources}\label{technology-resources-9}

Business capabilities rely on a range of tools, applications, systems,
and services for successful execution. Technology Resources use
Information/data to facilitate Business Processes. Such resources may
include:

\begin{itemize}
\tightlist
\item
  Modular software applications

  \begin{itemize}
  \tightlist
  \item
    Cloud or on-premise infrastructure
  \item
    Microservices
  \item
    Analytics
  \item
    Customer portal
  \end{itemize}
\end{itemize}

In this way we can clearly interrelate all of the MITA architecture
models and their individual components which allows us to reveal gaps
not only in the individual components of the architecture, but also
understand their impact on the integration of the architecture
components at the capability level.

\subsection{Relationship of MITA Capabilities to
Outcomes}\label{relationship-of-mita-capabilities-to-outcomes-9}

In the context of the Medicaid Information Technology Architecture
(MITA), outcomes are intrinsically linked to capabilities, as they
represent the tangible results achieved through the effective
integration and execution of various elements that constitute a
capability. In this sense, outcomes and capabilities define each other.

\begin{figure}[H]

{\centering \pandocbounded{\includegraphics[keepaspectratio]{media/capabilityModel/topLevelCapabilityMetamodelOutcomes.png}}

}

\caption{MITA Capability and Outcome Relationship Diagram}

\end{figure}%

\subsubsection{Outcomes}\label{outcomes-9}

MITA defines outcomes broadly to encompass CMS-required outcomes,
state-specific outcomes, and other outcomes not mandated as part of the
Advance Planning Document (APD) process. The sole criterion for an
outcome to meet this definition is that it must be a goal of a State
Medicaid Agency (SMA) and be achieved through a Medicaid Enterprise
System (MES) capability.

\begin{tcolorbox}[enhanced jigsaw, toprule=.15mm, colback=white, colframe=quarto-callout-note-color-frame, left=2mm, arc=.35mm, opacityback=0, rightrule=.15mm, breakable, bottomrule=.15mm, leftrule=.75mm]

\vspace{-3mm}\textbf{Key Definition}\vspace{3mm}

A MITA outcome is a goal of a State Medicaid Agency (SMA) that is
achieved by a Medicaid Enterprise System (MES) capability.

\end{tcolorbox}

\subsubsection{Measure}\label{measure-9}

Measure is a quantifiable metric used to assess the effectiveness and
efficiency of capabilities within a Medicaid Enterprise System (MES).
Measures provide quantifiable and qualitative values that help State
Medicaid Agencies (SMAs) track progress toward achieving specific
outcomes, such as CMS-required or state-specific goals. These indicators
might include metrics like processing times, error rates, or compliance
levels.

Measures are a measurement threshold by establishing a specific value or
level that must be met or exceeded to demonstrate successful
performance. For instance, a KPI might set a threshold for the maximum
allowable processing time for claims, ensuring that they are handled
within a specified timeframe to maintain compliance and eligibility for
enhanced federal funding. By monitoring these thresholds, organizations
can ensure they are meeting regulatory requirements and delivering
high-quality services to beneficiaries, while also identifying areas for
improvement.

\subsubsection{Measure Threshold}\label{measure-threshold-9}

A specific value or level of a measure that must be met or exceeded to
demonstrate the effective achievement of a capability's intended
outcome. This threshold serves as a benchmark for assessing whether the
processes, roles, and resources integrated within a Medicaid Enterprise
System (MES) are functioning optimally to meet the goals of a State
Medicaid Agency (SMA). For example, a measurement threshold might be set
for processing times, where claims must be processed within a certain
number of days to ensure compliance with CMS-required outcomes and
maintain eligibility for enhanced federal funding. By establishing and
monitoring these thresholds, organizations can ensure they are meeting
regulatory requirements and delivering high-quality services to
beneficiaries.

\subsubsection{Measurement}\label{measurement-9}

These outcomes and metrics are also used to ensure that healthcare
systems or modules comply with applicable federal regulations, forming
the baseline for system or module functionality. Achieving these
outcomes is essential for continuing to receive enhanced federal funding
for operations. Regular measurement and analysis of KPIs help
organizations demonstrate compliance and effectiveness, ensuring that
they meet regulatory requirements and continue to deliver high-quality
services to beneficiaries.

In this way we can clearly interrelate all of the MITA architecture
models and their individual components with the KPIs, thresholds, and
measurements that indicate whether our capability achieves our desired
outcome.

While models that help conceptualize the capabilities that achieve
CMS-required outcomes are the ones modeled for this version of MITA,
SMAs are encouraged to use these models as a reference to model
capabilities.

\section{Capability Mapping}\label{capability-mapping-9}

Capability mapping is a strategic tool that enables organizations, such
as State Medicaid Agencies (SMAs), to systematically identify, organize,
and visualize the key capabilities necessary to achieve their
objectives. Within the MITA framework, capability mapping provides SMAs
with a method of developing comprehensive views of the functions and
processes required to deliver Medicaid services effectively. To begin
the capability mapping process, SMAs should first identify the core
capabilities that align with their strategic objectives, focusing on
what the organization needs to achieve rather than how those goals are
accomplished. This involves listing all necessary capabilities and
understanding the desired outcomes they support. Next, these
capabilities should be organized into domains and areas that reflect
their strategic importance and interrelationships. Visualizing these
capabilities through diagrams or maps provides all stakeholders a common
view to understand the roles, processes, technology resources, and
information/data involved in executing each capability, as well as the
outcome each capability is designed to achieve. This structured approach
not only highlights areas for improvement or investment but also ensures
that organizational efforts are strategically aligned with desired
outcomes.

The benefits of capability mapping are multifaceted, offering SMAs a
clear pathway to strategic alignment and gap analysis. By visualizing
capabilities, organizations can identify operational gaps and determine
what new or enhanced capabilities are needed to close those gaps. This
visualization also improves communication among stakeholders by
providing a clear and concise representation of the organization's
functions. To refine capabilities, SMAs should analyze current
operations, assess the efficiency of underlying processes, and optimize
them to enhance capability effectiveness. Additionally, capability
mapping serves as a foundation for heat mapping, which assesses the MITA
Framework will utilize to visualize the maturity of each capability
evaluated in the State Self-Assessment. SMAs can overlay heat maps over
their capability maps to visualize many things other than maturity
levels, using color coding to indicate areas of strength and weakness.
Regular updates to these maps allow SMAs to monitor progress and ensure
resources are allocated effectively to achieve strategic goals. The MITA
framework includes examples of capability maps based on CMS-required
outcomes, serving as a reference model for SMAs to develop their own
capability maps tailored to state-specific goals and priorities. By
leveraging the reference models provided by MITA, SMAs can ensure their
capability mapping efforts are aligned with both federal requirements
and state-specific priorities.

\subsection{Organizing Capabilities}\label{organizing-capabilities-9}

To enhance the resolution and detail of a capability and provide a
unified view of all its components, a block diagram can be employed to
provide a common view of any MES. This diagram effectively links the
capability to business processes, roles, technical resources, and
information resources through functional decomposition. By breaking down
the capability into its constituent parts, the block diagram offers a
visual representation that highlights the interrelationships and
dependencies among these elements. This approach provides a clearer
understanding of how each component contributes to the overall
capability, facilitating more effective analysis, optimization, and
alignment with organizational objectives.

\pandocbounded{\includegraphics[keepaspectratio]{media/capabilityModel/capabilityOgranizationModel.drawio.png}}

We use this same method to present an this top level view of the
capabilities required to achieve CMS-required outcomes. From this view
increasingly detailed models can be constructed.

\pandocbounded{\includegraphics[keepaspectratio]{media/capabilityModel/mesModuleBasedCapabilities.drawio.png}}

\subsection{MITA Capability Models}\label{mita-capability-models-9}

The MITA framework represents capabilities visually through a layered
model that represent a capability of being composed of sub-capabilities
and the processes, roles, information and technology resources (PRIT)
that support the business in sustaining the capability. Each layer up
depicts increasingly strategic capabilities and each layer down depicts
the constituent elements that compose a capability in increasing
operational detail. It is not the intention of this version of MITA to
provide a full operational or tactical view of a capability, though SMAs
may consider using this approach to improve their organizational
awareness of their operations by developing further layers of their
capabilities through functional decomposition.

\pandocbounded{\includegraphics[keepaspectratio]{media/capabilityModel/capabilityLevels.png}}

\begin{itemize}
\tightlist
\item
  \textbf{Capability Domains:} The first layer of this model aims to
  group capabilities to organize the strategic view of an SMA's
  capabilities. In this view one or many capabilities can be grouped
  within a domain to indicate the pursuit of common outcomes. Each
  domain is denoted with a single number to help annotate each
  capability.

  \begin{itemize}
  \tightlist
  \item
    \textbf{Capability Areas:} The second layer of this model aims to
    provide a view of the groups of capabilities that compose a domain.
    They are organized to show capabilities that serve a specific group
    of similar outcomes and essential
  \item
    \textbf{Capabilities:} The third layer of this model provides a more
    detailed view view of
  \end{itemize}
\end{itemize}

\pandocbounded{\includegraphics[keepaspectratio]{media/capabilityModel/capabilityLevels2.png}}

\subsection{Relationship of MITA Capabilities to
Maturity}\label{relationship-of-mita-capabilities-to-maturity-9}

\begin{tcolorbox}[enhanced jigsaw, toprule=.15mm, leftrule=.75mm, colframe=quarto-callout-warning-color-frame, left=2mm, arc=.35mm, titlerule=0mm, rightrule=.15mm, opacitybacktitle=0.6, bottomtitle=1mm, toptitle=1mm, colbacktitle=quarto-callout-warning-color!10!white, bottomrule=.15mm, title=\textcolor{quarto-callout-warning-color}{\faExclamationTriangle}\hspace{0.5em}{Warning}, opacityback=0, breakable, colback=white, coltitle=black]

Under development.

\end{tcolorbox}

\begin{itemize}
\tightlist
\item
  \textbf{Levels of Maturity}

  \begin{itemize}
  \tightlist
  \item
    Description of the five levels of maturity in the MITA framework
  \item
    How capabilities evolve and mature over time
  \end{itemize}
\end{itemize}

\pandocbounded{\includegraphics[keepaspectratio]{media/capabilityModel/maturityModel.png}}

\subsection{Using Capability Maps for Heat Mapping Strategic Priorities
and Identifying Gaps with the MITA Maturity
Model}\label{using-capability-maps-for-heat-mapping-strategic-priorities-and-identifying-gaps-with-the-mita-maturity-model-9}

Capability maps are powerful tools that not only provide a visual
representation of an SMA's key capabilities but also serve as a
foundation for strategic analysis and planning. There are many
approaches to heat mapping capabilities, each offering unique insights
into organizational priorities and gaps. Here, we describe two
approaches: assessing maturity levels using the MITA Maturity Model and
prioritizing strategic outcomes.

\subsubsection{Identifying Gaps with the MITA Maturity
Model}\label{identifying-gaps-with-the-mita-maturity-model-9}

The MITA Maturity Model provides a framework for assessing the maturity
of an organization's capabilities across various dimensions, such as
business processes, information, and technology. By integrating the
maturity model with capability maps, SMAs can identify gaps between
their current state and desired maturity levels.

\paragraph{Example 1: Identifying Gaps in Data Management Maturity Using
the PRIT
Model}\label{example-1-identifying-gaps-in-data-management-maturity-using-the-prit-model-9}

An SMA is conducting an assessment of its data management capabilities
using the MITA Maturity Model, with a focus on the PRIT (Processes,
Roles, Information, and Technology) framework. The capability map
includes various data-related capabilities, such as ``Data
Integration,'' ``Data Quality Management,'' and ``Data Analytics.'' Each
of these capabilities is evaluated across the PRIT dimensions to
determine their maturity levels using the revised scale:

Processes: Level 1: Ad-Hoc Roles: Level 2: Compliant Information: Level
2: Compliant Technology: Level 2: Compliant The capability map is
updated to reflect the maturity assessment, with each dimension marked
with a color code: red for Level 1: Ad-Hoc, yellow for Level 2:
Compliant, green for Level 3: Efficient, blue for Level 4: Optimized,
and purple for Level 5: Pioneering. This visualization helps the SMA
prioritize strategic actions to enhance the ``Data Integration''
capability, such as standardizing processes, refining roles, improving
data quality, and upgrading technology.

\subsubsection{Heat Mapping Strategic
Priorities}\label{heat-mapping-strategic-priorities-9}

Heat mapping involves applying a color-coded overlay to a capability map
to visually represent the status or priority level of each capability.
This technique can be used to highlight areas of strength, weakness, or
strategic importance. For example, capabilities that are critical to
achieving CMS-required outcomes might be marked in one color, while
those needing immediate attention or improvement could be marked in
another. This visual representation helps stakeholders quickly grasp the
strategic landscape and make informed decisions about where to allocate
resources and focus efforts.

\paragraph{Example 2: Prioritizing Capabilities for CMS-Required
Outcomes}\label{example-2-prioritizing-capabilities-for-cms-required-outcomes-9}

An SMA is focused on achieving specific CMS-required outcomes related to
improving patient care and reducing administrative costs. The agency
creates a capability map that outlines all the capabilities necessary to
meet these outcomes. By applying a heat map, the SMA highlights
capabilities that are directly linked to these outcomes in green,
indicating they are of high strategic priority. Capabilities that are
indirectly related or less critical are marked in yellow, while those
that are currently underperforming or not aligned with strategic goals
are marked in red.

This visual representation allows the SMA to quickly identify which
capabilities require immediate attention and resources to ensure
compliance with CMS requirements. For instance, if the capability
related to ``Claims Processing Efficiency'' is marked in red, the agency
can prioritize initiatives to enhance this capability, such as investing
in new technology or streamlining processes.

\subsubsection{Other Uses for Capability Heat
Mapping}\label{other-uses-for-capability-heat-mapping-9}

Beyond assessing maturity levels and prioritizing strategic initiatives,
capability heat mapping can be applied in various other contexts to
enhance organizational effectiveness and alignment.

\paragraph{Example 3: Aligning Capabilities with State-Specific
Initiatives}\label{example-3-aligning-capabilities-with-state-specific-initiatives-9}

An SMA is working on a state-specific initiative to enhance telehealth
services for rural populations. The capability map includes capabilities
related to telehealth, such as ``Telehealth Infrastructure,'' ``Provider
Engagement,'' and ``Patient Access.'' The SMA uses a heat map to
highlight these capabilities in blue, indicating their alignment with
the state-specific initiative.

By analyzing the capability map, the SMA identifies that ``Provider
Engagement'' is a critical capability that requires further development
to support the telehealth initiative. The agency decides to invest in
training programs and outreach efforts to engage providers in rural
areas, ensuring that the telehealth services are effectively delivered
to the target population.

These examples demonstrate how capability maps, combined with heat
mapping and the MITA Maturity Model, can provide valuable insights for
strategic planning and gap analysis. By visualizing priorities and
maturity levels, SMAs can make informed decisions about where to focus
resources and efforts, ultimately enhancing their Medicaid Enterprise
Systems and achieving strategic objectives.

\begin{itemize}
\tightlist
\item
  \textbf{Capability Mapping}

  \begin{itemize}
  \tightlist
  \item
    Introduction to capability mapping and its significance
  \item
    How capabilities are organized and detailed at various levels of
    abstraction
  \end{itemize}
\end{itemize}

\section{Guidance on reuse of the MITA Capability
Model}\label{guidance-on-reuse-of-the-mita-capability-model-9}

\begin{itemize}
\tightlist
\item
  \textbf{Practical Application}

  \begin{itemize}
  \tightlist
  \item
    How to integrate the capability model into daily operations and
    strategic planning
  \item
    Tips for maximizing the benefits of the model
  \end{itemize}
\item
  \textbf{Continuous Improvement}

  \begin{itemize}
  \tightlist
  \item
    Encouragement for ongoing assessment and refinement of capabilities
  \item
    Leveraging feedback and performance data for model enhancement
  \end{itemize}
\item
  \textbf{Implementation Guidance}

  \begin{itemize}
  \tightlist
  \item
    Steps for adopting the capability model
  \item
    Resources and support available for SMAs
  \end{itemize}
\item
  \textbf{Performance Monitoring and Reporting}

  \begin{itemize}
  \tightlist
  \item
    Role of the capability model in tracking and enhancing performance
  \item
    Use of metrics and standards to measure capability effectiveness
  \end{itemize}
\end{itemize}

\chapter{MITA Capability Model}\label{mita-capability-model-10}

\section{Introduction to Business Capability
Models}\label{introduction-to-business-capability-models-9}

A capability model is a conceptual framework that outlines the key
capabilities an organization needs to achieve its strategic objectives.
It provides a comprehensive view of what an organization can do and
helps identify areas for improvement or investment. In the context of an
orchestra, a capability model might help the orchestra identify the set
of skills and resources, or other types of capabilities it needs to
perform a symphony. Just like an orchestra needs well practiced
musicians, sheet music, instruments, a conductor, and an audience to
produce a great symphony, a State Medicaid Agency (SMA) needs its
Medicaid Enterprise System (MES) to employ or develop specific
capabilities to deliver its services effectively, efficiently, and
economically to its enrollees and providers.

The concept of a business capability is extensively used within
enterprise architecture modeling and has been broadly used within
Business Capability Models as a tool to better align the business
strategy and information technology of both private sector and
governmental organizations since they emerged in the mid-2000s. One
example comes from the TOGAF Standard, a well-known standard in
enterprise architecture. Like most architecture frameworks TOGAF defines
a capability as something a business can do to meet its goals. This
focuses a strategic lens of an organization on ``what'' it needs to
achieve its goals, rather than ``how'' those goals are achieved. This
perspective allows for business planning from different viewpoints,
facilitating strategic alignment and operational efficiency.

SMA business architects, technologists, systems analysts, executives,
managers, and program staff can use this same modeling approach to
represent the functional components of their Medicaid Enterprise System
(MES) in ways that can help reveal gaps in their systems and provide
insights on what new or enhanced capabilities might be needed to close
those gaps.

By focusing on capabilities, SMAs can better align their information and
technology resources and processes with their strategic business goals,
ultimately improving their insight into how to improve the outcomes
their Medicaid Enterprise Architecture produces.

\subsection{Purpose}\label{purpose-14}

\begin{tcolorbox}[enhanced jigsaw, toprule=.15mm, colback=white, colframe=quarto-callout-note-color-frame, left=2mm, arc=.35mm, opacityback=0, rightrule=.15mm, breakable, bottomrule=.15mm, leftrule=.75mm]
\begin{minipage}[t]{5.5mm}
\textcolor{quarto-callout-note-color}{\faInfo}
\end{minipage}%
\begin{minipage}[t]{\textwidth - 5.5mm}

\vspace{-3mm}\textbf{Note}\vspace{3mm}

MITA 4.0 does not endeavor to specify all of the capabilities SMA's may
need to administer Medicaid programs; instead, this version of MITA
focuses on the capabilities that are most closely oriented towards
achieving the CMS-required outcomes.

\end{minipage}%
\end{tcolorbox}

Understanding the how the MITA Capability Model works is important to
obtaining the most value out of many of the other tools and artifacts in
the MITA framework, such as the MITA Maturity Model (MMM) and the
Business Process Model (BPM). The MITA Capability Model provides a
structured way for SMAs to identify, conceptually model, and improve the
capabilities needed for efficient Medicaid operations.

It is important to note that MITA 4.0 does not endeavor to specify all
of the capabilities SMA's may need to administer Medicaid programs;
instead, this version of MITA focuses on the capabilities that are most
closely oriented towards achieving the CMS-required outcomes. In this
way MITA 4.0 provides a reference model for SMAs to model other
capabilities that may be needed to achieve their other goals such as
state specific outcomes, or other state priorities while providing more
guidance within the MITA Framework to support modular.

\subsection{Update to MITA 3.0}\label{update-to-mita-3.0-10}

MITA 3.0 defined a capability as the level of maturity of a set of
business processes within a business category. By focusing on ``how''
MES operate MITA 3.0 helped SMA's identify ways to improve and mature
their business processes, but it did not link those processes with the
outcomes they are intended to achieve or ensure better alignment of the
information and technical architectures to business outcomes. The
addition of the MITA capability model to the MITA 4.0 business
architecture addresses that by providing the conceptual linkages needed
to elevate the strategic vantage point of the MITA Framework. To guide
this change, we present within this chapter a definition, description,
and approach to modeling business capabilities, based on the widely used
capability models contextualized for Medicaid Enterprises.

The business processes that operationalize MITA capabilities remain
foundational to characterizing the business architecture, and are by
definition a constituent part of any MITA capability. They provide
essential information on how capabilities are operationalize and should
continue to be a routinely utilized reference model for SMA business
process mapping. They are found with in the Business Process Model
chapter of this version of MITA.

\section{The MITA Definition of
Capability}\label{the-mita-definition-of-capability-10}

Within the context of MITA, a capability can be defined as the ability
or capacity of a State Medicaid Agency to achieve a desired outcome in
compliance with the
\href{https://www.ecfr.gov/current/title-42/chapter-IV/subchapter-C/part-433/subpart-C/section-433.112}{Standards
and Conditions within 42 CFR 433.112}. A capability may currently exist
in an operational state or be envisioned for future development. Through
careful planning, capabilities defined in this way can be matured and
refined over time to become more effective and efficient. They can be
organized and detailed at various levels of abstraction, providing
precise descriptions for operational purposes or more generalized views
for strategic planning.

\begin{tcolorbox}[enhanced jigsaw, toprule=.15mm, colback=white, colframe=quarto-callout-note-color-frame, left=2mm, arc=.35mm, opacityback=0, rightrule=.15mm, breakable, bottomrule=.15mm, leftrule=.75mm]

\vspace{-3mm}\textbf{Key Definition}\vspace{3mm}

\ldots a capability is defined as the ability or capacity of a SMA to
achieve a desired outcome\ldots{}

\end{tcolorbox}

To fully define a business capability, it is essential to understand how
it is realized through the integration of people, processes,
information, and technology resources of an SMA. While these elements of
the capability can change regularly, the capability itself is should
endure over longer planning horizons, supporting the long-term alignment
of business and IT and the achievement of increasingly beneficial
business outcomes.

\subsection{Structure of the MITA Capability
Model}\label{structure-of-the-mita-capability-model-10}

As depicted in the model below, the MITA Capability Model orients the
people, process, technology, and information resources to define a MITA
Capability. This means that to model a capability the appropriate
components of the information architecture and the technical
architecture must be brought together with the business architecture to
fully formulate any MITA Capability.

\begin{figure}[H]

{\centering \pandocbounded{\includegraphics[keepaspectratio]{media/capabilityModel/topLevelCapabilityMetamodelGraphic1.png}}

}

\caption{MITA Capability Relationship Diagram}

\end{figure}%

\subsubsection{Business Roles}\label{business-roles-10}

Business roles represent individual actors, stakeholders, or partners
involved in delivering a business capability. A single organizational
group or team may be wholly responsible for delivering the capability,
or multiple business entities may share the delivery of a particular
business capability. Business Roles perform Business Processes using
Technology Resources. They require skills and knowledge resources to
achieve outcomes, and should be actively engaged as partners in the
development or enhancement of any capability they help deliver.

\subsubsection{Business Processes}\label{business-processes-10}

Individual business capabilities may be enabled or delivered through a
range of business processes that detail the activities (the how)
associated with delivering the capability. Identifying and analyzing the
efficiency of the underlying processes helps to optimize the business
capability's effectiveness. Identifying the processes within a business
capability provides a focus for maturing the capability in concert with
the other capability components. Business Processes operationalize
Business Capabilities.

\subsubsection{Information/Data}\label{informationdata-10}

Information/data represents the business data, knowledge, and insight
consumed or produced by the business capability (as distinct from
IT-related data entities). This may also include information that the
capability exchanges with other capabilities to support the execution of
value streams. Examples include information about customers and
prospects, products and services, business policies and rules, sales
reports, and performance metrics. Information/data inform the Business
Capability, answering questions and supporting business rules.

\subsubsection{Technology Resources}\label{technology-resources-10}

Business capabilities rely on a range of tools, applications, systems,
and services for successful execution. Technology Resources use
Information/data to facilitate Business Processes. Such resources may
include:

\begin{itemize}
\tightlist
\item
  Modular software applications

  \begin{itemize}
  \tightlist
  \item
    Cloud or on-premise infrastructure
  \item
    Microservices
  \item
    Analytics
  \item
    Customer portal
  \end{itemize}
\end{itemize}

In this way we can clearly interrelate all of the MITA architecture
models and their individual components which allows us to reveal gaps
not only in the individual components of the architecture, but also
understand their impact on the integration of the architecture
components at the capability level.

\subsection{Relationship of MITA Capabilities to
Outcomes}\label{relationship-of-mita-capabilities-to-outcomes-10}

In the context of the Medicaid Information Technology Architecture
(MITA), outcomes are intrinsically linked to capabilities, as they
represent the tangible results achieved through the effective
integration and execution of various elements that constitute a
capability. In this sense, outcomes and capabilities define each other.

\begin{figure}[H]

{\centering \pandocbounded{\includegraphics[keepaspectratio]{media/capabilityModel/topLevelCapabilityMetamodelOutcomes.png}}

}

\caption{MITA Capability and Outcome Relationship Diagram}

\end{figure}%

\subsubsection{Outcomes}\label{outcomes-10}

MITA defines outcomes broadly to encompass CMS-required outcomes,
state-specific outcomes, and other outcomes not mandated as part of the
Advance Planning Document (APD) process. The sole criterion for an
outcome to meet this definition is that it must be a goal of a State
Medicaid Agency (SMA) and be achieved through a Medicaid Enterprise
System (MES) capability.

\begin{tcolorbox}[enhanced jigsaw, toprule=.15mm, colback=white, colframe=quarto-callout-note-color-frame, left=2mm, arc=.35mm, opacityback=0, rightrule=.15mm, breakable, bottomrule=.15mm, leftrule=.75mm]

\vspace{-3mm}\textbf{Key Definition}\vspace{3mm}

A MITA outcome is a goal of a State Medicaid Agency (SMA) that is
achieved by a Medicaid Enterprise System (MES) capability.

\end{tcolorbox}

\subsubsection{Measure}\label{measure-10}

Measure is a quantifiable metric used to assess the effectiveness and
efficiency of capabilities within a Medicaid Enterprise System (MES).
Measures provide quantifiable and qualitative values that help State
Medicaid Agencies (SMAs) track progress toward achieving specific
outcomes, such as CMS-required or state-specific goals. These indicators
might include metrics like processing times, error rates, or compliance
levels.

Measures are a measurement threshold by establishing a specific value or
level that must be met or exceeded to demonstrate successful
performance. For instance, a KPI might set a threshold for the maximum
allowable processing time for claims, ensuring that they are handled
within a specified timeframe to maintain compliance and eligibility for
enhanced federal funding. By monitoring these thresholds, organizations
can ensure they are meeting regulatory requirements and delivering
high-quality services to beneficiaries, while also identifying areas for
improvement.

\subsubsection{Measure Threshold}\label{measure-threshold-10}

A specific value or level of a measure that must be met or exceeded to
demonstrate the effective achievement of a capability's intended
outcome. This threshold serves as a benchmark for assessing whether the
processes, roles, and resources integrated within a Medicaid Enterprise
System (MES) are functioning optimally to meet the goals of a State
Medicaid Agency (SMA). For example, a measurement threshold might be set
for processing times, where claims must be processed within a certain
number of days to ensure compliance with CMS-required outcomes and
maintain eligibility for enhanced federal funding. By establishing and
monitoring these thresholds, organizations can ensure they are meeting
regulatory requirements and delivering high-quality services to
beneficiaries.

\subsubsection{Measurement}\label{measurement-10}

These outcomes and metrics are also used to ensure that healthcare
systems or modules comply with applicable federal regulations, forming
the baseline for system or module functionality. Achieving these
outcomes is essential for continuing to receive enhanced federal funding
for operations. Regular measurement and analysis of KPIs help
organizations demonstrate compliance and effectiveness, ensuring that
they meet regulatory requirements and continue to deliver high-quality
services to beneficiaries.

In this way we can clearly interrelate all of the MITA architecture
models and their individual components with the KPIs, thresholds, and
measurements that indicate whether our capability achieves our desired
outcome.

While models that help conceptualize the capabilities that achieve
CMS-required outcomes are the ones modeled for this version of MITA,
SMAs are encouraged to use these models as a reference to model
capabilities.

\section{Capability Mapping}\label{capability-mapping-10}

Capability mapping is a strategic tool that enables organizations, such
as State Medicaid Agencies (SMAs), to systematically identify, organize,
and visualize the key capabilities necessary to achieve their
objectives. Within the MITA framework, capability mapping provides SMAs
with a method of developing comprehensive views of the functions and
processes required to deliver Medicaid services effectively. To begin
the capability mapping process, SMAs should first identify the core
capabilities that align with their strategic objectives, focusing on
what the organization needs to achieve rather than how those goals are
accomplished. This involves listing all necessary capabilities and
understanding the desired outcomes they support. Next, these
capabilities should be organized into domains and areas that reflect
their strategic importance and interrelationships. Visualizing these
capabilities through diagrams or maps provides all stakeholders a common
view to understand the roles, processes, technology resources, and
information/data involved in executing each capability, as well as the
outcome each capability is designed to achieve. This structured approach
not only highlights areas for improvement or investment but also ensures
that organizational efforts are strategically aligned with desired
outcomes.

The benefits of capability mapping are multifaceted, offering SMAs a
clear pathway to strategic alignment and gap analysis. By visualizing
capabilities, organizations can identify operational gaps and determine
what new or enhanced capabilities are needed to close those gaps. This
visualization also improves communication among stakeholders by
providing a clear and concise representation of the organization's
functions. To refine capabilities, SMAs should analyze current
operations, assess the efficiency of underlying processes, and optimize
them to enhance capability effectiveness. Additionally, capability
mapping serves as a foundation for heat mapping, which assesses the MITA
Framework will utilize to visualize the maturity of each capability
evaluated in the State Self-Assessment. SMAs can overlay heat maps over
their capability maps to visualize many things other than maturity
levels, using color coding to indicate areas of strength and weakness.
Regular updates to these maps allow SMAs to monitor progress and ensure
resources are allocated effectively to achieve strategic goals. The MITA
framework includes examples of capability maps based on CMS-required
outcomes, serving as a reference model for SMAs to develop their own
capability maps tailored to state-specific goals and priorities. By
leveraging the reference models provided by MITA, SMAs can ensure their
capability mapping efforts are aligned with both federal requirements
and state-specific priorities.

\subsection{Organizing Capabilities}\label{organizing-capabilities-10}

To enhance the resolution and detail of a capability and provide a
unified view of all its components, a block diagram can be employed to
provide a common view of any MES. This diagram effectively links the
capability to business processes, roles, technical resources, and
information resources through functional decomposition. By breaking down
the capability into its constituent parts, the block diagram offers a
visual representation that highlights the interrelationships and
dependencies among these elements. This approach provides a clearer
understanding of how each component contributes to the overall
capability, facilitating more effective analysis, optimization, and
alignment with organizational objectives.

\pandocbounded{\includegraphics[keepaspectratio]{media/capabilityModel/capabilityOgranizationModel.drawio.png}}

We use this same method to present an this top level view of the
capabilities required to achieve CMS-required outcomes. From this view
increasingly detailed models can be constructed.

\pandocbounded{\includegraphics[keepaspectratio]{media/capabilityModel/mesModuleBasedCapabilities.drawio.png}}

\subsection{MITA Capability Models}\label{mita-capability-models-10}

The MITA framework represents capabilities visually through a layered
model that represent a capability of being composed of sub-capabilities
and the processes, roles, information and technology resources (PRIT)
that support the business in sustaining the capability. Each layer up
depicts increasingly strategic capabilities and each layer down depicts
the constituent elements that compose a capability in increasing
operational detail. It is not the intention of this version of MITA to
provide a full operational or tactical view of a capability, though SMAs
may consider using this approach to improve their organizational
awareness of their operations by developing further layers of their
capabilities through functional decomposition.

\pandocbounded{\includegraphics[keepaspectratio]{media/capabilityModel/capabilityLevels.png}}

\begin{itemize}
\tightlist
\item
  \textbf{Capability Domains:} The first layer of this model aims to
  group capabilities to organize the strategic view of an SMA's
  capabilities. In this view one or many capabilities can be grouped
  within a domain to indicate the pursuit of common outcomes. Each
  domain is denoted with a single number to help annotate each
  capability.

  \begin{itemize}
  \tightlist
  \item
    \textbf{Capability Areas:} The second layer of this model aims to
    provide a view of the groups of capabilities that compose a domain.
    They are organized to show capabilities that serve a specific group
    of similar outcomes and essential
  \item
    \textbf{Capabilities:} The third layer of this model provides a more
    detailed view view of
  \end{itemize}
\end{itemize}

\pandocbounded{\includegraphics[keepaspectratio]{media/capabilityModel/capabilityLevels2.png}}

\subsection{Relationship of MITA Capabilities to
Maturity}\label{relationship-of-mita-capabilities-to-maturity-10}

\begin{tcolorbox}[enhanced jigsaw, toprule=.15mm, leftrule=.75mm, colframe=quarto-callout-warning-color-frame, left=2mm, arc=.35mm, titlerule=0mm, rightrule=.15mm, opacitybacktitle=0.6, bottomtitle=1mm, toptitle=1mm, colbacktitle=quarto-callout-warning-color!10!white, bottomrule=.15mm, title=\textcolor{quarto-callout-warning-color}{\faExclamationTriangle}\hspace{0.5em}{Warning}, opacityback=0, breakable, colback=white, coltitle=black]

Under development.

\end{tcolorbox}

\begin{itemize}
\tightlist
\item
  \textbf{Levels of Maturity}

  \begin{itemize}
  \tightlist
  \item
    Description of the five levels of maturity in the MITA framework
  \item
    How capabilities evolve and mature over time
  \end{itemize}
\end{itemize}

\pandocbounded{\includegraphics[keepaspectratio]{media/capabilityModel/maturityModel.png}}

\subsection{Using Capability Maps for Heat Mapping Strategic Priorities
and Identifying Gaps with the MITA Maturity
Model}\label{using-capability-maps-for-heat-mapping-strategic-priorities-and-identifying-gaps-with-the-mita-maturity-model-10}

Capability maps are powerful tools that not only provide a visual
representation of an SMA's key capabilities but also serve as a
foundation for strategic analysis and planning. There are many
approaches to heat mapping capabilities, each offering unique insights
into organizational priorities and gaps. Here, we describe two
approaches: assessing maturity levels using the MITA Maturity Model and
prioritizing strategic outcomes.

\subsubsection{Identifying Gaps with the MITA Maturity
Model}\label{identifying-gaps-with-the-mita-maturity-model-10}

The MITA Maturity Model provides a framework for assessing the maturity
of an organization's capabilities across various dimensions, such as
business processes, information, and technology. By integrating the
maturity model with capability maps, SMAs can identify gaps between
their current state and desired maturity levels.

\paragraph{Example 1: Identifying Gaps in Data Management Maturity Using
the PRIT
Model}\label{example-1-identifying-gaps-in-data-management-maturity-using-the-prit-model-10}

An SMA is conducting an assessment of its data management capabilities
using the MITA Maturity Model, with a focus on the PRIT (Processes,
Roles, Information, and Technology) framework. The capability map
includes various data-related capabilities, such as ``Data
Integration,'' ``Data Quality Management,'' and ``Data Analytics.'' Each
of these capabilities is evaluated across the PRIT dimensions to
determine their maturity levels using the revised scale:

Processes: Level 1: Ad-Hoc Roles: Level 2: Compliant Information: Level
2: Compliant Technology: Level 2: Compliant The capability map is
updated to reflect the maturity assessment, with each dimension marked
with a color code: red for Level 1: Ad-Hoc, yellow for Level 2:
Compliant, green for Level 3: Efficient, blue for Level 4: Optimized,
and purple for Level 5: Pioneering. This visualization helps the SMA
prioritize strategic actions to enhance the ``Data Integration''
capability, such as standardizing processes, refining roles, improving
data quality, and upgrading technology.

\subsubsection{Heat Mapping Strategic
Priorities}\label{heat-mapping-strategic-priorities-10}

Heat mapping involves applying a color-coded overlay to a capability map
to visually represent the status or priority level of each capability.
This technique can be used to highlight areas of strength, weakness, or
strategic importance. For example, capabilities that are critical to
achieving CMS-required outcomes might be marked in one color, while
those needing immediate attention or improvement could be marked in
another. This visual representation helps stakeholders quickly grasp the
strategic landscape and make informed decisions about where to allocate
resources and focus efforts.

\paragraph{Example 2: Prioritizing Capabilities for CMS-Required
Outcomes}\label{example-2-prioritizing-capabilities-for-cms-required-outcomes-10}

An SMA is focused on achieving specific CMS-required outcomes related to
improving patient care and reducing administrative costs. The agency
creates a capability map that outlines all the capabilities necessary to
meet these outcomes. By applying a heat map, the SMA highlights
capabilities that are directly linked to these outcomes in green,
indicating they are of high strategic priority. Capabilities that are
indirectly related or less critical are marked in yellow, while those
that are currently underperforming or not aligned with strategic goals
are marked in red.

This visual representation allows the SMA to quickly identify which
capabilities require immediate attention and resources to ensure
compliance with CMS requirements. For instance, if the capability
related to ``Claims Processing Efficiency'' is marked in red, the agency
can prioritize initiatives to enhance this capability, such as investing
in new technology or streamlining processes.

\subsubsection{Other Uses for Capability Heat
Mapping}\label{other-uses-for-capability-heat-mapping-10}

Beyond assessing maturity levels and prioritizing strategic initiatives,
capability heat mapping can be applied in various other contexts to
enhance organizational effectiveness and alignment.

\paragraph{Example 3: Aligning Capabilities with State-Specific
Initiatives}\label{example-3-aligning-capabilities-with-state-specific-initiatives-10}

An SMA is working on a state-specific initiative to enhance telehealth
services for rural populations. The capability map includes capabilities
related to telehealth, such as ``Telehealth Infrastructure,'' ``Provider
Engagement,'' and ``Patient Access.'' The SMA uses a heat map to
highlight these capabilities in blue, indicating their alignment with
the state-specific initiative.

By analyzing the capability map, the SMA identifies that ``Provider
Engagement'' is a critical capability that requires further development
to support the telehealth initiative. The agency decides to invest in
training programs and outreach efforts to engage providers in rural
areas, ensuring that the telehealth services are effectively delivered
to the target population.

These examples demonstrate how capability maps, combined with heat
mapping and the MITA Maturity Model, can provide valuable insights for
strategic planning and gap analysis. By visualizing priorities and
maturity levels, SMAs can make informed decisions about where to focus
resources and efforts, ultimately enhancing their Medicaid Enterprise
Systems and achieving strategic objectives.

\begin{itemize}
\tightlist
\item
  \textbf{Capability Mapping}

  \begin{itemize}
  \tightlist
  \item
    Introduction to capability mapping and its significance
  \item
    How capabilities are organized and detailed at various levels of
    abstraction
  \end{itemize}
\end{itemize}

\section{Guidance on reuse of the MITA Capability
Model}\label{guidance-on-reuse-of-the-mita-capability-model-10}

\begin{itemize}
\tightlist
\item
  \textbf{Practical Application}

  \begin{itemize}
  \tightlist
  \item
    How to integrate the capability model into daily operations and
    strategic planning
  \item
    Tips for maximizing the benefits of the model
  \end{itemize}
\item
  \textbf{Continuous Improvement}

  \begin{itemize}
  \tightlist
  \item
    Encouragement for ongoing assessment and refinement of capabilities
  \item
    Leveraging feedback and performance data for model enhancement
  \end{itemize}
\item
  \textbf{Implementation Guidance}

  \begin{itemize}
  \tightlist
  \item
    Steps for adopting the capability model
  \item
    Resources and support available for SMAs
  \end{itemize}
\item
  \textbf{Performance Monitoring and Reporting}

  \begin{itemize}
  \tightlist
  \item
    Role of the capability model in tracking and enhancing performance
  \item
    Use of metrics and standards to measure capability effectiveness
  \end{itemize}
\end{itemize}

\part{Artifacts and References}

\chapter{Introduction to Standards and Conditions for Federal Financial
Participation
(FFP)}\label{introduction-to-standards-and-conditions-for-federal-financial-participation-ffp}

To ensure the effective use of Federal Financial Participation (FFP) at
the enhanced 90 percent rate, Medicaid systems must adhere to a set of
\textbf{standards} and meet specific \textbf{conditions} established by
the Centers for Medicare \& Medicaid Services (CMS). These standards
define the technical, operational, and compliance requirements for
system design and functionality, while the conditions outline
administrative, financial, and procedural obligations that states must
fulfill to qualify for funding. Together, these guidelines promote
efficiency, interoperability, accountability, and alignment with federal
regulations, ensuring that Medicaid systems effectively support program
goals and beneficiaries.

\subsection{\texorpdfstring{\textbf{Standards for Medicaid
Systems:}}{Standards for Medicaid Systems:}}\label{standards-for-medicaid-systems}

\begin{enumerate}
\def\labelenumi{\arabic{enumi}.}
\item
  \textbf{Efficiency and Effectiveness}: The system must demonstrate the
  ability to provide more efficient, economical, and effective
  administration of the State Medicaid plan.
\item
  \textbf{Compliance with Federal Standards}: The system must meet
  requirements, standards, and performance criteria outlined in the
  State Medicaid Manual and other applicable federal regulations.
\item
  \textbf{Medicare Compatibility}: The system must integrate with
  Medicare systems for eligibility verification and claims processing
  for dual-eligible individuals.
\item
  \textbf{Support for Quality Improvement}: The system must fulfill data
  requirements for quality improvement organizations under Title XI of
  the Act.
\item
  \textbf{Modular System Design}: The system must use modular, flexible
  development approaches, including open interfaces, exposed APIs, and
  separation of business rules.
\item
  \textbf{MITA Alignment}: The system must align with Medicaid
  Information Technology Architecture (MITA) standards to advance
  maturity in business, architecture, and data.
\item
  \textbf{Health IT Standards Compliance}: The system must comply with
  health IT standards, HIPAA regulations, accessibility standards, civil
  rights laws, and reporting protocols as required by federal law.
\item
  \textbf{Technology Sharing and Reuse}: The system must promote
  sharing, leveraging, and reuse of Medicaid technologies across States.
\item
  \textbf{Operational Efficiency}: The system must support accurate and
  timely processing, eligibility determinations, and effective
  communication with providers, beneficiaries, and stakeholders.
\item
  \textbf{Data Reporting and Transparency}: The system must produce
  transaction data, reports, and performance information to support
  program evaluation, operational improvement, and accountability.
\item
  \textbf{Interoperability}: The system must enable seamless
  coordination with the Marketplace, Federal Data Services Hub, health
  information exchanges, public health agencies, human services
  programs, and community organizations.
\item
  \textbf{MAGI-Based Functionality}: Eligibility and Enrollment (E\&E)
  systems must demonstrate acceptable MAGI-based functionality through
  performance testing.
\item
  \textbf{System Documentation}: Systems developed with 90 percent FFP
  must include documentation to ensure operability by various
  contractors or users.
\item
  \textbf{Alternate Hardware Strategies}: The State must consider
  strategies to minimize costs and challenges of operating software on
  alternate hardware or operating systems.
\end{enumerate}

\begin{center}\rule{0.5\linewidth}{0.5pt}\end{center}

\subsection{\texorpdfstring{\textbf{Conditions for Federal Financial
Participation
(FFP):}}{Conditions for Federal Financial Participation (FFP):}}\label{conditions-for-federal-financial-participation-ffp}

\begin{enumerate}
\def\labelenumi{\arabic{enumi}.}
\item
  \textbf{Advance Planning Document (APD) Approval}: The APD must be
  approved by CMS prior to the State incurring expenditures for system
  design, development, installation, or enhancement.
\item
  \textbf{State Ownership of Software}: The State must retain ownership
  of any software developed, installed, or enhanced with 90 percent FFP.
\item
  \textbf{Federal Licensing Rights}: The Department must have a
  royalty-free, non-exclusive, irrevocable license to use and authorize
  others to use software and documentation developed with 90 percent
  FFP.
\item
  \textbf{Cost Compliance}: System costs must adhere to federal cost
  principles outlined in 45 CFR 75, Subpart E.
\item
  \textbf{Commitment to System Use}: The State must agree in writing to
  use the system for the period specified in the approved APD or as
  determined by CMS.
\item
  \textbf{Data Safeguarding}: The State must ensure system data is
  safeguarded in compliance with federal privacy and security
  regulations.
\item
  \textbf{Failure Mitigation Plans}: The State must submit plans to
  address operational risks and consequences of failing to meet major
  milestones or functionality requirements.
\item
  \textbf{Personnel Identification}: The APD must identify key State
  personnel assigned to the project, including their roles and time
  commitments.
\item
  \textbf{Compliance with Additional Requirements}: The State must
  adhere to other statutory and regulatory requirements issued through
  formal guidance as determined by the Secretary.
\item
  \textbf{Eligibility for FFP}: FFP at 90 percent is available only for
  costs incurred for goods and services provided on or after April 19,
  2011.
\item
  \textbf{COTS Software Costs}: Costs for commercial off-the-shelf
  (COTS) software, including licensing, installation, configuration, and
  integration, must be expressly described in the approved APD.
\end{enumerate}

\begin{center}\rule{0.5\linewidth}{0.5pt}\end{center}

\subsection{\texorpdfstring{\textbf{Summary:}}{Summary:}}\label{summary-1}

\begin{itemize}
\tightlist
\item
  \textbf{Standards} focus on system design, functionality,
  interoperability, compliance, and operational efficiency.
\item
  \textbf{Conditions} outline administrative, financial, and procedural
  requirements for receiving FFP at the enhanced 90 percent rate.
\end{itemize}

\part{Help Improve this Content}




\end{document}
